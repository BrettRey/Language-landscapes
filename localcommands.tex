\newcommand*{\orcid}{}
\newcommand{\refp}[1]{(\ref{#1})}

\renewcommand{\ULdepth}{1.8pt}
\contourlength{0.8pt}

\newcommand{\myuline}[1]{%
  \uline{\phantom{#1}}%
  \llap{\contour{white}{#1}}}%

%----------------------------------------------------------------------
% Node labels in CGEL trees are defined with \Node, 
% which is defined so that \Node{Abcd}{Xyz} yields 
% a label with the function Abcd on the top, in small
% sanserif font, followed by a colon, and the category 
% Xyz on the bottom.
\newcommand{\Node}[2]{\small\textsf{#1:}\\{#2}}
% For commonly used functions this is defined with \(function)
\newcommand{\Head}[1]{\Node{Head}{#1}}
\newcommand{\Subj}[1]{\Node{Subj}{#1}}
\newcommand{\Comp}[1]{\Node{Comp}{#1}}
\newcommand{\Mod}[1]{\Node{Mod}{#1}}
\newcommand{\Det}[1]{\Node{Det}{#1}}
\newcommand{\PredComp}[1]{\Node{PredComp}{#1}}
\newcommand{\Crd}[1]{\Node{Coordinate}{#1}}
\newcommand{\Mk}[1]{\Node{Marker}{#1}}
\newcommand{\Obj}[1]{\Node{Obj}{#1}}
\newcommand{\Sup}[1]{\Node{Supplement}{#1}}

% Define high-contrast, colorblind-friendly colors
\definecolor{xGreen}{RGB}{27,158,119}  
\definecolor{xOrange}{RGB}{217,95,2}  
\definecolor{xPurple}{RGB}{117,112,179}
\definecolor{xPink}{RGB}{231,41,138} 

\newcommand{\am}[1]{{\ethiopic #1}}
\newcommand{\fa}[1]{{\persian #1}}

\newcommand{\annotate}[2]{
\uline{#1}\marginpar{\footnotesize\raggedright #2}}
