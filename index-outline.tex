abbreviation
!and plurals
!initialisms (see: initialism)
ability (modality)
absolute adjective
absolute construction
absolute grade
abstention (verb of)
abstract case
abstract noun
accent
!rhotic vs non-rhotic
accomplishment
accusative + infinitive
accusative case
!and subject
achievement
acronym
across the board rule
act-related (adjunct, etc.)
!position of
!subjective vs volitional
action
active voice
activity
!and prepositional meaning
addition
!and DP structure
!expressions of
!and numerals
addressee
adjectivalisation
adjective, adjective phrase (AdjP)
!adjective distinguished from:
!!adverb
!!determinative
!!noun
!!preposition
!!verb
!adjective/adjectival clause
!ascriptive
!attributive-only
!and clefts
!and comparison
!compound
!and conversion
!criterial properties
!functioning as
!!attributive
!!complement in NP
!!complement in PP
!!fused head
!!postpositive
!!predeterminer
!!predicative adjunct
!!predicative complement
!!subject
!!supplement
!never-attributive
!postpositive-only
!and preposing
!and raising/voice-neutrality
!semantic classes
!!absolute; age; colour; modal;
!!primacy; probability; shape;
!!size (adjective); worth
!structure of AdjP
!!complementation
!!!gaps in
!!modification of
!!→ descriptive; gradability; inflection (of
!!adjectives); limiting; participial adjective;
!!possessive; predicand (requirement)
AdjP → adjective, adjective phrase
adjunct
!and catenatives
!clause-oriented vs VP-oriented
!and constituent structure
!form of
!and gaps
!position of
!and preposition stranding
!and punctuation
!and relativisation
!semantic categories
!!act-related;
!!adversative; aspectual; cause; comitative;
!!comparative (adjunct); concession;
!!conditional; connective; degree; depictive;
!!direction; domain; duration; evaluative;
!!focusing modifier; frequency; instrument;
!!manner; means; metalinguistic; modal;
!!polarity; purpose; quantificational adjunct;
!!reason; recency; result; serial order;
!!situational adjunct; spatial extent; spatial
!!location; speech act-related; temporal
!!location
!→ bounding; complement; predicative; question;
!scope
adposition
adverb, adverb phrase (AdvP)
!adverb distinguished from:
!!adjective
!!determinative
!!preposition
!!pronoun
!adverb/adverbial clause
!and clefts
!and comparatives
!distinctive property
!functioning as
!!adjunct
!!complement in PP
!!complement in VP
!!focusing modifier
!!modifier
!!!in AdjP
!!!in AdvP
!!!in PP
!!peripheral modifier in NP
!!predicative complement (specifying)
!!supplement
!and grade
!heterogeneity of adverb category
!homonymy with adjectives
!morphology of adverbs
!position of
!structure of AdvP
!!complementation
!!modification
!→ inflection (of adverbs)
adverbial phrase
adversative (adjunct, etc.)
adversative coordination
adversity (and passives)
advice
!verbs, etc. of
AdvP adverb, adverb phrase
affirmation
affirmative (context, etc.)
affix, affixation
!Class I vs Class II affixes
!order of affixes
!→ negator; prefix; suffix; zero-affixation
affix-replacement
African American Vernacular English
afterthought
age
!adjective, etc.
!relation in NP
agent, agentivity
agentless passive
aggregate
agrammatical
agreement
!aspectual
!in case
!determiner-head
!pronoun-antecedent
!vs semantic compatibility
!subject-verb
!!and coordination
!!and existentials
!!and extraposition
!!and relatives
alternation
!in complementation
!→ morphological; morphophonological;
!phonological; spelling; voicing
alternative-additive determinative
alveolar plosive
ambiclipping
ambiguity
!vs indeterminacy
American English (AmE)
!regional variation
analysability (morphological)
anaphor
anaphora
!anticipatory
!first vs repeat mention
!and order of coordinates
!retrospective
!and stacking
!temporal
!and test for complements
!→ comparative, comparison
anaphoric chain
anaphoric marker
anchor
animacy
animal (noun/NP denoting)
answer (to question)
!and direction question
!and multi-variable question
!and polarity
!right answer
!→ response
antecedent
!integrated vs non-integrated
!missing
!realised by:
!!clause
!!AdjP/AdvP
!!nominal
!!!pro-nominal
antepenult
anteriority
!complex (vs simple)
apodosis
apology
apostrophe
appellation
appendage
apposition, appositive
!vs complement
!integrated
!marked, oblique
!and punctuation
!supplementary
appositional compound
appositive relative clause
approximate negator
approximating (degree)
approximation
!and coordination
!modifier in DP
!scaling modifier in NP
Arabic
argument (semantic)
article
!and strong forms
ascriptive compound
ascriptive vs specifying (construction, use of be, etc.)
asking (verb of)
aspect
!progressive
!system of
aspectual adjunct
!and polarity-sensitivity
!position of
aspectual verb
!lexical
aspectuality
!vs aspect
!and gerund-participial
!perfective vs imperfective
!and prepositions
assertion
assertive
assimilation
associated part (relation in NP)
asterisk
asymmetric coordination
asyndetic combinations of main clauses
asyndetic coordination coordination
asyndetic supplementation
atelic (situation, etc.)
attributive adjective
!associative
!of degree
!expressive
!locational
!modal
!particularising
!process-oriented
!quantifying
!and repetition/tautology
!structural constraints
!temporal
!transferred
attributive genitive
attributive modifier, etc.
attributive use of NP
attributive verb, VP
augmentative
Australian English (AusE)
authorial we
auxiliary verb
!analysis of (dependent vs catenative)
!combinations of
!and coordination
!core vs non-core uses
!definition
!distinctive properties
!and ellipsis
!and negation
!position of
!and position of adjuncts
!and preposing
!in reduced clause
!and relatives
!stranding of
!weak forms
!NICER diagnostics
!→ inflection (of auxiliaries); inversion
!(subject-auxiliary)
avoidance (verb, etc., of)
baby (noun/NP denoting)
baby-talk
back-clipping
back-formation
backgrounded element (in clefts)
backshift
!obligatory vs optional
backwards anaphora
bahuvrihi compound
bare
!comparative (complement); coordinate;
!existential clause; bare infinitival; passive clause;
!relative clause
bare NP
bare predication (in echo questions)
!and case
bare role NP
!functions of
base (morphological)
!bound vs free
base plural
be-passive
benefactive, beneficiary
!and relatives
benefit (and passives)
binary branching
bipartite (noun)
bivalent valency
blend (syntactic)
blending (morphological)
block quote
body-part
!noun/NP denoting
!relation in NP
!unexpressed
bold face
borrowing (of new words)
bound variable
boundary problems (in morphology)
bounded (situation)
bounding (adjunct, etc.)
!→ duration; frequency
bracket (square or round)
British English (BrE)
!regional variation
buildings (proper names of)
calendar time
Cambridge University
canonical vs non-canonical clause
capitalisation
cardinal numeral
!modification of
Carroll, Lewis
case
!double case-marking
!in fused relatives
!inflectional vs analytic
!and unbounded dependency
!→ accusative; coordination; dative; genitive;
!nominative
case agreement
cataphora
catenative (complement, construction, etc.)
catenative adjective
catenative verb
!as distinct type of complement
!for-complex
!genitive-complex
!oblique-complex
!plain-complex
!simple vs complex
!and understood subject
causation (verb of)
cause
!adjunct, etc.
!relation in NP
!→ purpose; reason
causer
central position (and adjuncts)
change of location
character (punctuation)
Churchill
citation
!of forms
citation from
clausal vs subclausal
!coordination; negation
clause
!vs sentence
!structure
!!adjuncts
!!complements
!→ canonical vs non-canonical clause; main
!clause; orientation of adjuncts; subordinate
!clause
clause polarity
!tests for
clause type
!and coordination
!and echo question
!and modality
!and punctuation
!and relatives
!in subordinate clauses
!and inversion (auxiliary)
cleft (clause)
!and demonstratives
!it-cleft
!!and case
!proper (vs pseudo-cleft)
!pseudo-cleft
!!basic vs reversed
!!and content clauses
!!vs it-cleft
!!and reflexives
clipping
clipping compound
clitic
!→ contraction; reduction of auxiliary
clock time
closed class
code
cognate object
cognition (verb of)
co-indexing
collection (relation in NP)
collective noun
collective (property of predicative)
colloquial style restrictions
colon
colour
!adjective, etc.
!relation in NP
combining form
!vs affix
!→ initial combining form
comitative
comma
!delimiting
!splice (see also: comma splice)
command
commentary
commercial name
commitment (verb of)
common noun
communication (verb of)
comparative, comparison
!and adjective-adverb homonymy
!and anaphora
!and case
!comparative adjunct
!comparative clause
!!vs content clause
!!reduction of
!!and tense/time
!!and unbounded dependency
!comparative complement
!!bare vs expanded
!!position of
!!single element
!comparative governor
!comparative grade
!!inflectional vs analytic
!comparative idioms
!comparative phrase
!correlative
!and degree determinatives
!in DP
!of equality
!and fused heads
!and indirect complements
!of inequality
!and manner adjuncts
!metalinguistic
!and modification
!non-scalar
!and NPIs
!and preposition stranding
!and prepositions
!and reanalysis
!and reflexives
!scalar
!!positive vs negative orientation
!!superiority vs inferiority
!set comparison
!term comparison
!!and constants vs variables
!!primary/secondary term
compass terms
complement, complementation
!in AdjP
!vs adjunct
!vs appositive
!in clause/VP
!!alternations
!!form of
!!position of
!external vs internal
!externalised vs internalised
!indirect
!vs modifier
!in NP
!!alternations
!!form of
!!position of
!oblique
!in PP
!!form of
!in VP (cross-ref: chapter scope)
complement clause
complement preposing
!of focus
!of non-focus
complex base
complex determinative
complex preposition
complex sentence
complex word
complex-intransitive (clause, etc.)
complex-transitive (clause, etc.)
!and content clauses
compliance (with directive)
composition
!modifier in NP
!relation in NP
compound, compounding
!coordinative
!and punctuation
!vs syntactic construction
compound adjective
compound base
compound determinative
compound noun
!noun-centred
!plural
!and reflexives
!verb-centred
compound preposition
compound sentence
compound verb
conative (verb)
concealed
!passive; question
concession, concessive adjunct/adverb
!form of
concessive affirmation
concessive implicature (in coordination)
concrete noun
condensed structure
condition (necessary vs sufficient)
conditional adjunct/construction
!and coordination
!fragment
!and futurate
!indirect
!and interrogatives
!inverted
!modifier in NP
!and NPI
!open
!and polarity
!remote
!with should
!and subjunctive
!and time, tense, modality
!→ exhaustive conditional
conditional implicature/interpretation
conjunct
conjunction
!logical
!part of speech
!subordinating
!as term for coordination
connective adjunct/adverb
!vs coordinator
!correlative
!and negation/polarity
!position of
!pure vs impure
consequence
!and coordination
consequent
consonant
!alternation
!doubling
!reduction
constituent structure
!in morphology
!→ tree (list of)
construction
consultation (verb of)
contact clause
containment (prepositional meaning)
content (relation in NP)
content clause
!and apposition
!and clause type
!and clefts
!vs comparative clause
!declarative
!!expandable
!!expanded
!!functions of
!!non-expandable
!!obligatory vs optional that
!!that declarative
!and ellipsis
!exclamative
!and extraposition
!and focusing modifier
!functioning as
!!adjunct
!!complement in AdjP
!!complement in AdvP
!!complement of like
!!complement in NP
!!complement in PP
!!complement in VP
!!indirect complement
!!subject
!!supplement
!interrogative
!(subordinate)
!and negation
!vs NP
!and passives
!and preposition stranding
!resultative
!→ comparative; subordinate clause
content-specifying supplement
contest (verb of)
continuative vs non-continuative perfect
contraction
!vs negative inflection
!of us
!→ clitic; weak form
contradictory (and negation)
contrary (and negation)
contrast
!and reflexives
control
controller
converse
conversion
!vs affixation
convey (meaning)
conveyed reaction (object of)
coordinate
!bare vs expanded
!expansion by modifier
!order of
!unlimited number of
coordination
!and agreement
!and alternative questions
!and analysis of auxiliaries
!and anaphora
!asymmetric (vs symmetric)
!asyndetic
!and bare NPs
!binary vs multiple
!and case
!clausal vs subclausal
!and clause type
!of clauses
!and comparatives
!vs complementation
!and constituent structure
!and content clauses
!correlative
!and deixis
!and determiners
!and discrete set interpretation of NPs
!and dislocation
!expanded
!of finite VPs
!and focusing modifiers
!and fused relatives
!and genitives
!of grammaticised words
!and idioms
!and imperatives
!of interrogative phrases
!joint vs distributive
!layered
!level of
!likeness between coordinates
!and modification
!and negation
!of nominals
!non-basic
!polysyndetic
!and punctuation
!range of
!and reciprocals
!and reduplication
!and reflexives
!and relativisation
!reversible vs irreversible
!and scope
!vs subordination
!of subordinators
!syndetic (vs asyndetic)
!and unbounded dependency
!of unlike categories
!at unlike levels
!and verb + preposition combinations
!of VPs
!of word-parts
!→ across the board rule; conjunction (logical);
!disjunction; marker of coordination
coordinator
!distinctive properties
!distinguished from:
!!connective adverb
!!preposition/subordinator
!position of
!sentence-initial
!in supplementation
!weak forms
copula, copular clause
core complement
core role (of interrogative items)
coreference
corpus
countability (count vs non-count nouns)
!and determiners
!test for
counter-expectation (verb, etc., of)
counterfactuality
counterpart
countries
!NPs denoting
!proper names of
covert mandative
creator (relation in NP)
current relevance
dance (noun/NP denoting)
dangling participle
Danish
dash
dates
dative case
dative of disadvantage
de-adjectival (noun, etc.)
declarative (clause)
!as directive
!vs imperative
!subordinate
!(declarative)
defective lexeme
defective verbs
definiens, definiendum
definition
definite article
!class use
!functions of
!non-referential use
definiteness (definite vs indefinite NPs)
!and existentials
!and genitives
!and presentationals
!and presupposition
!and relatives
definition
!of grammatical terms
!general vs language-particular
!definiens and definiendum NP interpretations
degree adjunct
!form of
!and negation
!position of
degree determinative
!and grade
!modification of
degree modification
!in comparatives
deictic centre
deictic expression
deictic marker
deictic shift
deixis
!and demonstratives
!and order of coordinates
!and reported speech
delayed right constituent
!in comparison
!in coordination
deliberative question
demand
demonstration
demonstrative determinative
!and clefts
!as fused head
denial
!verb, etc., of
denominal (verb, etc.)
denotation
deontic bias (in question)
deontic modality
deontic source
dependency relation
dependent
!internal vs external
!→ adjunct; complement; determiner;
!modification, modifier; predeterminer
dependent clause
dependent vs independent (forms/uses)
!genitive case
dephrasal compound
depiction (relation in NP)
depictive adjunct (vs catenative complement)
depictive predicative
derivation, derivative
derogatory term
description
!vs prescription
!vs theory
descriptive (NP interpretation)
descriptive adjective
descriptive genitive
descriptor
desententialisation
deserts (proper names of)
desiderative bias (in question)
determinative, determinative phrase (DP)
!and coordination
!criteria for
!dependents (DP structure)
!determinative distinguished from:
!!adjective
!!adverb
!!determiner
!!pronoun
!functioning as
!!determiner
!!fused head
!!marker of coordination
!!modifier/adjunct
!!predeterminer
!!predicative
!position of
!→ inflection (of determinatives)
determined (vs bare) NP
determiner
!basic
!and coordination
!and countability
!dependent, not head
!ellipsis of
!fused with head
!minor
!and noun-verb contrast
!and preposing
!and pro-form one
!and pronouns
deverbal (noun, etc.)
devoicing
diachronic description
diacritic
diaeresis
dialect
diffidence
digraph
dimensions of spatial extent
diminutive
direct object
direct reported speech
direction adjunct, etc.
!form of
direction of entailment
direction question
directive
!indirect/non-imperative
!and response
disambiguation
discontinuous constituents
discourse deixis
discrete set interpretation of NPs
disease (noun/NP denoting)
disjunction (logical)
!inclusive vs exclusive
disjunctive determinative
dislocation
!and case
displaced subject
!position of
distal (demonstrative, etc.)
distance
!noun/NP denoting
!and spatial extent
distinctiveness
distribution
distributive determinative
distributivity
ditransitive
!→ indirect object; transitivity
do-support
domain
!adjunct
!!position of
!modifier in NP
!noun/NP denoting
!and reciprocals
!and reflexives
donkey sentence
double case; genitive; negation; perfect;
!preposition (structure of PP); quotation
!mark
doubl-ing constraint
doubly remote construction
doubt (verb, etc., of)
downward entailing quantifier
DP→ determinative, determinative phrase
drink (noun/NP denoting)
Dryden
dual-gender noun
dual-transitivity
duality
dubitatives
dummy pronoun
dummy verb
duration (adjunct, etc.)
!bounding
!!overall vs terminal-point extent
!complement vs adjunct
!form of
!non-bounding
!position of
!and resultant state
durative (situation)
dvandva
dynamic modality
dynamic situation, dynamicity
e-deletion (spelling rule)
echoing
ellipsis
!of complement of lexical verbs
!and fused heads
!in imperatives
!in NP
!and preposition stranding
!vs pro-form
!in questions
!in VP
!→ auxiliary verb (stranding of); infinitival
!marker (stranding of); reduced
!clause/NP/VP; reduction
ellipsis points
ellipt
em-rule
embedding
!and presupposition
!→ subordinate clause
embellishment
emission (verb of)
emotion/attitude (verb of)
emotive modifier (of interrogative)
emphatic polarity
!in imperatives
-en form (of verb)
en-dash
en-rule
enclitic
end-attachment coordination
end position (and adjuncts)
endearment (term of)
ending
endocentric compound
endophora
endpoint (location, etc.)
English
!international
!Present-day
!Standard
!→ regional variation
entailment
!and ascriptive adjectives
!direction of
!and implicature
entreaty
epistemic adjunct
epistemic bias (in question)
epistemic modality
epistemic warrant
epithet
!as supplement
epitomisation
equal (relation in NP)
equative
errors (in speech)
established word
ethic dative
etymology
evaluative, evaluation
!adjunct, etc.
!!position of
!morphology
event (noun/NP denoting)
exception (modifier in NP)
exclamation
!non-exclamative form
!and statement
exclamation mark/point
exclamative
!clause
!!and extraposition
!!and inversion
!!and preposition fronting/stranding
!!subordinate
!!!distribution/functions of
!!!and focusing modifier
!!!vs open interrogative
!!and tags
!!and unbounded dependency
!!verbless
!word, phrase
exclusion (expressions of)
exclusive disjunction; person (1st)
exclusiveness
exhaustive conditional
!governed vs ungoverned
!and mood
!reduction of
!remote
!and tense/time
exhaustiveness
existential clause
!and agreement
!bare
!and case
!extended, extension
existential determinative
existential modality
existential perfect
existential quantification/quantifier
exocentric compound
expandable declarative clause
expansion of coordinate by modifier
experiencer (semantic role)
expletive
expository directive
express (meaning)
extendable achievement
extended existential
extent
!adjunct, etc.
!!form of
!overall vs terminal-point
!non-temporal
external dependent
external negation → negation (and modality)
externalisation (of object, etc.)
extranuclear (position)
!position, prenucleus
extraposed object
!as predicand
extraposed subject
!and content clauses
!as predicand
!and punctuation
extraposition
!in catenatives
!of NPs
!and passives
!and relatives
!vs right dislocation
factitive (theme)
factive (verb, etc.) factivity
failure (verb, etc., of)
false definite
familiarity status → information
FCF (final combining form)
feeling (relation in NP)
feminine (gender)
!and non-females
festival (noun/NP denoting)
final clause
final combining form (FCF)
finiteness (finite vs non-finite clause, etc.)
fish (noun/NP denoting)
focus
!and adjuncts
!informational vs scopal
!of negation
focus-frame
focusing modifier
!additive
!position of
!restrictive
food (noun/NP denoting)
foreclipping
foregrounded element (in clefts)
foreign plurals
form (relation in NP)
form-type (of non-finite clause)
formal style → style restrictions
formulaic frame
fossilised (expression, etc.)
fraction
fragment (clause)
free (vs fossilised) expression
free choice (meaning)
!and fused relatives
free direct speech
free indirect speech/style
free relative clause
French
frequency (adjunct, etc.)
!bounding vs non-bounding
!form of
!and negation
!in NP
!position of
!and progressive aspect
!and time
front position (in clause)
fronting
!and coordination
!of interrogative phrase
!of preposition/PP
!preposing
frozen (form, etc.)
full stop
!abbreviation use
!terminal use
function (grammatical)
!vs category
!derivative/secondary
fused-head construction
!and adjective
!types (partitive vs simple vs special)
fused relative (construction)
!and agreement
!and comparatives
!and exclamations
!vs interrogatives
!phrase vs clause
!and preposition fronting/stranding
!reduction of
!simple vs ever-type
futurate
!past
!progressive
future tense
future time, futurity
!→ conditional adjunct/construction
!(and time, tense, modality)
game (noun/NP denoting)
gap
!and antecedents
!in comparatives
!and factivity
!functions of
!location of
!parasitic
!and preposition stranding
!as subject
!and unbounded dependency
gapped coordination, gapping
!and case
!and negation
gender
!and agreement
!and morphology
!personal vs non-personal
general property (modifier in NP)
general question
generic (NP, etc.)
genitive case
!attributive
!and compound nouns
!and coordination
!and definiteness
!dependent vs independent
!descriptive
!and determiner/subject-determiner
!and dislocation
!double genitive
!and fused heads
!and gerund-participials
!head vs phrasal
!as inflectional case
!and locatives
!measure
!nominal
!and noun inflection
!oblique
!and ownership/possession
!and partitives
!predicative
!and pronouns
!and proper names
!and punctuation
!and reciprocals
!and reflexives
!and supplementation
genitive-complex (catenative)
geographical entity (noun/NP denoting)
German
Germanic
gerund
gerund-participial (clause, VP)
!as clause, not NP
!and coordination
!and exhaustive conditionals
!and existentials
!and extraposition
!and focusing modifier
!functioning as:
!!adjunct
!!catenative complement
!!complement vs non-complement
!!direct object
!!indirect object
!!modifier in NP
!!predicative complement
!!subject
!!supplement
!vs gerund-participle
!hollow
!and hybrid NP
!vs infinitival
!and negation
!and passives
!and present participle
!and progressive
!relative
!structure of
!subject of
gerund-participle (verb-form)
!in compounds
!and conversion
!functions of
!stranding of
gerundial noun
get-passive
goal
!form of
!relativisation of
!and state
govern
gradability, grading
grade
!inflectional vs analytic
grammar
grammatical function → function
grammatical word
grammaticalisation
grammaticised (word, etc.)
!and coordination
graphology
Great Vowel Shift
Greek
head
!headed vs non-headed construction
headless relative clause
heavy punctuation; weight
Hebrew
hendiadys
heterogeneity (of adverb category)
historic present
holistic interpretation
hollow clause
!gerund-participial
!and preposition stranding
!and unbounded dependency
homonymy
honorific
host
hot news
human property (relation in NP)
human topic (relation in NP)
hurting (verb of)
hybrid construction
hypallage
hypercorrection
hyphen
!hard vs soft
!lexical vs syntactic
!long vs ordinary
hyphen-character
hypocoristic
hyponymy (in compounds)
ICF initial combining form
identifiability
idiom
!anaphoric
!comparative
!and coordination
!and hollow clauses
!interrogative
!with it
!negative
!and NPIs
!and preposing
!prepositional
!verbal
!!and preposition stranding
illness (noun/NP denoting)
illocutionary force/meaning
!primary vs secondary
illocutionary verb
immediate constituent (IC)
imperative (clause)
!and conditionals
!and finiteness
!and focusing modifier
!and inflection
!let-imperatives
!!1st person inclusive
!!open
!and mood/modality
!and negation
!ordinary (vs let-imperatives)
!!and agentivity
!!as directive
!!and do
!!and passives
!!and subject
!and preposing
!and semantic role
!and subordination
!and tags
imperfective aspectuality
impersonal construction
implicature
!cancellation of
!conventional
!conversational
!scalar
imprecative retort
in situ (position in interrogatives)
inclusion (expressions of)
inclusive disjunction; person (1st)
incorporation of to
indefinite article
!functions of
independent clause
indeterminate (NP interpretation)
indexical
indexing act
indicative conditional
indicative mood
indicator
!in punctuation
!in supplementation
indirect complement complement
indirect object
!and clefts
!and gap
!position of
!relativisation
!ditransitive
indirect ostension
indirect question
indirect reported speech
indirect speech act
inference
inferiority
infinitival (clause, VP)
!bare
!and finiteness
!vs infinitive
!and negation
!and passive
!structure of
!to-infinitival
!!case of subject
!!as clause, not NP
!!and coordination
!!and direction questions
!!and existentials
!!and extraposition
!!functioning as:
!!!adjunct/modifier in clause
!!!catenative complement
!!!complement in AdjP
!!!complement in NP
!!!complement in PP
!!!complement in VP
!!!indirect complement
!!!modifier in NP
!!!object (preposed)
!!!predicative complement
!!!purpose adjunct
!!!subject
!!!supplement
!!vs gerund-participial
!!and imperative
!!as main clause
!!and position of adjuncts
!!vs PP
!!with and without subject
!!and verb inflection
!→ hollow clause; interrogative clause; relative
!clause; split infinitive
infinitival marker/subordinator
!stranding of
infinitive
!split infinitive
inflection
!of adjectives
!of adverbs
!vs analytic marking
!of auxiliaries
!!negative forms
!of determinatives
!and grade
!of nouns
!!genitive
!!plural
!of pronouns
!of verbs
!!3rd sg present
!!complex bases
!!and finiteness
!!gerund-participle
!!irregular
!!past participle
!!plain form
!!plain present tense
!!preterite
!!primary vs secondary forms
!!regular
!!traditional paradigm
inflectional form
inflectional morphology
!vs lexical word-formation
influence (verb of)
informal style
information familiarity status
!addressee-old vs new
!background knowledge
!discourse-old vs new
!old vs new
information packaging
!construction
informational focus → focus
-ing form (of verb)
inhabitant (noun/NP denoting)
inherent part (relation in NP)
initial combining form (ICF)
!→ archi; endo-; hyper-; hypo-; macro-;
!mega; meta-; micro-; peri
initialism
inquiry
instruction
instrument
!adjunct, etc.
!!form of
!!position of
!noun/NP denoting
integrated antecedent; apposition; relative clause;
!supplement
intensificatory repetition
intensificatory tautology
intensifier
intensifying meaning (of affix)
interjection
internal complement
!in clause
!!and content clauses
!in PP
internal dependent in NP
!fused with head
internal negation negation (and modality)
internalised complement
interpolation
interrogative clause
!closed
!!as primary clause type
!!exclamatory use
!!in exhaustive conditionals
!!and inversion
!!and polarity-sensitivity
!!and question-orientation
!complex-intransitive
!and coordination
!as directive
!and ellipsis
!gaps in
!infinitival
!!bare
!!and unbounded dependency
!and modal auxiliaries
!and NPIs
!open
!!as primary clause type
!!in exhaustive conditionals
!!and inversion
!!and preposition fronting/stranding
!!and subject
!!and unbounded dependency
!and polarity-sensitivity
!and presupposition
!and punctuation
!subordinate
!!and ellipsis
!!and focusing modifier
!!functioning
!!!in AdjP
!!!in AdvP
!!!as extraposed object
!!!as extraposed subject
!!!in NP
!!!as object
!!!in PP
!!!as predicative complement
!!!as subject
!!!as supplement
!!!in VP
!!vs NP
!!and passives
!!position of
!!and preposition stranding
!!reduction of
!!and tense/time
interrogative determinative
interrogative NP
interrogative pronoun
!and adjuncts
!and agreement
!and case
interrogative tag → tag
interrogative word, phrase
!and clitics
!modification of
intervening NP (in catenative construction)
intonation
intransitive → transitivity
intrusive /r/
invention (noun/NP denoting)
inversion
!be it
!in comparative clause
!subject-auxiliary
!!classification of constructions
!!and negation
!!vs subject postposing
!subject-dependent
inverted commas
invitation
Irish English
irrealis (mood)
irregular (inflection)
islands
!and gaps
!proper names of
Italian
italics
iteration
iterative (situation, etc.)
Japanese
Jespersen
job application
joint
!coordination (joint vs distributive)
justification of grammars
juxtaposed (vs hyphenated) base
kin
!noun/NP denoting
!relation in NP
labile ordering constraint
landmark
Latin
layered head (structure)
left dislocation
!vs preposing
let-imperative → imperative
letter (of alphabet)
lexeme
!variable vs invariable
lexical base (morphological)
lexical be
lexical category
lexical item
lexical modal (vs modal auxiliary)
lexical verb (vs auxiliary)
!and negation
lexical word-formation
lexicalisation
!and coordination
lexicon
liaison
licensing (of complements)
!of content clauses
!of exclamatives
!by idioms
!of interrogatives (subordinate)
!of mandatives
!multi-word
!vs semantic compatibility
light punctuation; weight
light verb
likeness between coordinates
liking (verb of)
limited duration→ situation (temporary)
limiting adjective
link (to prior discourse)
linking/r/
location
!in discourse
!relation in NP
!and word-formation
!→ change of location, spatial location, temporal
!location
locative (complement, etc.)
!and existentials
!and inversion
!and predicatives
!and preposing
!→ spatial location
logical subject
long passive passive
lower bound (in quantification)
main clause
!asyndetic combination of
!as complement
!as supplement
main verb
mandative (clause)
!covert
!and factivity
!functioning as
!!complement in AdjP
!!extraposed subject
!!internal complement
!!subject
!should-mandative
!subjunctive
!and tense
mandatory reflexive
manifestation (use of proper name)
manner (adjunct, etc.)
!and content clauses
!form of
!position of
!primary vs secondary
manufacture
!modifier in NP
!of words
marker of coordination
!form of
!position of
married relation (relation in NP)
masculine (gender)
mass noun
matched nouns (and bare NPs)
matrix (clause, etc.)
matrix licensing
maximal (degree)
meal (noun/NP denoting)
means (adjunct, etc.)
!form of
!position of
!relativisation of
measure (phrase, etc.)
measure genitive
media (noun/NP denoting)
meiosis
member (relation in NP)
memory (verb type)
metalinguistic adjunct
metalinguistic negation
metalinguistic use of language
metaphor (and locatives)
metonymy (and locatives)
middle intransitive
mid-interval
minimal (degree)
minor clause type
minor determiner
missing antecedent
modal adjective
modal adjunct/adverb
!position of
modal auxiliary
!and agreement
!distinctive properties
!and inflection
!and temporal specification
!defective
!as present tense
!→ auxiliary verb
modal harmony
modal modifier (in NP)
modal negation
modal preterite
modal remoteness
!→ doubly remote construction
modality
!adjunct modal adjunct
!and coordination
!degree of
!and futurity
!and infinitivals
!kind of
!!ability; deontic; dynamic;
!!epistemic; existential; necessity; possibility
!and mandatives
!vs mood
!and parentheticals
!strength of (strong, medium, weak)
!!semantic vs pragmatic
!subjective vs objective
!and word-formation
!→ negation; scope
mode of manufacture (modifier in NP)
moderate (degree)
modification, modifier
!in AdjP
!in AdvP
!in clause/VP
!and comparatives
!in DP
!modification of lexical base
!in NP
!!early vs residual
!!external vs internal
!!form of
!!as fused head
!!position of
!in PP
!→ degree modification; determinative; focusing
!modifier; predeterminer
money (noun/NP denoting)
monotransitive
!→ transitivity
monovalent
month (noun/NP denoting)
mood
!imperative
!(and inflection); indicative; irrealis;
!subjunctive
morphological alternation
morphological operation
morphological structure
morphology
morphophonological alternation
motional use of be
mountain ranges (proper names of)
mounting process
movement (verb of)
multal degree
multal quantification
multiple negation
multiple situation
!and duration
multiple-situation-bound (NP interpretation)
multiple terminals (punctuation)
multiplier
!functions of
multi-variable (interrogative/question)
music (noun/NP denoting)
musical instrument (noun/NP denoting)
mute e (and spelling alternations)
nation, nationality (noun/NP denoting)
natural source (relation in NP)
necessity (modal)
negation
!and adjuncts
!analytic vs synthetic
!and anaphora
!clausal vs subclausal
!and clefts
!contrastive
!and coordination
!with do
!double negative
!emphatic
!external vs internal negation and modality
!and focusing modifiers
!in imperatives
!increased specificity of
!and inversion
!metalinguistic (vs ordinary)
!and modality
!and morphology
!and non-finites
!non-verbal
!and presupposition
!and quantification
!and understood subject
!verbal
!!primary
!!secondary
!scope of
!→ focus; scope
negative-bound (NP interpretation)
negative clause
negative concord
negative determinative
negative interrogative
negative polarity item
negative retort
negative scaling modifier (in NP)
negative verb-form
negatively-oriented polarity-sensitive item (NPI)
!non-affirmative
negator
!absolute
!affixal
!approximate
!and NPIs
!position of
neo-classical compound
nested dependencies
neuter (gender)
!with animates
neutral (property of predicative, vs collective)
neutral question (vs biased)
neutral verb-form (vs negative)
new information information familiarity status
New Zealand English (NZE)
newspaper (proper names of)
nexus-question
NICE constructions
nominal clause
nominal relative clause
nominal (syntactic category)
!functions of
nominalisation
!action/state/process
nominative case
!and predicative
non-affirmative (context, etc.)
non-assertive
nonce-form, nonce-use
non-constituent coordination
non-core complement
non-core use of auxiliaries auxiliary verb
non-expandable declarative clause
non-finite clause
!and anaphora
!and clefts
!as directive
!and ellipsis
!functioning as
!!adjunct
!!catenative complement
!!complement in AdjP
!!complement in NP
!!complement in PP
!!indirect complement
!!modifier in clause (adjunct)
!!modifier in NP
!!object, extraposed object
!!subject, extraposed subject
!!supplement
!and futurity
!gaps in
!hollow
!and reflexives
!temporal interpretation
!→ finiteness, gerund-participial, infinitival,
!past-participial
non-focus
non-human property (relation in NP)
non-past tense
non-propositional meaning
non-truth-conditional meaning
non-punctual (situation, etc.)
non-referential → referentiality
non-specific (NP interpretation)
non-standard (dialect, etc.)
non-truth-conditional meaning
Norwegian
notional definition
noun, noun phrase (NP)
!classes of noun
!and clefts
!and conditionals
!and conversion
!and coordination
!distinctive properties
!and extraposition
!functions of noun/nominal
!!head of NP
!!modifier in NP
!functions of NP
!!adjunct
!!complement in AdjP
!!complement in clause
!!complement in NP
!!complement in PP
!!determiner
!!modifier in AdjP
!!modifier in AdvP
!!modifier in NP
!!modifier in PP
!!object
!!predicative
!!subject
!!subject-determiner
!!supplement
!!vocative
!and gaps
!and gradability
!noun distinguished from
!!adjective
!!verb
!noun used as adjective
!NP-bound interpretation
!semantic classes
!!abstract; animal; baby;
!!bipartite; body-part; buildings; collective;
!!commercial name; concrete; countability;
!!countries; dance; deserts; disease; distance;
!!domain; drink; event; festival; fish; food;
!!game; geographical entity; illness;
!!inhabitant; instrument; invention; islands;
!!kin; meal; media; money; month; mountain
!!ranges; music; musical instrument; nation,
!!nationality; newspaper; occupation;
!!particle; person; price; product name;
!!quantity; rate; relational; result; rivers; seas;
!!ship; species; status; time; transfer of
!!information; transport; weekday
!structure of NP
!!complementation
!!modification
!!order of elements
!→ compound noun; inflection (of nouns)
noun clause
NP
NPI
nucleus
!prenucleus
null complement anaphora
number
!and numerals
!singular vs plural
!!and coordination
!!neutralisation
!!and personal pronouns
!!in predicatives
!!respecification of
!→ agreement; plural; singular they
number-transparency
numeral
numerical (vs singulative) use of one
NZE New Zealand English
object
!in AdjP
!vs adjunct
!and case
!form of
!vs oblique
!position of
!vs predicative
!of preposition
!and subordinate clause
!unexpressed
!direct
!indirect
object postposing
objective modality; predicative complement
objective case
obligatory vs optional elements
oblique apposition; catenative; genitive; object;
!partitive; predicative
occupation (noun/NP denoting)
occurrence (vs state)
offer
Old English
Old French
old information
omission (verb, etc., of)
operator
opinion (verb, etc., of)
oppositeness (and word-formation)
opposition (and word-formation)
optative
oratio obliqua, oratio recta
order
!speech act (directive)
!syntactic (order/position of constituents)
!!of antecedent and anaphor
!!of auxiliaries
!!of coordinates
!!in imperatives
!!in NP structure
!!and scope
ordering (verb of)
ordering constraints in NP
ordinal numeral
!and fused heads
orientation
!of adjuncts to clause or VP
!of interrogatives to question or answer
!positive vs negative
!of predicative, etc., to subject or object
!and prepositional meaning
!time of
original (in clipping)
orthographic sentence
orthographic word
overall → duration (bounding); extent
override
!of agreement rule
!of polarity-sensitivity
!of reflexive rules
owner (relation in NP)
Oxford University
paradigm
paradigmatic relation
parallel structures
parasitic gap
parenthesis (punctuation)
parenthesised element
parenthetical
!interrogative
!and negative concord
part of speech
participial adjective
participial clause
participle
particle
!noun/NP denoting
!type of complement
!!position of
partitive (construction)
!with interrogatives
!oblique
!and relatives
!→ fused-head construction (types)
passive (clause, etc.)
!adjectival (vs verbal)
!and agent
!bare vs expanded
!be-passive vs get-passive
!and catenatives
!and comparatives
!concealed
!and content clauses
!and ditransitives
!!first vs second
!and imperatives
!and infinitivals
!long vs short
!and middle intransitives
!and non-finite clauses
!and object
!and past participle
!and polarity-sensitivity
!and scope
!and semantic roles
!and verbal idioms
!verbs restricted to
!→ internalised complement; prepositional
!passive
past-participial (clause/VP)
!and existentials
!and focusing modifier
!functioning as:
!!adjunct
!!catenative complement
!!complement in PP
!!modifier in NP
!!supplement
!and negation
past participle
!in compounds
!functions of
!passive vs perfect
!stranding of
past tense
!and modality
!primary vs secondary
!→ perfect; preterite
path (in change of location)
!form of
!relativisation of
patient
paucal degree
paucal determinative
paucal quantification
penult
perception (verb of)
percolation (upwards)
!and comparatives
!and exclamatives
!and interrogatives
!and mandatives
!and negation
!and number
!and questioned element
!and relatives
perfect (tense)
!and catenatives
!continuative vs non-continuative
!deictic use
!double
!experiential/existential
!in imperatives
!internal vs external
!and negation
!non-finite
!non-present
!omissibility of
!vs perfective
!and preposing
!of recent past
!resultative
!and scope
!with since
!and temporal specification
!→ backshift; scope
perfective aspectuality
performative (use of verb, etc.)
performer (relation in NP)
period (punctuation)
peripheral modifier (in NP)
perlocutionary effect
permission
!verb of
person
!grammatical system of
!!1st
!!!imperatives
!!!inclusive vs exclusive
!!2nd
!!3rd
!!and agreement
!!and order of coordinates
!noun/NP denoting
person deixis
personal determinative
personal gender
personal name
personal pronoun
!and case
!and dislocation
!ellipsis of
!as host
!and inversion
!and parasitic gaps
!with particles
!position of
perspective
!and deixis
!and prepositional meaning
!and reflexives
petrified (form, etc.)
phase
phoneme
phonological alternation
phonological modification
phonological reduction
phonological representation
phonologically-motivated compound
phonology
phrasal verb
phrase
picture noun (and reflexives)
plain case
plain-complex catenative
plain (vs embellished) clipping
plain form of verb
!stranding of
plain grade
plain present tense
plasticity of meaning
plea
pleonastic use
!of negative
!of rather
plural noun
!and compounds
!vs diminutive
!and genitive case
!and gerundial nouns
!and punctuation
!→ inflection of nouns; number; plural-only noun
plural-only noun
pluralia tantum
point location (prepositional meaning)
polarity concord
polarity (positive vs negative)
!adjunct
!and anaphora
!and order of coordinates
!→ negation; tag
polarity-sensitive item
!correspondences
!varying strength
!→ non-affirmative
Polish
politeness
polysemy
position (of constituents) → order (syntactic)
position/stance (verb of)
positive clause
positive grade
positively-oriented polarity-sensitive item (PPI)
possession
possessive adjective/pronoun
possibility (modal)
POSS-ing
postmodifier, postmodification
!and ellipsis
postposing
!of content clause
!in NP
!of relatives
!of subordinate interrogatives
!→ subject postposing
postposition
post-vocalic/r/
potential word
PP preposition, preposition phrase
pragmatic constraints (and information packaging)
pragmatics
precision
!modifier in DP
!scaling modifier in NP
predeterminer
!form of
!fused with head
predicand
!and number
!relation in NP
!requirement (test for adjectives)
predicate
!semantic (vs syntactic)
predication
!questioning of
predication property
predicative adjective/AdjP
predicative adjunct
predicative complement
!and case
!and clefts
!and coordination
!distinctive properties
!and distributivity
!and existentials
!form of
!and hollow clauses
!interrogative
!marked
!and number
!position of
!and quantification
!in PP
!and preposing in PP
!and questioning
!and reflexives
!and relativisation
!subjective vs objective
predicative genitive
predicative oblique
predicator
prefix
prenuclear position, prenucleus
!and grammatical function
preposing
!and anaphora
!and auxiliaries
!and clefts
!and content clauses
!and extraposition
!and locatives
!of PP
!in PP structure
!and preposition fronting/stranding
!and relatives
!and subject
!and unbounded dependency
preposition, preposition phrase (PP)
!and case
!and clefts
!and comparatives
!and compounds
!and coordination
!definition
!distinctive properties
!fronting vs stranding of preposition
!functions of PP
!!adjunct
!!complement in VP
!!dependent in AdjP
!!dependent in AdvP
!!dependent in DP
!!dependent in NP
!!dependent in PP
!!determiner
!!indirect complement
!!object
!!subject
!!supplement
!and gaps
!grammaticised
!idiomatic
!interrogative
!latent
!mobile vs fixed
!and negation
!and NPIs
!optional before interrogatives
!preposition and affix
!preposition distinguished from:
!!adjective
!!adverb
!!coordinator
!!gerund-participle
!!past participle
!!subordinator
!referential
!and reflexives
!and spatial location
!specified preposition
!structure of PP
!!complementation
!!!double complementation
!!modification
!!position of head
!and subjunctive
!transitive vs intransitive
!verb + preposition combinations
!weak forms
prepositional passive
prepositional verb
!and preposition stranding
prescriptive grammar and usage manuals
Present-day English
present participle
present perfect
!and backshift
present tense
!3rd sg
!and agreement
!for future time
!pragmatically unrestricted use
!vs non-past tense
!plain
!simple
!timeless use
!→ inflection (of verb)
presentational clause
presentational status
presupposition
!cancellation of
!and clefts
!existential
!pragmatic
!of questions (Q-A presuppositions)
!rejection of
preterit
preterite
!and agreement
!backshifted
!of modal auxiliaries
!modal remoteness use
!and past time (anteriority)
!simple
prevention (verb, etc., of)
price (noun/NP denoting)
primacy adjective
primary object
primary terminal (punctuation)
primary verb-form
probability (adjectives of)
process
!situation type
!syntactic
!in word-formation
processing (of language)
pro-clause
proclitic
product name
productivity
pro-form
!and ascriptive adjectives
!and polarity
progressive aspect
!and gerund-participials
!in imperatives
!interpretive use
!and inversion
!non-aspectual uses
!in non-finites
!after will
progressive aspectuality
!vs imperfectivity
prohibition (verb, etc., of)
promise
pro-nominal
pronominal adverb
pronominalisation
pronoun
!vs adverb
!and catenatives
!and coordination
!dependents of
!functions of
!and fused heads
!of laziness
!vs pro-form
!as subclass of noun
!weak forms
!anaphora; antecedent; inflection; personal
!pronoun; pro-form; reciprocal; reflexive
pro-NP
propensity
proper name
!and morphology
!and plural
!and relatives
!secondary use
!strong vs weak
proper noun
!and plural
!vs proper name
property (semantic role)
proportion (modifier in NP)
proportional construction
proportional quantification
proposition
!closed
!open
proposition affirmation
proposition assessment
proposition denial
proposition suspension
propositional content/meaning
propositional negation
pro-predicative
prosody → intonation, stress
protasis
prototypical (NP, etc.)
provenance (modifier in NP)
pro-verb
proximal (demonstrative, etc.)
!vs distal
!vs pre-/post-proximal
proximity (and agreement)
pseudo-passive
psychological state (verb of)
punctual (situation, etc.)
punctuation
!and coordination
!and indirect speech acts
!light vs heavy
!open vs closed
!and relatives
!rules of
!word-level
punctuation indicator
!functions of
punctuation mark
purportedly sex-neutral pronoun
purpose (adjunct, etc.)
!and clefts
!and content clauses
!form of
!and mood
!relativisation
Q-A presupposition
quantification
!and attributive adjective
!and existential clauses
!and frequency
!and negation
!and polarity-sensitivity
!in PP
!and predicatives
!and relatives
!and understood subject
!→ proportional; quantifier
quantificational adjunct
!position of
quantificational noun
quantifier
!existential; multal;
!paucal; quantification; universal
quantitative vs non-quantitative indefiniteness
quantity
!noun/NP denoting
!relation in NP
quasi-anaphora
quasi-modal use of be
quest (semantic role)
question
!and adjuncts
!alternative
!!vs polar question
!!presupposition of
!biased
!classification of
!closed
!concealed
!declarative
!direction
!echo
!!alternative
!!clarification type
!!vs ordinary question
!!polar
!!repetition type
!!variable
!embedded
!information
!interrogative
!kinds of
!neutral (vs biased)
!open
!ordinary (vs echo)
!polar
!!presupposition of
!polar-alternative
!and presupposition preservation
!rhetorical
!semantic vs pragmatic
!variable
!!presupposition of
question mark
question-orientation (vs answer-orientation)
quotation mark
!single vs double
raised complement
!object
!subject
raising adjective
raising verb
!raising to object (see also: raising verb)
ranking expression
rate (noun/NP denoting)
realignment (of functions/roles)
realisation, realise
reanalysis
reason (adjunct, etc.)
!form of
!and negation
!relativisation of
recency (adjunct)
recipient
!relation in NP
!semantic role
reciprocal (pronoun)
!implicit
!linear vs non-linear
!split vs compound
!symmetric vs asymmetric
recommendation
recursion
reduced clause
reduced NP
reduced VP
reducible clause
reduction
!abbreviation/contraction
!of auxiliary
!in correlative comparatives
!of exclamatives
!of exhaustive conditionals
!of fused relatives
!of interrogatives
!maximal finite
!and punctuation
!and questions
reduplication
reference
referent
referentiality (referential vs non-referential NP)
reflexive pronoun
!basic (vs override)
!complement use
!emphatic use
!functions of
!mandatory vs optional vs inadmissible
!omission of
!override
!as predicand
regional variation (AmE, BrE, etc.)
register
regular (inflection)
reinforcement (scaling modifier in NP)
relational noun
relative clause
!and AdjPs
!and agreement
!bare
!cleft
!vs comparative clause
!and conditionals
!vs content clause
!continuative use
!and definiteness
!and existentials
!formal types
!infinitival
!integrated
!non-wh
!and NPIs
!position of
!and preposition fronting/stranding
!relational types
!vs relative construction
!restrictive vs non-restrictive
!and superlative
!supplementary
!and tense/time
!and that
!and unbounded dependency
!wh
relative construction
relative degree
relative determinative
relative preposition
relative pronoun
!agreement with antecedent
!and case
!functions of
relative word, phrase
!with clausal antecedent
relativisation
relevance
remote, remoteness
!conditional; doubly
!remote; exhaustive conditional; modal
!remoteness
removal (prefix meaning)
reordering
repeated situation
repetition/iteration of words
reported speech
!direct vs indirect
!embedded vs non-embedded
reporting (verb of)
reporting frame
reproach
request
residual pre-head modifier
residue (in clipping)
resolution rule
respecification of number
response
!vs answer
!and ellipsis
!and polarity
!to question
restrictiveness (restrictive vs non-restrictive)
!focusing modifier; relative clause
restrictor
result
!adjunct, etc.
!!and content clauses
!!position of
!noun/NP denoting
!relation in NP
resultant state
resultative (predicative)
resumptive pronoun
reversal, reversative (and word-formation)
reversed pseudo-cleft
reversibility
!of coordinates
!in copular clauses
rhetorical question
rhotic vs non-rhotic accent
rhyming (in compounds)
right branching (structure)
right dislocation
right nonce-constituent coordination
rigid ordering constraint
rivers (proper names of)
Romance
root (in morphology)
root modality
Russian
Sanskrit
satisfaction of condition
saying (verb of) → communication
scalar affirmation
scalar change
scalar comparison → comparative
scalar implicature
scalar location
scaling modifier (in NP)
scopal focus
scope
!and adjuncts
!and comparatives
!and coordination
!having scope over
!and hyphens
!and interrogatives
!and linear order
!and modality
!and negation
!and only
!and passive
!and perfect tense
!and quantification
!and questions
!wide vs narrow
Scottish English
seas (proper names of)
secondary boundary marks
secondary ed formation
secondary object
secondary presupposition (of question)
secondary verb-form
segmental punctuation indicator
selection (and number)
selection restriction
selective (property)
semantic role
!→ agent;
!benefactive, beneficiary; causer;
!experiencer; factitive; goal; instrument;
!location/locative; path; patient; recipient;
!source; stimulus; theme
semantics
semicolon
sense (verb of) → perception
sentence
!and clause
!terminal
!type
serial order (adjunct, etc.)
!position of
serial state
set comparison comparative, comparison
sex-neutral pronoun
sexist language
shape (adjective denoting)
ship (noun/NP denoting)
short passive
sibilant
simple affirmation
simple agreement
simple base (morphologicial)
simple present tense / preterite
simple sentence
simple word
single-gender noun
single-variable interrogative/question
singular
singular they
singulary branching
singulative (vs numerical) use of one
situation
!kinds of
!singulary vs multiple
!→ accomplishment; achievement; activity;
!multiple situation; occurrence; process;
!punctual; state
situational adjunct
size
!adjective, etc., denoting
!and prepositional meaning
!relation in NP
slash (punctuation mark)
sloppy identity
sluicing
small capitals
small clause
solidus
sound emission (verb of)
source
!of agreement
!relation in NP
!semantic role
!!realisation
!!relativisation of
space (punctuation indicator)
Spanish
spatial extent (adjunct, etc.)
!form of
spatial location
!complement vs adjunct
!and deixis/anaphora
!and morphology
!and prepositional meaning
!and questions
!realisation
!and relatives
!temporal interpretation
speaker
special-purpose variety
specialised-modal construction
species (noun/NP denoting)
specifier
specifying (construction, etc.) → ascriptive vs
!specifying
speech act
!and felicity conditions
!→ illocutionary force/meaning; indirect speech
!act; performative
speech act-related
!adjunct
!!position of
speech and writing (relation between)
spelling
spelling alternation
spelling pronunciation
split antecedent
!and relatives
split infinitive
stacked modification, stacking
!and relatives
Standard English
starting-point (in change of location, etc.)
state
!and locational roles
!→ nominalisation; serial state
state-occurrence
statement
!and polarity
!and response
stative (verb, etc.)
status
!indication of
!marking (in punctuation)
!noun/NP denoting
!and prepositional meaning
stem
stimulus
!and echo question
!semantic role
stranding auxiliary verb; infinitival marker;
!preposition
strength hierarchy (punctuation)
stress, stressed vs unstressed forms
stress shift
stroke (punctuation mark)
strong forms
strong proper name
style restrictions (formal, informal, etc.)
subcategorisation
subclausal coordination; negation
subject
!as argument or non-argument
!and case
!and comparatives
!distinctive properties
!embedded (vs immediate)
!in exclamatives
!and familiarity status
!and gaps
!general definition
!and imperatives
!and passives
!position of
!and preposition stranding
!and punctuation
!realised by
!!AdjP
!!clause
!!NP
!!PP
!and reanalysis
!and reciprocals
!and reflexives
!and relatives
!and semantic role
!understood
!!controlled, raised or non-syntactic
!!interpretation
!→ displaced subject; dummy pronoun;
!extraposed subject
subject-determiner
!and clause subject
!vs PP
!and relatives
!→ genitive case
subject postposing
subjective
!act-related; modality; predicative
!complement
subjective case
subjectless non-finites
!→ subject (understood)
subjunctive (clause, etc.)
!distinctive properties
!and factivity
!and finiteness
!and inflection
!vs irrealis
!and mood/modality
!and negation
!in PP
!verb-form
subjunctive conditional
submodification
subordinate (relation in NP)
subordinate clause
!and anaphora
!classification
!and clause type
!as complement
!and ellipsis
!and modality
!non-embedded
!vs NP
!and passives
!and preposing
!and reflexives
!as subject
!and tags
!and tense/time
!→ comparative (clause); content clause;
!gerund-participial; infinitival; mandative;
!past-participial; relative clause;
!subjunctive
subordinating conjunction
subordination vs coordination
subordinative compound
subordinator
!and content clauses
!and coordination
!and infinitivals
!and interrogatives
!position of
!vs preposition
!and subordinating conjunctions
!weak forms
subset
!and ascriptive adjective
subtraction (modifier in NP)
sufficiency determinative
suffix
summation plural
superior (relation in NP)
superiority
superlative
!adverb
!and definiteness
!and fused heads
!grade
!!absolute vs relative
!!incorporated vs free
!!inflectional vs analytic
!and NPI
!phrase
!and prepositions
supplement, supplementation
!and agreement
!and apposition
!ascriptive vs specifying
!content-specifying
!and coordination
!vs dependent
!and genitive
!vs integrated construction
!interrogative
!and non-finite clauses
!position of
!predicative
!and punctuation
!and relatives
!and reported speech
!syndetic vs asyndetic
!→ relative clause
suppletion
support (prepositional meaning)
supportive do do-support
supreme example (relation in NP)
surface structure
surplus (in clipping)
syllabic consonant
syllepsis
symbol (written)
symmetric verb
synchronic description
syncretism
synonymy
syntactic category
syntagmatic relation
syntax
synthetic compound
system (grammatical)
taboo
tag (interrogative)
!constant polarity
!and exclamatives
!and imperatives
!and relatives
!reversed polarity
target (of agreement)
taste (and prescriptive grammar)
telic vs atelic (situation)
temporal deixis
temporal extent → duration
temporal inclusion (and coordination)
temporal interpretation of clause without primary
!tense
temporal location (adjunct, etc.)
!adjunct vs complement
!and aspectuality
!change of
!and existentials
!form of
!interval vs point
!and morphology
!position of
!and present perfect
!relativisation of
!and state vs occurrence
!as subject or object
!and tense/mood
temporal noun
!positional vs non-positional
temporal pronoun
temporal sequence (and coordination)
temporal specification in non-finites
temporary (situation)
tense
!absolute vs relative
!and clefts
!compound
!deictic vs non-deictic
!primary vs secondary
!system of
!vs time
!→ future tense; inflection (of verbs); past tense;
!perfect; present tense; preterite
tentative (use of modals)
term comparison → comparative, comparison
terminal-point extent
textual deixis
that declarative
theme
!semantic role
!primary vs secondary
!as term for topic
!→ factitive
theoretical framework
time
!deictic
!matrix
!noun/NP denoting
!of orientation
!referred to
!relation in NP
!of situation
!and tense
!semantic concept vs grammatical tense
!frequency; temporal location; temporal noun;
!temporal pronoun; tense
title
topic
topicalisation
total question
totality
tough movement
trajector
transfer of information (noun/NP denoting)
transferred epithet
transitivity (transitive vs intransitive)
!of clause, verb, etc.
!!and presentationals
!!and voice
!in logic
transport (noun/NP denoting)
tree (diagram)
!list of
trigraph
triple-gender noun
truncated it-cleft
truth conditions
!and negation
truth table
truth value
type
!modifier in NP
!relation in NP
ultimate constituent
ultimate head
unbounded dependency construction
!combinations
!major vs minor
unbounded dependency word
undergoer
!relation in NP
!semantic role
underlying structure
understatement
unfavourable evaluation (verb, etc., of)
uniqueness
universal determinative
universal personal pronoun
universal quantification/quantifier
unlimited state
un-passive
upper bound (in quantification)
upward entailing quantifier
upward percolation percolation
usage manuals
utterance
valency (monovalent, bivalent, trivalent)
value (semantic role)
variable
!bound; comparative
!(term comparison); lexeme; multi-variable;
!question (echo: variable); question
!(variable)
velar softening
verb, verb phrase (VP)
!attributive
!complementation of
!and coordination
!and conversion
!and focusing modifier
!and gaps
!and gradability
!as head of clause
!semantic classes of verb
!!abstention; advice;
!!asking; avoidance; causation; cognition;
!!commitment; communication; conative;
!!consultation; contest; counter-expectation;
!!denial; doubt; emission; emotion/attitude;
!!failure; hurting; influence; liking;
!!movement; omission; opinion; ordering;
!!perception; permission; position/stance;
!!prevention; prohibition; psychological state;
!!sound emission; unfavourable evaluation;
!!wanting
!verb distinguished from
!!adjective
!!noun
!verb + preposition combinations
!verb group
!VP deletion
!VP ellipsis
!VP preposing
!plain present (see present tense)
!→ compound verb; factive; inflection (of verbs);
!orientation of adjuncts; passive (verbs
!restricted to); performative; stative;
!transitivity
verbalisation
verbless clause
!and case
!comparative
!as directive
!exclamative
!interrogative
!and negation
!in PP
!as supplement
virgule (punctuation)
vocative
!and imperatives
voice (grammatical system)
!in NP
!voice-neutral vs voice-sensitive (catenative, etc.)
voicing
volition
vowel
!vowel alternation/change
!vowel reduction/shortening
VP verb, verb phrase
wanting (verb, etc., of)
warning
waxing and waning
weak form
weekday (noun/NP denoting)
weight (heavy vs light constituents)
wh-cleft
wh relative relative clause
wh word
wish
Wodehouse
word
!inflection; lexical
!word-formation
word boundary
worth (adjective denoting)
writing system
y-replacement (spelling rule)
yes/no question
Yiddish, Yiddishism
zero-affixation,
zero-point
zeugma
zone
