\addchap{\lsPrefaceTitle} \label{ch:preface}

This book is not a traditional grammar guide or a collection of lesson plans, but a foundation for understanding language and a catalyst for informed teaching. It doesn't merely complement existing knowledge -- it lays the groundwork for novices to understand the landscape of language, while offering seasoned educators new vantage points from which to view familiar territory.

Each chapter aims to provide you with tools to navigate a different aspect of English:

\begin{itemize}
    \item We start by challenging common assumptions about language categories and encouraging critical thinking about how we define and understand linguistic concepts.

    \item We explore the nuances of Standard English and grammaticality, moving beyond the typical approaches to get at what ``correctness'' in language could really mean.

    \item We look into the relationships between grammatical categories (like noun, verb, phrase, and clause) and functions (like subject, modifier, head, and dependent), revealing the many-to-many connections that often confuse learners and teachers alike.

    \item We take a fresh look at lexical categories like adverbs and prepositions, challenging traditional classifications and revealing a simpler underlying system while recognizing that even words as seemingly simple as \textit{on} can't be fully understood without an inside perspective.

    \item We introduce syntax trees not as an English-teaching tool, but as a way for teachers to deepen our own understanding of sentence structure.

    \item We examine the sound system of English, focusing on often-overlooked aspects like suprasegmentals and connected speech processes.

    \item We consider the vocabulary, exploring word frequency, definitions, strategies for deliberate practice, and the complexities of idiomatic expressions.

    \item We unpack the intricacies of English writing systems, including phonics and spelling patterns.

    \item We explore various grammatical constructions, including questions, negation, and relative clauses, revealing their underlying structures and shared features.

    \item We discuss the concept of fluency and flow in language use, often overshadowed by an emphasis on accuracy. We examine how fluency manifests across different language skills and the various factors that influence it, providing insights to help balance fluency and accuracy in language teaching.

    \item We examine discourse structure, coherence, and information packaging in English, providing insights into how meaning is constructed beyond the sentence level.

    \item Finally, we look at the dynamics of conversation in English, exploring turn-taking, pragmatics, and cultural norms.
\end{itemize}

Throughout the book, you'll find that it often prioritizes areas that are typically underrepresented in TESL training. I'll point out common pitfalls and misunderstandings, not to criticize, but to help you avoid the ruts that many language courses get stuck in.

As you explore this linguistic landscape, remember that language is an ever-changing ecosystem. My goal is not always to provide definitive answers, but to equip you with the fundamental knowledge and critical thinking skills to navigate the terrain of English as you encounter it in your teaching journey.

This book maps the terrain; it's up to you to explore it.

\paragraph*{A note about the grammatical framework}

The framework provided here follows that of \textit{The Cambridge grammar of the English language} (\textit{CGEL} \cite{Huddleston2002}) very closely, but has been simplified for accessibility. For instance, I use a two-level noun and noun phrase analysis instead of \textit{CGEL}'s three-level approach, employ a smaller set of functions, and avoid the issue of function fusion. This book focuses on providing the essential elements and ideas that teachers need to embark on their English-teaching journey, rather than attempting to cover all constructions dealt with in \textit{CGEL}. For a more in-depth treatment, consider \textit{A student's introduction to English grammar} \citep{huddleston2022} or \textit{CGEL} itself.

\paragraph*{A note about Canadian English}

While most of the book applies to any English-language teaching situation, the phonology section presents vowels as spoken in southern Ontario, Canada. Many of these will align with General American English, but I include features such as Canadian raising (e.g., \href{https://en.wiktionary.org/wiki/File:En-ca-raising-house.ogg}{/hʌʊs/},\footnote{Click to go to audio recordings on \textit{English Wiktionary}.} as opposed to the British or General American \href{https://en.wiktionary.org/wiki/File:en-us-house-noun.ogg}{/haʊs/}), or indeed any other pronunciation. This choice reflects my own dialect and provides a specific reference point for phonological discussions.