\chapter{Categories and functions: \\An overview} \label{ch:categories}

\epigraph{In the architecture of speech,\\
every beam must bear its load,\\
every word know its place,\\
and still the structure breathes.}{}

With some understanding of the concepts of standards, \is{Standard English!focus of book}Standard English(es), grammaticality, and normativity under our belts, I turn back to the ideas we started to develop in Chapter \ref{ch:1}. For the reasons laid out in Chapter 2, as well as for practical reasons, the focus will be on Standard English in Chapter 3 and most of the book. Here, I'll explain how \is{category, linguistic!vs function}categories and \is{function, syntactic!vs category}syntactic functions interact (Section \ref{sec:cat-funct}) in Standard English. I'll then provided explanations of the following categories: adjectives and adjective phrases in Section \ref{sec:adjs+adjPs}, pronouns in Section \ref{sec:pronouns}, and determinatives and determinative phrases in Section \ref{sec:DPs}.

\section{Categories and functions: An overview}\label{sec:cat-funct}
\subsection*{A garden metaphor}

I'm in my front garden in late May. The garden is lush, and the peonies are particularly notable as they begin to set their buds. Soon, they will bloom into large flowers. But a closer examination of their stems reveals that they aren't very thick or strong, despite being fairly long. In fact, these peonies grow to about three feet in height. Given the size of the blooms, which can rival that of a newborn's head, their slender stems often struggle to support the weight, and the flowers droop.

Stems play a crucial role in plants. They provide structural rigidity, ensuring the plant remains upright. In the case of these peonies, the stems might not be up to the task. But in the same garden, there's a rosebush. Its stems are sturdy and can easily support the weight of its flowers. While roses and peonies are different types (or \textsc{categories}) of plants, their stems serve a similar \textsc{function}: transporting nutrients and providing support.

But stems aren't the only plant parts with important roles. Consider the roots. While they anchor the plant into the ground, offering some structural support, their primary function is nutrient absorption. So, while roots prioritize nutrient absorption and offer some support, stems focus more on support while also aiding in nutrient transport.

In my garden, there's also a challenge with weeds, especially on the pathways. To combat this, I have spread mulch. Interestingly, it is made of wood, which functioned as a tree's ``stem'' (or trunk and branches). In its original role, the wood provided structural support to the tree and aided in nutrient transport. Now, repurposed as mulch, its function has shifted to weed suppression and moisture retention in the garden. Different organisms might see other uses for this wood: insects might view it as a food source, while a salamander might see it as a potential home.

The garden serves as a microcosm, illustrating the multifunctionality of natural elements. We categorize plant parts into stems, roots, flowers, and leaves, and each of these categories can serve multiple functions, from structural support to nutrient transport. Similarly, in language, words belong to lexical categories: nouns, verbs, adjectives, etc. These categories serve various linguistic functions, from modification to heading a phrase or acting as a subject or determiner.

The key takeaway is the flexibility and multifunctionality inherent in both nature and language. There isn't always a one-to-one relationship between categories and functions~-- a single function can be realized by multiple categories, just as a single category can fulfill various roles.

\subsection{Many-to-many relationships} \label{sec:many-to-many}
\is{many-to-many relationships!between categories and functions|(}To reiterate, there's no one-to-one relationship between \is{category, linguistic!many-to-many relationship with functions}categories and \is{function, syntactic!many-to-many relationship with categories}functions, and the belief that there is is akin to the \hyperref[sec:fallacy-of-monosemy]{fallacy of monosemy}\is{fallacy of monosemy}. If there were, then we would have a much easier time defining the categories. For instance, if nouns and only nouns functioned as subjects, then we could simply define a noun as any word that functions as a subject, and we wouldn't have to bother with the stuff about coming after \textit{the} and \textit{a}, adjective phrases as modifiers, plurals, and the rest of it (Section \ref{sec:nouns}). But that's not how language works.

This echoes the point that was being made in the soup sketch: no single criterion will define a category. And yet, assuming one-to-one relationships is more or less what \is{traditional grammar!conflation of categories and functions}traditional English school grammars have always tried to do. They say that an adjective is any word that modifies a noun. Here's an example from the 11th edition of the very popular \textit{The big blue book of grammar and punctuation} \citep{straus2014}.

\begin{quote}
    \begin{itemize}[noitemsep]
        \item A \textbf{noun} is a word or set of words for a person, place, thing, or idea (p.~1).
        \item An \textbf{adjective} is a word or set of words that \textbf{modifies} (i.e., describes) a noun or pronoun. Adjectives may come before the word they modify (p.~15).
    
        \textit{\textbf{Examples}:}
        \begin{itemize}[noitemsep]
            \item \textit{That is a \textbf{cute} puppy.}
            \item \textit{She likes a \textbf{high school} senior.}
        \end{itemize}
    \end{itemize}
\end{quote}
    
Of course, this can't be right. A high school is a place (or a thing?), which should make \textit{high school} a noun under the definition above, but in the example sentence, it's supposed to be an adjective. In other cases, such as \textit{the Trudeau government}, Trudeau is a person but the word is, in the phrasing of \textit{The blue book of grammar and punctuation}, ``modifying the noun'' \textit{government}. So is \textit{Trudeau} a noun or an adjective? Or is it in some kind of quantum superposition of being both at the same time? In fact, both \textit{high school} and \textit{Trudeau} are just nouns, full stop (check the properties in Section \ref{sec:nouns} yourself if you doubt this).

This illustrates how we can better understand language by recognizing that the relationships between categories (like noun and adjective) and functions (like subject or modifier) are many-to-many. \is{traditional grammar!category mistake in}Traditional grammars often try to resolve this complexity with explanations like "it's a noun functioning as an adjective," but that's like saying that the woody stem of a rhododendron is ``functioning as a rose stem,'' or that poppies' long, slender stems ``function as peony stems.'' These stems all function as (better or worse) supports, but the only stem that ``functions as a rose stem'' is a rose stem. Claiming that a noun ``functions as an adjective'' is a \is{category mistake!defined}category mistake: \textsc{adjective} is not a function.

So to be perfectly clear, almost all categories perform multiple functions, and almost all functions can be performed by more than one category. Noun phrases, being no exception, have multiple functions, and not every subject is a noun phrase. In fact, it's quite common for nouns to function as modifiers in NPs as \textit{Trudeau} does in \textit{the Trudeau government}.

Make a list of other noun–noun combinations (e.g., \textit{soccer ball}, \textit{faculty office}), where the first is a modifier and the second is the head, and then check the following examples.

\begin{itemize}[noitemsep]
    \item \textit{health care}, \textit{world war}, \textit{law enforcement}, \textit{climate change}, \textit{air force}, \textit{family member}, \textit{video clip}, \textit{credit card}, \textit{breast cancer}, \textit{interest rate}, \textit{justice department}, \textit{tea party}, \textit{blood pressure}, \textit{Saturday night}, \textit{security council}, \textit{heart attack}, \textit{task force}, \textit{press conference}, \textit{stock market}, \textit{school district}, \textit{death penalty}
    \item \textit{cell phone}, \textit{phone call}, \textit{call centre}, \textit{centre stage}, \textit{stage manager}, \textit{manager position}, \textit{position paper}, \textit{paper towel}, \textit{towel rack}
    \item \textit{community college}, \textit{college student}, \textit{student loan}, \textit{loan guarantee}, \textit{guarantee program}, \textit{program director}, \textit{director level}, \textit{level change}
\end{itemize}\is{many-to-many relationships!between categories and functions|)}

\subsection{Categories}

\is{category, linguistic!defined}We have two fundamental types of categories: \textsc{lexical categories} and \textsc{phrasal categories}. These categories can often be identified out of context. For instance, if you see the word \textit{happy}, you can categorize it as an adjective. You can look it up in a dictionary, and it will say ``adjective''. There is no need for any more context. In some cases, the category will be ambiguous, as we saw with verbs/nouns like \textit{run}, \textit{jump}, and \textit{walk} in Section \ref{sec:verbs}. It's not the case that each of these words belongs to two categories; the verb \textit{walk} and the noun \textit{walk} are different words, belonging to different lexical categories. It's simply the case that we may not have enough information to say which of the two words we're looking at in a particular case. It is clear, however, that \textit{walk} is not an adjective, adverb, preposition, or anything other than a noun or a verb.

\textsc{Functions}, on the other hand, are relational. \is{function, syntactic!relational nature of}We can't ever look up a word in a dictionary and find out whether it's a modifier, a subject, a head, or a dependent. These function can only be determined in a context by seeing how the phrases relate to those around them.

\subsubsection*{Lexical categories}

\is{category, linguistic!lexical vs phrasal}\textsc{Lexical categories} contain individual words. Usually, in English, these are set off by spaces in writing, although there are some words that contain an internal space (e.g., \textit{each other}). So far, we have considered nouns and verbs. The other lexical categories we will look at in this book are adjectives, pronouns (a kind of noun), determinatives, prepositions, adverbs, coordinators, and subordinators. The only lexical category we won't eventually look at is interjections.\footnote{If you'd like to learn about them, I suggest starting with the Wikipedia entry for English interjections, for which, as of this writing, I am the primary author: \href{https://en.wikipedia.org/wiki/English\textunderscore interjections}{https://en.wikipedia.org/wiki/English\textunderscore interjections}.}

\subsubsection*{Phrasal categories}

Most of the lexical categories mentioned so far have a related phrasal category, as shown in Figure \ref{fig:projections}. For nouns, there are noun phrases (or NPs); for verbs there are VPs; for adjectives, AdjPs, etc. There's one additional phrasal category, clause, which is headed by a VP as we saw in Section \ref{sec:verbs}.\is{determinative, determinative phrase (DP)}\is{noun, noun phrase (NP)}\is{verb, verb phrase (VP)}\is{adverb, adverb phrase (AdvP)}\is{preposition, preposition phrase (PP)}\is{adjective, adjective phrase (AdjP)}\is{clause}


\begin{figure}[h]
    \centering
    \begin{tikzpicture}[node distance=0.4cm]
        % Lexical row
        \node (noun) {Noun};
        \node[right=of noun] (verb) {Verb};
        \node[right=of verb] (adj) {Adjective};
        \node[right=of adj] (det) {Determinative};
        \node[right=of det] (adv) {Adverb};
        \node[right=of adv] (prep) {Preposition};
        
        % Phrasal row
        \node[above=of noun, anchor=base] (np) {NP};
        \node[above=of verb, anchor=base] (vp) {VP};
        \node[above=of adj, anchor=base] (adjp) {AdjP};
        \node[above=of det, anchor=base] (detp) {DP};
        \node[above=of adv, anchor=base] (advp) {AdvP};
        \node[above=of prep, anchor=base] (prepP) {PP};
        
        % Top row for Clause
        \node[above=of vp, anchor=base] (clause) {Clause};
        
        % Draw the light vertical lines
        \draw[gray, thin] (noun) -- (np);
        \draw[gray, thin] (verb) -- (vp);
        \draw[gray, thin] (adj) -- (adjp);
        \draw[gray, thin] (det) -- (detp);
        \draw[gray, thin] (adv) -- (advp);
        \draw[gray, thin] (prep) -- (prepP);
        \draw[gray, thin] (vp) -- (clause);
    \end{tikzpicture}
    \caption{Some lexical and phrasal categories}
    \label{fig:categories}
\end{figure}\label{fig:projections}

As we saw in chapter 1, the everyday meaning of \textit{phrase} happens to be ``a group of words that have a particular meaning when used together,'' but for our purposes, all we need is a head word. Often, this comes with dependents, but one-word phrases are also possible. This is another example of the difference between the everyday meaning and the technical definition. So, here, \textit{go to school} is a \textsc{VP}, but so is \textit{go} in \textit{please, go}. We can have a long NP such as \textit{all the very interesting people who I met yesterday}, but we can also have single-word NPs such as \textit{they} or \textit{water}.

\subsection{Syntactic functions}

\is{function, syntactic!defined}\textsc{Syntactic functions} (or just \textsc{functions}) are broadly split into \textsc{Head} and \textsc{Dependent}. There is only one kind of head, but we'll divide dependents into \textsc{Subject}, \textsc{Modifier}, \textsc{Determiner}, and \textsc{Complement}. We'll also include a special kind of complement called an \textsc{Object}. \is{function, syntactic!list of}There are some other functions that I'll introduce later, but we'll stick with these for now.

Once again, the following explanations may include terms that you're not yet familiar with. As I've said, because this is a system, this is difficult to avoid, but the terms will be explained soon, at which point, you may wish to review these definitions.

\subsubsection*{Head}\label{sec:head}

\is{head (function)}The head is the function of the most important element in a phrase. The phrase is almost always named after the lexical category of its head (e.g., an AdjP has an adjective as its head), and every phrase has a head. Heads are often the only element in a phrase. For instance, in \textit{rain fell}, the subject NP contains just the head noun \textit{rain} and the head VP contains just the head verb \textit{fell}.

I will often shorten \textsc{Head} to H.

\subsubsection*{Dependent}\label{sec:Dep}

\is{dependent (Dep)!defined}A dependent is any element in the structure of a phrase or clause other than the head: \textit{\uline{the} \uline{best} kind} (NP headed by \textit{kind} with two dependents); \textit{won \uline{an award}} (VP headed by \textit{won} with one dependent); \textit{\uline{more} important} (AdjP headed by \textit{important} with one dependent). Subjects, complements, modifiers, and determiners are all kinds of dependents.

I will often shorten \textsc{Dependent} to Dep.

\paragraph*{Subject} \label{sec:subjects} (a type of Dependent)

\begin{itemize}[noitemsep]
    \item \is{subject (Subj)!defined}The subject typically functions as the topic of the clause and is syntactically a dependent of the VP in a clause.
    \item It most often precedes the VP in declarative clauses (e.g., \textit{\uline{A sunny day}} [\textit{should be savoured}]).
    \item In questions with subject--auxiliary inversion, the subject follows the auxiliary (e.g., \textit{Could \uline{they} be right?}), a structure further explored in Section \ref{sec:basic-int}.
    \item Semantically, the subject often denotes the agent in active sentences, but this is not universally the case, setting the stage for exceptions discussed in Section \ref{sec:subjects2}.
    \item \is{agreement!subject--verb}Agreement with the verb in person and number is a key characteristic of the subject, particularly visible in present tense verbs (e.g., \textit{I like it} vs \textit{She like\uline{s} it}).
    \item While typically an NP, subjects can also be clauses, VPs, or even preposition phrases (PPs), highlighting the flexibility of this grammatical function.
\end{itemize}


I will often shorten \textsc{Subject} to Subj.

\paragraph*{Determiner} \label{sec:determine} (a type of Dependent)

\begin{itemize}[noitemsep]
    \item \is{determiner (Det)!defined}Determiners are found only in NPs.
    \item A Head noun will never have more than one determiner \\(e.g., \textit{*\uline{my} \uline{the} class}).
    \item Determiners are usually determinative phrases (DPs) or possessive NPs \\(e.g., \textit{the}, \textit{almost all}, \textit{ some},\textit{ each},\textit{ my},\textit{ Brett's}).
    \item A singular countable noun needs a determiner to form an NP \\(e.g., \textit{I need a}/\textit{the}/\textit{each}/\textit{every pen}, but not \textit{*I need pen}).
    \item Determiners come before modifiers in the NP \\(e.g., \textit{\uline{the} Trudeau government}).
\end{itemize}

I will often shorten \textsc{Determiner} to Det. \textsc{Determin\uline{er}} and \textsc{modifi\uline{er}} are syntactic functions. The \textit{--er} tells you it's a function. In contrast, \textsc{Determina\uline{tive}} and \textsc{adjec\uline{tive}} are lexical categories.

\paragraph*{Modifier} \label{sec:modifier}(a type of Dependent)

\begin{itemize}[noitemsep]
    \item \is{modification, modifier!defined}Most heads allow one or more modifiers.
    \item Typical modifiers are AdjPs and AdvPs, but almost any phrase can be a modifier.
    \item Modifiers can often move position \\(e.g., \textit{he \uline{quickly} jumped the fence}, \textit{he jumped the fence \uline{quickly}}).
    \item Modifiers can often be removed without drastic or unexpected changes of meaning (e.g., \textit{a \uline{nice, furry} dog was sitting there} $\rightarrow$ \textit{a dog was sitting there}).
    \item Similarly, modifiers can almost always be added or removed without affecting grammaticality.
    \item In theory, the number of modifiers is unlimited.
    \item Unfortunately, the boundary between complements and modifiers can be very fuzzy.
\end{itemize}

I will often shorten \textsc{Modifier} to Mod.

\paragraph*{Complement} \label{sec:complement}(a type of Dependent)

\begin{itemize}[noitemsep]
    \item \is{complement, complementation!defined}Complements ``complete'' a phrase. This meaning is reflected in the spelling \textit{compl\uline{e}ment}. (\textit{Compl\uline{i}ments}, in contrast, are nice things you say about somebody.)
    \item Almost any kind of phrase can be a complement, though AdvP and DP complements are very rare.
    \item Complements usually follow their head. In contrast, modifiers can often move around (e.g., the modifier \textit{quickly} can go at the beginning, middle, and end of \textit{she changed direction}).
    \item Complements are ``selected'' by a head \\(e.g., \textit{listen \uline{to the music}} not \textit{*listen \uline{about the music}}; \textit{listen} selects a \textit{to} PP.).
    \item If a phrase is indifferent about a head, it's not a complement. For example, \textit{on Monday} can go with almost any verb \\(e.g., \textit{eat}/\textit{play}/\textit{work}/\textit{think}/\textit{sit \uline{on Monday}}).
    \item There are zero, one, two, or sometimes three complements, but never four.\footnote{If you think of subjects as a kind of complement, then four is possible.}
    \item Unfortunately, the boundary between complements and modifiers can be very fuzzy.
\end{itemize}

I will often shorten \textsc{Complement} to Comp.

\paragraph*{Object} \label{sec:object} (a type of Complement)

\begin{itemize}[noitemsep]
    \item \is{object (Obj)!defined}Objects are a kind of complement.
    \item Objects are always dependents in VPs or in PPs \\(e.g., \textit{grab \uline{the bag}} or \textit{in \uline{the bag}}).
    \item Objects are almost always NPs.
    \item If there's an action, the object is usually what is acted upon \\(e.g., \textit{I threw \uline{the ball}).}
    \item The object of an active sentence is usually the subject of the related passive sentence (e.g., \textit{A friend made \uline{the cake}} $\rightarrow$ \textit{\uline{the cake} was made by a friend}).
    \item Verbs like \textit{be},\textit{ become},\textit{ seem},\textit{ appear},\textit{ feel} (e.g., \textit{I feel good}), \textit{look} (e.g., \textit{it looks easy}), etc. take complements, but not the special kind called objects.
\end{itemize}

I will often shorten \textsc{Object} to Obj.

\begin{tcolorbox}[title=Exercises, colback=white]
    Underline the whole subject in each quotation. If there is more than one subject, underline the first two. Ignore any further subjects.
    \begin{enumerate}[noitemsep]
        \item ``Undressing her was an act of recklessness, a kind of vandalism, like releasing a zoo full of animals, or blowing up a dam.'' \phantom{xxxxxxxxxxxxxx} \hfill--Michael~Chabon
       \item ``The war in Zagreb began over a pack of cigarettes.''\hfill--Sara Nović
       \item ``Jack put his arm out the window, waving his hat like a visiting dignitary, backed into the street, and floated away, gentling the gleaming dirigible through the shadows of arching elm trees, light dropping on it through their leaves like confetti as it made its ceremonious passage.'' \phantom{xxxxxxxx} \hfill--Marilynne~Robinson
     \item ``A sudden warm rainstorm washes down in sweet hyphens.''\phantom{xxxxxxxx} \hfill--J.~M.~Ledger
     \item ``And as the ax bites into the wood, be comforted in the fact that the ache in your heart and the confusion in your soul means that you are still alive, still human, and still open to the beauty of the world, even though you have done nothing to deserve it.''\hfill--Paul Harding
     \item ``Within seconds of that thought, the train entered Washington, where she was to come to her end more than sixty-eight years later, a mother to seven living and two dead, a grandmother to twenty-one living and three dead, a great-grandmother to twelve, a great-great grandmother to twins.''\hfill--Edward~P.~Jones
     \item ``We were all a little drunk with spring, like the fat bees reeling from flower to flower, and a strange insurrectionary current ran among us.''\hfill--Tobias Wolff
     \item ``When he was dry, he believed it was alcohol he needed, but when he had a few drinks in him, he knew it was something else, possibly a woman; and when he had it all~-- cash, booze, and a wife~-- he couldn't be distracted from the great emptiness that was always falling through him and never hit the ground.''\hfill--Denis Johnson
     \item ``Lizards skit like quick beige sticks.''\hfill--Richard Beard
    \end{enumerate}
\end{tcolorbox}
\newpage
\begin{tcolorbox}[title=Answer key, colback=white]

\begin{enumerate}[noitemsep]
    \item \textit{Undressing her}
    \item \textit{The war}
    \item \textit{Jack} \dots \textit{ it}
    \item \textit{A sudden warm rainstorm}
    \item \textit{the ax \dots~the ache in your heart and the confusion in your soul} 
    \item \textit{the train \dots~she}
    \item \textit{We \dots~a strange insurrectionary current}
    \item \textit{he \dots~he}
    \item \textit{Lizards}
\end{enumerate}
\end{tcolorbox}

\section{Adjectives and adjective phrases}\label{sec:adjs+adjPs}

Now that you have a better understanding of categories and verbs, it's time to look at the characteristics of \is{adjective, adjective phrase (AdjP)!characteristics of}adjectives and \is{adjective, adjective phrase (AdjP)!characteristics of}adjective phrases.

Before we start, I think it's useful to restate that these characteristics are often related to teaching points. For example, the issue of nouns being countable is actually an issue that causes a great deal of confusion for students. So, with that said, \dots

\begin{enumerate}[noitemsep]
    \item \is{adjective, adjective phrase (AdjP)!denotation of (properties, characteristics)}Adjectives typically denote properties or characteristics, but not objects or actions.
    \item \is{adjective, adjective phrase (AdjP)!head}Adjectives always head adjective phrases.
    \item \is{adjective, adjective phrase (AdjP)!as modifier (in NP and AdjP)}Adjective phrases typically function as modifiers in NPs (\textit{a \uline{nice} day}) and \is{adjective, adjective phrase (AdjP)!as complement (in VP and PP)}complements in VPs (\textit{was \uline{nice}}). Sometimes they also function as modifiers in AdjPs, though not in VPs or most other phrase types, and complements in PPs (\textit{for \uline{free}}).
    \item \is{adjective, adjective phrase (AdjP)!tests for (complement to \textit{be}, \textit{become}, \textit{seem})}Any phrase that can function as complement to all three of \textit{be}, and \textit{become}, and \textit{seem} is an adjective phrase.
    \item \is{adjective, adjective phrase (AdjP)!tests for (modification by very, too, so)}Any word that can be modified by all three of \textit{very}, and \textit{too}, and \textit{so} is an adjective (or an adverb).
    \item \is{adjective, adjective phrase (AdjP)!gradability of (-\textit{er}/-\textit{est}, \textit{more}/\textit{most})}Many adjectives can take \textit{--er} and \textit{--est} endings, but so can a number of commonly used adverbs.
    \item Many adjectives can be gradable with \textit{more} and \textit{less}, and \textit{most} and \textit{least}, but so can many adverbs.
    \item \is{adjective, adjective phrase (AdjP)!formation of adverb from (-\textit{ly})}Many adjectives can take an \textit{--ly} ending and become an adverbs.
\end{enumerate}

Here are some examples.

\subsection*{Denotation}

As in Section \ref{sec:nouns}, the terms used here are just everyday descriptions. They're not technical terms, nor are they meant to be exhaustive.

\begin{itemize}[noitemsep]
    \item \textbf{evaluative properties:} \textit{good}, \textit{great}, \textit{helpful}, \textit{interesting}
    \item \textbf{scalar properties:} \textit{big}, \textit{heavy}, \textit{high}, \textit{old}, \textit{thick}
    \item \textbf{relationship properties:} \textit{Canadian}, \textit{different}, \textit{mutual}, \textit{ninth}, \textit{other}
    \item \textbf{material properties:} \textit{ceramic}, \textit{metallic}, \textit{plastic}, \textit{wooden}, \textit{woolen}
    \item \textbf{state properties:} \textit{amorphous}, \textit{aqueous}, \textit{crystalline}, \textit{gaseous}, \textit{viscous}
    \item etc.
\end{itemize}

\subsection*{Heading adjective phrases}
\begin{itemize}[noitemsep]
    \item \textit{\uline{happy}}, \textit{very \uline{happy}}, \textit{very \uline{happy} with the outcome}
    \item \textit{\uline{nice}}, \textit{not \uline{nice}}, \textit{no \uline{nicer} than the other one}
    \item \textit{\uline{interesting}}, \textit{easily more \uline{interesting} than mine}
    \item \textit{slightly too \uline{big}}
\end{itemize}

\subsection*{AdjPs as modifiers in NPs}
\begin{itemize}[noitemsep]
    \item \uline{\textit{civil}} \textit{rights}, \textit{some \uline{quite good} news},  \uline{\textit{national}} \textit{security}, \textit{the \uline{whole} thing}, \uline{\textit{recent}} \textit{years}, \textit{our \uline{mental} health}, \textit{her \uline{very best} friend}, \textit{a \uline{really big} deal}, \textit{a \uline{supremely important} responsibility}
\end{itemize}

\subsection*{AdjPs as complements to \textit{be}, \textit{become}, and \textit{seem}}
\begin{itemize}[noitemsep]
    \item \textit{is}/\textit{became}/\textit{seems} \textit{clear},  \textit{aware}, \textit{available}, \textit{pregnant},  \textit{familiar to me},  \textit{popular with the ladies}, and \textit{very interested in the difference}
\end{itemize}

\subsection*{AdjPs with \textit{very}, \textit{too}, and \textit{so} as modifiers}
\begin{itemize}[noitemsep]
    \item \textit{very}/\textit{too}/\textit{so} \textit{good}, \textit{important}, \textit{late}, \textit{different}, \textit{difficult}, and \textit{nice}
    \item(Many adverbs work here too: \textit{well}, \textit{importantly}, \textit{differently}, \textit{nicely})
\end{itemize}

\subsection*{Taking \textit{--er} and \textit{--est} suffixes}
\begin{itemize}[noitemsep]
    \item \textit{bigger}, \textit{biggest}; \textit{greater}, \textit{greatest}; \textit{older}, \textit{oldest}; \textit{lower}, \textit{lowest}; \textit{yonger}, \textit{youngest}
    \item (A number of commonly used adverbs work here too: \textit{stay longer}, \textit{leave earlier}, \textit{run faster}, \textit{work harder}, \textit{jump higher})
\end{itemize}

\subsection*{Graded with \textit{more} and \textit{less}, and \textit{most} and \textit{least}}
\begin{itemize}[noitemsep]
    \item \textit{more}/\textit{less}/\textit{most}/\textit{least} \textit{likely}, \textit{expensive}, \textit{favourite}, \textit{important}, \textit{restrictive}, \textit{effective}, \textit{able}, \textit{possible}
    \item (Again, adverbs like \textit{frequently}, \textit{environmentally}, \textit{politically}, \textit{often}, etc. work here too)
\end{itemize}

\subsection*{Take \textit{--ly} to form adverbs}
\begin{itemize}[noitemsep]
    \item \textit{quick} $\rightarrow$ \textit{quickly}; \textit{real} $\rightarrow$ \textit{really}; \textit{actual} $\rightarrow$ \textit{actually}; \textit{probable} $\rightarrow$ \textit{probably}; \textit{final} $\rightarrow$ \textit{finally}; \textit{simple} $\rightarrow$ \textit{simply}; \textit{exact} $\rightarrow$ \textit{exactly}; \textit{certain} $\rightarrow$ \textit{certainly}
\end{itemize}

\subsection*{Exercise}
Apply the properties to identify the adjectives in this excerpt from ``A very old man with enormous wings'' by Gabriel García Márquez, translated by Gregory Rabassa. If you're not sure how to do this, have a look at the worked answers that follow.
\begin{quote}
On the following day, everyone knew that a flesh-and-blood angel was held captive in Pelayo's house. Against the judgment of the wise neighbor woman, for whom angels in those times were the fugitive survivors of a celestial conspiracy, they did not have the heart to club him to death. Pelayo watched over him all afternoon from the kitchen, armed with his bailiff's club, and before going to bed he dragged him out of the mud and locked him up with the hens in the wire chicken coop. In the middle of the night, when the rain stopped, Pelayo and Elisenda were still killing crabs. A short time afterward, the child woke up without a fever and with a desire to eat. Then they felt magnanimous and decided to put the angel on a raft with fresh water and provisions for three days and leave him to his fate on the high seas. But when they went out into the courtyard with the first light of dawn, they found the whole neighborhood in front of the chicken coop having fun with the angel, without the slightest reverence, tossing him things to eat through the openings in the wire as if he weren't a supernatural creature but a circus animal.
\end{quote}

\paragraph*{\textit{Following}}
\textit{On the \uline{following} day everyone knew that a flesh-and-blood angel was held captive in Pelayo's house.}

\begin{itemize}[noitemsep]
    \item \textit{Following} is a modifier in the noun phrase \textit{the following day}, but it is not modifiable by \textit{very}, \textit{so}, or \textit{too}.
    \item There is no \textit{more}/\textit{most}/\textit{less}/\textit{least following}.
    \item We cannot say something is \textit{the followingest}.
    \item There is no adverb \textit{followingly}.
    \item It cannot function as the complement of \textit{become} or \textit{seem} \\(e.g., \textit{the day became following}).
\end{itemize}
Therefore, \textit{following} is probably not an adjective, or at best it is a very bad example of an adjective. Since it ends in \textit{--ing} consider whether it might be a verb.

\paragraph*{\textit{Those}}
\textit{Against the judgement of the wise neighbor woman, for whom angels in \uline{those} times were the fugitive survivors of a celestial conspiracy, they did not have the heart to club him to death.}
\begin{itemize}[noitemsep]
    \item \textit{Those} comes before the noun \textit{times}, but it doesn't denote a property. Instead, it has a kind of pointing meaning.
    \item It is not gradable with \textit{--er}, \textit{more}, or \textit{less}.
    \item It cannot be modified by \textit{very}.
    \item It cannot function as complement of \textit{become} or \textit{seem}.
    \item There is no word \textit{*thosely}.
\end{itemize}
So, \textit{those} is probably not an adjective either.

\paragraph*{\textit{Wire}}
\textit{Pelayo watched over him all afternoon from the kitchen, armed with his bailiff's club, and before going to bed he dragged him out of the mud and locked him up with the hens in the \uline{wire} chicken coop.}

\begin{itemize}[noitemsep]
    \item \textit{Wire} functions as a modifier in the noun phrase \textit{the wire chicken coop}, but it denotes a physical object, not a property.
    \item We can say \textit{that seems like wire}, but not \textit{*that seems wire}.
    \item It has none of the other characteristics of an adjective.
    \item It can modify a verb (e.g., \textit{a \uline{wire wrapped} guitar string}).
\end{itemize}
Again, \textit{wire} is probably not an adjective. Given that it denotes a physical object, consider whether it might be a noun.

\paragraph*{\textit{Magnanimous}}
\textit{Then they felt \uline{magnanimous} and decided to put the angel on a raft with fresh water and provisions for three days and leave him to his fate on the high seas.}

\begin{itemize}[noitemsep]
    \item \textit{Magnanimous} doesn't modify a noun, but it functions as a complement in a VP headed by \textit{felt}.
    \item It denotes a characteristic of the people (they).
    \item There isn't a word \textit{*magnanimouser}, but we do have \textit{more} or \textit{less magnanimous}.
    \item It can be modified by \textit{very}.
    \item There is an adverb \textit{magnanimously}.
    \item There's nothing wrong with \textit{it seems magnanimous}.
\end{itemize}
So, \textit{magnanimous} seem to be an adjective.

\paragraph*{\textit{Fresh}}
\textit{Then they felt magnanimous and decided to put the angel on a raft with \uline{fresh} water and provisions for three days and leave him to his fate on the high seas.}

\begin{itemize}[noitemsep]
    \item \textit{Fresh} is a modifier in the noun phrase \textit{fresh water}.
    \item It denotes a scalar property.
    \item It can be graded (e.g., \textit{fresher}, \textit{freshest}, \textit{least fresh}).
    \item Something can be \textit{very fresh}.
    \item There is an adverb \textit{freshly}.
    \item Something can \textit{seem fresh}. And while it's kind of unusual for something to \textit{become fresh}, just because things tend to go from fresh to stale and not the other way around, there's no grammatical problem with \textit{become fresh}.
\end{itemize}
Clearly, \textit{fresh} is an adjective.

Now, have another look at the passage with these examples in mind.

\begin{tcolorbox}[title=Summary and review, colback=white]
    Fill in the gaps with \textit{noun}(\textit{s}), \textit{verb}(\textit{s}), or \textit{adjective}(\textit{s}).

    \phantom{a}

    \phantom{zzz}A word that denotes a person, place, or thing is a/an \uline{\hspace{1.5cm}}.
    \uline{\hspace{1.5cm}} It can typically follow \textit{the} or \textit{a}. A word that is countable is a/an \uline{\hspace{2cm}}.
    Almost every \uline{\hspace{2cm}} can follow the infinitive marker \textit{to}.
    A noun phrase is typically headed by a/an \uline{\hspace{2cm}}.
    An adjective phrase is always headed by a/an \uline{\hspace{2cm}}.
    \uline{\hspace{2cm}} phrases can perform a variety of functions, including subject and object. Any phrase that can function as complement to \textit{be}, and \textit{become}, \textbf{and} \textit{seem} is \\a/an \uline{\hspace{2cm}} phrase. Almost every \uline{\hspace{2cm}} has an \textit{--ing} form.
    \uline{\hspace{2cm}} commonly denote actions and states, and they do not denote physical objects.

    \phantom{zzz}Many \uline{\hspace{2cm}} can take \textit{--er} and \textit{--est} endings, but so can many adverbs.
    \uline{\hspace{2cm}} typically denote properties or characteristics, but not objects or actions.
    \uline{\hspace{1.4cm}}~phrases can typically be possessive. Typically,
    \uline{\hspace{2cm}} can be modified by adjective phrases. Many \uline{\hspace{2cm}} can take an \textit{--ly} ending and become an adverbs. Almost every \uline{\hspace{2cm}} has a third-person singular \textit{--s} form. A/an \uline{\hspace{2cm}} is often in an agreement relationship with a phrase functioning as its subject. A/an \uline{\hspace{2cm}} cannot be modified by an adjective phrase. The head of a clause is typically a/an \uline{\hspace{2cm}} phrase.
    \uline{\hspace{2cm}} can denote almost anything from objects to actions to characteristics.

    \phantom{zzz}Any word that can be modified by \textit{very}, \textit{too}, and \textit{so} is a/an \uline{\hspace{2cm}} (or an adverb).
    \uline{\hspace{2cm}} cannot typically be modified by adverb phrases. A word that has a past-tense form is a/an \uline{\hspace{2cm}}. Many \uline{\hspace{2cm}} can be gradable with \textit{less} and \textit{least}, and so can many adverbs.
    \uline{\hspace{2cm}} phrases modify nouns and \uline{\hspace{2cm}}, but not verbs. Almost any word that is plural is a/an \uline{\hspace{2cm}}.
    \uline{\hspace{2cm}} phrase must be headed by a verb.

    \phantom{a}

    \phantom{zzz}Keep working on recalling the properties of nouns and verbs, and add adjectives to your practice.
\end{tcolorbox}

\section{Pronouns, and determinatives and DPs}

\subsection{Pronouns}\label{sec:pronouns}
\begin{itemize}[noitemsep]
    \item Did you ever notice that we sometimes say things like, \textit{look at the baby. It's so cute}. but we'd never say, \textit{look at her boyfriend. *It's so cute}.
    \item Did you ever notice that plural \textit{they} is used to refer to anything from teachers to trees but singular \textit{they} is only for persons? If you doubt me, give it a try.
    \item Did you ever notice that in some uses, \textit{whose} is used to refer to anything from children to chambers (e.g., \textit{The chasm ran through a chamber whose ceiling was lost in the darkness above}), but \textit{who} is only for persons?
    \item Did you ever notice that in some cases \textit{which} can't be used for persons (e.g., *\textit{that's the guy which sold me the car})?
    \item Of course, everyone's aware of the masculine/feminine gender pronouns, but did you ever think about personal/non-personal gender pronouns?
\end{itemize}

\is{pronoun!defined}Pronouns are a special kind of noun. They typically refer to another noun phrase that has come before or that comes later (e.g., in \textit{Xi saw \uline{the egg} fall but caught \uline{it} before \uline{it} smashed}, the pronoun \textit{it} refers to the egg). There are only a small number of pronouns. The main ones are set out in Table \ref{tab:pronouns}.

\begin{sidewaystable}
\centering
\caption{The main pronouns in Modern English}
\label{tab:pronouns}
\begin{tabular}{l l l c c c c c}
\multicolumn{8}{c}{\textbf{Pronoun Forms}} \\
\hline
\textbf{Type} & \textbf{Number} & \textbf{Gender} & \textbf{Nom} & \textbf{Accusative} & \textbf{Reflexive} & \textbf{Ind Gen} & \textbf{Dep Gen} \\
\hline
1\textsuperscript{st}-per & Singular & Personal & \textit{I} & \textit{me} & \textit{myself} & \textit{mine} & \textit{my} \\
1\textsuperscript{st}-per & Plural & Personal & \textit{we} & \textit{us} & \textit{ourselves} & \textit{ours} & \textit{our} \\
2\textsuperscript{nd}-per & Singular & Personal & \textit{you} & \textit{you} & \textit{yourself} & \textit{yours} & \textit{your} \\
2\textsuperscript{nd}-per & Plural & Personal & \textit{you} & \textit{you} & \textit{yourselves} & \textit{yours} & \textit{your} \\
3\textsuperscript{rd}-per & Singular & Masculine & \textit{he} & \textit{him} & \textit{himself} & \textit{his} & \textit{his} \\
3\textsuperscript{rd}-per & Singular & Feminine & \textit{she} & \textit{her} & \textit{herself} & \textit{hers} & \textit{her} \\
3\textsuperscript{rd}-per & Singular & Impersonal & \textit{it} & \textit{it} & \textit{itself} & & \textit{its} \\
3\textsuperscript{rd}-per & Singular & Personal & \textit{they} & \textit{them} & \textit{themselves} & \textit{theirs} & \textit{their} \\
3\textsuperscript{rd}-per & Plural & Universal & \textit{they} & \textit{them} & \textit{themselves} & \textit{theirs} & \textit{their} \\
3\textsuperscript{rd}-per generic & Singular & Personal & \textit{one} & \textit{one} & \textit{oneself} & & \textit{one's} \\
Interrogative & Singular & Personal & \textit{who} & \textit{whom} & & \textit{whose} & \textit{whose} \\
Relative & Flexible & Personal & \textit{who} & \textit{whom} & &  & \textit{whose} \\
Reciprocal & Singular & Universal &  & \textit{each other} & & \textit{each other's} & \textit{each other's} \\
Reciprocal & Singular & Universal &  & \textit{one another} & & \textit{one another's} & \textit{one another's} \\
Dummy & Singular & Impersonal & \textit{it} & & & & \\
Dummy & Flexible & None & \textit{there} & & & & \\
\hline
\end{tabular}
\end{sidewaystable}

A \is{dummy pronoun!defined}\textsc{dummy pronoun} is one which does not refer, as in \textit{\uline{There} are two dummy pronouns} or \textit{\uline{It}'s windy today.}\label{sec:dummy} There is no semantic reason for \textit{there} and \textit{it} to appear in those sentences. Its role is purely grammatical, because English demands a subject in this context.

For discussion of interrogative and relative pronouns, see Sections \ref{sec:interrogative-phrases} and \ref{sec:focused-articulated}.

When it comes to the possessive pronouns, \is{agreement!pronoun-possessor (English vs French)}gender agreement can be an issue. In some languages, they agree in gender with the head noun, while in English, \textit{his}, \textit{her\op\textit{s}\cp}, and \textit{its} agree with the possessor. So, for the grandmother of a man, English would use the masculine \textit{his} because he is a man, while French\il{French} would use the feminine \textit{sa} because the grandmother is a woman.\is{cross-linguistic influence}

\ea \label{ex:his-grandmother}
    \ea[]{\textit{It's \uline{his} grandmother.} [masculine] \textit{C'est \uline{sa} grand-mère.}\hfill[feminine]}
    \ex[]{\textit{It's \uline{his} grandfather.} \phantom{..}[masculine] \textit{C'est \uline{son} grand-père.}\hfill[masculine]}
    \z
\ex \label{ex:her-grandmother}
    \ea[]{\textit{It's \uline{her} grandmother.} \phantom{..}[feminine] \textit{C'est \uline{sa} grand-mère.}\hfill[feminine]}
    \ex[]{\textit{It's \uline{her} grandfather.} \phantom{....}[feminine] \textit{C'est \uline{son} grand-père.}\hfill[masculine]}
    \z
\z

This can be a challenging concept to adjust to if the language you grew up speaking (L1) makes the opposite choice. Even for speakers whose L1 doesn't make any such grammatical distinction, it may be worth pointing out how English gender agreement works in such cases.

One interesting property of pronouns is that only a pronoun (but not all pronouns) can appear in a \textsc{tag question}, such as \textit{that's very interesting, \ob isn't \uline{it}\cb?} or \textit{there's more, \ob isn't \uline{there}\cb?}\is{tag question}\is{question!tag}

\subsection{Determinatives and DPs}\label{sec:DPs}\is{determinative, determinative phrase (DP)|(}\is{determinative, determinative phrase (DP)!vs determiner (function)}\is{category, linguistic!vs function}

There is a good deal of confusion about the terms \textit{determiner} and \textit{determinative}. The former is a function and the latter a category. Unfortunately, some books use them the other way around.\footnote{Those using them flipped around include \textit{A comprehensive grammar of English}, English and Simple English \textit{Wiktionary}, and \textit{Wikipedia}.} I will use  \textit{determin\uline{er}} as a function (like  \textit{modifi\uline{er}}) and \textit{determina\uline{tive}} as a category (like \textit{adjec\uline{tive}}). In general, a reliable dictionary for checking whether a word is a determinative is the \href{https://simple.wiktionary.org/wiki/Category:Determiners}{\textit{Simple English Wiktionary}}, although, confusingly, it calls them ``determiners''. It will help if you can keep the following terminological reminder in mind.

\begin{itemize}[noitemsep]
    \item \textsc{Determin\uline{er}} and \textsc{modifi\uline{er}} are syntactic functions. The \textit{--er} tells you it's a function.
    \item \textsc{Determina\uline{tive}} and \textsc{adjec\uline{tive}} are lexical categories.
\end{itemize}

For the most part, determinative phrases (DPs)\footnote{The term DP here refers simply to phrases headed by determinatives (e.g., \textit{some}, \textit{many more}, \textit{less than five}). This differs from how some linguists use ``DP" to refer to what we call noun phrases.} tend to be single-word phrases. Occasionally though, a DP includes a modifier. In the following examples, the modifier is underlined.

\begin{itemize}[noitemsep]
    \item \textit{\ob\uline{many} more{\cb} people},  \textit{\ob\uline{that} many{\cb} days},  \textit{\ob\uline{all} these{\cb} sandwiches}, \textit{\ob\uline{almost} every{\cb} time}, \textit{\ob\uline{at least} two{\cb} times}
\end{itemize}

The function that DPs typically perform is that of determiner. Only NPs have determiners, and a head noun may only have one determiner (\textit{the ball}) or none (\textit{balls}), never two or more (\textit{*the my ball}). A singular countable noun typically \textbf{requires} a determiner to be a full NP, and without one, it is usually ungrammatical (e.g., \textit{*she passed me ball}).

Determinatives also commonly function as modifiers as in \textit{I feel \ob\uline{much} better{\cb}}, \ob\textit{good \uline{enough}{\cb}}, \textit{\ob\uline{this} good{\cb}}, etc.

\subsubsection*{Determinatives vs Adjectives}\is{determinative, determinative phrase (DP)!vs adjective}

In traditional English grammar, there is no concept of determiner or determinative. There was a special category for \textit{the} and \textit{a} called articles, but all the other determinatives were just called ``adjectives'' when they preceded a noun. For instance, the word \textit{some} in \textit{some people} is categorized as an adjective in the \textit{Merriam-Webster dictionary}, even though it has none of the other characteristics of adjectives. There are two reasonably good ways to distinguish whether a word is a determinative or an adjective.

\begin{itemize}[noitemsep]
    \item \is{adjective, adjective phrase (AdjP)!vs determinative}Most adjectives are happy to follow \textit{the}, but it's quite rare for determinatives to do so.
    \item \is{determinative, determinative phrase (DP)!vs adjective}Most determinatives can function in the frame \_\_ \textit{of the} \textsc{Noun} (e.g., \textit{\uline{some} of the people}, \textit{\uline{all} of the time}, \textit{\uline{each} of the tests}, \textit{\uline{many} of the changes}), but adjectives never do.
\end{itemize}

And, of course, you can simply consult the list of determinatives in the \href{https://simple.wiktionary.org/wiki/Category:Determiners}{\textit{Simple English Wiktionary}} or the appendix \ref{ch:appendix} because there aren't very many. A sample is presented in Table \ref{tab:determinatives}.

\begin{table}
\centering
\caption{A sample of determinatives in Modern English}
\label{tab:determinatives}
\begin{tabular}{l l l l l}
\hline
\textbf{Determinative} & \textbf{Type} & \textbf{Number} & \textbf{Gender}\textsuperscript{\textdagger} & Definiteness \\
\hline
    \textit{a}/\textit{an}   & Article & Singular & & Indefinite \\
    \textit{the}   & Article & Flexible & & Definite \\
    \textit{this} & Demonstrative & Singular &Impersonal&  Definite \\
    \textit{these} & Demonstrative & Plural &Impersonal& Definite \\
    \textit{that} & Demonstrative & Singular &Impersonal& Definite \\
    \textit{those} & Demonstrative & Plural &Impersonal& Definite \\
    \textit{each} & Distributive & Singular && Definite \\
    \textit{every} & Distributive & Singular && Definite \\
    \textit{any} & Existential & Flexible && Indefinite \\
    \textit{some} & Distributive & Flexible && Indefinite \\
    \textit{either} & Disjunctive & Singular && Indefinite \\
    \textit{neither} & Disjunctive & Singular && Indefinite \\
    \textit{no} (\textit{none}) & negative & Flexible && Definite \\
    \textit{few}/\textit{fewer}/\textit{fewest} & Paucal & Plural & Impersonal & Indefinite\\
    \textit{a little} & Paucal & Singular & Impersonal &  Indefinite \\
    \textit{what} & Interrogative & Flexible & Impersonal &  \\
    \textit{which} & Interrogative & Flexible &  &  \\
    \textit{what} & Relative & Flexible & Impersonal & Definite\\
    \textit{which} & Relative & Flexible & Impersonal &  Definite \\
    \textit{one} & Numeral & Singular && Indefinite\\
    \textit{two} & Numeral & Plural &&  Indefinite \\
\hline
\multicolumn{5}{p{0.87\textwidth}}{\textsuperscript{\textdagger}Only when used without a head noun (e.g., \textit{What person is this?} but not \textit{What is this?} for a person.) The demonstratives are not always impersonal (e.g., \textit{This is my friend.} or \textit{Those who try win.})}
\end{tabular}
\end{table}


\subsubsection*{Determinatives vs Pronouns}\is{determinative, determinative phrase (DP)!vs pronoun}

Like pronouns, determinatives can often refer to NPs that appear elsewhere in the discourse (e.g., \textit{I bought \uline{a bag of chocolate chip cookies}, and I ate \uline{some} on the way home}). For this reason, they are often categorized as pronouns. For example, this use of the word \textit{some} is categorized as a pronoun in the \textit{Merriam-Webster dictionary}, even though it has none of the other characteristics of pronouns. For one thing, \is{determinative, determinative
phrase (DP)!vs pronoun}determinatives don't have the various forms that most pronouns have and, like adjectives, pronouns cannot appear in the frame \_\_ \textit{of the} \textsc{Noun}, shown above (e.g, \textit{*she of the students} or \textit{*us of the teachers}).

\subsection{Definite vs indefinite} \label{sec:definite-v-indef}\is{determinative, determinative phrase (DP)!and definiteness}

A given NP is either \is{definiteness!defined}\textsc{definite} or \textsc{indefinite}. In English, this is usually marked by the choice of determin\uline{er} (the function). Certain determinatives, such as \textit{the} and \textit{this} along with possessive (genitive) NPs, such as \textit{my} and \textit{Brett's}, mark the NP as definite. Other determinatives such as \textit{a}, \textit{any}, and \textit{some} mark the NP as indefinite, and most NPs without a determiner are also indefinite. The main exceptions here are proper names and most pronouns, which are definite.

\is{definiteness!test for}Definiteness is one of those concepts that seems obvious at first blush but which, on second thought, turns out to be less clear. A useful if somewhat simplified way to think about it is found in \textit{CGEL}, which says definiteness can be understood as pre-empting a question with \textit{which}? 

\begin{quote}
    Compare, for example:
    \begin{enumerate}
        \item 
        \begin{enumerate}
            \item \textit{Where did you park the car?}
            \item \textit{The father of one of my students rang me up last night.}
            \item \textit{The first person to run the mile in under four minutes was Roger Bannister.}
        \end{enumerate}
    \end{enumerate}
    Example (i) illustrates the frequent case where the addressee can be assumed to be familiar with the referent of the definite NP: you have been driving the car and presumably know a good deal more about it than that it is a car --what colour and make of car it is, and so on. You thus don't need to ask \textit{Which car?}: you know which one I'm referring to. \\\hfill\citep[368]{Huddleston2002}
\end{quote}

\noindent(If you did respond to my utterance of [2i] by asking \textit{Which one?}, it would indicate that there had been a breakdown in communication, that I had been mistaken in assuming that you'd be able to identify which car I was referring to.)

\is{definiteness!as issue for learners}Definiteness can be a trouble point for learners of English because definiteness in English is always marked (by a definite determiner or by virtue of being a pronoun or proper name), but only optionally marked in many languages. Moreover, the choice of determiner is only partially conditioned by definiteness. Countability, quantity, number, proximity, and possession all factor in.

\begin{figure}
    \centering
    \begin{forest}
for tree={grow'=0, parent anchor=east, child anchor=west, l=0cm, 
    anchor=west, calign=child edge, inner sep=0pt}
[D
    [definite
        [demonstrative
            [singular
                [\itshape this]
                [\itshape that]
            ]
            [plural
                [\itshape these]
                [\itshape those]
            ]
        ]
%        [personal
%            [2nd-person
%                [\itshape you]
%            ]
%            [1st-person
%                [\itshape us]
%                [\itshape we]
%            ]
%        ]
        [general
            [\itshape the]
        ]
        [temporal
            [\itshape next-(week)]
            [\itshape last-(week)]
        ]
    ]
    [quantificational
        [grade
            [abundant
                [basic
                    [\itshape many]
                    [\itshape much]
                ]
                [comparative
                    [\itshape more]
                    [\itshape most]
                ]
            ]
            [scarce
                [mass/non-count
                    [\itshape little]
                    [\itshape least]
                ]
                [unit/count
                    [\itshape few]
                    [\itshape fewer]
                    [\itshape fewest]
                ]
            ]
        ]
        [universal
            [\itshape all]
            [\itshape each]
            [\itshape every]
            [\itshape no]
        ]
        [dual
            [\itshape both]
        ]
        [singular
            [\itshape a/an]
            [\itshape another]
        ]
        [partial
            [positive
                [\itshape a little]
                [\itshape a few]
            ]
            [sufficing
                [\itshape enough]
                [\itshape sufficient]
            ]
            [dual
                [\itshape either]
                [\itshape neither]
            ]
            [existential
                [\itshape any]
                [\itshape some]
            ]
        ]
        [cardinal
            [singular
            [\itshape one]]
            [plural
                [\itshape zero]
                [\itshape two]
                [\itshape three]
                [etc.]
            ]
        ]
    ]
    [interrogative
        [\itshape what]
        [\itshape which]
    ]
]
\end{forest}
    \caption{The determinatives and their semantics}
    \label{fig:determinative-semantics}
\end{figure}\is{determinative, determinative phrase (DP)|)}

\begin{tcolorbox}[title=From ``The love of a good woman'' by Alice Munro, colback=white]

Each NP is in bracketed (there are some smaller NPs inside these, but we'll ignore them for now). If there is a phrase that functions as Det in an NP, underline it.

\phantom{a}

    [These women] aren't so much older than [Kath] and [Sonje]. But [they] 've reached [a stage in [life] that [Kath] and [Sonje] dread]. [They] turn [the whole beach] into [a platform]. [Their burdens], [their strung-out progeny and maternal poundage], [their authority], can annihilate [the bright water], [the perfect small cove with [the red-limbed arbutus trees]], [the cedars growing crookedly out of [the high rocks]]. [Kath] feels [their threat] particularly, since [she] 's [a mother] now [herself]. When [she] nurses [her baby] [she] often reads [a book], sometimes smokes [a cigarette], so as not to sink into [a sludge of [animal function]]. And [she]'s nursing so that [she] can shrink [her uterus] and flatten [her stomach], not just provide [the baby]---[Noelle]---with [precious maternal antibodies].

    \phantom{xxx}[Kath] and [Sonje] have [their own Thermos of [coffee]] and [their extra towels, rigged up as [a shelter] for [Noelle]]. [They] have [their cigarettes] and [their books]. [Sonje] has [a book by [Howard Fast]]. [Her husband] has told [her] that, if [she] has to read [fiction], [she] should read [Fast]. [Kath] is reading [the short stories of [Katherine Mansfield]] and [the short stories of [D. H. Lawrence]]. [Sonje] has got into [the habit of putting down [her own book] and picking up [whichever book of [Kath's] that [Kath] is not reading at [the moment]]]. [She] limits [herself] to [one story] and then goes back to [Howard Fast].
\end{tcolorbox}

\section{NPs as determiners}

examples such as \textit{my mother's sister}

\section{Summary}

This chapter introduced several fundamental concepts in English grammar:

\begin{itemize}[noitemsep]
    \item \textit{Categories} and \textit{functions} have a many-to-many relationship, complicating simplistic definitions of lexical categories.
    \item \textit{Lexical categories} include nouns, verbs, adjectives, determinatives, and others, while \textit{phrasal categories} are their expanded forms (e.g., noun phrases, verb phrases). But \textit{phrase} doesn't mean `more than one word'.
    \item Key \textit{syntactic functions} discussed were heads, subjects, modifiers, determiners, complements, and objects.
    \item \textit{Adjectives} and \textit{adjective phrases} were examined in detail, including their typical functions and distinguishing characteristics.
    \item \textit{Pronouns} were presented as a special class of nouns, with attention to their various forms and uses.
    \item \textit{Determinatives} and \textit{determinative phrases} were distinguished from adjectives and pronouns, with focus on their role in marking definiteness.
\end{itemize}

Teachers should be prepared for the challenge of explaining these concepts to learners clearly and simply but not simplistically. That doesn't mean we need to deal with all the exceptions and edge cases from the beginning, but it does mean that we shouldn't be setting traps for ourselves to fall into later.