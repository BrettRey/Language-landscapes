\appendix \label{ch:appendix}

\chapter{Understanding language proficiency}
\label{ch:proficiency}

If you've taught English for any length of time, you've probably been asked to assess someone's ``level'' or help a student reach a particular benchmark. Maybe you've wondered why one student who speaks fluently in class discussions can't pass a writing test, or why another who aces grammar exercises freezes up in conversation. These puzzles point to something fundamental: proficiency isn't the straightforward concept that scales and frameworks make it seem.

\section{What is proficiency?}

When we talk about someone's English proficiency, what exactly are we measuring? The question seems simple enough~-- after all, we make these judgments constantly. We know when someone speaks English ``well'' or ``poorly''. But dig a little deeper and the concept starts to fracture.

\ea A new colleague contributes sophisticated ideas in meetings using academic vocabulary and complex structures. But at lunch, when conversation turns to weekend plans or TV shows, she seems lost. What's her proficiency level?
\z

\ea A taxi driver navigates the city expertly, chatting with passengers about routes, weather, and local news. He's never studied English formally and his grammar is noticeably inconsistent. But he's never had a passenger complain. How proficient is he?
\z

These aren't trick questions~-- they're the reality of language proficiency. It's not a single skill that develops evenly. It's multidimensional, context-dependent, and deeply tied to what someone needs to do with language.

Most frameworks divide proficiency into skills: speaking, listening, reading, writing. But even ``speaking'' includes wildly different activities~-- giving presentations, telling jokes, comforting friends, negotiating conflicts. Each requires different resources. Then there's the fluency dimension we explored in Chapter \ref{ch:fluency}~-- that sense of flow and ease that doesn't always correlate with accuracy or complexity. Someone might speak with an easy flow despite grammatical inconsistencies, while another produces perfect sentences but sounds stilted and unnatural.

And we haven't even touched on pragmatic competence~-- knowing not just what you can say but what you should say. The ability to make requests politely, refuse without offending, or read between the lines of indirect communication. These pragmatic skills often determine real-world success more than grammatical accuracy, but they're notoriously difficult to evaluate.

\section{How frameworks see proficiency}

Proficiency frameworks represent a classic case of what James \textcite{scott1998} calls ``seeing like a state''~-- the need to make messy human realities legible for institutional purposes. Just as states create standardized maps that flatten complex territories into manageable abstractions, proficiency scales flatten the wild diversity of human communication into neat levels and descriptors.

This isn't necessarily bad. Institutions need ways to make decisions: who qualifies for university admission, who can practice medicine, who gets citizenship. Teachers need ways to group students and track progress. Learners themselves often want concrete goals and clear markers of achievement. The frameworks serve these legitimate needs.

But like Scott's simplified maps, proficiency scales capture some things while obscuring others. They make certain aspects of language visible and measurable while rendering others invisible or irrelevant.

\subsection{Canadian Language Benchmarks (CLB)}

The CLB emerged from a specific institutional need: helping adult immigrants integrate into Canadian society. It divides proficiency into 12 benchmarks across four skills, grouped into three stages:

\begin{itemize}[noitemsep]
\item Stage I (Benchmarks 1~--4): Basic language ability
\item Stage II (Benchmarks 5~--8): Intermediate ability  
\item Stage III (Benchmarks 9~--12): Advanced ability
\end{itemize}

The CLB's task-based approach specifies concrete things learners can do:

\begin{itemize}[noitemsep]
    \item At CLB 5 Speaking: ``Give simple informal warnings (e.g., \textit{Watch out! That's hot!})''
    \item At CLB 7 Reading: ``Follow instructions in a manual on how to operate familiar equipment''
\end{itemize}

This specificity makes the framework practical for teaching and assessment. But it also reveals what the framework values: workplace communication, consumer interactions, institutional engagement. The CLB maps the territory of ``successful integration'' as economic participation and system navigation. Other ways of living in Canada~-- maintaining home languages, creating art, building alternative communities~-- fall outside its view.

\subsection{Common European Framework of Reference (CEFR)}

The CEFR emerged from Europe's particular needs: facilitating mobility across European borders and educational systems. Its six-level structure (A1--C2) has become globally influential, perhaps because its descriptors are more abstract and thus more adaptable:

\begin{itemize}[noitemsep]
\item A1--A2: Basic user
\item B1--B2: Independent user
\item C1--C2: Proficient user
\end{itemize}

Where CLB specifies tasks, CEFR describes general capabilities. A B2 speaker ``can interact with a degree of fluency and spontaneity that makes regular interaction with native speakers quite possible without strain for either party''. This leaves room for interpretation~-- what counts as fluency? Which native speakers? But it also allows the framework to travel across contexts more easily than the CLB's specific Canadian tasks.

The 2018 update added scales for online interaction and plurilingual competence, recognizing that people don't keep their languages in separate boxes but draw on all their resources strategically. This is progress, but the framework still fundamentally sees like an institution that needs to sort people into levels.

\section{What gets lost}

When we see proficiency as levels and descriptors, what disappears?

First, the dynamic nature of communication. Real interaction involves constant adjustment, repair, negotiation. A ``B1 speaker'' might function like C1 when talking about their professional expertise but drop to A2 when their child needs medical attention and anxiety kicks in. The frameworks present proficiency as a stable attribute rather than a dynamic performance.

Second, the full range of communicative resources people use. Multilingual speakers don't just switch between languages~-- they mesh them creatively, use gesture and visual aids, recruit help from others. The grandmother who gets her prescription filled through a mix of basic English, pointing, and help from her grandchild is successfully communicating, but proficiency scales can't capture this resourcefulness.

Third, the social dimension of communication. Whether an interaction succeeds depends as much on the listener as the speaker. The same utterance might work perfectly with a patient, experienced interlocutor but fail with someone impatient or prejudiced. Proficiency isn't just in the speaker~-- it's in the interaction.

\section{Using frameworks wisely}

Given these limitations, we need to work with proficiency frameworks wisely.

First, remember what they're for. Frameworks are institutional tools for sorting and decision-making. They're not complete descriptions of human communication. When you need to assign students to classes or prepare them for tests, use the frameworks. When you're helping someone develop as a communicator, look beyond them.

Second, pay attention to what's not being measured. If you're teaching test preparation, be explicit about it: ``This is how to succeed at IELTS writing, which is different from real-world writing.'' Help learners develop both test-taking skills and actual communication abilities.

Third, recognize uneven profiles as normal. Someone might be CLB 8 in reading but CLB 5 in speaking. This isn't a problem to fix~-- it reflects their learning history and current needs. Work with these profiles rather than against them.

Fourth, consider alternative ways of documenting proficiency. Portfolios of actual work, recordings of real interactions, reflective writing about communication experiences~-- these capture dimensions that tests miss. Even if institutional requirements mean you need test scores, you can supplement them with richer evidence.

Finally, remember that proficiency is always partial and situated. The question isn't ``How proficient is this person?'' but ``Proficient enough for what?'' A taxi driver needs different English than a graduate student. Success looks different in different contexts.

\section{Conclusion}

Proficiency frameworks try to make the unmappable mappable~-- to create stable, measurable categories from the fluid, context-dependent reality of human communication. They succeed well enough for institutional purposes, which is why they persist. But as teachers, we need to see both what they illuminate and what they obscure.

The CLB and CEFR aren't going away. Your students will need to achieve certain levels for their goals. But you can help them understand these requirements as institutional hurdles rather than measures of their worth as communicators. You can validate the full range of their communicative resources, not just their test performance. And you can keep the messy, magnificent reality of human interaction at the center of your teaching, even as you prepare students for the simplified world of proficiency scales.




\chapter{List of Prepositions} \label{sec:preps-list}

The following is a thorough (but necessarily incomplete) list of the lexical items that would be classified as prepositions in the \textit{CGEL} framework. See \S\ref{sec:prepositions} for details about prepositions and PPs.

\begin{multicols}{3}
\begin{enumerate}[noitemsep]
    \item \textit{à la}
    \item \textit{a priori}
    \item \textit{abaft}
    \item \textit{aboard}
    \item \textit{about}
    \item \textit{above}
    \item \textit{aboveboard}
    \item \textit{abreast}
    \item \textit{abroad}
    \item \textit{absent}
    \item \textit{according} [+\textit{to} PP]
    \item \textit{across}
    \item \textit{adjacent} [+\textit{to} PP]
    \item \textit{adrift}
    \item \textit{afloat}
    \item \textit{aft}
    \item \textit{after}
    \item \textit{afterward}(\textit{s})
    \item \textit{against}
    \item \textit{ago}
    \item \textit{ahead}
    \item \textit{akimbo}
    \item \textit{albeit}
    \item \textit{aloft}
    \item \textit{along}
    \item \textit{alongside}
    \item \textit{although}
    \item \textit{amid}(\textit{st})
    \item \textit{among}(\textit{st})
    \item \textit{ante}
    \item \textit{anti}
    \item \textit{apart} [+\textit{from} PP]
    \item \textit{aplenty}
    \item \textit{apropos} [+\textit{of} PP]
    \item \textit{around}
    \item \textit{as}
    \item \textit{as for}
    \item \textit{as from}
    \item \textit{as if}
    \item \textit{as long as}
    \item \textit{as of}
    \item \textit{as per}
    \item \textit{as soon as}
    \item \textit{as though}
    \item \textit{as to}
    \item \textit{ashore}
    \item \textit{aside} [+\textit{from} PP]
    \item \textit{aslant}
    \item \textit{assuming}
    \item \textit{astride}
    \item \textit{at}
    \item \textit{athwart}
    \item \textit{atop}
    \item \textit{away}
    \item \textit{awry}
    \item \textit{back}
    \item \textit{backstage}
    \item \textit{backward}(\textit{s})
    \item \textit{bar}
    \item \textit{barring}
    \item \textit{based} [+\textit{on} PP]
    \item \textit{because} [+\textit{of} PP]\is{preposition, preposition phrase (PP)!because@\textit{because}}
    \item \textit{before}
    \item \textit{beforehand}
    \item \textit{behind}
    \item \textit{below}
    \item \textit{beneath}
    \item \textit{beside}(\textit{s})
    \item \textit{between}
    \item \textit{betweixt}
    \item \textit{beyond}
    \item \textit{but}
    \item \textit{by}
    \item \textit{by virtue}
    \item \textit{ceilingward}(\textit{s})
    \item \textit{chez}
    \item \textit{circa}
    \item \textit{clear} [+\textit{of} PP]
    \item \textit{close}(\textit{er}/\textit{est}) [+\textit{to} PP]
    \item \textit{come}
    \item \textit{compared} [+\textit{to} PP]
    \item \textit{concerning}
    \item \textit{consequent} [+\textit{on} PP]
    \item \textit{considering}
    \item \textit{contra}
    \item \textit{contrary} [+\textit{to} PP]
    \item \textit{counter} [+\textit{to} PP]
    \item \textit{counting}
    \item \textit{coupled} [+\textit{to} PP]
    \item \textit{depending}  [+\textit{on} PP]
    \item \textit{despite}
    \item \textit{down}
    \item \textit{downhill}
    \item \textit{downstage}
    \item \textit{downstairs}
    \item \textit{downstream}
    \item \textit{downtown}
    \item \textit{downward}(\textit{s})
    \item \textit{downwind}
    \item \textit{due} [+\textit{to} PP]
    \item \textit{during}
    \item \textit{earthward}(\textit{s)}
    \item \textit{effective}
    \item \textit{elsewhere}
    \item \textit{ere}
    \item \textit{except}(\textit{ing})
    \item \textit{excluding}
    \item \textit{exclusive} [+\textit{of} PP]
    \item \textit{failing}
    \item \textit{far}(\textit{ther}/\textit{thest})
    \item \textit{following}
    \item \textit{for}
    \item \textit{for all}\newline[+\textit{that} relative clause]
    \item \textit{fore}
    \item \textit{forth}
    \item \textit{forward}(\textit{s})
    \item \textit{fro}
    \item \textit{from}
    \item \textit{further}
    \item \textit{given}
    \item \textit{going} [+\textit{on} PP]
    \item \textit{granted}
    \item \textit{hence}
    \item \textit{henceforth}
    \item \textit{here}
    \item \textit{home}
    \item \textit{homeward}(\textit{s})
    \item \textit{if}
    \item \textit{immediately}\newline(in some dialects)
    \item \textit{in}
    \item \textit{in addition} [+\textit{to} PP]
    \item \textit{in case} [+\textit{of} PP]
    \item \textit{in charge} [+\textit{of} PP]
    \item \textit{in consequence} [+\textit{of} PP]
    \item \textit{in front} [+\textit{of} PP]
    \item \textit{in league} [+\textit{with} PP]
    \item \textit{in line} [+\textit{with} PP]
    \item \textit{in order} [+\textit{on} PP]
    \item \textit{in spite}\newline [+\textit{to} infinitival clause]
    \item \textit{in step} [+\textit{with} PP]
    \item \textit{in view} [+\textit{of} PP]
    \item \textit{inasmuch} [+\textit{as} PP]
    \item \textit{including}
    \item \textit{indoors}
    \item \textit{inland}
    \item \textit{inside}
    \item \textit{instead}
    \item \textit{into}
    \item \textit{inward}(\textit{s})
    \item \textit{irrespective} [+\textit{of} PP]
    \item \textit{landward}(\textit{s})
    \item \textit{leeward}(\textit{s})
    \item \textit{leftward}(\textit{s})
    \item \textit{less}
    \item \textit{lest}
    \item \textit{like}
    \item \textit{minus}
    \item \textit{modulo}
    \item \textit{near}(\textit{er}/\textit{est})
    \item \textit{nearby}
    \item \textit{next}
    \item \textit{no matter}
    \item \textit{north}
    \item \textit{northeast}
    \item \textit{northward}(\textit{s})
    \item \textit{northwest}
    \item \textit{notwithstanding}
    \item \textit{o'clock}
    \item \textit{of}
    \item \textit{off}
    \item \textit{offline}
    \item \textit{offshore}
    \item \textit{on}
    \item \textit{on account} [+\textit{of} PP]
    \item \textit{on board}
    \item \textit{on condition} \newline[+\textit{that} content clause]
    \item \textit{on top} [+\textit{of} PP]
    \item \textit{once}
    \item \textit{online}
    \item \textit{only}\newline(in some dialects)
    \item \textit{onshore}
    \item \textit{onto}/\textit{on to}
    \item \textit{onward}(\textit{s})
    \item \textit{opposite}
    \item \textit{other than}
    \item \textit{out}
    \item \textit{outdoors}
    \item \textit{outside}
    \item \textit{outward}(\textit{s})
    \item \textit{over}
    \item \textit{overboard}
    \item \textit{overhead}
    \item \textit{overland}
    \item \textit{overnight}
    \item \textit{overseas}
    \item \textit{owing} [+\textit{to} PP]
    \item \textit{pace}
    \item \textit{past}
    \item \textit{pending}
    \item \textit{per}
    \item \textit{plus}
    \item \textit{point}
    \item \textit{post}
    \item \textit{pre}
    \item \textit{preliminary} [+\textit{to} PP]
    \item \textit{preparatory} [+\textit{to} PP]
    \item \textit{previous} [+\textit{to} PP]
    \item \textit{prior} [+\textit{to} PP]
    \item \textit{pro}
    \item \textit{provided}/\textit{providing}
    \item \textit{pursuant}
    \item \textit{qua}
    \item \textit{re}
    \item \textit{rearward}(\textit{s})
    \item \textit{regarding}
    \item \textit{regardless}
    \item relative [+\textit{to} PP]
    \item \textit{respecting}
    \item \textit{rightward}(\textit{s})
    \item \textit{round}
    \item \textit{sans}
    \item \textit{save}(\textit{ing})
    \item \textit{seaward}(\textit{s})
    \item \textit{seeing} [+\textit{as} PP]
    \item \textit{shorward}(\textit{s})
    \item \textit{short} [+\textit{of} PP]
    \item \textit{sideward}(\textit{s})
    \item \textit{sideway}(\textit{s})
    \item \textit{since}
    \item \textit{skyward}(\textit{s})
    \item \textit{so}
    \item \textit{so as}
    \item \textit{south}
    \item \textit{southeast}
    \item \textit{southward}(\textit{s})
    \item \textit{southwest}
    \item \textit{spite} (in some dialects)
    \item \textit{subsequent} [+\textit{to} PP]
    \item \textit{sunward}(\textit{s})
    \item \textit{supposing}
    \item \textit{take away}
    \item \textit{than}
    \item \textit{then}
    \item \textit{there}
    \item \textit{thereabout}(\textit{s})
    \item \textit{thereby}
    \item \textit{therefore}
    \item \textit{thereof}
    \item \textit{through}
    \item \textit{throughout}
    \item \textit{till}
    \item \textit{times}
    \item \textit{to}
    \item \textit{together} [+\textit{with} PP]
    \item \textit{touching}
    \item \textit{toward}(\textit{s})
    \item \textit{turning} [+\textit{to} PP]
    \item \textit{unbeknown}(\textit{st})
    \item \textit{under}
    \item \textit{underfoot}
    \item \textit{underground}
    \item \textit{underneath}
    \item \textit{unless}
    \item \textit{unlike}
    \item \textit{until}
    \item \textit{unto}
    \item \textit{upon}
    \item \textit{upstage}
    \item \textit{upstairs}
    \item \textit{upstream}
    \item \textit{upwind}
    \item \textit{via}
    \item \textit{vis-à-vis}
    \item \textit{versus}
    \item \textit{wanting}
    \item \textit{west}
    \item \textit{westward}(\textit{s})
    \item \textit{when}
    \item \textit{whence}
    \item \textit{when}(\textit{so})\textit{ever}
    \item \textit{where}
    \item \textit{whereabouts}
    \item \textit{whereafter}
    \item \textit{whereas}
    \item \textit{whereby}
    \item \textit{wherein}
    \item \textit{whereupon}
    \item \textit{where}(\textit{so})\textit{ever}
    \item \textit{while}
    \item \textit{whither}
    \item \textit{windward}(\textit{s})
    \item \textit{with}
    \item \textit{within}
    \item \textit{without}
    \item \textit{worth}
\end{enumerate}
\end{multicols}


\chapter{List of determinatives}

The following is a thorough (but necessarily incomplete) list of the lexical items that would be classified as determinatives by \textit{CGEL}. See \S\ref{sec:DPs} for details about determinatives and DPs.

The numerals here are given up to 20, and then by tens and hundreds, but all the numerals belong to the category. (They also belong to the  category of nouns. See \textcite{reynolds:unpublished} for more about number words.)

\begin{multicols}{3}
\begin{enumerate}[noitemsep]
    \item\textit{a}/\textit{an}
\item\textit{a certain}
\item\textit{a few}
\item\textit{a little}
\item\textit{all}
\item\textit{another}
\item\textit{any}
\item\textit{anybody}
\item\textit{anyone}
\item\textit{anything}
\item\textit{anytime}
\item\textit{anywhere}
\item\textit{both}
\item\textit{certain}
\item\textit{each}
\item\textit{eight}
\item\textit{eighty}
\item\textit{either}
\item\textit{eleven}
\item\textit{enough}
\item\textit{every}
\item\textit{everybody}
\item\textit{everyone}
\item\textit{everything}
\item\textit{everywhere}
\item\textit{few}/\textit{fewer}/\textit{fewest}
\item\textit{fifteen}
\item\textit{fifty}
\item\textit{five}
\item\textit{forty}
\item\textit{four}
\item\textit{fourteen}
\item\textit{hundred}
\item\textit{last \op night}, \textit{week}, etc.\cp
\item\textit{little}/\textit{less}/\textit{least}
\item\textit{many}/\textit{more}/\textit{most}
\item\textit{many a}
\item\textit{million}
\item\textit{much}/\textit{more}/\textit{most}
\item\textit{neither}
\item\textit{next \op week}, \textit{month}, etc.\cp
\item\textit{nine}
\item\textit{nineteen}
\item\textit{ninety}
\item\textit{no}/\textit{none}
\item\textit{no one}
\item\textit{nobody}
\item\textit{nothing}
\item\textit{nowhere}
\item\textit{once}
\item\textit{one}
\item\textit{said}
\item\textit{seven}
\item\textit{seventeen}
\item\textit{seventy}
\item\textit{several}
\item\textit{six}
\item\textit{sixteen}
\item\textit{sixty}
\item\textit{some}
\item\textit{somebody}
\item\textit{someday}
\item\textit{someone}
\item\textit{something}
\item\textit{sometime}
\item\textit{somewhat}
\item\textit{somewhere}
\item\textit{sufficient}
\item\textit{ten}
\item\textit{that}/\textit{those}
\item\textit{the}
\item\textit{thirteen}
\item\textit{thirty}
\item\textit{this}/\textit{these}
\item\textit{thousand}
\item\textit{three}
\item\textit{thrice}
\item\textit{twelve}
\item\textit{twenty}
\item\textit{twice}
\item\textit{two}
\item\textit{umpteen}
\item\textit{us}
\item\textit{various}
\item\textit{we}
\item\textit{what}
\item\textit{whatever}
\item\textit{whatsoever}
\item\textit{which}
\item\textit{whichever}
\item\textit{you}
\item\textit{zero}
\end{enumerate}
\end{multicols}

\newpage
\chapter{Ordered list of teachable graphemes} \label{sec:gpcs-list}

The following is a list of high-frequency, reliable spelling patterns that are worth teaching to students who lack alphabetic literacy or who otherwise have trouble reading words in English. It includes individual graphemes, onsets, rimes, codas, and a few morphemes, such as \textit{-tion}, and \textit{--ed}. See \S\ref{sec:phonics} for details about graphemes and phonemes. The majority of the data is from \citet{Fry2004a}, supplemented by data from \citet{Gontijo2003} and my own queries of \textit{COCA} \citep{COCA}.

I have omitted items like $\langle$--ain$\rangle$, which, despite being fairly common, are unreliable: compare $\langle$again$\rangle$ /əˈɡ\uline{ɛn}/, $\langle$certain$\rangle$ /ˈsɜrt\uline{ən}/, and $\langle$main$\rangle$ /m\uline{eɪn}/. The items combine well, so that right from the beginning, you get highly decodable words, like \textit{it}, \textit{in}, \textit{at}, \textit{an}, \textit{tin}, \textit{tan}, \textit{did}, \textit{din}, \textit{rid}, \textit{rat}, \textit{ran}, \textit{drat}, \textit{end}, \textit{den}, \textit{ten}, \textit{net}, \textit{tent}, \textit{tend}, \textit{trend}, \textit{trim}, \textit{Tim}, \textit{mint}, etc. Most of these words are quite common and may be familiar even to beginning English-language learners.

\begin{longtable}{lllll}
\caption{Frequency of Graphemes and Corresponding Phonemes} \label{tab:gpcs} \\
\hline
\textbf{Number} & \textbf{Graphemes} & \textbf{Frequency} & \textbf{Example} & \textbf{Phonemes} \\
\hline
\endfirsthead
\hline
\textbf{Number} & \textbf{Graphemes} & \textbf{Frequency} & \textbf{Example} & \textbf{Phonemes} \\
\hline
\endhead
\endfoot

\hline \hline
\endlastfoot
1 & $\langle$t$\rangle$ & 7,080 & \textit{time} & /t/ \\
2 & $\langle$n$\rangle$ & 6,319 & \textit{not} & /n/ \\
3 & $\langle$i$\rangle$ & 5,346 & \textit{in} & /ɪ/ \\
4 & $\langle$r--$\rangle$ & 5,293 & \textit{real} & /r/ \\
5 & $\langle$l$\rangle$ & 4,894 & \textit{love} & /l/ \\
6 & $\langle$a$\rangle$ & 4,192 & \textit{at} & /æ/ \\
7 & $\langle$--an$\rangle$ & 4,016 & \textit{can} & /æn/ \\
8 & $\langle$--r$\rangle$ & 3,821 & \textit{car} & /r/ \\
9 & $\langle$d$\rangle$ & 3,592 & \textit{day} & /d/ \\
10 & $\langle$e$\rangle$ & 3,316 & \textit{end} & /ɛ/ \\
11 & $\langle$m$\rangle$ & 3,304 & \textit{more} & /m/ \\
12 & $\langle$p$\rangle$ & 3,296 & \textit{people} & /p/ \\
13 & $\langle$c$\rangle$ & 3,049 & \textit{case} & /k/ \\
14 & $\langle$s$\rangle$ & 2,582 & \textit{still} & /s/ \\
15 & $\langle$b$\rangle$ & 2,242 & \textit{back} & /b/ \\
16 & $\langle$--er$\rangle$ & 1,979 & \textit{her} & /ər/ \\
17 & $\langle$--nd$\rangle$ & 1,970 & \textit{fund} & /nd/ \\
18 & $\langle$s$\rangle$ & 1,956 & \textit{these} & /z/ \\
19 & $\langle$o$\rangle$ & 1,876 & \textit{open} & /oʊ/ \\
20 & $\langle$--ing$\rangle$ & 1,832 & \textit{doing} & /ɪŋ/ \\
21 & $\langle$--y$\rangle$ & 1,801 & \textit{funny} & /i/ \\
22 & $\langle$e$\rangle$ & 1,765 & \textit{me} & /i/ \\
23 & $\langle$o$\rangle$ & 1,723 & \textit{other} & /ʌ/ \\
24 & $\langle$o$\rangle$ & 1,558 & \textit{not} & /ɑ/ \\
25 & $\langle$u$\rangle$ & 1,509 & \textit{up} & /ʌ/ \\
26 & $\langle$v$\rangle$ & 1,485 & \textit{view} & /v/ \\
27 & $\langle$--at$\rangle$ & 1,462 & \textit{cat} & /æt/ \\
28 & $\langle$a$\rangle$ & 1,438 & \textit{ago} & /ʌ/ \\
29 & $\langle$i$\rangle$ & 1,347 & \textit{animal} & /ʌ/ \\
30 & $\langle$f$\rangle$ & 1,070 & \textit{from} & /f/ \\
31 & $\langle$c--$\rangle$ & 1,067 & \textit{city} & /s/ \\
32 & $\langle$a$\rangle$ & 1,002 & \textit{agent} & /eɪ/ \\
33 & $\langle$u$\rangle$ & 907 & \textit{unit} & /ju/ \\
34 & $\langle$g$\rangle$ & 842 & \textit{again} & /g/ \\
35 & $\langle$--tion$\rangle$ & 820 & \textit{action} & /ʃən/ \\
36 & $\langle$--ly$\rangle$ & 815 & \textit{lovely} & /li/ \\
37 & $\langle$a-e$\rangle$ & 790 & \textit{ate} & /eɪ/ \\
38 & $\langle$e$\rangle$ & 763 & \textit{effect} & /ʌ/ \\
39 & $\langle$--st$\rangle$ & 759 & \textit{fast} & /st/ \\
40 & $\langle$h--$\rangle$ & 746 & \textit{house} & /h/ \\
41 & $\langle$--nt$\rangle$ & 730 & \textit{want} & /nt/ \\
42 & $\langle$g--$\rangle$ & 647 & \textit{gem} & /d͡ʒ/ \\
43 & $\langle$--le$\rangle$ & 620 & \textit{able} & /l/ \\
44 & $\langle$pr--$\rangle$ & 605 & \textit{program} & /pr/ \\
45 & $\langle$k$\rangle$ & 601 & \textit{like} & /k/ \\
46 & $\langle$w--$\rangle$ & 578 & \textit{week} & /w/ \\
47 & $\langle$st--$\rangle$ & 566 & \textit{stop} & /st/ \\
48 & $\langle$i-e$\rangle$ & 555 & \textit{ice} & /aɪ/ \\
49 & $\langle$i$\rangle$ & 554 & \textit{item} & /aɪ/ \\
50 & $\langle$--ld$\rangle$ & 499 & \textit{old} & /ld/ \\
51 & $\langle$--ll$\rangle$ & 489 & \textit{bell} & /l/ \\
52 & $\langle$ar(e)$\rangle$ & 474 & \textit{are} & /ɑr/ \\
53 & $\langle$--ay$\rangle$ & 460 & \textit{way} & /eɪ/ \\
54 & $\langle$--ss$\rangle$ & 442 & \textit{pass} & /s/ \\
55 & $\langle$--ns$\rangle$ & 416 & \textit{plans} & /nz/ \\
56 & $\langle$sh$\rangle$ & 389 & \textit{fish} & /ʃ/ \\
57 & $\langle$fr--$\rangle$ & 387 & \textit{fresh} & /fr/ \\
58 & $\langle$o-e$\rangle$ & 370 & \textit{home} & /oʊ/ \\
59 & $\langle$ou$\rangle$ & 366 & \textit{double} & /ʌ/ \\
60 & $\langle$th$\rangle$ & 363 & \textit{other} & /ð/ \\
61 & $\langle$ng$\rangle$ & 362 & \textit{thing} & /ŋ/ \\
62 & $\langle$i-e$\rangle$ & 339 & \textit{give} & /ɪ/ \\
63 & $\langle$--or$\rangle$ & 321 & \textit{labour} & /ər/ \\
64 & $\langle$--or$\rangle$ & 312 & \textit{for} & /ɔr/ \\
65 & $\langle$tr--$\rangle$ & 300 & \textit{train} & /tr/ \\
66 & $\langle$--all$\rangle$ & 295 & \textit{ball} & /ɑl/ \\
67 & $\langle$u-e$\rangle$ & 290 & \textit{tune} & /(j)u/ \\
68 & $\langle$ch$\rangle$ & 272 & \textit{each} & /t͡ʃ/ \\
69 & $\langle$--nce$\rangle$ & 261 & \textit{once} & /ns/ \\
70 & $\langle$ee$\rangle$ & 249 & \textit{keep} & /i/ \\
71 & $\langle$ea$\rangle$ & 245 & \textit{eat} & /i/ \\
72 & $\langle$--ore$\rangle$ & 237 & \textit{more} & /ɔr/ \\
73 & $\langle$ur$\rangle$ & 234 & \textit{turn} & /ər/ \\
74 & $\langle$--ill$\rangle$ & 233 & \textit{will} & /ɪl/ \\
75 & $\langle$--ight$\rangle$ & 230 & \textit{right} & /ʌɪt/ \\
76 & $\langle$j$\rangle$ & 229 & \textit{just} & /d͡ʒ/ \\
77 & $\langle$gr--$\rangle$ & 229 & \textit{great} & /gr/ \\
78 & $\langle$ou$\rangle$ & 227 & \textit{out} & /ʌʊ/ \\
79 & $\langle$z$\rangle$ & 222 & \textit{size} & /z/ \\
80 & $\langle$y$\rangle$ & 211 & \textit{my} & /aɪ/ \\
81 & $\langle$ai$\rangle$ & 208 & \textit{rain} & /eɪ/ \\
82 & $\langle$rr$\rangle$ & 207 & \textit{carry} & /r/ \\
83 & $\langle$--ir$\rangle$ & 204 & \textit{sir} & /ɜr/ \\
84 & $\langle$--ack$\rangle$ & 203 & \textit{back} & /æk/ \\
85 & $\langle$pl--$\rangle$ & 203 & \textit{please} & /pl/ \\
86 & $\langle$x$\rangle$ & 202 & \textit{experience} & /ks/ \\
87 & $\langle$ull$\rangle$ & 200 & \textit{pull} & /ʊl/ \\
88 & $\langle$qu$\rangle$ & 191 & \textit{quick} & /kw/ \\
89 & $\langle$qu$\rangle$ & 191 & \textit{technique} & /k/ \\
90 & $\langle$br--$\rangle$ & 190 & \textit{bring} & /br/ \\
91 & $\langle$a-e$\rangle$ & 187 & \textit{village} & /ɪ/ \\
92 & $\langle$--ls$\rangle$ & 180 & \textit{levels} & /lz/ \\
93 & $\langle$sp--$\rangle$ & 178 & \textit{special} & /sp/ \\
94 & $\langle$oo$\rangle$ & 173 & \textit{moon} & /u/ \\
95 & $\langle$ar$\rangle$ & 168 & \textit{dollar} & /ər/ \\
96 & $\langle$al(l)$\rangle$ & 165 & \textit{all} & /ɔl/ \\
97 & $\langle$--ell$\rangle$ & 164 & \textit{well} & /ɛl/ \\
98 & $\langle$--nk$\rangle$ & 150 & \textit{thank} & /ŋk/ \\
99 & $\langle$a-e$\rangle$ & 147 & \textit{dance} & /æ/ \\
100 & $\langle$au$\rangle$ & 146 & \textit{auto} & /ɔ/ \\
101 & $\langle$--ake$\rangle$ & 144 & \textit{make} & /ek/ \\
102 & $\langle$ea$\rangle$ & 139 & \textit{head} & /ɛ/ \\
103 & $\langle$dr--$\rangle$ & 132 & \textit{drive} & /dr/ \\
104 & $\langle$ay$\rangle$ & 131 & \textit{day} & /eɪ/ \\
105 & $\langle$thr--$\rangle$ & 127 & \textit{through} & /θr/ \\
106 & $\langle$oa$\rangle$ & 126 & \textit{oat} & /oʊ/ \\
107 & $\langle$ow$\rangle$ & 124 & \textit{own} & /oʊ/ \\
108 & $\langle$--ame$\rangle$ & 124 & \textit{same} & /em/ \\
109 & $\langle$o$\rangle$ & 123 & \textit{off} & /ɔ/ \\
110 & $\langle$ow$\rangle$ & 119 & \textit{owl} & /aʊ/ \\
111 & $\langle$str--$\rangle$ & 114 & \textit{street} & /str/ \\
112 & $\langle$oo$\rangle$ & 114 & \textit{look} & /ʊ/ \\
113 & $\langle$o(n)$\rangle$ & 112 & \textit{son} & /ʌ/ \\
114 & $\langle$--ink$\rangle$ & 106 & \textit{think} & /ɪŋk/ \\
115 & $\langle$bl--$\rangle$ & 106 & \textit{black} & /bl/ \\
116 & $\langle$ir$\rangle$ & 104 & \textit{girl} & /ər/ \\
117 & $\langle$a(n)$\rangle$ & 100 & \textit{danger} & /eɪ/ \\
118 & $\langle$y$\rangle$ & 100 & \textit{system} & /ɪ/ \\
119 & $\langle$--ide$\rangle$ & 97 & \textit{side} & /aɪd/ \\
120 & $\langle$oi$\rangle$ & 92 & \textit{oil} & /ɔɪ/ \\
121 & $\langle$--ps$\rangle$ & 89 & \textit{groups} & /ps/ \\
122 & $\langle$fl--$\rangle$ & 86 & \textit{floor} & /fl/ \\
\hline
\end{longtable}

