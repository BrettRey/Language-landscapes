\chapter{Coherence and cohesion} \label{ch:coherence}

\epigraph{I once thought thought branched\\\phantom{~~~~~~~~~} like winter-bare maple.\\
Now I know better: it spreads\\\phantom{~}\hfill 
like mycorrhizal networks beneath our feet.}{}

\section{Introduction}
Consider the following excerpt from the short story ``Quantum convention'' by Eric Schlich:
\ea \label{ex:parents}
    \ea[]{\textit{Both my parents \uline{were} high school teachers~-- my mother English, my father History.}}
    \ex[]{\textit{They \uline{used to come} home with stacks of student essays.}}
    \ex[]{\textit{Sometimes they'\uline{d take} sick days just to catch up on the grading.}}
    \z
\z

This is a simple description of the narrator's parents and their travails. But it pulls you right into the story because of its structure.

In Chapter \ref{ch:info-package}, we looked at how to topicalize various elements within the sentence, but a topic, having been established in one sentence, has to be unfurled across those that follow. It can neither simply be left to drop nor hoisted so often that the text becomes choppy or repetitive.

\is{topic!chain|(}
The first sentence (\ref{ex:parents}a) gives us \textit{both my parents}. That \textit{both} is then bisected into \textit{my mother} and \textit{my father}, and finally simplified to \textit{they} in (b \& c). Pronouns like \textit{they} create coherence by linking to and renewing previously invoked referents \citep{HallidayHasan1976,BrownYule1983}.

Oddly enough, we can even create a sense of coherence in what is unsaid. Consider the ill-advised rewrite in (\ref{ex:parents-were-teachers-revised}).

\ea \label{ex:parents-were-teachers-revised}
    \ea[]{\textit{Both my parents had jobs as high school teachers.}}
    \ex[]{\textit{My mother worked as a high school English teacher, and my father worked as a high school History teacher.}}
    \ex[]{\textit{One part of their job was teaching high school students.}}
    \ex[]{\textit{There were a lot of high school students and, high school students write essays as part of their studies.}}
    \ex[]{\textit{There were a lot of essays produced.}}
    \ex[]{\textit{The teaching and the writing happened at the school.}}
    \ex[]{\textit{Part of my parents' jobs as high school teachers was grading essays that are produced by students at school, and jobs are supposed to be performed at the workplace. \dots}}
    \z
\z

Instead of this tediously detailed exposition suggesting all sorts of digressions, Schlich says only what we need to know. He strips (\ref{ex:parents-were-teachers-revised}b) down to \textit{my mother English, my father History}, a bare topic--comment construction that nevertheless recalls key elements~-- my parents and teachers~-- from the foregoing clause.

The connection between teachers and students is so obvious that it literally goes without saying. Including the details in (c) could only imply something less than straightforward. Likewise, the reason for bringing the essays home is left to us to infer, a process that requires the reader to draw on what has gone before. These mental passages back and forth among the ideas provide the warp and weft that allow the passage in (\ref{ex:parents}) to cohere so tightly.
\is{topic!chain|)}

\bigskip

\is{past tense|(}\is{tense!past|(}
There is more to this than coherence. Consider the constructions used to manage past time reference. The underlined \textit{were} in (\ref{ex:parents}a) uses the simple-past tense construction to establish the background. But establishing this foundation implies that a structure is to follow.

\is{use to@\textit{used to}|(}\is{habitual \textit{use to}|(}
The second sentence (b) starts building by switching to the \textit{used to} construction. \textit{Used to} has a past habitual meaning, but in the discourse context, it frames the beginning of a story.
\is{use to@\textit{used to}|)}

\is{would@\textit{would}|(}
The third sentence (c) shifts to the past-tense modal \textit{would} (contracted as \textit{'d}) combined with the \textit{take} VP. This \textit{would}, again means habitual past action, similar to \textit{used to}. But in the discourse, \textit{'d} is used to elaborate the details of the story.
\is{would@\textit{would}|)}

Taken together, these verb constructions create a layered discourse structure that funnels the reader into the details of the parents' work life and challenges \citep{Suh1992}.

\begin{enumerate}[noitemsep]
    \item Background (simple past)
    \item Story frame (\textit{used to})
    \item Story elaboration (\textit{would})
\end{enumerate}
\is{habitual \textit{use to}|)}\is{past tense|)}\is{tense!past|)}

Such precision requires very particular structural choices, though not necessarily ones most speakers or writers could articulate. Discourse analysis like that above uncovers the subtle ways in which language is used to shape the structure and coherence of texts; it explores what the linguistic structures achieve in a text and how they do so.

\subsection{What is discourse?}
\is{discourse|(}

\textsc{Discourse} is another of those terms that has both an everyday meaning and a technical one. In everyday use, it refers to spoken or written communication, discussion, or debate, as in \textit{the discourse around climate change has become highly politicized}. We might talk about \textit{public discourse}, \textit{political discourse}, or \textit{academic discourse}, usually to highlight the ways language is being used in a particular domain or to achieve certain purposes. There's often a sense that discourse is not just casual talk, but language used in a more deliberate, purposeful, or even strategic way.

In linguistics, \textit{discourse} has come to mean any stretch of language longer than a sentence, especially when considered in its full communicative and social context. In other words, it's conversations, speeches, stories, essays, articles, and other runs of language. Unlike isolated examples crafted solely to illustrate grammatical rules, discourse involves language actively used for communication, action, and interaction in the world.

So when we do discourse analysis, we're looking at language from observed, recorded, transcribed, or collected texts. These texts could be \textsc{monologic} (produced by one person) or \textsc{dialogic} (involving two people), or they could involve interactions among larger groups. We're interested in how meanings are constructed and negotiated in that text, how social relationships and identities are enacted, how power structures are reinforced or challenged, all through the minute details of the language used.

Discourse analysts ask questions like: How is the topic being introduced and developed? What roles and identities are the participants taking on? What assumptions or biases are being revealed? How are arguments being structured and evidence presented? What rhetorical strategies are being employed to persuade or manipulate? By carefully examining the linguistic choices made at every level~-- from the word to the clause to the larger text structure~-- discourse analysis can provide a rich, nuanced understanding of how language works in context.

As language teachers, we can use discourse as one more lens through which to reflect on English and to help our students to do so. After all, our learners' goal is not just to master isolated bits of grammar and vocabulary, but to communicate effectively. This is not to say that all aspects of English discourse are particular to English; the use of pronouns, for example, is common across languages, but the past-tense $\rightarrow$ \textit{used to} $\rightarrow$ \textit{would} structure of getting into a story may not no have direct translations in other languages.

Students may benefit from knowing how to structure writing for various purposes, how to participate in a job interview, how to tell an engaging story, how to express disagreement politely, how to interpret the hidden meanings in an advertisement or political speech. This is discourse competence: the ability to understand and deal with text in a way that's not just grammatically correct, but socially appropriate and communicatively effective for a given purpose and context.

But more than the students, it may be us teachers who really need an understanding of discourse. It's easy to tell students how to form a given construction. But without an awareness of when and why it's deployed in texts (including conversations), we risk reducing language classes to the study of a set of mechanical, purposeless rules.
\is{discourse|)}

\section{Coherence in discourse}

\subsection{What is coherence?}
\is{coherence|(}

\textsc{Coherence} is the property that distinguishes a well-formed text from a random collection of sentences. It's what makes a text ``hang together'' as a unified whole, allowing the reader to follow the flow of ideas and grasp the overall meaning. 

But what exactly creates this sense of unity and flow? Is it the presence of explicit linking words like \textit{moreover}, \textit{however}, and \textit{therefore}? Many writing textbooks would have you believe so. They often present these connectors as the magic glue that holds a text together.

While linking words can certainly be useful signposts, they are not the main drivers of coherence. In fact, an overreliance on them can make writing seem artificial and formulaic. The real key to coherence lies elsewhere: in the careful structuring of information and the strategic use of lexical cohesion~-- that is, the way words and phrases are repeated, varied, and related across the text \citep{HallidayHasan1976}.
\is{lexical cohesion|(}

To understand how this works, let's consider a simple example that may not look so simple.

\ea[]{\textit{\textcolor{xGreen}{The Renaissance} marked \textcolor{xGreen}{a profound \uline{transformation}} in European \uuline{art} and \textcolor{xPurple}{culture}. 
    This \textcolor{xPurple}{cultural} \textcolor{xGreen}{\uline{transformation}} was particularly evident in the \uline{flourishing} of visual \uuline{arts}. 
    \textcolor{xOrange}{\uuline{Artists}} of the \textcolor{xGreen}{era}, such as \textcolor{xOrange}{Michelangelo and Leonardo da Vinci}, \uline{revolutionized} \uuline{artistic} \textcolor{xPink}{techniques}. 
    \textcolor{xOrange}{Their} \uline{innovative} \textcolor{xPink}{approaches} heralded a \uline{new} \textcolor{xGreen}{era} in the \textcolor{xPink}{representation} of \textcolor{xPurple}{human experience}.}}\label{ex:history}
\z


The example in (\ref{ex:history}), while a little visually hectic with its colours and underlines, actually demonstrates a powerful way lexical cohesion works. Notice how the different colours create distinct chains, each focusing on a key aspect of the Renaissance. We see a green chain denoting the era itself, a purple chain linking culture to human experience, and an orange one highlighting the artists driving the change.  The pink chain draws our eye to the core of the artistic shift~-- techniques, approaches, and representation.  Finally, careful single underlining marks the words conveying a sense of dynamism and newness, while double underlines picks out those related to art. You can also see a clustering of green at the top, orange in the middle, and pink near the end. Together, these chains create a multi-layered picture of the Renaissance, leading the reader from the broad concept of transformation to the specific outcomes in artistic innovation.

This is the essence of coherence: the logical progression of ideas, supported by the strategic repetition and variation of key words and phrases. I'll return to this example a number of times in this section, so there's no need to understand each aspect yet.

While coherence is the overall sense that a text hangs together, cohesion refers to the specific linguistic tools writers and speakers use to create this effect. We can have one without the other, but they usually go together.

An example of a coherent paragraph with almost no cohesive devices is (\ref{ex:cohere-wout-cohesives}).

\ea \label{ex:cohere-wout-cohesives}
\textit{A runner woke before sunrise. A gardener prepared compost behind a shed. Children discovered an old boat stranded near pebbles. A tailor examined patterns under faint light. Someone arranged cutlery in a distant kitchen. Nearby, a violinist practiced unfamiliar scores. Another figure observed silent machinery from a hidden alcove. In short, the world went on without him.}
\z

\noindent One employing cohesive devices but which is nevertheless incoherent is (\ref{ex:incoherent}).

\ea \label{ex:incoherent}
\textit{The cat sat on the mat. The mat was blue, a primary colour, and rainbows include all the colours. They appear after rain and are the stuff of folklore. Cats don't like the rain, but they also commonly appear in folklore. Mats, however, don't, not even blue ones, but flying carpets are associated with Arabian folklore, and they're a kind of mat.}
\z
\is{lexical cohesion|)}
\is{coherence|)}

\section{Cohesive devices}
\is{cohesion|(}

\textsc{Cohesion} refers to the linguistic means by which \textsc{coherence} is achieved. In other words, it's the visible manifestation of the underlying semantic ties that hold a text together. There are several types of cohesive devices, each contributing in its own way to the overall unity and flow of a text.

The most basic way we create connections in text is through reference~-- using words that point to other elements in the discourse.

\subsubsection*{Reference, pronouns, and pro-forms}
\is{reference|(}\is{pronoun|(}\is{pro-form|(}

\textsc{Reference} is a broad term that encompasses the use of words to point to other elements in a text or to the world outside the text. Pronouns are a common type of referential device, but they are not the only one.

It's important to distinguish between pronouns as a category of words and pro-forms as a type of meaning relation. Pro-forms are words or phrases that express the same content as another element in the text, with the meaning being recoverable from the context. They ``stand in for'' the other element, forming cohesion chains and reducing verbatim repetitions. This is the definition usually given for pronouns, and while it's typical (semantic) pronoun behaviour, it's not a good definition, as we discussed back in Section \ref{sec:pronouns}.

Most pronouns function as pro-forms, as in (\ref{ex:Ethan}).

\ea[]{\textit{\uline{Ethan} is an excellent teacher. \uline{He} always explains things clearly.}} \label{ex:Ethan}
\z

Here, the pronoun \textit{he} refers to (``stands in for'') the noun \textit{Ethan}, creating a cohesive link between the sentences. This act of referring back to something already mentioned is called \textsc{anaphora}, and the word or phrase it refers to is the \textsc{antecedent}.
\is{anaphora|(}\is{antecedent|(}

But, not all pronouns have antecedents. That is, they're not all pro-forms. Interrogative \textit{who}, for instance, does not stand in for any other element; instead, it signals that the speaker is asking about the identity of a person, as in (\ref{ex:who-is}; See Section \ref{sec:interrogative-phrases}).\footnote{Don't confuse this with relative \textit{who}, which does refer. See Section \ref{sec:focused-articulated}.}

\ea[]{\textit{\uline{Who} is an excellent teacher?}}\label{ex:who-is}
\z

Conversely, not every case of anaphora involves pronouns. Words from other lexical categories can also function as pro-forms, as \textit{did so} does in (\ref{ex:did-so}).

\ea[]{\textit{She told me to \uline{begin}, and so I \uline{did so}.}}\label{ex:did-so}
\z
\is{anaphora|)}\is{antecedent|)}

\paragraph*{Reflexive pronouns} Reflexives (such as \textit{myself}, \textit{yourself}, \textit{himself}, \textit{herself}, \textit{itself}, \textit{ourselves}, \textit{yourselves}, and \textit{themselves}) play a crucial role in maintaining topicality within a clause, particularly when the subject of an action is also its object.
\is{reflexive pronoun|(}\is{topic|(}\is{pronoun!reflexive|(}

Consider the examples in (\ref{ex:Brett-himself}), each of which involves just one person.

\ea\label{ex:Brett-himself}
\ea[\textsuperscript{?}]{\textit{Brett hurt Brett.} \hfill[clashing topical implications]}
\ex[\textsuperscript{?}]{\textit{Brett hurt him.} \hfill[clashing topical implications]}
\ex[]{\textit{Brett hurt himself.} \hfill[maintains topicality]}
\z
\z

In (a), the repetition of \textit{Brett} in the object position conflicts with the assumption that topics need not be reintroduced within a clause, suggesting there are two Bretts, even though this interpretation has been ruled out. In (b), \textit{him} avoids repetition but implies a different person, again clashing with the assumption that Brett remains the sole topic.

The reflexive pronoun \textit{himself} resolves these issues, maintaining Brett as the clear topic throughout the clause without reintroducing the noun phrase:

\begin{enumerate}[noitemsep]
    \item It preserves Brett as the consistent topic, even in the object position.
    \item It signals explicitly that the subject and object are the same, maintaining topical coherence.
    \item It respects the ``given'' status of the information without disrupting the flow.
\end{enumerate}
\is{topic|)}\is{reflexive pronoun|)}\is{pronoun!reflexive|)}

This use of reflexive pronouns is particularly important with transitive verbs like \textit{hurt}, which require an object. When the action reflects back on the subject, the reflexive pronoun allows us to maintain a single, clear topic throughout the clause.

For many speakers, some reflexive pronouns (in particular \textit{myself} and \textit{yourself}) have acquired a secondary, socially driven function where their original discourse meaning~-- coreferentiality within a clause~-- is sidelined in favour of signalling formality or self-reference in a more distant, impersonal sense, as in (\ref{ex:social-myself}). This socially marked use can cause some confusion or even resistance among speakers who expect reflexives to serve only their coreferential, discourse-specific roles \citep{Gilman1989}.

\ea\textit{Please, contact \uline{myself} with any questions.}\label{ex:social-myself}
\z

Reflexives also have the meaning of `alone' or `without the help of others', as in \textit{He did it himself}. This has little to do with coherence and is another reminder that exclusively one-to-one form--meaning pairings are rare.

\paragraph*{Definiteness}
\is{definiteness|(}\is{deixis|(}
Another important concept related to reference is \textsc{definiteness}. Noun phrases can be either definite or indefinite, depending on whether the speaker assumes the listener can identify the referent. One way you know your listener can identify the referent is if it's already been mentioned in the discussion. In other words, if an NP has an antecedent, it's going to be definite. For this reason, most pronouns are definite.

That may not be obvious. We often think of definite noun phrases as being those marked with the definite article \textit{the}, and those marked by the indefinite article \textit{a/an} or no article as being indefinite. And this is generally true, but expressing definiteness or indefiniteness is a more complex matter than that. First, there are a lot of other indefinite determinatives and a few other definite ones too (see Section \ref{sec:definite-v-indef}).

This interplay between definiteness and pronoun use is an important aspect of creating coherence in a text, at least in a English one. By using pronouns to refer back to already-mentioned entities, and by marking new entities as indefinite, writers can guide readers through the flow of information and help them track the main participants in the discourse. Keep in mind, though, that many languages create perfectly coherent texts without marking definiteness. It can be done, but in English, failing to mark definiteness is generally seen as unhelpful at best and often as ungrammatical.

It's worth noting that pronouns and pro-forms are often \textsc{deictic}, meaning they point to something, including something in the discourse. For example, the meanings of \textit{I} and \textit{you} change depending on who is speaking and who is being addressed. Similarly, the prepositions \textit{here} and \textit{now} point to the time and place where the speaker is, and \textit{last week}, \textit{yesterday}, \textit{this morning}, \textit{tomorrow}, and other related words all point to times relative to the speaker's point in time.

Most languages use pronouns and pro-forms in the ways described, so such use isn't anything students will be unfamiliar with, but the processing demands of dealing with a second language may be such that it's easier for students to simply repeat a given phrase verbatim instead of switching to a pronoun or another pro-form. In writing, they may get distracted and fail to use a pronoun where they could or sometimes leave in a pronoun where the antecedent was never established or has been edited out. And in listening or reading, it's not unusual to lose track of the antecedent of a given pronoun.

In summary, pronouns are a central part of the reference system in English, and as pro-forms, they work with definiteness and deixis to create coherence in texts. By attending to these various aspects of reference, writers~-- it's often impractical and of less urgency to address these topics in speech~-- can create texts that are both cohesive and coherent, guiding readers smoothly through the development of ideas.
\is{deixis|)}\is{definiteness|)}

\paragraph*{Pronouns vs deictic determinatives}
\is{deixis!determinative|(}
\textit{This}, along with \textit{that} (and even \textit{which}), are used without head noun to refer a preceding topic (and item, idea, situation), sentence, or paragraph. Many years ago, this was regularly criticized (the \textit{this} in this sentence is an example), but today, such criticism is rare and usually unwarranted. Occasionally, the reference is improved by the addition of the head noun, but often its omission is the better choice.

The distinction between pronouns and deictic determinatives hinges significantly on their function within discourse coherence and attention management. Pronouns, such as \textit{it}, primarily sustain focus on established referents, seamlessly maintaining coherence with minimal cognitive load. This characteristic is crucial in texts where continuity with previously introduced entities is paramount.

Deictic determinatives, notably \textit{this}, typically shift, broaden, or narrow the discourse focus. They are often employed to highlight new propositions or aspects deriving from but not identical to prior text elements. This shift can introduce a subtle complexity, requiring readers to integrate or reinterpret elements within the discourse dynamically. Such usage not only enriches the narrative but also invites deeper engagement by challenging readers to form connections that are not immediately obvious.

As a result, a typical pattern is antecedent topic/idea/sentence/paragraph $\rightarrow$ \textit{this} $\rightarrow$ \textit{it}, as in (\ref{ex:this-to-it}).

\ea[]{
    \textit{\uline{Cloud computing offers flexibility and scalability that was once impossible}. \uline{This} has revolutionized how businesses store data and manage resources. \uline{It} allows companies to focus on innovation instead of infrastructure.}
}\label{ex:this-to-it}
\z

You can see that \textit{this} refers to the first sentence and topicalizes it. This specification is necessary because, from the first sentence on its own, we might assume that cloud computing generally is the topic. But \textit{this} implies that it's the whole proposition about cloud computing, rather than cloud computing itself that is the topic. Once the topic has been established, then \textit{it} maintains it.

The determinatives \textit{this} and \textit{that} are deictic~-- they have a pointing function~-- and this is what allows them to shift, focus, or broaden the topic.\footnote{The subordinator \textit{that} is a different matter. It has no meaning at all. See Chapter \ref{ch:subord-coord}.}

Here's another example.

\ea \label{ex:Canada-it-that}
    \ea[]{\textit{What's Canada's comparative advantage in geopolitics and foreign policy, and how should \uline{it} work?}}
    \ex[]{\textit{What's Canada's comparative advantage in geopolitics and foreign policy, and how should \uline{it} leverage \uline{that}?}}
    \z
\z

The focus interrogative topicalizes Canada's comparative advantage in geopolitics and foreign policy, and the most natural antecedent for a pronoun like \textit{it} is the topic. That's exactly what we see in  (\ref{ex:Canada-it-that}a): \textit{it} refers to the topic. In (b), though, the verb is \textit{leverage} and usually agents, not policies leverage things. Also, we have a second anaphoric relationship in \textit{that}. Putting these together, \textit{it} picks out an agent (Canada), then then \textit{that} shifts the focus back to the previous topic.
\is{deixis!determinative|)}

\bigskip

While reference creates connections through pointing, lexical cohesion builds meaning through word choice itself.

\subsection{Lexical cohesion}
\is{lexical cohesion|(}

Lexical cohesion is the use of word choice~-- repetition, variation, and semantic relationships~-- to create a network of meaning across a text. It is perhaps the most powerful form of cohesion, relying on the reader's understanding of the relationships between words and their ability to follow the threads of meaning woven through the text. Among these threads of lexical cohesion, we find:

\begin{itemize}[noitemsep]
    \item \textbf{Repetition}: Using the same word or phrase multiple times across a text. This can help maintain a clear focus, especially for key concepts. \is{repetition}
    \item \textbf{Variation}: Using words that are morphologically or semantically related: categorial variations, \textsc{synonyms}, \textsc{antonyms}, and \textsc{hyper}/\textsc{hyponyms} (words in a super/subordinate relationship). This allows for variety, contrast, generalizing, and focussing, while still maintaining a sense of unity. \is{synonymy}\is{antonymy}\is{hypernym}\is{hyponym}
    
    \item \textbf{Collocation}: Using words that tend to co-occur frequently, creating a sense of naturalness and coherence. \is{collocation}
\end{itemize}

Consider (\ref{ex:history}; reproduced here as \ref{ex:history2}) with the words coloured and underlined to illustrate these types of lexical cohesion more clearly:

\ea[]{\textit{\textcolor{xGreen}{The Renaissance} marked \textcolor{xGreen}{a profound \uline{transformation}} in European \uuline{art} and \textcolor{xPurple}{culture}. 
    This \textcolor{xPurple}{cultural} \textcolor{xGreen}{\uline{transformation}} was particularly evident in the \uline{flourishing} of visual \uuline{arts}. 
    \textcolor{xOrange}{\uuline{Artists}} of the \textcolor{xGreen}{era}, such as \textcolor{xOrange}{Michelangelo and Leonardo da Vinci}, \uline{revolutionized} \uuline{artistic} \textcolor{xPink}{techniques}. 
    \textcolor{xOrange}{Their} \uline{innovative} \textcolor{xPink}{approaches} heralded a \uline{new} \textcolor{xGreen}{era} in the \textcolor{xPink}{representation} of \textcolor{xPurple}{human experience}.}}\label{ex:history2}
\z
Verbatim repetition is seen in the dual appearances of \textit{\textcolor{xGreen}{transformation}} and \textit{\textcolor{xGreen}{era}}, respectively. Repetition with variation is even more abundant in this passage.

\begin{itemize}[noitemsep]
    \item \textit{\uuline{art}}, \textit{\uuline{arts}}, \textit{\uuline{Artists}}, and \textit{\uuline{artistic}} are all variations of the same root word, creating a strong morphological chain throughout the passage.
    
    \item \textit{\textcolor{xGreen}{The Renaissance}} and \textit{\textcolor{xGreen}{a profound transformation}} in the first sentence are semantically related, the Renaissance being a specific instance of a profound transformation.
    
    \item \textit{\textcolor{xPurple}{culture}}, \textit{\textcolor{xPurple}{cultural}}, and \textit{\textcolor{xPurple}{human experience}} are semantically related.
    
    \item \textit{\uline{transformation}}, \textit{\uline{flourishing}}, \textit{\uline{revolutionized}}, \textit{\uline{innovative}}, and \textit{\uline{new}} all convey the idea of change and innovation, forming another lexical chain.
    
    \item \textit{\textcolor{xPink}{techniques}}, \textit{\textcolor{xPink}{approaches}}, and \textit{\textcolor{xPink}{representation}} are related to the methods and practices of art.

    \item And, of course, \textcolor{xOrange}{Michelangelo} and  \textcolor{xOrange}{Leonardo da Vinci} are specific \textcolor{xOrange}{artists}, to which \textcolor{xOrange}{their} refers.
\end{itemize}
These semantic relationships allow the writer to avoid repetition while still maintaining clear connections between ideas.

\textbf{Collocation} is seen in phrases like \textit{profound transformation}, \textit{visual arts}, and \textit{human experience}. These combinations create subtle but powerful links across a text, contributing to its cohesion in ways that might not be immediately obvious. (See, though, Section \ref{sec:collocations}.)

\textit{Mark} and \textit{profound} collocate with \textit{ transformation}, \textit{flourishing }with\textit{ arts}, \textit{revolutionize }with\textit{ techniques}, \textit{innovative }with\textit{ approaches}, and \textit{herald }with \textit{a new era}. All are established word partnerships that speakers of English recognize, even if unconsciously. Each combination reinforces the text's theme of dramatic cultural change while creating natural-sounding prose.

Where much lexical cohesion is backwards looking, collocations suggest to us what's coming next. When you read \textit{The Renaissance marked a profound}, you're primed for \textit{transformation}. You may not get always exactly that, but it's often helpful in leading us to predict the next word.

\bigskip

It's important to note that these types of lexical cohesion often work together. For example, the combination of \textit{\textcolor{xGreen}{The Renaissance}} and \textit{\textcolor{xGreen}{era}} is an example of repetition (of the concept of a time period) and variation (using a more general term in place of the specific name).
\is{lexical cohesion|)}

Sometimes the strongest connections in discourse come not from what we say, but from what we leave unsaid.

\subsection{Ellipsis}\label{sec:ellipsis}
\is{ellipsis|(}

\textsc{Ellipsis} is the omission of linguistic material that is recoverable from context. Unlike deletion for brevity, ellipsis creates cohesion by forcing readers or listeners to actively connect the elliptical structure to its antecedent. This mental work of recovery strengthens the coherence of the text.

\subsubsection*{Types and patterns}

English licenses ellipsis in several syntactic environments, each with its own constraints:

\textsc{Subject ellipsis} occurs primarily in coordinate\is{coordination} structures and informal registers:
\is{subject (Subj)!ellipsis|(}

\ea
\ea[]{\textit{She picked up the phone and \xout{she} called her sister.}}
\ex[]{\textit{\xout{I} Hope you're doing well.} \hfill [informal written/spoken]}
\ex[]{\textit{A: Where's Ahmad?\\B: \xout{He} Went home early.}}
\z
\z
While coordination freely allows subject ellipsis when the subjects are coreferential, conversational ellipsis follows more complex patterns. First-person subjects drop most readily (\textit{Think so}, \textit{Don't know}), while third-person dropping typically requires immediate prior mention.
\is{subject (Subj)!ellipsis|)}

\textsc{VP ellipsis} removes entire verb phrases after auxiliaries:
\is{verb, verb phrase (VP)!ellipsis|(}

\ea
\ea[]{\textit{Maya can speak French, and Dmitri can \xout{speak French} too.}}
\ex[]{\textit{A: Will you be attending?\\B: I might \xout{be attending}.}}
\ex[]{\textit{Someone should tell her, but I won't \xout{tell her}.}}
\z
\z
The auxiliary carries tense and agreement information, making the full VP recoverable. This construction is particularly useful for avoiding repetition while maintaining grammatical parallelism.
\is{verb, verb phrase (VP)!ellipsis|)}

\textsc{Gapping} removes the verb (and sometimes more) from coordinate structures:\is{coordination}
\is{gap, gapping|(}

\ea[]{\textit{John ordered coffee, and Mary \xout{ordered} tea.}}
\z
\is{gap, gapping|)}

Unlike VP ellipsis, gapping requires parallelism between the clauses and typically occurs only in coordination.

\textsc{Bare argument ellipsis} reduces clauses to a single constituent:
\is{bare-argument ellipsis|(}

\ea
\ea[]{\textit{A: Who can help with this?\\B: Maybe Carlos \xout{can help with this}?}}
\ex[]{\textit{They're serving pasta, \xout{they're serving} probably lasagna.}}
\z
\z
\is{bare-argument ellipsis|)}

\subsubsection*{Discourse functions}

Ellipsis serves several discourse functions beyond mere economy:

\textbf{Maintaining topicality}: By omitting given information, ellipsis keeps focus on new information while reinforcing the active status of the topic. The very act of omission signals `this element is so salient it needs no repetition'.

\textbf{Managing participation}: In conversation, elliptical responses show engagement with prior talk:

\begin{dialogue}
\item[A] \textit{When does the library close?}
\item[B1] \textit{The library closes at eight.} \hfill [sounds pedantic or uncooperative]
\item[B2] \textit{At eight.} \hfill [natural, cooperative response]
\end{dialogue}

\textbf{Creating cohesion across distance}: Ellipsis can link elements across intervening material:

\ea[]{\textit{The committee \uline{proposed increasing fees}~-- a move that upset many students and sparked protests~-- but the board ultimately \uline{didn't} \xout{increase fees}.}}
\z

\subsubsection*{Constraints and challenges}

Not all material can be elided. Ellipsis requires:
\begin{itemize}[noitemsep]
\item \textbf{Recoverability}: The antecedent must be sufficiently salient \is{recoverability}
\item \textbf{Syntactic identity}: The elided material must match its antecedent structurally (with some permitted mismatches) \is{syntactic identity}
\item \textbf{Appropriate context}: Some ellipsis types are restricted to specific registers or constructions \is{register}
\end{itemize}

These constraints explain certain errors:

\ea[*]{\textit{Mary can speak French, but I don't think John \xout{can speak French}.}}
\z
Here the ellipsis fails because \textit{can} cannot be recovered after \textit{John}. We would need \textit{John can}.

\subsubsection*{Cross-linguistic variation}
\is{cross-linguistic variation|(}

Languages differ dramatically in their ellipsis patterns. Japanese and Korean regularly omit subjects and objects when recoverable from context \citep{Kuno1973}. Spanish and Italian allow more subject-dropping than English but require rich verbal morphology to identify the subject \citep{JaeggliSafir1989}. Mandarin\il{Chinese!Mandarin} permits both subject and object dropping but relies heavily on discourse context \citep{li_thompson_1979}.
\il{Japanese}\il{Korean}\il{Spanish}\il{Italian}\il{French}\il{German}

English occupies a middle ground~-- more ellipsis than French or German, less than many East Asian languages. This creates predictable challenges:

\begin{itemize}[noitemsep]
\item Japanese speakers may overuse ellipsis: *\textit{Yesterday bought book}
\item French speakers may underuse it: \textit{I think that it is possible}
\item Spanish speakers may drop subjects inappropriately: *\textit{Is raining}
\end{itemize}
\is{cross-linguistic variation|)}
\is{ellipsis|)}

\begin{tcolorbox}[title=``Use complete sentences'', colback=white, colframe=blue!75!black, fonttitle=\bfseries]\label{sec:complete-sentences}
Few language teachers consider discourse, and this leads to awkward expectations being placed on students. For instance teachers often urge or require their students, especially in the beginning levels, to ``use complete sentences.''

\phantom{~~~~}This well-intentioned directive reveals a fundamental misunderstanding of how discourse actually works. In natural communication, ``complete sentences'' are often neither necessary nor appropriate. Consider a typical conversation:

\begin{dialogue}
   \item[A] \textit{Where are you headed?}
   \item[B] \textit{The store. Need anything?}
   \item[A] \textit{Just milk.}
\end{dialogue}

Each response here is perfectly coherent and communicatively effective, despite being ``incomplete'' by certain classroom standards. The discourse context provides all the necessary information for these fragments to be meaningful. In fact, responding with complete sentences would be terribly artificial and potentially even uncooperative:

\begin{dialogue}
   \item[A] \textit{Where are you headed?}
   \item[B] \textit{I'm headed to the store. Do you need anything at the store?}
   \item[A] \textit{Yes, I do need something at the store. I need milk at the store.}
\end{dialogue}

This disconnect between classroom expectations and real-world language use can be particularly problematic for language learners. When we insist on complete sentences in all contexts, we:

\begin{enumerate}[noitemsep]
   \item Create artificial barriers to fluent communication
   \item Ignore the crucial role of discourse context in determining appropriate responses
   \item Miss opportunities to teach authentic language use
   \item Risk developing students who sound robotic or overly formal in casual conversation
\end{enumerate}

The challenge is helping students recognize when ellipsis is appropriate (casual conversation, parallel structures) versus when fuller forms are expected (formal writing, initial introductions of topics). This requires attention to register, genre, and the specific communicative goals of each interaction.
\end{tcolorbox}


\subsection{Connectives}
\is{connective adverb/adjunct|(}

Connectives, also known as ``linking words'' or ``transitional phrases'', are used to signal logical relations between clauses and sentences. They're a collection of words and phrases from various lexical categories, each with its own unique properties and functions.

\textbf{Coordinators}, such as \textit{and}, \textit{but}, and \textit{or}, are used to join words, phrases, or clauses of equal grammatical status. Despite what some may have heard, there's nothing wrong with starting a sentence with a coordinator. In fact, it can be an effective way to create a sense of continuity or to emphasize a contrast:
\is{coordinator|(}

\ea[]{\textit{\textbf{But} let's not forget the importance of practice. \textbf{And} that's where the real challenge lies.}}
\z
The key, as with other connectives, is not to overuse them in this position.
\is{coordinator|)}

Prepositions such as \textit{because}, \textit{if}, \textit{though}, and \textit{while}, license subordinate clauses as complements and indicate their logical relationship to the main clause. They can signal cause and effect, conditionality, concession, simultaneity, or other relationships:

\ea[]{\textit{I enjoy learning languages \textbf{because} it allows me to connect with people from different cultures.}}
\z
Here, \textit{because}\is{preposition, preposition phrase (PP)!because@\textit{because}} introduces a clause that gives the reason for the main assertion.

There are also various more or less set preposition phrases that function as connective adjuncts. Those such as \textit{for example}, \textit{in addition}, \textit{in contrast},   \textit{on the other hand}, \textit{as a result}, and \textit{in conclusion}, are invaluable, but also often over used.

\textbf{Adverbs} like \textit{however}, \textit{moreover}, \textit{therefore}, and \textit{nevertheless} are also common connectives, along with the plainer \textit{also} and \textit{too}. They can appear at the beginning of a clause or sentence, or they can be inserted within the clause, usually set off by commas:

\ea[]{\textit{Learning a new language can be challenging. \textbf{However}, it's also incredibly rewarding. \textbf{Moreover}, it opens up new opportunities for work, travel, and cultural exchange.}}
\z
Here, \textit{however} signals a contrast between the two ideas (challenging vs rewarding), while \textit{moreover} signals the addition of further benefits.

It's worth noting that some adverbs, like \textit{however} and \textit{therefore}, are often used as connectives, while others, like \textit{quickly} or \textit{loudly}, typically modify verbs or adjectives. This highlights the functional diversity of the adverb category.

Used judiciously, connectives can help clarify the structure of an argument and guide the reader through the logic of a text. But they are not the primary source of coherence. A text that leans too heavily on connectives can feel choppy, formulaic, or just somewhat off.  Unfortunately, many writing textbooks give the impression that they are the imprimatur of academic writing, and many teachers focus on them because they're rather easier to address~-- and assess~-- than clarity, voice, style, and having something interesting to say. But if used simply as a way to dress up weak writing in academic cloth, connectives will impress few readers and fool fewer still.

A prime example of this pedagogical approach is the ``FANBOYS'' mnemonic (\textit{for}, \textit{and}, \textit{nor}, \textit{but}, \textit{or}, \textit{yet}, \textit{so}), which is presented in many composition textbooks as a complete and authoritative list of coordinating conjunctions. In fact, the FANBOYS list is neither internally coherent nor exhaustive, and its rigid application often relies on questionable punctuation rules \citep{Reynolds2011}
\is{FANBOYS}
\is{connective adverb/adjunct|)}

In summary, connectives are just one tool in the cohesion toolbox. They can be helpful in signalling logical relations, but they should be used in conjunction with, not as a replacement for, the other types of cohesive devices discussed in this section. By understanding the unique properties and functions of each type of connective, and by using them judiciously, writers can create texts that are both cohesive and engaging.

Beyond these basic connecting words, language also provides a rich set of expressions for managing the flow of discourse itself.

\subsection{Discourse markers and pragmatic particles}\label{sec:discourse-markers}
\is{discourse marker|(}\is{pragmatics!particle|(}

While pronouns and other referential devices create explicit connections in text, language also employs a set of words and phrases that manage the flow of discourse more broadly. These \textsc{discourse markers}~-- also called pragmatic particles or pragmatic markers~-- help speakers navigate the complex social and cognitive demands of communication \citep{Schiffrin1987}. Far from being mere verbal tics, they perform essential discourse work: managing turn-taking, signalling stance, repairing misunderstandings, checking comprehension, and marking relationships between utterances.
\is{turn-taking}\is{stance}\is{repair!conversational}\is{hedging}

Consider this exchange:

\begin{dialogue}
\item[A] \textit{Like, I was thinking about what you said yesterday?}
\item[B] \textit{Right, about the project deadline.}
\item[A] \textit{Yeah, well, I'm not sure we can finish by Friday.}
\item[B] \textit{I mean, we could ask for an extension.}
\item[A] \textit{You know, that's probably what we should do.}
\end{dialogue}

Each discourse marker here serves a distinct function. Initial \textit{like} frames what follows as a thought in progress. \textit{Right} acknowledges a shared understanding while transitioning to business. \textit{Yeah} confirms receipt of information before \textit{well} signals a shift to potentially problematic content. \textit{I mean} introduces a clarification or suggestion. And \textit{you know} invites agreement while marking the proposition as mutually accessible.

\subsubsection*{Positional patterns and functions}

Discourse markers cluster strongly at utterance boundaries, particularly in initial position where they frame what follows:

\begin{itemize}[noitemsep]
   \item \textbf{\textit{So}}~-- marks conclusions, summaries, or topic shifts: \textit{So what you're saying is\dots}
   \item \textbf{\textit{Well}}~-- signals various transitions, often to dispreferred responses: \textit{Well, I'm not sure about that}
   \item \textbf{\textit{Now}}~-- introduces new topics or returns focus: \textit{Now, about those reports\dots}
   \item \textbf{\textit{Look/Listen}}~-- demands attention, often preceding disagreement: \textit{Look, we need to be realistic}
   \item \textbf{\textit{Actually}}~-- corrects or contradicts expectations: \textit{Actually, that's not what happened}
\end{itemize}

Final position markers check understanding or soften assertions:

\begin{itemize}[noitemsep]
   \item \textbf{\textit{right?/yeah?}}~-- seek confirmation: \textit{We're meeting at three, right?}
   \item \textbf{\textit{though}}~-- qualifies preceding content: \textit{It's not a bad idea, though}
   \item \textbf{\textit{or whatever/or something}}~-- mark vagueness: \textit{We could get pizza or whatever}
\end{itemize}

Medial markers often manage information flow or hedge commitment:

\begin{itemize}[noitemsep]
   \item \textbf{\textit{like}}~-- marks approximation or exemplification: \textit{It was, like, really crowded}
   \item \textbf{\textit{I mean}}~-- self-repairs or clarifies: \textit{She's~-- I mean~-- she was my supervisor}
   \item \textbf{\textit{sort of/kind of}}~-- hedge precision: \textit{It's sort of blue-green}
\end{itemize}

\subsubsection*{Prosodic sensitivity}
\is{prosody}\is{intonation}

The meaning of discourse markers depends crucially on intonation. Consider \textit{well}:

\begin{itemize}[noitemsep]
   \item Long falling tone: disagreement or bad news (\textit{We::ll $\downarrow$, that's not quite right})
   \item Level tone: neutral transition (\textit{Well $\rightarrow$, let's see what happens})
   \item Rising tone: consideration or uncertainty (\textit{Well $\uparrow$, maybe?})
\end{itemize}

Similarly, \textit{so} varies by prosody:
\begin{itemize}[noitemsep]
   \item Emphatic stress: strong conclusion (\textit{SO you knew all along!})
   \item Unstressed: topic management (\textit{So anyway, about lunch\dots})
   \item Lengthened: processing time (\textit{So:::, what should we do?})
\end{itemize}

\subsubsection*{Register and social meaning}
\is{register}

Discourse markers carry social meaning beyond their discourse functions. Some markers index informality (\textit{like}, \textit{you know}), while others suggest formality (\textit{indeed}, \textit{moreover}). Age and social group also matter: \textit{lowkey} and \textit{highkey} have emerged among younger speakers as stance markers, with \textit{lowkey} meaning something like `actually/honestly' in \textit{Lowkey this lasagna tastes awful}.

The frequency and clustering of markers also carries social meaning. Heavy use of \textit{like} may index youth or uncertainty, while sparse marker use can seem abrupt or unfriendly. The key is appropriateness to context rather than simple presence or absence.

\subsubsection*{Cross-linguistic challenges}

Discourse markers present particular challenges for language learners because:

\begin{enumerate}[noitemsep]
   \item Semantic elusiveness: Their meaning emerges from their discourse function rather than propositional content
   \item False friends: Similar forms across languages often have different functions (Japanese \textit{sou} /soː/ vs English \textit{so})
   \item Prosodic integration: Correct form with wrong intonation changes the meaning
   \item Register sensitivity: Appropriate use varies dramatically by context
   \item Rapid change: New markers emerge and spread quickly through speech communities
\end{enumerate}
\is{pragmatics!particle|)}\is{discourse marker|)}

\section{Abductive bridging and coherence}\label{sec:bridging}
\is{bridging inference|(}\is{abduction (inference)|(}\is{coherence!and abduction|(}

The cohesive devices we've examined~-- pronouns, ellipsis, lexical chains~-- create explicit connections across texts. But much of what makes discourse coherent lies not in what's explicitly marked but in what readers infer. Consider this brief narrative:

\ea
   \textit{Sarah walked into the restaurant. The waiter showed her to a table by the window.}
\z
Nothing explicitly links \textit{the waiter} to \textit{the restaurant}, but we effortlessly understand that this waiter works at this restaurant. This interpretive leap~-- called a \textsc{bridging inference}\is{inference, bridging}~-- illustrates how coherence emerges through active hypothesis construction rather than passive decoding.

\subsection{The abductive engine}
\ia{Peirce, Charles Sanders}

When we encounter \textit{the waiter} with its definite article signalling identifiability, we face an interpretive puzzle: what makes this waiter identifiable when none has been mentioned? Following Peirce's notion of abduction (inference to the best explanation), we unconsciously generate hypotheses:

\begin{itemize}[noitemsep]
   \item Restaurants typically employ waiters
   \item This waiter is part of the restaurant scenario
   \item The definite article signals this expected relationship
\end{itemize}

The hypothesis that best explains the definite marking is that the waiter belongs to the restaurant frame. This abductive process happens automatically, drawing on our knowledge of typical situations (restaurants have waiters), linguistic conventions (definite articles mark identifiable referents), and discourse expectations (new information should connect to given information).

Bridging inferences can span various semantic relationships:

\ea
   \ea[]{\textit{We entered the church. The \uline{altar} was covered in flowers.}
         \hfill[part--whole]}
   \ex[]{\textit{I tried to call you yesterday. The \uline{line} was busy.}
         \hfill[event--instrument]}
   \ex[]{\textit{John fell off his bike. The \uline{doctor} said nothing was broken.}\\
         \hfill[cause--consequence]}
   \ex[]{\textit{The picnic was wonderful. The \uline{ants} were the only problem.}\\
         \hfill[scenario--typical element]}
   \z
\z
In each case, coherence emerges not from explicit marking but from our ability to construct plausible connections. The definite article triggers a search for an antecedent; finding none explicitly mentioned, we abduce one from context.

\subsection{Coherence without cohesion}

This abductive capacity explains why texts can cohere even with minimal cohesive marking. Consider:

\ea
   \textit{Max fell. Blood. Ambulance. Hospital. Recovery.}
\z
Despite lacking pronouns, connectives, or even complete sentences, this telegraphic sequence coheres perfectly. We abduce causal and temporal links: falling causes injury (blood), injury requires medical attention (ambulance), ambulances transport to hospitals, hospitals enable recovery. The coherence emerges from our ability to construct a plausible narrative that connects these fragments.

This principle extends to more complex discourse. Academic writing in some disciplines favours what might seem like abrupt transitions:

\ea
   \textit{Inflation rose 3\% last quarter. The central bank meets Thursday.}
\z
No explicit connective links these sentences, but economists readily infer the connection: rising inflation typically triggers central bank action. The coherence relies on shared disciplinary knowledge that makes the bridge obvious to insiders.

\subsection{Cross-linguistic variation in bridging}
\is{cross-linguistic variation|(}

Languages differ in how much they leave to inference. Japanese\il{Japanese} and Chinese\il{Chinese} texts often require more bridging than English texts, with readers expected to supply logical connections that English would mark explicitly \citep{Hinds1987}. Korean\il{Korean} academic writing traditionally favoured a \textit{ki-sung-chon-kyul} structure (intro\-duction--development--turn--conclusion) that requires readers to infer the connection between the ``turn'' and the conclusion \citep{Kaplan1966}.

English occupies a middle position: more explicit than many East Asian languages but less so than German\il{German} academic prose, which traditionally marks logical relationships meticulously. This variation creates predictable challenges:

\begin{itemize}[noitemsep]
   \item Japanese writers may produce English texts that seem ``disconnected'' to English readers
   \item German writers may produce English texts that seem ``over-connected''
   \item English readers may struggle with texts requiring unfamiliar bridging patterns
\end{itemize}
\is{cross-linguistic variation|)}



\section{Weaving it all together} 

Cohesion isn't just about linking words or following grammatical rules~-- it's about choosing expressions that resonate with each other and with the larger discourse. As we've seen throughout the chapter, cohesive devices work together to guide readers through the flow of ideas and create a sense of unity.

While languages differ in how they achieve cohesion, the basic principles we've discussed tend to hold across languages. Japanese and Korean, for instance, rely heavily on ellipsis, often omitting subjects and objects that would need to be expressed in English. Spanish and Italian also allow more ellipsis than English but make greater use of verb morphology to maintain reference. English, while perfectly comfortable with some ellipsis, tends to favour explicit pronouns and other pro-forms.\il{Japanese}\il{Korean}\il{Spanish}\il{Italian}

But despite these differences in specific devices, the underlying principles~-- managing information flow, maintaining reference, creating semantic links~-- are remarkably similar across languages. This means students often bring valuable intuitions about cohesion from their first language. They already know that texts need to hang together, that topics need to be maintained across sentences, and that information needs to be packaged in accessible ways. The challenge is learning the specific ways English handles these universal tasks.

As language teachers, particularly in the teaching of reading and writing, we can help students develop their command of English cohesive devices while drawing on their existing discourse competence. This might involve:

\begin{itemize}[noitemsep]
    \item Exploring how their first language achieves similar cohesive effects
    \item Analyzing the differences in when and how English uses pronouns vs ellipsis
    \item Practicing with lexical cohesion patterns typical of English
    \item Working with authentic texts to see how expert writers create coherence
\end{itemize}

One activity that can be particular helpful is the text puzzle, which is created by separating each sentence and then having students use the cohesive devices to put them back in the (or a) logical order.

Another is to mark up the text to show lexical connections as in (\ref{ex:history}). 

The goal isn't just to master individual cohesive devices, but to develop an intuition for how they work together in English to create meaning. This means moving beyond isolated exercises to working with whole texts in meaningful contexts, helping students see how English-specific patterns serve universal discourse needs.
\is{cohesion|)}\is{coherence|)}

\begin{tcolorbox}[title=Exercise: Short-Answer Questions, colback=white, colframe=blue!75!black, fonttitle=\bfseries]

\begin{enumerate}[noitemsep]
    \item What is the difference between coherence and cohesion?
    \item Explain how pronouns contribute to cohesion in a text.
    \item Give an example of ellipsis and explain how it creates cohesion.
    \item Describe the concept of lexical cohesion and provide two examples of how it can be achieved.
    \item Why is understanding discourse important for language teachers?
    \item Explain the role of definiteness in creating coherence.
    \item Give an example of how variation can be used to create lexical cohesion.
    \item What are discourse markers and how do they differ from connectives?
    \item Why is it important to teach students about the appropriate use of discourse markers?
    \item Discuss the potential challenges that language learners might face with discourse markers.
\end{enumerate}

\end{tcolorbox}

\begin{tcolorbox}[title=Answer Key 1--7, colback=white, colframe=blue!75!black, fonttitle=\bfseries]

\begin{enumerate}[noitemsep]
    \item Coherence is the overall quality of a text that makes it understandable and logical, while cohesion refers to the specific linguistic features that connect sentences and ideas within a text.
    \item Pronouns contribute to cohesion by referring to previously mentioned NPs, creating a chain of reference, and avoiding unnecessary repetition. This helps to maintain topicality and guide the reader's understanding of the relationships between entities in the text.
    \item Example of ellipsis: \textit{Did you finish your homework?} \textit{Yes, I did.} In the response, the verb phrase \textit{finish my homework} is omitted because it is understood from the context of the question. This avoids repetition.
    \item Lexical cohesion refers to the use of word choice to create connections between ideas in a text. Examples: a) Repetition: Using the same key words or phrases throughout a text to emphasize a particular concept. b) Synonymy: Using words with similar meanings to create variation while maintaining semantic links.
    \item Understanding discourse is important for language teachers because it helps them to teach language in a more communicative and context-aware manner. By focusing on how language is used in real-world situations, teachers can help students develop their discourse competence and communicate more effectively.
    \item Definiteness contributes to coherence by signaling whether a noun phrase refers to something already known to the listener or reader. Definite noun phrases typically refer to entities that have been previously introduced or are assumed to be shared knowledge, while indefinite noun phrases introduce new entities into the discourse.
    \item Example of variation: "The ancient city was a marvel of engineering. Its intricate network of aqueducts and roads testified to the ingenuity of its builders." Here, the words "marvel," "intricate," and "ingenuity" are all variations on the concept of impressive skill and complexity, creating lexical cohesion without directly repeating the same words.
\end{enumerate}
\end{tcolorbox}

\begin{tcolorbox}[title=Answer Key 8--10, colback=white, colframe=blue!75!black, fonttitle=\bfseries]

\begin{enumerate}[noitemsep]\setcounter{enumi}{7}
    
    \item Discourse markers are expressions like \textit{well}, \textit{you know}, and \textit{like} that function primarily at the discourse level to manage the flow of conversation, signal stance, and check comprehension. Connectives, on the other hand, typically signal logical relationships between clauses or sentences, such as cause and effect, contrast, or addition.
    \item They play a crucial role in natural and fluent communication.
    \item Challenges with discourse markers: a) Meaning: Their meaning is often context-dependent and difficult to define precisely. b) Variation: Their use can vary widely across dialects and registers. c) Transfer: Learners may overuse or misuse markers based on similar but not identical functions in their first language.
\end{enumerate}
\end{tcolorbox}
\begin{tcolorbox}[title=Essay questions, colback=white, colframe=blue!75!black, fonttitle=\bfseries]
\begin{enumerate}[noitemsep]
    \item Analyze how reference, ellipsis, lexical cohesion, and connectives work together to create coherence in a specific text of your choice.
    \item Discuss the challenges of teaching discourse competence to language learners, and suggest some effective strategies for addressing these challenges in the classroom.
    \item Explain the role of discourse markers, focusing on their functions, positioning, and the potential pitfalls for language learners.
    \item Compare lexical cohesion in different genres, such as academic writing, fiction, and advertising. How do the choices of expressions reflect the purpose and audience of each genre?
    \item Discuss the concept of topicality and its importance for understanding how information is structured and presented in discourse. Illustrate your points with examples of how topicality is managed through various linguistic features.
\end{enumerate}
\end{tcolorbox}
