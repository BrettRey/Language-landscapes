\chapter{Information packaging} \label{ch:info-package}

\epigraph{Some truths arrive\\
in the order of their weight\\
light things first\\
heavy things lingering}{}

\section{What is information packaging?}

When we communicate, it's not just about what we say, but also about how we structure and present that information. This is what linguists call \textsc{information packaging}.

Think of it like packing a suitcase. You have a variety of items you want to bring (the information), but how you arrange those items, what you put on top for easy access, what you tuck away at the bottom - that's all part of your packing strategy. And just as different packing styles can make it easier or harder to access what you need from your suitcase, different information packaging choices can make your message clearer or murkier for your audience.

In English, we have a toolbox full of linguistic strategies for information packaging. We can choose what to put in the subject of a clause, what to emphasize with intonation, what to keep together, and what to delay until the end. Each of these choices subtly shapes how our message will be understood.

Why does this matter for language learning and teaching? Because these packaging strategies aren't just stylistic frills~-- they're key to effective communication. Choosing the right packaging helps us highlight what's new or important, connect ideas to things already mentioned, and steer our listener's attention to the main point. Conversely, packaging things awkwardly can leave the audience confused, distracted, or missing they key ideas entirely.

So in the rest of this chapter, we'll unpack some of the main tools English offers for packaging information, and explore what they mean for the language classroom. We'll look at given vs. new information, topic and focus, cleft sentences, fronting and postponing strategies, and the discourse functions of active and passive. Along the way, the goal is to develop a sharper eye and ear for how these information-packaging strategies work, so you can help your students make wiser choices in their own English communication.

\subsection{Given vs New Information}

One fundamental pragmatic aspect of information packaging is the interplay between given (or known) information and new information. \textsc{Given information} is what you can reasonably assume your listener is already aware of, based on the prior discourse or general knowledge. \textsc{New information} is what you're introducing for the first time, or highlighting as especially important or newsworthy.

We tend to structure our sentences so that given information comes earlier and new information comes later. This ``given--new'' ordering helps listeners process the message more easily, moving from what they're familiar with to what's being introduced. Consider (\ref{ex:hybrid}).

\ea[]{I bought a new car yesterday. It's a hybrid.\textit{}}\label{ex:hybrid}
\z

In the first sentence, \textit{a new car} denotes new information~-- I'm telling you about it but I haven't mentioned it before. But \textit{I} denotes old information. The very fact that you're hearing or reading those words, means you know I'm addressing you. So \textit{I} comes first and \textit{a new car} comes later. If I said \textit{A new car was bought by me}, flipping the order, that would just be weird.

In the second sentence, \textit{it} refers back to this car, which is not longer new information, having now been established in the discourse by the previous sentence. The new information in the second sentence is the type of car, a hybrid. Again, I could say, \textit{a hybrid is the type of car}, but again, it's a strange way to package the information.

We could even try going so far as to reverse order of the sentences.

\ea[?]{\textit{It's a hybrid. I bought a new car yesterday.}}\label{ex:hybrid-reversed}
\z

Again, the result is very odd in (\ref{ex:hybrid-reversed}), \textit{it} doesn't connect to any previously mentioned referent or anything obvious in the environment, so the pronoun feels abrupt and disconnected. The new information \textit{a new car} comes awkwardly late.

So in general, people tend to instinctively structure their sentences and discourse in a given-to-new arc. Of course, this is a flexible principle, not an absolute rule. Speakers can intentionally depart from given-to-new order for special effects like dramatic emphasis. But most of the time, following this default makes the discourse feel more natural and easy to process.\footnote{\citet[16]{Lambrecht1994} expresses it like this: ``In the unmarked case, a clause-initial subject will have a topic relation and a clause-final object a focus relation to the proposition.''}

As an English teacher, attuning yourself and your students to this given--new flow may not be a big problem. It's the normal way to do things in any language, but sometimes thinking explicitly about it can help students structure larger pieces of discourses more coherently.

Recall the discussion of first-person subject dropping in English (see Section \ref{sec:subject-dropping}). That's just a more extreme form of this. The subject \textit{I} is so totally ``given'', so much ``on the table'' that it doesn't even bear mentioning. But if we dropped the new information, then nobody would have any idea what we were saying.

Ultimately, the given--new principle is a key foundation for the other information packaging strategies. Choosing what to present as given or new, and in what order, remains a core concern even as we turn to other tools like clefts and frontings. Keeping this foundation in mind will help make sense of the larger information packaging system.

\subsection{Topic and Focus} \label{sec:topic+focus}

Closely related to the notion of given vs. new information is the semantic distinction between topic and focus. The \textsc{topic} of a sentence is what the sentence is about, the starting point for the message. The \textsc{focus} is the most salient or newsworthy part of the message, the key piece of information the speaker wants to highlight.

In English, topics are often (but not always) given information, while focused elements tend to be new information. And just as given information usually comes before new, topics typically appear early in the sentence (often in subject position) while focused elements appear later. Consider (\ref{ex:Yokos-laptop}).

\ea[]{Q: \textit{What did Yoko buy at the store?}\\A: \textit{She bought a new laptop.}}\label{ex:Yokos-laptop}
\z

In the answer, Yoko is the topic~-- she's the one the sentence is about, and she's also given information carried over from the question. A new laptop is the focus, conveying the key new information requested by the question.

Now imagine the different context in (\ref{ex:whose-laptop}).

\ea[]{Q: \textit{Who bought a new laptop?}\\A: \textit{Yoko bought it.}}\label{ex:whose-laptop}
\z

Here, the new laptop is given in the question, while Yoko is the focused answer to \textit{who}. This moves \textit{Yoko} into the stressed, focal position at the beginning of the sentence. By doing so, it also inverts our usual given--new information order~-- \textit{Yoko} is new and \textit{it} is given.

We've already used the term \textsc{focus} for \textsc{focused questions} in Section \ref{sec:focused interrogatives}. In focused questions, the (semantic) focus is the interrogative phrase. Sometimes this phrase is also the (syntactic) subject as it is in \textit{Who left}, but usually it's not. When it's not, we focus it by moving it out of the position the answer would have been in a declarative and putting it first, as in (\ref{ex:steves-laptop}b). 

\protectedex{
\ea\label{ex:steves-laptop}
    \ea[]{\textit{\phantom{Who did }Yoko bought a new laptop for \uline{Steve}.}\hfill[Declarative]}
    \ex[]{\textit{\uline{Who} did Yoko buy\phantom{ght} a new laptop for \uline{\phantom{Steve}}?}\hfill[Interrogative]}
    \z
\z
}

Usually, the \textit{for} PP has an object, as it does in \textit{for Steve}. And we could ask \textit{he bought it for who?} but we've focused the interrogative phrase by putting \textit{who} at the front of the question/interrogative. Whether stuff appears at the front or the back, the very fact that it's not in its usual spot can focus it.

Of course, in speech, we can also put something into focus by stressing it, as we saw in Section \ref{sec:stress}. (We often represent this in writing with capital letters, bold, underlining, or some other typographical method.) This focus stress tells the listener where to direct their attention for the key information.

Apart from subject-auxiliary inversion, English has other syntactic strategies for marking focus, like clefts (see Section \ref{sec:clefts}), such as (\ref{ex:laptop-cleft}).

\ea\label{ex:laptop-cleft}
    \ea[]{\textit{It was \uline{Yoko} who bought a new laptop.}}
    \ex[]{\textit{It was \uline{for Steve} that she bought it.}} 
    \z
\z

Clefts allow speakers to focus almost any element by making it a post-head complement of the VP, even elements that wouldn't normally appear in that position, such as \textit{for Steve} in (b).

As for topics, English speakers have a few common ways of setting them up. One is simply to place the topic in subject position, as in \textit{The weather is lovely}. Another is to use topic-announcing phrases like \textit{as for}, \textit{regarding}, \textit{speaking of}, etc. (as in \ref{ex:broken-laptop}; mentioned in Section \ref{sec:subject-denotation}).

\ea[]{\textit{Regarding broken laptop, blame Steve.}}\label{ex:broken-laptop}
\z

This phrase explicitly marks the broken laptop as the topic around which the following sentence will revolve.

The topic and focus, in tandem with given--new ordering, form the backbone of information structure in English. By tuning in to these patterns, learners can understand why and when they might choose to use one of the information packaging constructions in the following sections.

\section{Information packaging constructions}

\subsection{The passive voice} \label{sec:passive}

The passive voice is perhaps the most familiar of the information-packaging constructions. The active voice is the default, unmarked choice,\footnote{In linguistics, \textsc{unmarked} denotes the basic, default form of expression in a language~-- the version that requires no special grammatical modification.} the subject of the sentence often being an agent/actor. The passive voice, allows the speaker to avoid any mention of that subject and instead topicalize and focus the patient or recipient. Consider (\ref{ex:new-manager}).

\ea\label{ex:new-manager}
    \ea[]{\textit{The company hired a new manager on Monday.}\hfill[\textsc{active}]}
    \ex[]{\textit{Farah was hired on Monday.}\hfill[\textsc{passive}]}
    \z
\z

In (\ref{ex:new-manager}a), the topic and focus is the company, which is the agent and the old information. The new manager and the fact of her employment is new information. In (b), the topic and focus shifts to Farah and the fact of her being hired. Farah is also the old information while the fact of her hiring is new. The company isn't mentioned, possibly because it's too obvious (it's an internal announcement) or because it's not important or perhaps isn't known. Compare that to (\ref{ex:new-manager-by-phrase}).

\ea\label{ex:new-manager-by-phrase}
    \ea[]{\textit{SpaceX hired a new manager on Monday.}\hfill[\textsc{active}]}
    \ex[]{\textit{Farah was hired as a manager by SpaceX on Monday.}\hfill[\textsc{passive}]}
    \z
\z

This time, in the passive, the topic is Farah, but the focus is arguably the remarkable position she landed. So the optional \textit{by} phrase can be used to de-topicalize, but it can also be used for focus.

The structure of the passive construction in English most typically involves the use of the auxiliary verb \textit{be} and the past participle of the lexical verb. The object of the active clause becomes the subject of the passive, while the subject of the active sentence is typically dropped. Alternatively, it can become the object in an optional \textit{by} phrase.

This shift in focus can serve various discourse functions. One is to allow the topic of the passage to remain in the subject position, even if it's not the semantic agent of the action. Consider a paragraph like (\ref{ex:XYZ-co}).

\ea[]{\textit{XYZ Corporation has been a leader in the tech industry for over a decade. Founded in 2010, the company quickly gained a reputation for innovation and quality. In 2015, XYZ Corp was awarded a prestigious industry prize for its groundbreaking software. The company was praised for its commitment to customer satisfaction.}}\label{ex:XYZ-co}
\z

Here, XYZ Corporation is the main topic of the paragraph, and the passive voice in the third and fourth sentences allows it to remain in the subject position throughout, even when it's the recipient of actions like being awarded a prize or being praised. This maintains a clear, consistent topic chain and preserves the old--new information structure.

The passive voice can also help us restructure a sentence to keep the head of the subject close to the verb. Consider (\ref{ex:CNN}a) from \textit{Wikipedia} where the two are right next to each other. Now compare the active counterpart (b) where they are separated by more than 20 words.

\ea\label{ex:CNN}
    \ea[]{\textit{A convoluted neural network \uline{architecture} \uline{is} formed by a stack of distinct layers that transform the input volume into an output volume (e.g. holding the class scores) through a differentiable function.}\\\hfill[\textsc{passive}]}
    \ex[]{\textit{A \uline{stack} of distinct layers that transform the input volume into an output volume (e.g. holding the class scores) through a differentiable function \uline{forms} a convoluted neural network architecture.}\hfill[\textsc{active}]}\z
\z

You may have been told that you should avoid the passive, but no matter how many times you've heard such thing, I think you'll agree that the passive is better here.

The passive can also be used to intentionally de-emphasize or omit the agent, either because it's unknown or irrelevant, or because the speaker wants to avoid assigning explicit blame or responsibility:

\ea\label{ex:broken-window}
    \ea[]{\textit{Someone seems to have broken the window.}\hfill[\textsc{active}]}
    \ex[]{\textit{The window seems to have been broken.}\hfill[\textsc{passive}]}
    \z
\ex\label{ex:mishandled-crisis}
    \ea[]{\textit{The government mishandled the crisis.}\hfill[\textsc{active}]}
    \ex[]{\textit{The crisis was mishandled.}\hfill[\textsc{passive}]}
    \z
\z

In (\ref{ex:broken-window}b), the agent is omitted entirely, shifting the focus away from the fact that someone did it, which seems fairly obvious, to the window itself. In (\ref{ex:mishandled-crisis}b), no mention is made of the government, perhaps to avoid a directly accusatory tone.

Of course, the passive voice can also be overused or used inappropriately, but so can the active.

For English learners, understanding the discourse functions of the passive can help them make better choices in their own language production. Instead of blindly following a foolish dictum like ``always use active voice,'' they can learn to see the passive as a tool for managing information flow and emphasis.

As a teacher, you can help students develop this strategic awareness through targeted exercises and discussions. For example, you can have present students with a paragraph~-- perhaps from a text they've recently read~-- in which all the passive clauses have been made active and ask them to passivize those where they think the passive would be better. In taking this up, you'd ask students for the reasons for their choices: to defocus the subject of the active, which is unimportant or unknown (e.g., \ref{ex:broken-window}); to maintain a consistent topic (e.g., \ref{ex:XYZ-co}); to avoid separating the head of the subject from the head verb (e.g., \ref{ex:CNN}), or for some other topic or focus purpose.

\subsection{Cleft constructions} \label{sec:clefts}

Cleft constructions allow the speaker to split a simple sentence into two parts, each with its own information structure. The two main types of clefts are \textit{it}-clefts and pseudo-clefts.

\subsubsection*{\textit{It}-clefts}
\textsc{\textit{It}-clefts} have the following structure: \textit{it} + \textit{be} + \textsc{focus} + relative clause.

\ea \label{ex:Laptop-it-cleft}
    \ea[]{\textit{Yoko bought a new laptop.}\hfill{\textsc{[non-cleft}]}}
    \ex[]{\textit{It was Yoko who bought a new laptop.}\hfill{\textsc{[\textit{it}-cleft}]}}
    \ex[]{\textit{It was a new laptop that Yoko bought.}\hfill{\textsc{[\textit{it}-cleft}]}}
    \ex[]{\textit{It was for Steve that Yoko bought a new laptop.}\hfill{\textsc{[\textit{it}-cleft}]}}
    \z
\z

In (\ref{ex:Laptop-it-cleft}b), Yoko is the focus; in (c), it's the new laptop; and in (d), it's who it was purchased for. The \textit{it}-cleft structure allows the speaker to place anything in the focus position in the main clause, where it receives the most stress and prominence.

\subsubsection*{Pseudo-clefts}
\textsc{Pseudo-clefts}, also known as \textit{wh}-clefts, have the following structure: relative clause + \textit{be} + \textsc{focus}.


\ea \label{ex:Jun-cleft}
    \ea[]{\textit{Jun will visit Paris next summer.}\hfill{\textsc{[non-cleft}]}}
    \ex[]{\textit{What Jun will do next summer is visit Paris.}\hfill{\textsc{[pseudo cleft}]}}
    \ex[]{\textit{Where Jun will go next summer is Paris.}\hfill{\textsc{[pseudo cleft}]}}
    \ex[]{\textit{How it will go is unpredictable.}\hfill{\textsc{[pseudo cleft}]}}
    \z
\z

In (\ref{ex:Jun-cleft}b), the event of the visit is focused, while in (c), the location Paris is focused. And in (d), the unpredictability of the venture is focused. The pseudo-cleft allows the speaker to front the old information in the relative phrase, while placing the focus in the complement in the \textit{be} VP.

Both \textit{it}-clefts and pseudo-clefts serve to highlight or emphasize a particular element of the sentence. They're often used to contrast the focused element with some other possible alternative, as in (\ref{ex:cleft-contrast}).

\ea \label{ex:cleft-contrast}
    \ea[]{\textit{It was the red car, not the blue one, that she bought.}}
    \ex[]{\textit{What I want is a vacation, not more work.}}
    \z
\z

Adjuncts can be difficult to focus in their normal positions, but clefts allow them to be moved to the focus position.

\ea\label{ex:adjuct-focus}
    \ea[]{\textit{We meet our friends tomorrow.}\hfill[\textsc{non-cleft}]}
    \ex[]{\textit{It's tomorrow that we meet our friends.}\hfill[\textsc{\textit{it}-cleft}]}
    \z
\ex\label{ex:adjunct-focus2}
    \ea[]{\textit{We're meeting at the park.}\hfill[\textsc{non-cleft}]}
    \ex[]{\textit{Where we're meeting is at the park.}\hfill[\textsc{pseudo cleft}]}
    \z
\z

While passives are common, especially in academic texts, clefts are less so. They're also mainly spoken constructions, with pseudo clefts being mostly limited to conversation \citep[938]{Biber1999}.

\subsection{Preposing and Postposing}

Preposing (or fronting) and postposing involve moving elements from their default positions to the beginning or end of the clause for specific discourse effects.

\subsubsection*{Preposing}\label{sec:preposing}

In a \textsc{preposing} construction, a constituent that would normally come later in the sentence is moved to the front, either because it is old information or to focus it. Focused interrogatives are one example of preposing (see Section \ref{sec:focused interrogatives}). 

While it's possible to prepose an object or other complements in a VP, it's much more common with adjuncts or objects of prepositions. Examples are given in (\ref{ex:preposing}--\ref{ex:preposing-end}).

\ea \label{ex:preposing}
    \ea[]{\textit{She took her own, but she left his.}\hfill [\textsc{default}]}
    \ex[]{\textit{She took her own, but \uline{his} she left.}\hfill  [\textsc{preposed object}]}
    \z
\ex 
    \ea[]{\textit{I could open it with a hammer.}\hfill  [\textsc{default}]}
    \ex[]{\textit{\uline{With a hammer}, I could open it.}\hfill  [\textsc{preposed PP adjunct}]}
    \z
\begin{samepage}
\ex
    \ea[]{\textit{I'll go if you go.}\hfill  [\textsc{default}]}
    \ex[]{\textit{\uline{If you go}, I'll go too.}\hfill  [\textsc{preposed PP adjunct}]}
    \z
\end{samepage}
\ex 
    \ea[]{\textit{I haven't looked at the report yet.}\hfill  [\textsc{default}]}
    \ex[]{\textit{\uline{The report}, I haven't looked at yet.}\hfill  [\textsc{preposed object in PP}]}
    \z
\ex 
    \ea[]{\textit{We watched a movie last night.}\hfill  [\textsc{default}]}
    \ex[]{\textit{\uline{Last night}, we watched a movie.}\hfill  [\textsc{preposed NP adjunct}]}
    \z
\ex \label{ex:preposing-end}
    \ea[]{\textit{It worked because of all the practice.}\hfill  [\textsc{default}]}
    \ex[]{\textit{\uline{Because of all the practice}, it worked.}\hfill  [\textsc{preposed PP adjunct}]}
    \z
\z

Just as a fronted interrogative phrase triggers subject-auxiliary inversion or \textit{do} support, sometimes, a fronted adjunct will do the same as in (\ref{ex:fronting-inversion}--\ref{ex:fronting-inversion-end}). This is particularly common with negative adjuncts. 

\ea \label{ex:fronting-inversion}
    \ea[]{\textit{I rarely wake up this early.}\hfill  [\textsc{default}]}
    \ex[]{\textit{\uline{Rarely} do I wake up this early.}\hfill  [\textsc{preposed AdvP adjunct}]}
    \z
\ex 
    \ea[]{\textit{They have never admitted it.}\hfill  [\textsc{default}]}
    \ex[]{\textit{\uline{Never} have they admitted it.}\hfill  [\textsc{preposed AdvP adjunct}]}
    \z
\ex 
    \ea[]{\textit{She not only got up, but she even left.}\hfill  [\textsc{default}]}
    \ex[]{\textit{\uline{Not only} did she get up, but she even left.}\hfill  [\textsc{preposed AdvP adjunct}]}
    \z
\ex 
    \ea[]{\textit{You should not open it under any circumstances.}\hfill  [\textsc{default}]}
    \ex[]{\textit{\uline{Under no circumstances} should you open it.}\hfill  [\textsc{preposed PP adjunct}]}
    \z
\ex \label{ex:fronting-inversion-end}
    \ea[]{\textit{It passed so quickly that we missed it.}\hfill  [\textsc{default}]}
    \ex[]{\textit{\uline{So quickly} did it pass that we missed it.}\hfill  [\textsc{preposed AdvP adjunct}]}
    \z
\z

\subsubsection*{Postposing}
As you would expect, \textsc{postposing} constructions place later a constituent that would normally come earlier. This is often done with longer or ``heavy'' constituent to improve clarity and processing. It's also another way of shifting the focus. See the examples in (\ref{ex:postposing-start}--\ref{ex:postposing-end}).

\ea \label{ex:postposing-start}
\ea[]{\textit{The fact that the project was behind schedule worried the manager.} \\\hfill [\textsc{default}]}
\ex[]{\textit{It worried the manager \uline{that the project was behind schedule}.} \\\hfill [\textsc{postposed subject}]}
\z

\begin{samepage}
\ex
\ea[]{\textit{The book that he'd been working on for years was finally published.} \\\hfill [\textsc{default}]}
\ex[]{\textit{It was finally published, \uline{the book that he'd been working on for years}.} \\\hfill [\textsc{postposed subject}]}
\z
\end{samepage}

\ex
\ea[]{\textit{He gave all the leftovers from the dinner to the dog.} \hfill [\textsc{default}]}
\ex[]{\textit{He gave to the dog \uline{all the leftovers from the dinner}.} \\\hfill [\textsc{postposed direct object}]}
\z

\ex
\ea[]{\textit{What we do with the rest is a bigger problem.} \hfill [\textsc{default}]}
\ex[]{\textit{A bigger problem is \uline{what we do with the rest}.} \hfill [\textsc{postposed subject}]}
\z

\ex
\ea[]{\textit{The decision had been so difficult that I'd delayed it for months.} \\\hfill [\textsc{default}]}
\ex[]{\textit{So difficult had been \uline{the decision} that I'd delayed it for months.} \\\hfill [\textsc{postposed subject}]}
\z

\ex 
\ea[]{\textit{The question of where we're staying could be raised.} \hfill [\textsc{default}]}
\ex[]{\textit{The question could be raised \uline{of where we're staying}.} \\\hfill [\textsc{postposed modifier}]}
\z

\ex 
\ea[]{\textit{Somebody who seemed to be in a hurry arrived with a package for you.} \\\hfill [\textsc{default}]}
\ex[]{\textit{Somebody arrived with a package for you \uline{who seemed to be in a hurry}.} \\\hfill [\textsc{postposed modifier}]}
\z

\ex \label{ex:postposing-end}
\ea[]{\textit{You asked me to tell them, and I told them. }\hfill [\textsc{default}]}
\ex[]{\textit{You asked me to tell them, and tell them \uline{I did}.} \hfill [\textsc{postposed subject}]}
\z
\z

It's important to note that while preposing and postposing are possible in English, for the most part, they're marked structures and shouldn't be overused. Most sentences will follow the default word order, and deviations from this should be motivated by specific discourse needs.

\subsection{Extraposed subject} \label{sec:extraposition}

Subject extraposition packages information by moving a ``heavy" subject to the end of the clause. It's particularly common with clausal subjects~-- those that contain a full clause, often introduced by \textit{that}, \textit{whether}, or \textit{if}. The extraposed subject's original position is filled with the dummy pronoun \textit{it}.

As with other information packaging constructions, extraposition can serve different discourse functions. One is simply to make sentences easier to process by moving long, complex subjects to the end. But it also helps maintain a natural flow of information from given to new, as in (\ref{ex:extrapose-worry}).

\ea\label{ex:extrapose-worry}
    \ea[]{John: \textit{I hope there are no surprises.}}
    \ex[]{Minh: \textit{It worries me \uline{that Al hasn't replied yet}.}\hfill[\textsc{extraposed}]}
    \ex[\textsuperscript{?}]{Minh: \textit{\uline{That Al hasn't replied yet} worries me.}\hfill[\textsc{non-extraposed}]}
    \z
\z

Minh's response in (\ref{ex:extrapose-worry}b) exemplifies how extraposition helps maintain the given-to-new flow of information. The conversation has already established concern about possible surprises, so starting with \textit{it worries me} connects to this given information before introducing the new information about Al. The non-extraposed version in (c) begins abruptly with new information, making it feel less connected to the ongoing discourse.

This reordering for processing and information flow is especially common with expressions of evaluation or stance, like \textit{appear}, \textit{seem}, \textit{be clear}, \textit{be obvious}, \textit{be important}, and so on:

\ea\label{ex:extrapose-seem}
    \ea[]{\textit{\uline{To finish this project by Friday} seems impossible.}\hfill[\textsc{default}]}
    \ex[]{\textit{It seems impossible \uline{to finish this project by Friday}.}\hfill[\textsc{extraposed}]}
    \z
\ex\label{ex:extrapose-clear}
    \ea[]{\textit{\uline{That everyone agreed} was obvious.}\hfill[\textsc{default}]}
    \ex[]{\textit{It was obvious \uline{that everyone agreed}.}\hfill[\textsc{extraposed}]}
    \z
\z

In fact, in many cases, the extraposed version sounds much more natural than keeping the clausal subject in its original position. The non-extraposed versions can feel awkward and overly formal. In other cases, such as (\ref{ex:extrapose-revise}), extraposition is obligatory.

\ea\label{ex:extrapose-revise}
    \textit{After reviewing the client's feedback...}
    \ea[]{\textit{...it seems \uline{that we need to revise our proposal}.}\hfill[\textsc{extraposed}]}
    \ex[*]{\textit{...\uline{that we need to revise our proposal} seems.}\hfill[\textsc{non-extraposed}]}
    \z
\z

Infinitival subjects are also commonly extraposed, especially when they're modified by additional phrases that make them ``heavier'':

\ea\label{ex:extrapose-infinitive}
    \ea[]{\textit{\uline{To work with such talented people} is a pleasure.}\hfill[\textsc{default}]}
    \ex[]{\textit{It is a pleasure \uline{to work with such talented people}.}\hfill[\textsc{extraposed}]}
    \z
\z

It's important to be aware that an ``extraposed subject'' is a type of complement; it's not the subject of the clause. In (\ref{ex:extrapose-infinitive}), for instance, the one and only subject is \textit{it}, even though it's semantically empty.

\subsection{Right dislocation} \label{sec:right-dislocation}

Right dislocation is an information packaging construction, mostly limited to spoken English, where we place a nominal expression at the end of the clause while leaving a pronoun in its normal position. It's especially common in conversation, where it helps clarify referents and manage information flow:

\ea\label{ex:right-dis-basic}
   \ea[]{\textit{She's really smart, \uline{that new professor}.}}
   \ex[]{\textit{They're impossible to find, \uline{those books you mentioned}.}}
   \z
\z

The dislocated phrase at the end serves to clarify or elaborate on the pronoun earlier in the clause. This is particularly useful when the speaker realizes the listener might need help identifying the referent, or when the speaker wants to add emphasis.

Right dislocation differs from extraposition in important ways. With extraposition, the \textit{it} is semantically empty~-- it's just holding the subject position until the real subject appears. But in right dislocation, the pronoun has full referential meaning, and the phrase at the end is providing additional information about that referent.

The construction is especially useful for adding commentary or evaluation while maintaining a smooth information flow:

\ea\label{ex:right-dis-comment}
   \ea[]{\textit{He's always late, \uline{your brother}.}}
   \ex[]{\textit{It's beautiful, \uline{this time of year}.}}
   \ex[]{\textit{They're quite expensive, \uline{these organic vegetables}.}}
   \z
\z

In each case, the evaluation comes first, followed by clarification of what's being evaluated. This ordering lets speakers lead with their main point while ensuring their reference is clear.

Right dislocation can also help manage longer stretches of discourse by keeping the reference clear:

\ea\label{ex:right-dis-discourse}
   \ea[]{\textit{I couldn't believe it when I saw it in the store. It was enormous, \uline{that cake you were telling me about}.}}
   \ex[]{\textit{They've been giving me trouble all week, \uline{those new students in my afternoon class}.}}
   \z
\z
