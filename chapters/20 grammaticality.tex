\chapter{The Nature of Grammaticality} \label{ch:grammaticality}
\is{grammaticality|(}\is{form--meaning!pairing|(}\is{processing limitations|(}\is{language change|(}\is{community!speech communities|(}\is{Standard English}

\epigraph{They come to the edge of grammar and stop,\\
words do. Beyond this is silence,\\
though not a silence you can trust.}{}

Many people believe that grammaticality is like the rule that stops play in soccer when the ball exits the pitch. In other words, they believe that something is either in bounds or out. But if we're analogizing to the realm of soccer, then grammaticality is much more like the handball rule, with its nuanced application and interpretation.\is{analogy!handball vs grammar}\is{gradient acceptability}

At the core, both handball violations and grammatical errors involve a relationship between form and meaning.\is{form--meaning!pairing} A handball isn't just about physical contact (form) but how that contact affects play (meaning). Similarly, grammaticality is at its core about form-meaning pairings. An interrogative clause typically has the meaning of a question, and the past-tense usually means past time.\is{interrogative clause}\is{tense!past}

But form-meaning pairings aren't universal, and neither is the handball rule. If you want to be recognized as playing by FIFA's laws of the game, then your handball rule needs to be FIFA's. Similarly, if you want to be seen as speaking Standard English, then it's important that you follow its syntax and morphology.\is{dialect!standard vs nonstandard}\is{morphology!community conventions}

But if you're playing American football or rugby, then different rules apply, just as they do in French and German.\is{cross-linguistic variation} And if a local children's soccer league modifies the handball rule or a local community accepts and uses double modals (\textit{you might could do that}), then the soccer players, league officials, and the \textit{might could} users aren't failing to follow rules~-- they're following different but equally systematic ones.\is{double modal} What matters is that within any given system or community, the rules are clear and consistently applied.

Keep in mind that there aren't just contemporaneous differences. Both domains evolve over time through use and reinterpretation.\is{language change!use and reanalysis} The handball rule has transformed considerably due to new technologies like video assistant referee (VAR), changing gameplay dynamics, and the practical limitations of what players and officials can process in real time. Language similarly shifts through usage, reanalysis, and evolving perceptions of what's acceptable. The futurate \textit{going to} idiom, for instance, evolved from expressions of physical movement.\is{grammaticalization!going to@\textit{going to}-future}

Violations in both domains are judged across multiple levels. A handball decision considers positioning, intent, impact, and game context. Similarly, grammatical judgments involve syntax, sound patterns, meaning, and social appropriateness.\is{acceptability judgment!multi-factor}\is{pragmatics!social meaning} Sometimes violations are intentionally overlooked for functional effect, like letting play continue for advantage in soccer or indicating some social meaning in language.

Processing limitations affect both domains. A referee might miss a handball due to an obstructed view or game speed, just as a listener might misinterpret a complex sentence or overlook an error in rapid conversation.\is{processing limitations!performance vs competence} Both demonstrate how human cognition grapples with complex, dynamic rule systems.

Importantly, neither strict handball rules nor perfect grammar are required for success. An unintentional handball that doesn't affect play might go unpunished; an ungrammatical construction like \textit{It very good} might communicate perfectly well.\is{communication!and grammaticality} What matters is how the violation affects the larger goal~-- whether that's fair play or clear communication.

The severity of judgments varies in both domains based on context and impact. A handball might result in anything from ``play on'' to a red card; a grammatical error might pass unnoticed or cause serious misunderstanding.\is{register!effects}\is{stakes and judgments} These judgments reflect not just rule violations but their effect on the broader activity.

In essence, both handball rules and grammar serve as frameworks shaped by human convention, cognitive constraints, and practical needs. A language's grammatical requirements may be somewhat arbitrary, but they're not unmotivated~-- they're evolving systems that balance formal requirements against functional demands, and they reflect what each community considers consistently important or relevant.\is{functional pressures}\is{cognitive load}

\section{Form--Meaning Pairings: The Foundation}
\is{construction!defined}\is{semantics!truth-conditional}\is{pragmatics!discourse functions}\is{social meaning}\is{register}

At its most basic, a grammatical construction is one with an accepted form--meaning pairing within a specific language community, dialect, register, or situation. These pairings involve both semantic components and socio-pragmatic aspects. For instance, \textit{Try it and you'll regret it} isn't just a conditional statement~-- it often conveys a threat.\is{conditionals!threat/ultimatum use}

The meaning element of grammatical constructions extends beyond simple semantics. At the most basic level, a construction's meaning includes its truth-conditional semantics~-- what has to be true in the world for it to be true. But constructions can also encode pragmatic meaning (like social distance, as when \textit{did you want that} is more formal than \textit{do you want that}), social meaning (like a speaker's relationship to standard language norms, as when \textit{I ain't doing nothin'} marks non-standardness in a way that \textit{I'm not doing anything} doesn't), and discourse meaning (like managing turn-taking or expressing emotion).\is{politeness strategies}\is{negation!double negation} Some constructions, like interjections, may have little semantic content but rich pragmatic, social, or discourse meaning~-- consider how \textit{oh} can mark surprise, manage turn-taking, or signal a change in the speaker's knowledge state, depending on its context and intonation.\is{interjection}\is{discourse marker!oh@\textit{oh}}

And~-- I promise, this is the last time in the book that I'll make this point (See Section \ref{sec:fallacy-of-monosemy})~-- it's not the case that a given form must have a single meaning or that a meaning can only be conveyed by a unique form.\is{fallacy of monosemy}\is{monosemy}\is{polysemy} Consider how the progressive construction can indicate ongoing action (\textit{she's working}), future plans (\textit{we're moving next month}), or habitual irritating behavior (\textit{he's always leaving dirty dishes}).\is{aspect!progressive (uses)} And future time can be expressed through the progressive (\textit{I'm leaving tomorrow}), \textit{going to} (\textit{it's going to rain}), modal \textit{will} (\textit{she will arrive at 6}), or simple present (\textit{the train leaves at noon}).\is{future time, futurity}\is{futurate}

In fact, the relationship between form and meaning is even more flexible than these examples suggest.

\paragraph*{Meaning Without Grammar}\is{meaning!without grammar}\is{gesture}\is{paralinguistic cues}

It's critical to realize that, although grammatical forms carry meaning, meaning can be conveyed and understood perfectly well without grammar. Though the following utterances employ neither syntax nor morphology~-- not even lexis in the case of speaker B in example (c)~-- there can be no mistaking their meaning.

\ea
\ea[]{\textit{Hungry. Please food. Help.}}
\ex[]{\textit{Coffee hot! Careful! No drink!}}
\ex[]{A: \textit{Where's the book?} \\B: (shrugs, raises eyebrows, and grunts \textit{unnnhunnh})}
\z\z

This raises an important distinction between grammaticality and communicative effectiveness.\is{communication!and grammaticality} Grammaticality arises from and depends on form--meaning pairs~-- it needs meaning~-- but meaning is insufficient. What makes a construction grammatical is its acceptance within a specific community as a conventional way to express a meaning. And different communities make radically different choices, not just about what forms to use, but about which meanings must be grammatically encoded at all.

\subsection{Context and Community}
\is{community!speech communities}\is{dialect}\is{register}\is{aspect!progressive vs simple}\is{evidentiality}\is{calque}\is{code-switching}\is{bilingualism}

Different language communities encode different distinctions as grammatically relevant. English requires marking progressive aspect in certain contexts (\textit{She works} vs.\ \textit{She is working}) while French does not. Using the simple present when progressive is required (*\textit{Look, he runs}) or progressive when simple present is appropriate (\textit{*The sun is rising in the east every morning}) violates English grammaticality. A French speaker, though, can use \textit{Elle travaille} for both meanings, adding another construction if the distinction is important.

Similarly, Turkish\il{Turkish|(} mandates marking evidentiality~-- how the speaker knows something~-- in contexts where English leaves this information optional. Consider:

\ea
\ea \textit{Goran yemek yiyor}\\`Goran is eating (I see him eating)'
\ex \textit{Goran yemek yiyormuş}\\`Goran is eating (I heard/inferred he's eating)'
\z\z

Turkish grammar generally requires speakers to specify whether they directly witnessed an event or learned about it secondhand~-- a distinction that might seem oddly specific to English speakers. (``You mean I \strong{have} to indicate whether I actually saw Goran eating or just heard about it?!'') But the mandatory nature of English progressive marking might seem equally puzzling to Turkish speakers (``You're telling me I have to signal whether or not I'm considering the end of Goran's eating?!'') A language's grammatical requirements may be arbitrary, but they're not unmotivated~-- they reflect what that speech community considers consistently worth marking. Just as Turkish's evidentiality marking reveals the cultural significance of information sources, English's progressive aspect shows that the community of Standard English speakers conceptualize an action's temporal boundedness as relevant enough that we've grammaticalized its expression.

The notion of a community of speakers itself bears consideration. Such groups can split along innumerable geographical and social borders, such as those between region, age, profession, or even shared interests, each fostering its own linguistic norms and expectations.

For most of us, \textit{lift} and \textit{use} are strictly transitive verbs. Bodybuilders, though, use intransitive \textit{lift} to mean `lift weights', while drug users employ intransitive \textit{use} to mean `use drugs'. These specialized meanings emerge because the distinctions they encode are consistently relevant to these communities. 

Similarly, bilingual communities may mix languages when certain distinctions from one remain communicatively important, handy, or just familiar. A Chinese--English bilingual family might say \textit{open the light}, reflecting Chinese 開燈 (kāi dēng)\il{Chinese}, or a Spanish-English family might use \textit{she's more tall than me}, carrying over Spanish comparative patterns.\il{Spanish}

But these communities go beyond simple calques to develop more complex patterns of language mixing. Turkish-English bilinguals I know mix Turkish question marking within English syntax. For instance:

\ea \textit{Comedin mi?} \z

In this example, the Turkish question particle \textit{mi} and second-person singular past-tense marker \textit{-din} are combined with the English root \textit{come}, adapting it to Turkish grammatical structure to mean `did you come?' This type of mixing allows speakers to draw on familiar patterns from Turkish while using English vocabulary, creating a hybrid form that feels natural within their bilingual context.Turkish\il{Turkish|)}

Similarly, a Vietnamese\il{Vietnamese} students tells me that \textit{She eat rồi} (`She has eaten already') is something Vietnamese--English bilinguals would say.

And all of these are grammatical. They're not Standard English, but they don't purport to be. They're accepted form-meaning pairs within the respective speech communities.\is{community!norms} Of course, Standard English happens to be a very large, diverse, and powerful community (See Section \ref{sec:standard}), whose shadow can be difficult to escape~-- especially when speakers of other dialects are bilingual with SE.

This interplay between general cognitive mechanisms and specific linguistic contexts reflects a broader pattern: domain-general biases often have domain-specific effects.\is{domain-general biases} While humans possess general cognitive preferences~-- like the bias toward simpler, more compressible patterns~-- these preferences manifest differently across domains. In language, they interact with linguistic representations and communicative pressures to create effects that appear highly specialized.

Consider how a general cognitive preference for simplicity plays out across different aspects of language. In phonology, it might lead to simpler syllable structures. In morphology, it pushes toward regular rather than irregular forms. In syntax, it favors consistent ordering patterns. Though the underlying bias is the same, its effects depend crucially on the specific structures involved. What counts as ``simple'' in each case depends on the representational system in question.\is{simplicity bias}

\section{When Things Go Wrong: Types of Ungrammaticality}
\is{ungrammaticality|(}\is{form--meaning!mismatch}\is{historic present|(}\is{present tense!historic|(}\is{tense!present}

That doesn't mean that anything goes. Things fall apart~-- forms can fail to pair with any meaning, meanings can clash with the demands of other forms, and processing demands can exceed cognitive limits.

\subsection{Form--Meaning Mismatches}
A construction becomes ungrammatical when the mapping between form and meaning breaks down in one of three fundamental ways:
\begin{enumerate}[noitemsep]
    \item The grammatical form lacks any viable meaning interpretation
    \item The intended form--meaning pairing fails to align with conventional usage
    \item A non-standard form is used without pragmatic motivation
\end{enumerate}

\paragraph*{Type 1: Meaningless Forms} When a grammatical construction fails to map to any coherent meaning, we have the most severe type of ungrammaticality. Consider the utterance *\textit{Can the have running}. This sequence of words follows no interpretable pattern in English syntax, making it impossible for speakers to construct any meaning from it. Unlike other types of ungrammaticality, such constructions are rejected not because they express meaning incorrectly, but because they fail to express any meaning at all.\is{nonsense string}

\paragraph*{Type 2: Meaning Misalignment} This type encompasses cases where the form could potentially carry meaning, but that meaning clashes with either:
\begin{itemize}[noitemsep]
    \item The speaker's intended meaning
    \item Other meaning elements in the utterance
\end{itemize}

For example, when a French\il{French} speaker says *\textit{I have 35 years} (calqued\is{calque} from French \textit{J'ai 35 ans}) to express their age, the utterance is ungrammatical not because it's syntactically malformed, but because its intended meaning (`I am 35 years old') fails to align with how English conventionally expresses this concept. Similarly, *\textit{I go there yesterday} demonstrates a temporal contradiction between the present-time meaning of the verb tense and the past-time meaning of \textit{yesterday}.\is{tense–time mismatch}

But the same string of words can become grammatical when pragmatically motivated:
\ea[]{\textit{So, I go there yesterday, and suddenly...}}
\z

Here, the present tense form serves a legitimate discourse function~-- the ``historical present" used for vivid narration~-- resolving what would otherwise be a meaning clash. This interpretation is facilitated by \textit{so}, meaning `I'm about to tell you a story'.\is{historic present|)}\is{present tense!historic|)}\is{tense!present}

A particularly interesting case of meaning misalignment occurs with question formation. Consider the sentence \textit{Kim believes the claim that Pat saw someone}. When we try to ask for the identity of `someone' by moving the word to the front, we get the ungrammatical *\textit{Who does Kim believe the claim that Pat saw?}\is{island constraint}\is{complex NP constraint} This illustrates how information-structural requirements can clash: the construction starting with \textit{who} is a focus\is{focus} construction (see Section \ref{sec:focused interrogatives}). Part of its meaning is something  like `pay attention to the element in front'. But the \textit{the claim that...} construction includes the meaning `consider the content of this claim as a whole'. As \citet{cuneo_goldberg_2022} argue, the resulting ungrammaticality arises from the clash between the discourse meaning of the syntactic structures.

\paragraph*{Type 3: Unmotivated Non-Standard Forms} The third type involves using non-standard forms without any clear communicative purpose. Consider *\textit{We sheared three sheeps}. While English typically forms plurals with \textit{-s}, \textit{sheep} is conventionally invariant. The form \textit{sheeps} could be grammatical in specific contexts:
\begin{itemize}[noitemsep]
    \item Distinguishing types of sheep: \\\textit{We raise various sheeps, including Merino and Dorset.}
    \item Referring to sports team members \\\textit{The Sheffield Sheep won again~-- those Sheeps are unstoppable!}
    \item In metalinguistic discussion \\\textit{How many }sheep\textit{s did you count in the text?}
\end{itemize}

Without such pragmatic motivation, though, using \textit{sheeps} represents a form--meaning mismatch~-- the speaker has departed from conventional form without any communicative justification.\is{plural!irregular}\is{metalinguistic pluralization}

These types demonstrate that grammaticality judgments must consider multiple levels of linguistic structure:
\begin{itemize}[noitemsep]
    \item Syntactic well-formedness
    \item Semantic coherence
    \item Pragmatic appropriateness
    \item Information structural constraints
\end{itemize}\is{acceptability judgment!multi-factor}

A complete account of grammaticality must therefore examine not just syntactic form, but the entire context of form--meaning mappings, including speakers' communicative intentions and the discourse environment in which utterances occur.

\subsection{Processing Issues}
\is{processing limitations!false negatives}\is{processing limitations!false positives}\is{garden-path sentence}\is{center embedding}\is{comparative illusion}\is{agreement attraction}

Processing limitations can lead to both false negatives (judging grammatical constructions as ungrammatical) and false positives (judging the ungrammatical as grammatical). These mismatches between actual and perceived grammaticality reveal how cognitive constraints influence our linguistic judgments.

\paragraph*{False Negatives}
A grammatical construction can be wrongly judged ungrammatical when:
\begin{itemize}[noitemsep]
    \item The structure is misperceived due to garden path effects (e.g., \textit{The old man the phones}, meaning `The elderly are responsible for handling the phones')
    \item Processing capabilities are exceeded by structural demands (e.g., \textit{The bread the baker the apprentice helped made is delicious})
\end{itemize}

Consider garden path sentences like \textit{The horse raced past the barn fell} and \textit{The old man the phones}. While syntactically well formed, such constructions often receive negative judgments because their initial parsing leads readers down an incorrect interpretive path. In \textit{The horse raced}, readers typically interpret \textit{raced} as a past-tense head verb. When they hit \textit{fell}, the clause seems like a two-headed monstrosity. This happens because \textit{the horse raced past the barn} is entirely plausible as a main clause and seems much likelier than for \textit{raced past the barn} to be a modifier in the NP headed by \textit{horse} (equivalent to \textit{The horse that was raced past the barn fell}).

Similarly, in \textit{The old man the phones}, readers initially interpret \textit{man} as a noun rather than a verb because NPs like \textit{the old man} are so much more common, making it difficult to realize that the subject is just \textit{the old}. The strong initial activation of the simpler, more frequent interpretations can suppress recognition of the correct but less common readings, leading to perceived ungrammaticality.

A sentence like \textit{The bread the baker the apprentice helped made is delicious} employs the usual form for centre-embedded relative clauses, but the burden of one inside another inside a third likely exceeds working memory capacity for most speakers.

\paragraph*{False Positives}
Conversely, an ungrammatical construction can be wrongly judged grammatical when processing capabilities are exceeded by semantic demands. Consider:
\ea[*]{\textit{More people found Mr Bush's ads negative than they did Mr Kerry's.}}
\z

This sentence, appears to be a standard comparative construction. When you look carefully at it, though, you find that it doesn't actually mean anything: its structure entails something like comparing the number of people who had one perception to the degree of another perception. To see this more clearly, consider the grammatical alternatives:

\ea[]{\textit{More people found Mr Bush's ads negative than did Mr Kerry's.}\\\hfill[compares number of people]}
\ex[]{\textit{People found Mr Bush's ads more negative than they did Mr Kerry's.}\\\hfill[compares negativity]}
\z

But the form up to \textit{than} is likely enough to evoke the right meaning, and with that accomplished, we don't seem to notice that the complement in the \textit{than} PP goes off the rails.\is{comparative illusion}

Such false positives aren't limited to comparative constructions. When multiple semantic features compete for our attention, even basic grammatical requirements like subject--verb agreement can be overlooked. Consider (\ref{ex:agreement-error}) from \citet{corbett2016}.

\ea[*]{\textit{The patchwork of laws governing background checks, assault-weapons limits, and open-carry practices help explain why people continue to be wounded and killed.}}\label{ex:agreement-error}
\z

Again, this strikes most people as fully grammatical on a first read, but the subject is singular (its head being \textit{patchwork}), and so the verb should be \textit{helps}, not \textit{help}. There's just so much plurality going on that we're overloaded and we simply accept \textit{help} as the correct choice.\is{subject--verb agreement}\is{agreement attraction}

These cases highlight how grammaticality judgments involve both immediate processing and considered analysis.

\subsection{Judgment Severity Range}
\is{acceptability judgment!gradient}\is{register!effects}\is{register!violation}

Not all grammaticality judgments are equal in severity. While some errors strike listeners or readers as glaringly ungrammatical, others may be perceived as minor deviations that still allow the intended meaning to come through. The range of grammaticality judgments reflects both the type and context of a violation and includes factors such as syntax, semantics, pragmatics, and social appropriateness.

\paragraph*{Highly Ungrammatical}

At one end of the spectrum are utterances that are unambiguously and severely ungrammatical, often because they fail to convey any coherent meaning. These forms generally violate core syntactic rules and lack interpretable form–meaning mappings:\textit{Can the have running.} The lack of coherent structure here makes interpretation impossible, and speakers would universally reject such forms as ungrammatical.

\paragraph*{Mildly Ungrammatical}

Mildly ungrammatical utterances, by contrast, involve less severe violations that may be overlooked in casual conversation or certain dialects. These forms often convey meaning clearly but deviate subtly from the expected form, for instance, \textit{I go there yesterday.} This construction contains a clash between the verb's present-tense form and the past-time adverb \textit{yesterday}. Although non-standard, it remains comprehensible, especially if pragmatic cues indicate a narrative or historical present.

\paragraph*{Context-Dependent Severity}

The perceived severity of an error often depends on the communicative context. In formal written English, a deviation like *\textit{She don't know} would likely be judged more harshly than in informal spoken English, where the same form might be accepted as part of a particular dialect or register. Similarly, grammaticality judgments may shift based on the expectations of a specific linguistic community. What might seem ungrammatical in Standard English could be completely acceptable in another dialect or regional variety, reflecting that grammaticality is not fixed but socially and situationally flexible.

\paragraph*{Range in Grammaticality Judgments}

The gradations in judgment severity indicate that grammaticality is not a simple binary between acceptable and unacceptable but a spectrum. This range reflects the interaction of cognitive constraints, social expectations, and communicative goals. A sentence like \textit{Each one are different} may feel mildly ungrammatical, as the meaning is clear despite the subject–verb disagreement, whereas *\textit{Each con\textbf{tract} is different}, with stress on the second syllable, disrupts interpretation by confusing lexical meaning and pronunciation.

In sum, grammaticality judgments depend on the degree of deviation from expected form–meaning pairings, the clarity of the intended meaning, and the social context in which the construction appears. This spectrum of severity shows that grammaticality is not about strict adherence to form alone, but rather about balancing form, meaning, and context to achieve effective communication.
\is{ungrammaticality|)}

\section{Evolution and Change}
\is{language change|(}\is{grammaticalization}\is{negation!French \textit{pas}}\is{do-support@\textit{do}-support}\is{word order!Old English}\is{case!loss of marking}\is{social evaluation of forms}\is{auxiliary do@auxiliary \textit{do}}

Like the handball rule we started with, grammaticality shifts over time through use, reanalysis, and evolving needs. These changes don't happen randomly~-- they follow predictable paths driven by semantic, social, and structural pressures.

\subsection{Motivations for Change}

\subsubsection{Semantic Opportunities}
Sometimes a metaphorical connection suggests itself as a handy way to express something, and speakers seize the opportunity. The French\il{French} negative \textit{pas} illustrates this perfectly. It started as the word for `step', and speakers noticed that saying \textit{Je ne marche pas} ('I don't walk a step') was a particularly vivid way to emphasize negation~-- after all, if you're not even taking one step, you're really not walking at all. This minimal unit of movement proved so useful for emphasizing negation that speakers extended it to other verbs: \textit{Je ne parle pas} (`I don't speak a step'). Though the metaphor makes less literal sense here, the negative force was clear. Eventually, \textit{pas} became so useful as a negative marker that it started taking over from \textit{ne} entirely.\is{Jespersen's cycle}

The development of \textit{going to} in English shows a similar opportunistic adoption. When you're going somewhere to do something, your movement and your intention are bound together~-- \textit{I'm going to the store to buy milk}. Speakers noticed this connection between movement and future action could be metaphorically extended. First it spread to cases where the movement was followed by another action, like \textit{I'm going to help you}, and then to situations that didn't involve any movement: \textit{It's going to rain}. What started as a handy metaphorical extension for expressing planned actions became fully grammaticalized as a general future marker.\is{grammaticalization!going to@\textit{going to}-future}

\subsubsection{Structural Pressures} 
Sometimes grammar changes because the current system exceeds speakers' processing capacities. In Old English\il{English!Old}, word order was flexible because nouns inflected to show their functions in the sentence (as personal pronouns still do today). For instance, \textit{Se hund bit þone mann} and \textit{þone mann bit se hund} both meant `The dog bites the man' because \textit{se hund} marked the subject form and \textit{þone mann} marked the object form, regardless of their position in the sentence.

When English lost these special endings, speakers had to find a new way to show who was doing what. The solution was to have the subject appear first, right before its verb. But this created a problem: making questions had long involved moving the verb around (\textit{Saw you him?}). You couldn't keep subjects reliably before their verbs if you kept moving the verbs.

This wasn't just complicated~-- it had become unworkable. The new requirement to show who was doing what through word order clashed directly with the old system of moving verbs around to form questions and negatives. The language needed a solution that could satisfy both needs at once.

That solution emerged in the form of \textit{do} as a stand-in that could move instead of the lexical verb. As \citet{culicover2008rise} shows, this innovation spread gradually and unevenly. By 1650--1700, \textit{do}-support appeared in 92\% of negative questions but only 46\% of plain negatives. This pattern makes sense: questions required moving something to the front to be understood, while negation could be expressed through other means. The differential adoption rates suggest that speakers implemented the new construction most quickly where it was most needed.\is{do-support@\textit{do}-support!diffusion}\is{auxiliary do@auxiliary \textit{do}}

\subsubsection{Social Factors}
Sometimes grammar changes because of who uses a construction. As speakers shift from \textit{I'll write it tomorrow} to \textit{I'm writing it tomorrow}, they aren't solving any structural problem. Instead, certain groups~-- often younger or urban speakers~-- start using the construction to mark themselves as modern or sophisticated. If these speakers have social influence, others begin adopting the pattern too.\is{progressive aspect!social spread of futurate}

The opposite can also happen~-- constructions associated with lower status groups often become stigmatized and die out among those seeking higher status. Double negatives like \textit{I don't want nothing} were once standard in English but became marked as uneducated. The same happened to \textit{ain't} (e.g., \textit{I ain't tired}), which was once used by all social classes but gradually became socially restricted as it acquired a stigma.\is{negation!double negation}\is{ain't@\textit{ain't}}

Social factors help explain why seemingly equivalent constructions end up with different fates. English could have solved its word order pressures in various ways, but \textit{do}-support caught on in part because it first appeared in the writing of educated speakers. Whether a construction spreads or dies often has as much to do with who uses it as how well it works grammatically.\is{prestige dialects}

\subsection{Domain-General Biases in Language Change}
\is{domain-general biases}\is{simplicity bias}\is{regularization}\is{head--dependent order}\is{culture!and transmission}

Recent research suggests that many changes in language arise from the interaction between domain-general cognitive biases and the specific properties of linguistic systems \citep{culbertson2016simplicity}. Of particular importance is the bias toward simplicity~-- a general cognitive preference for patterns that can be represented more concisely. This domain-general bias can have highly specific effects on language when it interacts with linguistic representations and communicative pressures.

This simplicity bias helps explain several phenomena discussed earlier:
\begin{itemize}[noitemsep]
    \item A tendency toward regularization of irregular forms (as with English verbs moving from \textit{clomb} and \textit{pled} to \textit{climbed} and \textit{pleaded})
    \item A preference for consistent head-before-dependent ordering across phrase types (e.g., in NPs \textit{knowledge {\ob}that you said it{\cb}}, VPs \textit{know {\ob}you said it{\cb}}, and PPs \textit{because {\ob}you said it{\cb}})
    \item A tendency for language to favour building new meanings by combining existing words/parts rather than creating entirely new words for each concept
\end{itemize}

In each case, the general cognitive preference for simplicity interacts with specific linguistic structures to shape how languages change over time.

Importantly, these biases don't need to be strong to have significant effects. Through cultural transmission~-- as language is learned and used across generations~-- even weak cognitive biases can be amplified to create strong patterns \citep{culbertson2016simplicity}. This helps explain why languages show robust regularities despite the absence of hard constraints in our language faculty. A slight preference for regular forms in learning, for instance, can lead to near-complete regularization over generations of transmission.

This perspective helps unify our understanding of language change. Rather than positing separate explanations for different types of change, we can often trace them back to the interaction of domain-general biases with specific linguistic structures. The apparent diversity of linguistic patterns emerges not from a multitude of specialized constraints, but from how general cognitive preferences play out across different aspects of linguistic structure.

\subsection{Processing and Dependency Distances}
\is{dependency distance|(}\is{right-branching}\is{auxiliaries!processing benefits}

Processing limitations play an important role in grammatical evolution, particularly through \textit{dependency distance}~-- the ``distance'' between syntactically related words. Long dependency distances place strain on working memory, which, over time, exerts pressure on languages to favour structures that minimize these distances.

\paragraph*{Dependency Distance and Grammatical Change}

Dependency distance measures how far apart syntactically connected words are within a sentence. For example:

\ea
\textit{The coach believes the team can win.}
\z

Here, the dependency distance between \textit{coach} and \textit{believes} is short, allowing for smooth processing. But what about a more complex sentence?

\ea
\textit{The coach who was the most popular since the Scorpions last made the semi-finals in 1992 believed the team could win.}
\z

The dependency distance between \textit{coach} and \textit{believed} is longer, requiring more cognitive effort to track. Over time, languages tend to avoid or rephrase structures with such long dependencies, leading to shifts toward shorter, more manageable dependency distances. We saw this with the passive construction (See Section \ref{sec:passive}) and again with (\ref{ex:agreement-error}).\is{processing limitations!dependency cost}

\paragraph*{Structural Adjustments from Processing Pressures}

Several grammatical structures reflect a preference for reduced dependency distance:
\begin{itemize}[noitemsep]
    \item \textbf{Right-Branching Structures}: In languages like English, right-branching structures (where additional information follows the main clause, e.g., \textit{I met Rachel in early Spring last year, in Paris, where she was giving a presentation on trolleybus systems}) ease processing by keeping dependencies short.
    
    \item \textbf{Avoidance of Long Dependency Chains}: English and other languages often avoid structures with extended dependencies, favoring rephrasing that keeps related elements closer together.
    
    \item \textbf{Emergence of Auxiliary Constructions}: The development of \textit{do}-support in English questions and negatives reduces dependency load by allowing lexical verbs to remain in place, simplifying the processing of complex sentences.
\end{itemize}

\paragraph*{Impact on Language Evolution}

Processing limitations, particularly related to dependency distance, influence grammatical change by favouring structures that align with human cognitive capacities. This pressure shapes grammatical systems over time, leading languages to adapt toward forms that facilitate efficient processing.
\is{dependency distance|)}

\subsection{Usage-Based Theories of Change}
\is{usage-based theory|(}\is{frequency}\is{entrenchment}\is{analogy}\is{grammaticalization}

I've talked about established or conventional forms, but how do they become convention? It's mostly just exposure. Over time, high-frequency usage can lead to the entrenchment of certain forms, while less-used forms may fade or change.

\paragraph*{Frequency and Entrenchment}

The frequency with which a structure is used strengthens its presence in the language. For example, the frequent use of certain verb-noun combinations, such as \textit{take a look} or \textit{make a decision}, reinforces these specific phrases and encourages speakers to treat them as conventional units. This process of entrenchment means that high-frequency patterns become more stable, while less frequently used forms are more susceptible to change or simplification.

\paragraph*{Analogy and Extension}

High-frequency constructions often serve as models for other structures. For instance, the regularization of English past tense verbs, as seen with the analogy-driven shift from \textit{clomb} to \textit{climbed}, reflects speakers' tendency to extend well-established patterns to new words. Usage-based theories emphasize how analogy allows speakers to generalize patterns across the language, gradually reshaping grammatical structures over time.

\paragraph*{Grammaticalization from Repeated Usage}

Usage-based theories explain grammaticalization~-- the process by which lexical items become grammatical markers~-- as resulting from patterns of frequent use. While the \textit{going to} construction eventually shifted from expressing physical movement to marking future tense, this was a rare metaphorical extension that took hold within the grammar. New metaphors arise in language constantly, but only a few become entrenched enough to change grammar. 

For a metaphor to grammaticalize, it must occur frequently in contexts that reinforce its reinterpretation. In the case of \textit{going to}, speakers consistently used it in ways that emphasized intention or future action, rather than movement. Over time, this frequent usage allowed it to shed its literal meaning and take on a purely grammatical role. But most metaphorical expressions, like \textit{grasp the idea} or \textit{kick the habit}, remain lexical phrases because they don't occur in contexts that reinterpret them as grammatical markers. 

This selective process shows how grammaticalization requires high-frequency use in specific communicative contexts, where meanings shift incrementally until the original lexical item becomes a stable grammatical marker.
\is{usage-based theory|)}\is{language change|)}\is{processing limitations|)}\is{form--meaning!pairing|)}\is{community!speech communities|)}\is{grammaticality|)}

\section{Conclusion}

Grammaticality, much like the nuanced handball rule in soccer, is not a simple matter of right or wrong but a complex interplay of form, meaning, context, and community conventions. This chapter has explored how grammaticality arises from accepted form–meaning pairings within specific linguistic communities, dialects, registers, or situations. These pairings are not universal; they vary widely across different languages and social groups, reflecting the diversity and adaptability of human language.

At its core, grammaticality is about more than just following a set of prescriptive rules. It's about using language in ways that are understood and accepted by a particular community. This means that what is considered ungrammatical in one context may be perfectly acceptable in another. For example, double negatives or certain non-standard verb forms may be grammatically correct within specific dialects or social groups, even if they deviate from Standard English norms.

We have also seen that meaning can be conveyed without strict adherence to grammatical structures. Utterances lacking conventional syntax or morphology can still effectively communicate intended messages, highlighting the distinction between grammaticality and communicative effectiveness. This underscores that while grammatical forms carry meaning, meaning itself is not confined to these forms.

Ungrammaticality arises when there's a breakdown in the mapping between form and meaning. This can occur through meaningless forms that fail to convey any interpretable message, meaning misalignments where intended meanings clash with conventional usage, or the use of non-standard forms without pragmatic motivation. But grammaticality judgments aren't binary; they exist on a spectrum from highly ungrammatical constructions that universally fail to convey meaning to mildly ungrammatical ones that may still be understood in context.

Processing limitations further complicate our perceptions of grammaticality. Cognitive constraints can lead to false negatives, where grammatical constructions are judged ungrammatical due to misprocessing, and false positives, where ungrammatical constructions are mistakenly accepted. These phenomena illustrate how human cognition grapples with complex and dynamic linguistic systems.

Language is inherently dynamic, evolving over time through use, reanalysis, and changing communicative needs. Semantic opportunities, structural pressures, and social factors drive grammatical change, leading to the emergence of new constructions and the obsolescence of others. Usage-based theories emphasize how frequency and context of use contribute to the entrenchment of certain forms, shaping grammatical conventions over time.

In essence, grammaticality is a complex dynamic system, shaped by the balance between form, meaning, community norms, and cognitive capacities. Recognizing this, we can approach language teaching and analysis with a deeper appreciation for its fluidity, adapting our methods to embrace linguistic diversity and change.
