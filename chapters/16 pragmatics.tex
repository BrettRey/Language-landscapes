\chapter{Pragmatics} \label{ch:pragmatics}

\epigraph{We learn to read\\
the weather in each other's voices,\\
the barometric pressure\\
of an almost-spoken word}{}

\section{Introduction to Pragmatics}
\is{pragmatics|(}\is{context|(}\is{inference|(}

Consider this exchange:

\begin{dialogue}
\item[A] \textit{I'm on my way.}
\item[B] \textit{The door's open.}
\end{dialogue}

Nothing in the literal meaning of B's response addresses A's statement, but the exchange makes perfect sense. We can imagine a context in which A is likely going to B's house, and B is saying A can come right in without knocking. The study of how we make such inferences~-- how we derive meaning beyond what is literally said~-- is \textsc{pragmatics}.

Even a seemingly straightforward utterance like \textit{it's cold} can mean radically different things depending on context. Imagine the following situations:

\ea
    \ea[]{[At home, near an open window] \textit{It's cold.}\hfill[Request to close window]}
    \ex[]{[Tasting soup] \textit{It's cold.}\hfill[Complaint about food]}
    \ex[]{[At the bus stop] \textit{It's cold.}\hfill[Just making conversation]}
    \z
\z

The same three words can function as a request, a complaint, or an ice-breaker. The linguistic form remains constant, but the meaning shifts dramatically based on context. This is the domain of pragmatics: how context shapes meaning.

But pragmatics isn't just about context. When I say \textit{some students arrived late}, you understand that not all students were late, even though \textit{some} literally means `at least one, possibly all'. Similarly, if you ask \textit{do you have the time?} I understand that you are asking an open question (`What time is it?'), not a closed question (`Do you or do you not know what time it is?'). And when I respond \textit{three fifteen}, you know I mean it's currently 3:15, not that the baseball score is 3 to 15 or that the number of students in the school is 315. These meanings arise not from the words themselves, but from shared assumptions about how we use language.

The study of these shared assumptions forms another key aspect of pragmatics. We expect speakers to be informative but not overly so, truthful, relevant, and clear. When they appear to violate these expectations, we search for additional meaning. If someone asks \textit{is Pat a good teacher?} and receives the response \textit{Pat has neat handwriting}, the apparent irrelevance triggers a search for implied meaning~-- in this case, that Pat is not a particularly good teacher.

So while semantics deals with the relationship between linguistic forms and their conventional meanings, pragmatics examines how context, shared knowledge, and principles of communication combine to create meaning in actual language use. For language teachers and learners, understanding pragmatics is crucial because so much of what we communicate goes beyond what we literally say.
\is{inference|)}\is{context|)}

\subsection{Why Pragmatics Matters for Language Teaching}
\is{language teaching|(}\is{pragmatics!competence|(}\is{Nation's Four Strands|(}\ia{Nation, Paul}

Crucially for you, this may be the chapter where the learners' instincts are often better than the teachers'. After all, learners bring with them sophisticated pragmatic knowledge from their first language. They know that literal meanings aren't always intended meanings, that social relationships affect how we speak, and that context shapes interpretation. They've been making these kinds of pragmatic inferences their whole lives.

Recall the discussion of subject dropping in Section \ref{sec:subject-dropping}. When a student says \textit{Got it} or \textit{Need help?}, they're using perfectly standard English. But teachers, perhaps focused on an artificially formal register or misunderstanding what constitutes Standard English, may incorrectly mark these as errors. Similarly, the insistence on ``complete sentences'' (Section \ref{sec:complete-sentences}) often reflects teachers' misconceptions about standard usage rather than any real grammatical principle. A response like \textit{On the table} to the question \textit{Where's my phone?} is entirely standard~-- in fact, answering with a complete sentence like \textit{Your phone is on the table} could seem oddly formal or even uncooperative.

The issue isn't always that learners are breaking rules of Standard English, but rather that teachers sometimes operate under various misconceptions: conflating formality with accuracy, idealizing the written language, and failing to appreciate that more isn't always better. This may stem from their own education, where formal academic language was prioritized, or from textbooks that present an artificially rigid version of the language. It might reflect an over-reaction against genuine student errors, such as dropping heads, determiners, and other elements required in a particular context.

This highlights the need for a balanced approach to language development, encompassing Nation's Four Strands of language learning. While learners implicitly absorb pragmatic norms through Meaning-focused Input (listening and reading authentic interactions) and test their understanding through Meaning-focused Output (speaking and writing in context), Language-focused Learning (direct instruction on specific pragmatic features like politeness markers or speech acts) plays a crucial role in making these often-subtle aspects explicit. Furthermore, Fluency Development (practising known pragmatic routines until they become automatic) ensures learners can use these features smoothly in real-time communication. A well-rounded approach recognizes that pragmatic competence, like vocabulary or grammar, develops most effectively when all four strands are nurtured, rather than relying solely on exposure or isolated instruction.

Indeed, research has demonstrated that instruction in pragmatic features significantly improves learners' outcomes compared to mere exposure to the language \autocite{JeonKaya2006, RenLiLu2022}. For teachers, this means consciously incorporating pragmatic awareness into lesson planning rather than assuming students will simply absorb these conventions.

Keep this in mind as we explore the various aspects of English pragmatics in this chapter. Ask yourself whether you're insisting on more elaborate forms because they're more appropriate or just because they contain the structure you want students to practice. There's nothing wrong with the latter~-- explicit practice of forms is often necessary~-- but don't confuse it with teaching appropriate language use. And don't let it lead you to mark pragmatically appropriate responses as errors just because they don't showcase the target structure.
\is{Nation's Four Strands|)}\is{pragmatics!competence|)}

\section{Cooperation and Implicature}\label{sec:implicature}
\is{implicature|(}\is{cooperative principle|(}\ia{Grice, Paul}\is{Gricean maxims|(}

Consider this exchange from Section \ref{sec:fallacy-of-monosemy}:\is{fallacy of monosemy}

\begin{dialogue}
\item[Child] \textit{Can I go to the bathroom?}
\item[Adult] \textit{I don't know, can you?}
\end{dialogue}

This tired form of hazing dressed up as a joke depends on affecting to misunderstand the child's intended meaning. The child is making a request for permission, but the adult pretends to interpret \textit{can} as a question about ability. The adult is being uncooperative, violating the normal expectations we have about how conversation works.

Here's a similarly uncooperative conversation:

\begin{dialogue}
\item[A] \textit{Do you know where JB 110 is?}
\item[B] \textit{Yes.}
\end{dialogue}
Compare that to how most people would react:

\begin{dialogue}
\item[A] \textit{Do you know where JB 110 is?}
\item[B] \textit{That's building J, there. And }B\textit{ is for basement.}
\end{dialogue}

B's response doesn't literally answer the yes/no question that was asked, but it's exactly what we expect. B understands that A is asking for directions, not conducting a survey of people's knowledge of campus buildings. This kind of inference~-- deriving unstated meaning from what is said~-- is called \textsc{implicature}.

Implicature generally works because we assume speakers are trying to be helpful. When someone says \textit{I hear you're a good cook}, they're usually angling for an invitation. When a colleague mentions \textit{I didn't receive your email}, they're asking you to send it again. When someone responds to \textit{How are you?} with \textit{I've had better days}, they're saying things aren't going well.

The philosopher Paul Grice formalized these expectations into what he called the \textsc{cooperative principle}: speakers try to make their contributions appropriate to the conversation. This principle breaks down into four basic maxims that we unconsciously follow and expect others to follow:

\ea
   \ea[]{Be appropriately informative (not too little, not too much)}
   \ex[]{Be truthful}
   \ex[]{Be relevant}
   \ex[]{Be clear}
   \z
\z

Much of what we call implicature arises when these maxims appear to be violated. If I ask \textit{how was the movie?} and you respond \textit{well, the popcorn was good}, I understand you didn't enjoy the film. Your seemingly irrelevant response triggers a search for hidden meaning. Similarly, if I say \textit{some students passed the test}, you understand that not all students passed~-- if everyone had passed, saying \textit{some} would violate the maxim of being appropriately informative.

Grice didn't see the maxims as prescriptive rules about how we ought to speak. Rather, he meant them as descriptions of what speakers typically do and what listeners typically expect. When we deviate from these expectations, he observed, we usually do so deliberately, with the intention of creating additional meaning:

\ea
   \ea[]{[Asked \textit{How did you get here?}]\\ \textit{I placed one foot in front of the other, repeatedly transferring my weight from the back foot to the front foot, while maintaining my balance and adjusting my trajectory as needed to avoid obstacles and ensure arrival at my intended destination.}\hfill[too much information]}
   \ex[]{[About a student's excuse] \\\textit{Yes, and I'm the Queen of England.}\hfill[not true]}
   \ex[]{[Asked if they liked the gift] \\\textit{Well, it's Thursday today.}\hfill[not relevant]}
   \z
\z

In each case, the speaker is overtly breaching normal expectations about communication, but in a way that creates clear meaning. The absurdly detailed description of walking mocks the obvious question. The claim to be the Queen of England marks the student's excuse as equally incredible. And the complete non-sequitur about Thursday effectively communicates dislike of the gift without saying so directly.

These principles of cooperation operate in all languages, but the ways speakers flout them vary considerably. What counts as ``too much information'' in one culture might be ``appropriately thorough'' in another. What seems ``admirably direct'' in some contexts might seem ``unnecessarily blunt'' in others. 

Even routine greetings may work differently: English speakers expect \textit{how are you?} to receive only a brief, positive response. I recently heard of an interaction somebody had with a Russian in which the reply was ``Fine, like everyone in Canada.''
\il{Russian}

Indonesian speakers use \textit{sudah mandi?} (`have you showered yet?') as a similar kind of greeting that requires no literal answer about bathing (something I didn't at first understand when I was in Ambon). The extent to which speakers mark their non-cooperation varies too. English speakers often use elaborate irony or sarcasm~-- \textit{Oh, GREAT, that's EXACTLY what I needed right now}~-- while speakers of other languages might prefer more subtle signals.
\il{Indonesian}
\is{Gricean maxims|)}\is{cooperative principle|)}\is{implicature|)}

\subsubsection{Abduction as the engine of implicature}\label{sec:abduction-implicature}
\is{abduction (inference)|(}\is{implicature!and abduction|(}

The maxims help us recognize when additional meaning might be present, but they don't explain how we actually arrive at specific interpretations. When someone says \textit{the popcorn was good} in response to a question about a movie, how do we leap from this apparent non-sequitur to ``the movie was bad''? The answer lies in what the philosopher Charles Sanders Peirce\ia{Peirce, Charles Sanders} called \textsc{abduction}~-- inference to the best explanation.

Consider again the exchange:
\ea
   \textbf{A:} \textit{How was the movie?}\\
   \textbf{B:} \textit{Well, the popcorn was good.}
\z
The interpretive process isn't a matter of applying memorized rules. Instead, we unconsciously ask: what would make B's response sensible? We generate hypotheses:
\begin{itemize}[noitemsep]
   \item B misheard the question
   \item B has an unusual obsession with cinema snacks
   \item B is avoiding saying something negative about the film
\end{itemize}

Given the context~-- the hesitation marker \textit{well}, the assumption that B is being cooperative, the social norm against unnecessary negativity~-- the third hypothesis best explains the utterance. This abductive leap from surprising utterance to best available explanation happens automatically and rapidly.

This same abductive capacity that helps us bridge gaps in written texts (Section \ref{sec:bridging}) operates in conversation, but with crucial differences. In reading, we typically bridge from given information to new information, inferring unstated connections between explicitly mentioned elements. In conversation, we bridge from literal meaning to speaker meaning, inferring unstated intentions behind utterances.

For language learners, this means implicature isn't fundamentally about learning new rules but about applying existing reasoning skills to new linguistic and cultural data. When a student hears \textit{I'm on my way} followed by \textit{the door's open}, they're not consulting a mental rulebook that says ``mentions of open doors following travel announcements indicate permission to enter''. Instead, they're asking what would make this response relevant and helpful~-- and inferring that the speaker must be facilitating entry.

This abductive process explains why pragmatic competence can transfer surprisingly well across languages despite surface differences. The basic cognitive machinery~-- the ability to reason from puzzling communicative behaviour to likely intentions~-- is universal. What varies is the cultural knowledge that constrains which explanations count as ``best''. In some contexts, \textit{nice weather today} might be small talk; in others, it might be an indirect suggestion to go outside. The abductive process is the same; only the background assumptions differ.

Understanding implicature as abduction also clarifies where explicit instruction helps: when it provides learners with the language and cultural premises they need to generate appropriate hypotheses. When we teach that English speakers often use past-tense modals for politeness (\textit{could you...} rather than \textit{can you...}), we're not teaching a mechanical rule but enriching the hypothesis space learners can draw from when interpreting indirect speech.
\is{abduction (inference)|)}\is{implicature!and abduction|)}

\section{Speech Acts and Performatives}\label{sec:speech-acts}
\is{speech act|(}\is{performative|(}\ia{Austin, J. L.}

Consider what happens when a judge says \textit{I sentence you to five years}. These words don't just describe something~-- they do something. The act of sentencing is accomplished through the utterance itself. Or consider a priest saying \textit{I now pronounce you husband and wife}. Again, the words themselves perform the action. These are clear examples of what J. L. \citet{austin1962} called \textsc{performative utterances}.

But speech accomplishes actions in less obvious ways too. When we say things like \textit{I promise to return it tomorrow} or \textit{I apologize for being late}, we're not just making statements~-- we're performing the acts of promising and apologizing. Even apparently simple statements like \textit{it's cold in here} often perform actions~-- in this case, perhaps, requesting that someone close a window.

Austin's key insight was that all utterances perform actions, not just those with explicit performative verbs like \textit{promise}, \textit{apologize}, or \textit{sentence}. When we say \textit{the bus leaves at six}, we might be making a prediction, stating a fact, issuing a command, making a suggestion, or offering an excuse, depending on the context.

The same speech act can be realized by different syntactic constructions. This becomes particularly clear when we look at requests/directions:

\ea
   \ea[]{\textit{Please close the window.}\hfill[imperative directive]}
   \ex[]{\textit{Would you mind closing the window?}\hfill[interrogative directive]}
   \ex[]{\textit{It's freezing in here.}\hfill[declarative directive]}
   \z
\z

Each accomplishes the same basic task~-- trying to get the window closed~-- but does so through different types of speech acts.

When I say \textit{It's cold}, I perform:
\begin{itemize}[noitemsep]
   \item A \textbf{locutionary act}: producing words with a certain meaning \is{locutionary act}
   \item An \textbf{illocutionary act}: making a statement, request, etc. \is{illocutionary act}
   \item A \textbf{perlocutionary act}: having some effect on the hearer \is{perlocutionary act}
\end{itemize}

The locutionary act is simply saying that the temperature is low. The illocutionary act might be requesting that someone close a window. And the perlocutionary act would be actually getting them to close it. Of course, perlocutionary acts can fail~-- my hint about the temperature might be ignored~-- but the illocutionary act of requesting is still performed whether or not it achieves its intended effect.

These three levels of acts help explain many pragmatic misunderstandings, particularly across cultures. Consider the contrast between English \textit{I know} and \textit{I see} responses:

\ea[]{Speaker: \textit{The meeting is at 3pm tomorrow.}\\Listener:}
\ea{\textit{I know.}}
\ex{\textit{I see.}}
\z
\z
While both responses have similar locutionary meaning (expressing the listener's knowledge of teh meeting time), they perform dramatically different illocutionary acts. \textit{I know} often functions as a conversation-stopper that signals impatience or can even be perceived as rude, while \textit{I see} acknowledges receiving valuable new information. It's not uncommon for language learners to use \textit{I know} when they mean \textit{I see}, creating perlocutionary effects (annoying or confusing the speaker) they never intended. This pragmatic distinction, rarely taught explicitly, causes frequent cross-cultural misunderstandings and deserves attention.

\begin{tcolorbox}[title=Practice: Analyzing Speech Acts, colback=white, parbox]
\setlength{\parindent}{1.5em}
\noindent For each of the following exchanges, identify:\\
\phantom{~~~}1. The literal meaning (locutionary act)\\
\phantom{~~~}2. The intended meaning (illocutionary act)\\
\phantom{~~~}3. The expected outcome (perlocutionary act)\\
\phantom{~~~}4. Any relevant cultural considerations
\begin{enumerate}[noitemsep]
    \item In response to ``Would you like some coffee?'':\\
    ``I've already had five cups today.''
    
    \item Teacher to student arriving late:\\
    ``Nice of you to join us.''
    
    \item At a dinner party:\\
    Host: ``Would anyone like the last piece?''\\
    Guest: ``Oh, I couldn't.''
    
    \item In an office:\\
    ``Do you have a minute?''
    
    \item On the phone:\\
    ``I should let you go.''
\end{enumerate}

Consider how these exchanges might differ in other languages/cultures you're familiar with. How would you help students understand and produce appropriate responses in English?
\end{tcolorbox}
\is{performative|)}\is{speech act|)}

\section{Politeness and Face}\label{sec:politeness}
\is{politeness|(}\is{face (social)|(}

Consider these ways of asking someone to move:
\ea
    \ea[]{\textit{Move.}\hfill[imperative]}
    \ex[]{\textit{Could you move, please?}\hfill[modal interrogative]}
    \ex[]{\textit{I wonder if you'd mind moving a bit?}\hfill[subordinate interrogative]}
    \z
\z

All convey basically the same message, but they differ dramatically in politeness. The imperative is direct to the point of rudeness in most contexts, while the subordinate interrogative is almost apologetic in its indirectness. Understanding these differences requires understanding the concept of \textsc{face}.

\subsection{Face and Face-Threatening Acts}
\is{face (social)!threats to|(}

\textsc{Face}, in pragmatics, refers to our public self-image~-- how we want others to see us and how we see ourselves. It has two aspects:
\begin{itemize}[noitemsep]
    \item \textbf{Positive face}: our desire to be liked, appreciated, and approved of \is{positive face}
    \item \textbf{Negative face}: our desire for autonomy and freedom from imposition \is{negative face}
\end{itemize}

Many speech acts inherently threaten either positive or negative face. Requests threaten negative face by imposing on the hearer. Criticisms threaten positive face by expressing disapproval. Refusals threaten both faces~-- they deny a request (threatening the asker's positive face) while also creating social discomfort (threatening the refuser's positive face).

Consider these refusals:
\ea \label{ex:refusals}
    \ea[]{\textit{No.}\hfill[bare negative]}
    \ex[]{\textit{I'm afraid I can't.}\hfill[regret expression]}
    \ex[]{\textit{I'd love to, but I'm away.}\hfill[positive preface + excuse]}
    \z
\z

The bare negative in (\ref{ex:refusals}a) does nothing to mitigate the face threat. Each subsequent example adds more face-saving elements: expressions of regret, positive feelings, and explanations. These structures work to preserve both the speaker's and hearer's face.
\is{face (social)!threats to|)}

\subsection{Politeness Strategies}
\is{politeness!strategies|(}\is{negative politeness}\is{positive politeness}

English speakers manage face threats in two main ways. Sometimes we try to minimize the imposition on others (negative politeness), and sometimes we try to emphasize social bonds and shared goals (positive politeness).

Consider this exchange between colleagues:

\begin{dialogue}
\item[A] \textit{I was wondering if you might possibly have time to look at this report?}
\item[B] \textit{Let's go through it together~-- we can figure out what needs changing.}
\end{dialogue}

A's request shows classic negative politeness: the past progressive \textit{was wondering}, the past-tense modal \textit{might}, the hedge about possibly having time. Each element works to minimize the imposition. It's as if A is saying ``I know this threatens your autonomy, and I'm doing everything I can to show I respect that.''

B's response, in contrast, exemplifies positive politeness. Instead of maintaining distance, B emphasizes connection and shared purpose through \textit{let's} and \textit{we}. By framing it as a collaborative activity rather than a favour, B reduces the face threat while strengthening social bonds.

Different cultures and contexts favor different mixes of these strategies. Some Asian cultures, for example, tend to use more negative politeness in professional contexts, maintaining greater distance even between close colleagues. American English speakers often use more positive politeness in professional settings, actively working to reduce distance. Neither approach is inherently more polite~-- they're just different ways of managing face threats.

\subsection{Power, Distance, and Imposition}\label{sec:power}
\is{power (social)}\is{distance (social)}\is{imposition}

The choice of politeness strategy depends on three main factors:
\begin{itemize}[noitemsep]
    \item \textbf{Power}: relative social status of speaker and hearer
    \item \textbf{Distance}: degree of familiarity between participants
    \item \textbf{Imposition}: how big a request or face threat is involved
\end{itemize}

As power difference and social distance increase, so does the complexity of the request structure, even though the basic imposition remains the same.

The relationship between these factors can be seen in the interaction patterns of modal verbs:
\ea \label{ex:modal-patterns}
    \ea[]{\textit{can/will} \(\rightarrow\) low distance, equal power}
    \ex[]{\textit{could/would} \(\rightarrow\) medium distance/power difference}
    \ex[]{\textit{might/wonder if could} \(\rightarrow\) high distance/power difference}
    \z
\z

\begin{tcolorbox}[title=Cross-cultural Pragmatic Failure, colback=white]
\is{pragmatics!failure|(}\is{pragmalinguistic failure}\is{sociopragmatic failure}
When learners transfer L1 politeness strategies directly to English, it can lead to pragmatic failure:\is{cross-linguistic influence}
\begin{itemize}[noitemsep]
    \item Being too direct (\textit{Give me pen})
    \item Being too indirect (\textit{If it's not too much trouble, I'm hoping that maybe...})
    \item Using inappropriate register (\textit{Yo, Professor!})
    \item Missing expected markers (\textit{I want coffee} vs \textit{I'd like a coffee, please})
\end{itemize}

Help students understand that:
\begin{itemize}[noitemsep]
    \item Politeness norms vary by culture
    \item Direct translation often doesn't work
    \item Context determines appropriate forms
    \item Practice with authentic situations helps
\end{itemize}

AI conversation tools can provide safe environments for students to practice politeness strategies across different contexts without the anxiety of real-world consequences. These tools can simulate various power dynamics and social distances, giving students valuable practice with feedback.
\end{tcolorbox}
\is{pragmatics!failure|)}

\subsubsection*{Addressing the teacher}
\is{address term|(}\is{vocative|(}

The choice of address terms in educational settings provides a clear example of how power, distance, and imposition shape linguistic interactions. Using an inappropriate term (like \textit{Teacher} by an adult learner) can threaten both the learner's face (by appearing childish) and potentially the teacher's face (by not using a preferred or role-appropriate term).

Consider the context of adult English language instruction in Ontario, such as LINC programs. What term should instructor encourage adult learners to use when addressing them? Several options exist, each carrying different social meanings and potential face implications:

\begin{itemize}[noitemsep]
    \item First Name: In many contemporary Canadian educational contexts, particularly adult learning and increasingly in post-secondary institutions, using the instructor's first name is encouraged or is the norm. This reduces social distance, fosters a more egalitarian atmosphere, and minimizes the power differential. However, this informality might conflict with the politeness norms of learners from cultures where formal address is required to show respect, potentially causing discomfort for the student.
    
    \item \textit{Teacher}: It's common for learners transferring patterns from other languages to use \textit{teacher} as a direct address form (e.g., \textit{Excuse me, teacher!}) or as an honorific with either first or last names (e.g., \textit{teacher Brett}). The first is strongly associated with children's usage, making its use by adults potentially infantilizing, while the second is rarely used at all.

    \item \textit{Ms.}/\textit{Mr.} + Last name: This remains a standard, respectful form of address that clearly marks the institutional roles. It maintains a greater social distance than first names. While polite, in some adult learning environments aiming for reduced hierarchy, it might be perceived as somewhat formal or even old-fashioned compared to first-name usage, though it remains a safe default for students unsure of the instructor's preference.
    
    \item \textit{Professor}: This title is generally reserved for instructors in colleges and universities and marks a specific academic rank and role. While appropriate as a title or vocative in that context (unless the instructor prefers first names), it would typically not be used in community-based adult ESL programs like LINC.
\end{itemize}

The ``best'' choice is highly dependent on the specific classroom culture, the instructor's preference (which should ideally be communicated clearly), and the learners' own cultural backgrounds and comfort levels. Discussing these options helps learners understand that terms of address are not just labels but active choices that negotiate relationships and reflect social norms. Misunderstanding or misapplying these norms can lead to pragmatic failure, reinforcing the importance of context in determining appropriate language use (see the Cross-cultural Pragmatic Failure box). Instructors can facilitate this by explicitly discussing address terms and establishing clear expectations suitable for an adult learning environment.
\is{vocative|)}\is{address term|)}\is{politeness|)}

\section{Context and Meaning}
\is{context|(}\is{common ground|(}

Consider what happens when someone says \textit{the door's open}. Without more information, we have no way to know whether this is a warning about security, an invitation to enter, a complaint about a draft, an explanation for noise, or a suggestion to close it.

And yet, in actual conversation, we rarely have trouble determining which meaning is intended. When A says \textit{I'm on my way} and B responds \textit{the door's open}, likely, B is telling A they can walk right in. When someone gets up to investigate a noise and says \textit{oh, the door's open}, we know they've found the source. 

What's interesting isn't just that context determines meaning~-- that's obvious~-- but how quickly and reliably speakers and hearers converge on the same interpretation. Consider a longer example:

\ea
   \textbf{A:} \textit{Did you get my email?}\\
   \textbf{B:} \textit{My laptop died.}\\
   \textbf{A:} \textit{So you need me to send it again?}\\
   \textbf{B:} \textit{No, I saw it on my phone. I'll respond tonight.}
\z

B's first response isn't a simple ``no,'' which would be true but misleading. Nor is it a full explanation like ``yes, I got it on my phone, but I haven't responded because my laptop died and I prefer to write longer emails on a proper keyboard.'' Instead, B offers just one piece of relevant context~-- the laptop problem~-- trusting A to work out its implications. A proposes one interpretation, which B then corrects while filling in the rest of the story.

This kind of collaborative meaning-making is the norm in conversation. Consider another example:

\ea
   \textbf{A:} \textit{Coffee?}\\
   \textbf{B:} \textit{I've had five cups already.}\\
   \textbf{A:} \textit{Tea then?}\\
   \textbf{B:} \textit{Water's fine.}
\z

A doesn't ask ``are you declining my offer of coffee because you've already had too much caffeine, or is there some other reason you're mentioning your coffee consumption?'' The meaning is clear from context. Similarly, B doesn't spell out ``no thank you, I don't want tea either because I've already had too much caffeine today, but I would accept a non-caffeinated beverage.'' Each turn provides just enough information for the other person to infer the relevant meaning.
\is{common ground|)}\is{context|)}

\section{Teaching Pragmatic Competence}
\is{pragmatics!competence|(}\is{explicit instruction}\is{implicit instruction}\is{opportunity cost}

When we interpret language, we draw on several types of context: linguistic (surrounding text), physical (immediate environment), social (relationships and roles), and cognitive (shared knowledge). This context-sensitivity is crucial for pragmatic competence.

Successful communication depends on what we call \textsc{common ground}~-- the knowledge, beliefs, and assumptions shared by speaker and hearer. When pragmatic expectations aren't met, communication can break down through pragmalinguistic failure (using language that doesn't convey the intended meaning) or sociopragmatic failure (violating social norms).

For language learners, pragmatic failure can be more serious than grammatical errors because:
\begin{itemize}[noitemsep]
    \item Grammar errors are usually recognized as learning errors
    \item Pragmatic errors may be interpreted as rudeness
    \item Feedback on pragmatic errors is often indirect or absent
    \item L1 pragmatic patterns can be deeply ingrained
\end{itemize}


While earlier findings often suggested a distinct advantage for explicit instruction \autocite{JeonKaya2006}, more recent large-scale analysis indicates that the difference compared to well-designed implicit instruction may not always be statistically significant \autocite{RenLiLu2022}, highlighting the effectiveness of various focused pedagogical techniques. Whether you lean more on explicit or implicit pragmatic instruction, you should approach the topic intentionally. The ideas presented in the chapter should provide you with a solid base from which to build.

Beyond the type of instruction, research syntheses indicate other factors also influence pragmatic learning. Notably, the duration of instruction appears crucial; both meta-analyses suggest that longer periods of focused instruction tend to yield greater gains in pragmatic competence compared to shorter interventions, reinforcing the benefit of sustained attention rather than brief, isolated lessons. As always, the opportunity costs (Section \ref{sec:opportunity-cost}) of time spent should be considered.

Learner characteristics also play a role, with evidence that intermediate-level learners, and instruction in foreign language settings where classroom input is paramount, showing the biggest impact.

How learning is measured can affect the observed outcomes; for instance, gains might appear larger on written tasks compared to more complex oral production tasks. It's also important to recognize that these meta-analyses have limitations, often constrained by the number and methodological rigour of available primary studies, and frequently focusing on specific pragmatic features like speech acts. So, while research clearly supports intentional pragmatics teaching, applying these findings requires considering the specific learners, context, duration, and assessment methods involved.

\subsection{Using AI to Make Pragmatics Visible}
\is{AI for language learning}\is{simulation (pedagogical)}\is{feedback (pedagogical)}

One of the challenges in teaching pragmatics has always been making implicit norms explicit. Conversational AI tools offer unique advantages in this regard:
\begin{itemize}[noitemsep]
\item \textbf{Controlled variation}: AI can demonstrate how the same basic message changes under different conditions of power, distance, and imposition. For example, showing students how a request transforms across contexts:
\ea
   \ea[]{[To close friend] \textit{Close the window?}\\\hfill[direct, minimal politeness]}
   \ex[]{[To classmate] \textit{Could you close the window, please?}\\\hfill[conventional politeness]}
   \ex[]{[To professor] \textit{I'm sorry to trouble you, but could I ask you to close the window?}\hfill[high deference]}
   \z
\z

\item \textbf{Simulated consequences}: AI can respond differently to pragmatically appropriate versus inappropriate utterances, providing immediate feedback without real-world social costs.

\item \textbf{Analytical explanation}: Unlike human interlocutors who may simply follow pragmatic norms unconsciously, AI can often explicitly identify the implicit rules being followed or broken and explain them to learners.

\end{itemize}
For teachers, AI tools can help illuminate pragmatic patterns that might otherwise remain invisible. By analyzing language samples and highlighting pragmatic features~-- hedges, politeness markers, indirectness strategies~-- AI can make these patterns more salient for both teachers and students.
The most effective approach integrates AI as a supplement to, not a replacement for, human interaction. After all, the ultimate goal is successful communication with other people, not with machines. But AI's ability to make explicit what is usually implicit offers a valuable window into the normally hidden workings of pragmatic norms.

\begin{tcolorbox}[title=Teaching Pragmatic Competence, colback=white]
To help students develop pragmatic competence:
\begin{itemize}[noitemsep]
    \item Expose them to authentic language use in varied contexts
    \item Explicitly highlight context-dependent meanings
    \item Practice with realistic scenarios that vary power, distance, and imposition
    \item Discuss cross-cultural differences in pragmatic norms
    \item Provide specific feedback on pragmatic appropriateness
    \item Use AI conversation tools to create safe practice environments where students can experiment with different pragmatic strategies
\end{itemize}
Remember that students' L1 pragmatic knowledge is an asset~-- help them understand how to apply it appropriately in English.
\end{tcolorbox}

By making pragmatics an explicit part of language instruction rather than an incidental byproduct, teachers can help students develop the communicative competence they need to navigate real-world interactions successfully. The goal isn't perfect native-like pragmatic performance, but rather a flexible awareness that allows students to make informed choices about how they use language in different contexts.
\is{pragmatics!competence|)}\is{pragmatics|)}
