\chapter{Adverbs and prepositions}

\epigraph{Slowly, carefully, precisely\\Between here and there}{}

\noindent
While traditional grammatical analysis drew seemingly clear lines between preposition phrases and adverb phrases, this apparent simplicity masked significant problems. The rigid classifications often led to contradictions and failed to capture important distinctions in how these elements actually function in English. If you've ever felt confused about whether a word like \textit{before} is an adverb, a ``conjunction'', or a preposition depending on the sentence, you're not alone.

Many dictionaries and traditional grammars adhere to what we might call the ``NP-complement view'' of prepositions---defining them primarily as words that must precede and govern a noun phrase. This often leads to categorizing words like \textit{before} or \textit{down} as adverbs when they appear without a following noun phrase (\textit{I've seen him before}, \textit{He fell down}) or as ``conjunctions'' when followed by a clause (\textit{before you arrived}). But this approach just creates unnecessary complexity and often contradicts the dictionaries' own examples \autocite{reynolds2025}.

This chapter aims to untangle this confusion by presenting a clearer, more consistent framework based on syntactic behavior---a ``flexible-complement view.'' This view recognizes prepositions as heads of prepositional phrases (PPs) that can take various types of complements (including noun phrases, other PPs, clauses, or even no complement at all). We argue that adopting this perspective not only aligns better with linguistic evidence but also simplifies the grammar, ultimately providing a more robust and pedagogically useful understanding for ESL teachers. We will explore the characteristics and functions of adverbs and prepositions, examine the evidence for the flexible-complement view, address common misclassifications, and discuss specific challenges these categories present for language learners. Our goal is to equip you with the tools to navigate this complex landscape with confidence.

\section{Adverbs and adverb phrases}
The most typical adverbs are those like \textit{happily} where you start with an adjective and you add \textit{--ly}. But there are many others. These include: \textit{again}, \textit{best}, \textit{due}, \textit{even}, \textit{far}, \textit{hard}, \textit{just}, \textit{long}, \textit{maybe}, \textit{never}, \textit{often}, \textit{pretty}, \textit{quite}, \textit{right}, \textit{seldom}, \textit{thus}, and \textit{very}.

\subsection{The characteristics of adverbs and adverb phrases}

\subsubsection*{Adverbs}
\begin{itemize}
    \item often denote manner, frequency, and degree
    \item function as the head of AdvP
    \item are often formed from an adjective + \textit{--ly}
    \item are often gradable with \textit{more}, \textit{less}, \textit{--er}, and \textit{--est}
    \item are usually modifiable by \textit{very} 
\end{itemize}

\subsubsection*{Adverb phrases (AdvPs)} \label{sec:advps}

\begin{itemize}
    \item function as modifier of almost any constituent, including NP, but not noun
    \item rarely function as complements
    \item rarely take complements
    \item are headed by an adverb
\end{itemize}

\subsubsection*{What they denote}

The meanings of adverbs as a group are quite diverse.

\begin{itemize}
    \item \textbf{Manner:} \textit{easily}, \textit{quickly}, \textit{silently}
    \item \textbf{Time and frequency:} \textit{already}, \textit{always}, \textit{often}, \textit{rarely}, \textit{temporarily}
    \item \textbf{Degree:} \textit{almost}, \textit{thoroughly}, \textit{very}
    \item \textbf{Order:} \textit{initially}, \textit{last}, \textit{next}
    \item \textbf{Certainty:} \textit{definitely}, \textit{likely}, \textit{probably}
    \item \textbf{Domain:} \textit{politically}, \textit{scientifically}, \textit{technically}
    \item \textbf{Evaluation:} \textit{fortunately}, \textit{happily}, \textit{well}
    \item \textbf{Speech act:} \textit{confidentially}, \textit{frankly}, \textit{honestly}
    \item \textbf{Connective:} \textit{furthermore}, \textit{however}, \textit{thus}.
    \item etc.
\end{itemize}

\subsubsection*{The heads of AdvPs}
The underlined adverbs are the heads of the following adverb phrases: \textit{\uline{clearly}}, \textit{\uline{very}}, \textit{very \uline{clearly}}, \textit{not so \uline{well}}, etc.

\subsubsection*{Adjective + \textit{--ly}}
The following are quite common examples: \textit{exactly}, \textit{recently}, \textit{usually}, \textit{quickly}, \textit{clearly}, \textit{particularly}, \textit{completely}.


\subsubsection*{Grade}

Like adjectives, adverbs are often gradable: \textit{well}, \textit{better}, \textit{best}, \textit{fast}, \textit{faster}, \textit{fastest}, \textit{more skillfully}, \textit{least helpfully}, etc.

\subsubsection*{Modifiable by \textit{very}}
Also like adjectives, adverbs can often be modified by \textit{very}: \textit{very well}, \textit{very quickly}, \textit{very often}, \textit{very hard}, \textit{very long}, \textit{very soon}, etc.

\subsubsection*{Adverb phrases as modifiers}

Most typically, AdvPs function as modifiers in a wide range of contexts:

\begin{itemize}
    \item \textbf{In VPs}: \textit{She sings \uline{beautifully}}. \textit{He ran \uline{quickly} to the store}.
    \item \textbf{In AdjPs}: \textit{The movie was \uline{incredibly} boring}. \textit{She is \uline{almost} ready}.
    \item \textbf{In other AdvPs}: \textit{He runs \uline{very} quickly}.  \textit{She sings \uline{quite} beautifully}.
    \item \textbf{In clauses}: \textit{\uline{Hopefully}, she will arrive soon.} \textit{\uline{Surprisingly}, he didn't enjoy the concert.}
    \item \textbf{In PPs}: \textit{He lives \uline{right} in the city center}. \textit{She sat \uline{just} behind the curtain}
    \item \textbf{Before NPs}: \textit{\uline{Even} the thought of it is frightening.} \textit{Everyone's presence was comforting, \uline{particularly} yours.}
    \item \textbf{But not in NPs:} *\textit{The \uline{quickly} runner won.} *\textit{She bought the \uline{beautifully} dress.}\footnote{\citet{payne2010} document rare exceptions such as \textit{the demand \uline{internationally} for lithium}.}
\end{itemize}

\subsubsection*{AdvPs and complements} \label{sec:AdvPs+Comps}
It's quite unusual for AdvPs to function as complements, but they do occur. \textit{She }[\textit{treated it \uline{carefully}}]\textit{.} \textit{He }[\textit{worded his response \uline{very precisely}}]\textit{.} \textit{Wait }[\textit{until \uline{later}}]\textit{.} \textit{{\ob}Before \uline{long}{\cb}, they were ready}

Similarly, AdvPs rarely allow complements of any kind, but there are some adverbs that do: [\textit{\uline{happily} for them}]\textit{ it had worked}; [\textit{\uline{independently} of his parents}]; [\textit{\uline{appropriately} for the situation}]. In some cases, the complements are indirect: \textit{\uline{more} important \uline{than that}}, \textit{\uline{as} good \uline{as you}}, \textit{\uline{so} bad \uline{that it's scary}}, \textit{\uline{too} late \uline{to change}}.

\begin{tcolorbox}[title=Practice, colback=white]
Explore the characteristics and denotations of adverbs:
\begin{enumerate}
    \item List three adverbs that denote manner.
    \item Provide five adverbs, as different from each other as possible, formed by adding \textit{--ly} to an adjective.
    \item Provide an example of an adverb that denotes frequency.
    \item How do adverb phrases (AdvPs) typically function in a clause?
    \item Identify three adverbs that cannot be modified by  \textit{very}.
    \item For each of \textit{purposefully}, \textit{purposely}, \textit{friendly}, \textit{weekly}, \textit{brotherly}, \textit{tally}, \textit{gingerly}, \textit{burly}, and \textit{partly}, decide if it is always, sometimes, or never an adverb.
\end{enumerate}
\end{tcolorbox}

\subsection{Not actually adverbs}
Unfortunately, almost all published dictionaries and grammar books are quite confused about the distinction between adverbs and prepositions. They commonly categorize words like \textit{after}, \textit{away}, \textit{before}, \textit{here}, \textit{home}, \textit{now}, \textit{then}, \textit{there}, \textit{when}, \textit{where} as adverbs when they're really prepositions. The main discussion will be in Section \ref{sec:likebefore}, but for now, consider that in Standard English, typical adverbs don't allow modification by \textit{right}, though some other dialects do, so that \textit{she came right quickly} is easily identified as a non-Standard dialect (see Section \ref{sec:standard}). In contrast, \textit{right} commonly functions as a modifier in preposition phrases such as \textit{right into the hole}. Notably, it also does so in phrases headed by the list above. That's just one piece of evidence that these are prepositions, not adverbs.

\begin{tcolorbox}[title=Practice, colback=white]
Differentiate between adverbs and prepositions:
\begin{enumerate}
    \item Identify a word that is traditionally misclassified as an adverb~-- other than those listed above~-- but is actually a preposition.
    \item Construct a sentence where \textit{before} would traditionally have been seen as an adverb.
    \item Provide an example of an adverb phrase functioning as a modifier in a VP.
\end{enumerate}
\end{tcolorbox}

\section{Prepositions and preposition phrases (PPs)} \label{sec:prepositions}

So, what are prepositions then? Here's a sample (those marked with \textsubscript{o} have related words in other lexical categories):
\textit{about}, \textit{before}, \textit{concerning}\textsubscript{o}, \textit{during}, \textit{except}\textsubscript{o}, \textit{for}\textsubscript{o}, \textit{given}, \textit{home}\textsubscript{o}, \textit{in}\textsubscript{o}, \textit{like}\textsubscript{o}, \textit{minus}\textsubscript{o}, \textit{near}\textsubscript{o}, \textit{of}, \textit{plus}\textsubscript{o}, \textit{qua}, \textit{regarding}\textsubscript{o}, \textit{since}, \textit{there}, \textit{under}, \textit{versus}, \textit{with}, and \textit{yonder}. For a more thorough list, see Section \ref{sec:preps-list}.

%%%%%%%%%%%
%reference  to the appendix is numbered incorrectly
%%%%%%%%%%%

Prepositions are some of the most common words in English. Behind only \textit{the} and \textit{and}, \textit{of} occurs roughly once every 50 words. Other extremely common prepositions are \textit{in}, \textit{to}, \textit{for}, \textit{with}, \textit{on}, \textit{at}, \textit{from}, and \textit{by}, all of which are among the 30 most frequently used words in English. 

Part of the reason for this is that along with their basic meanings, they tend to have grammatical functions. For example, \textit{by} indicates the agent in passive constructions, like \textit{The book was written \myuline{by her}}. Similarly, \textit{of} can denote possession, origin, or indeed any relationship, as in \textit{The sound \myuline{of the waves}.} If we can say \textit{I gave her the note} or \textit{I gave the note \myuline{to her}}, how much meaning can \textit{to} really be adding?

\subsection{The characteristics of prepositions}\label{sec:preps}

English prepositions have the following characteristics.

\begin{itemize}
    \item Prepositions typically denote relations in space or time.
    \item They functions as the head of a preposition phrase (PP). (Remember, our technical meaning of \textsc{phrase} allows for single-word phrases.)
    \item PPs often function as modifiers in NPs and VPs.
    \item PPs often function as complements in a range of phrases.
    \item Prepositions usually have no suffixes or prefixes.
    \item And they usually have only one form.
    \item Preposition phrases may include a Mod:AdvP, especially \textit{even}, \textit{just}, and \textit{almost}; and distinctively \textit{right}, \textit{straight}, \textit{clear}, and \textit{way}.
    \item They are almost never modified by degree modifiers like \textit{very}, \textit{too}, and \textit{pretty}.
    \item They most typically take Obj:NP complements.
    \item But they also take a variety of other complements, including PP (\textit{out \uline{of the box}}), Clause (\textit{before \uline{I arrived}}), and AdjP (\textit{for \uline{free}}), or no complement at all.
\end{itemize}

\subsubsection*{Denotation of Prepositions}

Prepositions primarily denote relations. These relations can be:

\begin{itemize}
    \item \textbf{Spatial:} \textit{in}, \textit{on}, \textit{under}, \textit{between}
    \item \textbf{Temporal:} \textit{before}, \textit{after}, \textit{during}, \textit{until}
    \item \textbf{Directional:} \textit{to}, \textit{from}, \textit{towards}, \textit{away}
    \item \textbf{Causal:} \textit{because}, \textit{due}, \textit{since},
    \item etc.
\end{itemize}

\noindent Often prepositions can denote different types of relations (e.g., \textit{before 6:00}; \textit{before the mirror}).

\subsubsection*{Functioning as the Head of a PP}

Every preposition phrase has a preposition at its head:

\begin{itemize}
    \item \textit{\uline{In} the park}, \textit{\uline{during} the movie}, \textit{\uline{under} the table}
\end{itemize}

\subsubsection*{PPs as Modifiers and Complements}

Preposition phrases function as modifiers in NPs and VPs:

\begin{itemize}
    \item Modifying NPs: \textit{the book \uline{on the shelf}}, \textit{the man \uline{with the hat}}
    \item Modifying VPs: \textit{slept \uline{beside a tree}}, \textit{studies \uline{during the day}}
\end{itemize}

They can also function as complements:

\begin{itemize}
    \item In VPs: \textit{She is \uline{in the garden}}. \textit{He talked \uline{to us}}. \textit{I received it \uline{from them}}. 
    \item In NPs: \textit{knowledge \uline{of finance}}, \textit{interest \uline{in movies}}, \textit{desire \uline{for love}}.
    \item In AdjPs: \textit{eager \uline{for work}}, \textit{happy \uline{about the change}}, \textit{proud \uline{of you}}.
    \item In PPs: \textit{down \uline{from the cottage}}, \textit{due \uline{to the time}}, \textit{except \uline{for that}}.
\end{itemize}

\subsubsection*{Form and Structure of Prepositions}

Most prepositions, like those in (\ref{ex:bareprep}), have only one form and lack prefixes or suffixes, but there are exceptions.

\ea 
    \ea \textit{at}, \textit{by}, \textit{from}, \textit{home}, \textit{in}, \textit{now}, \textit{on}, \textit{of}, \textit{over}, \textit{through}, \textit{to}, \textit{under}, \textit{where}, \textit{with} \label{ex:bareprep}
    \ex \textit{apart}, \textit{beside}, \textit{ceilingward}, \textit{depending} \label{ex:compoundprep}
    \ex \textit{in front}, \textit{on board}, \textit{so as} \label{ex:complexprep}
    \ex \textit{close}, \textit{far}, \textit{near} \label{ex:compareprep}
    \z
\z

\noindent Prepositions like those in (\ref{ex:compoundprep}) have internal structure. For example \textit{apart} is, historically at least, \textit{a--} + \textit{part}. Sometimes, single prepositions are written with a space, like those in (\ref{ex:complexprep}). Despite the space, there's good reason (which I won't go into here; see \cite[622]{Huddleston2002}) to think of these as single words. Finally, the three prepositions in (\ref{ex:compareprep}) are a little like adjectives and adverbs in that they have \textit{--er} and \textit{--est} forms. Again, there are good reasons to think they are prepositions, but for those, you should consult \textcite[609]{Huddleston2002}.

\subsubsection*{Modifiers in Preposition Phrases}

Preposition phrases can include modifiers, especially adverbs \textit{even}, \textit{just}, and \textit{almost}; and distinctively \textit{right}, \textit{straight}, \textit{clear}, and \textit{way}. They also include measure phrases like \textit{two metres} as modifiers, but rarely degree modifiers.

\begin{itemize}
    \item \textit{\uline{almost} to the door}, \textit{\uline{even} without it},  \textit{\uline{just} before noon}.
    \item \textit{\uline{right} in front of the house}, \textit{\uline{straight} up the tree}, \textit{\uline{clear} to the horizon}, \textit{\uline{way} out west}.
    \item \textit{\uline{two hours} before he arrived}, \textit{\uline{a kilometre} down the road}.
    \item *\textit{\uline{very} in the room}, *\textit{\uline{so} down the well}, *\textit{\uline{too} on the table}.
\end{itemize}

\subsubsection*{Complements of Prepositions}\label{sec:PP-complementation}

While Obj:NP is the most common complement in PPs, they can take various complements.

\begin{itemize}
    \item Obj:NP: \textit{on \uline{the table}}, \textit{with \uline{John}}
    \item Comp:VP \textit{before \uline{going to bed}}
    \item Comp:PP: \textit{out \uline{of the box}}, \textit{from \uline{behind the clock}}
    \item Comp:Clause: \textit{before \uline{I arrive}}, \textit{of \uline{whether it works or not}}
    \item Comp:AdjP: \textit{for \uline{free}}, \textit{as \uline{possible}}
    \item (no complement): \textit{above}, \textit{home}, \textit{here}, \textit{now}\, \textit{then}, \textit{where}
\end{itemize}

\subsection{PPs and their complements}

A major confusion in almost all discussions of prepositions is the mistaken idea that they obligatorily take Obj:NP complements. This oversimplified view, while pedagogically convenient, is at odds with the empirical data of the language.

Consult any standard dictionary and you will likely find inconsistency in the Obj:NP dogma. While their definitions of \textit{preposition} might suggest that prepositions exclusively take noun phrases as their complements, the very same dictionaries often provide counter-examples that defy this convention. Even the venerable \textit{Oxford English dictionary} falls into this trap. It defines \textit{preposition} as ``An indeclinable word or particle governing (and usually preceding) a noun, pronoun, etc.,'' But under the entry for \textit{for} (categorized as a preposition), you can find ``VI.18.a.ii. With an adjective as complement. Now chiefly in set expressions, as in \textit{to give a person up for lost}, \textit{to leave a person for dead}, \textit{to take for granted}, etc.'' But the problem goes well beyond this edge case.

\subsubsection*{The problem of words like \textit{before}} \label{sec:likebefore}

The word \textit{before} serves as a prime example of the challenges in categorizing prepositions. Traditional grammar sometimes labels \textit{before} as an adverb in contexts like \textit{I've seen it \uline{before}} with no complement, as a preposition in \textit{before \uline{dinner}}, and as a so-called ``subordinating conjunction'' in \textit{before \uline{he arrives}}, where there is a clausal complement. This tripartite classification is not only cumbersome but also conceptually problematic.

\begin{tcolorbox}[title=Words like \textit{before}, colback=white]
    There aren't actually that many  ``words like \textit{before}'', those allowing three types of complement: Obj:NP, clause, or none. Apart from \textit{before}, we have \textit{after} and \textit{since}. But the list of prepositions that take clausal complements is much larger. It includes: \textit{after}, \textit{although}, \textit{as}, \textit{assuming}, 
        \textit{because}, \textit{before}, 
        \textit{considering}, 
        \textit{despite} + \textit{that}, 
        \textit{except} + \textit{that}, 
        \textit{for},
        \textit{given}, 
        \textit{if}, \textit{in} + \textit{that}, \textit{in case}, \textit{in order} + \textit{that}, 
        \textit{lest}, \textit{notwithstanding} + \textit{that}, \textit{now} + \textit{that}, \textit{on condition}, \textit{once}, 
        \textit{provided}, \textit{seeing}, \textit{since}, \textit{so}, \textit{supposing}, 
        \textit{though}, \textit{till}, 
        \textit{unless}, \textit{until}, 
        \textit{when}, \textit{whenever}, \textit{where}, \textit{whereas}, and \textit{while}.

    ~~~The main preposition that allow no complement (at least some of the time) include:
    \textit{a priori}, \textit{aboard}, \textit{about}, \textit{above}, \textit{ago}, \textit{back}, \textit{below}, \textit{beneath}, \textit{down}, \textit{downstairs}, \textit{downtown}, \textit{east}, \textit{here}, \textit{home}, \textit{in}, \textit{indoors}, \textit{inside}, \textit{leftwards}, \textit{next}, \textit{overseas}, \textit{then}, \textit{there}, \textit{when}, \textit{where}, and \textit{within}.
\end{tcolorbox}

If we accept that prepositions can take various complements, including clauses, then there's no need to posit a long list of ``subordinating conjunctions''. We also keep the category of adverbs from becoming too unruly. Recall from Section \ref{sec:advps} that AdvPs rarely allow or function as complements. What then are we to make of \textit{before \uline{now}} if both \textit{before} and \textit{now} are adverbs? 

The core adverbs are the \textit{--ly} words like \textit{really}, \textit{probably}, \textit{actually}, and \textit{simply}. Such words almost never function as complements in \textit{be} VPs. For example, you can't say *\textit{that was probably} or *\textit{the concert is really}. But there's no problem with \textit{that was before} or \textit{the concert is now}. This requires some kind of extraordinary explanation if you think that \textit{before} and \textit{now} are adverbs in these examples. In contrast, if you understand that they're prepositions, then everything simply follows from that and no exceptional explanation is needed.

Most English teachers subscribe to the widespread view of prepositions being words that always take Obj:NP complements, and when they're exposed to the view of prepositions I'm explaining here, they typically reject it as ``difficult''. If you already have one way of looking at things, it is clearly difficult to change your mind. There's no question about that. But for English-language learners, this is the simpler system. It dramatically reduces the number of ``exceptions'' and it's obviously easier for students to learn that \textit{after} is always a preposition than it is to try and understand when it's an adverb, when it's a preposition, and when it's a ``subordinating conjunction''. The choice then becomes one between the system that's easier for the teachers and the one that's easier for their students.\footnote{It's also hardly an innovation. The idea was explained as early as the 18th century \citep{Hunter1784}.}


\subsection{Prepositions present challenges for learners}

Even if grammarians seem to have had a lot of difficulty analyzing prepositions, learning to use them might not seem so hard. Unfortunately, it turns out to be another aspect of English that is actually quite a minefield for learners.

\subsubsection*{Selecting the right preposition in a complement}

One problem is the selection of heads for PP complements. For example, it may seem obvious to you that one graduates \uline{from} a university, but the meaning and structure associated with the verb \textit{graduate} has evolved over time. Originally, \textit{graduate} meant `admit to a university degree' and took an Obj:NP denoting the student(s) (\textit{The university graduated \uline{a large class} this year}). Later, \textit{graduate} came to mean `complete a course of study at high school or college and obtain the qualification', and it licensed a \textit{from} PP complement denoting the school (\textit{I graduated \uline{from high school} in 1989}). In its most recent usage, \textit{graduate} means `successfully exit an educational institution upon completion of the course of study' and takes an Obj:NP denoting the school (\textit{I graduated \uline{high school} in 1989}).

A speaker of English in the 1800s would likely have found the later uses arbitrary and ungrammatical. To somebody learning English, they will seem similarly arbitrary, despite how natural they may feel to you.

The verb \textit{listen} is another that has seen changes in its complementation patterns over time. In its earlier usage, \textit{listen} could take either a direct object (\ref{ex:listen}) or a \textit{to} PP complement (\ref{ex:listento}).

\ea \textit{At which I ceas't, and listen'd \uline{them} a while.} (1634)\label{ex:listen}
\z
\ea \textit{The holy man lestned well \uline{to all hir confession}.} (c. 1450)\label{ex:listento}
\z
    
In contemporary usage, a direct object is no longer possible, and a \textit{to} PP complement has become essentially obligatory. But how did it change from a direct object to a \textit{to} PP? The metaphorical reasoning behind the choice of \textit{to} can be seen in the phrase \textit{give ear to}. It's obvious when you know it, but otherwise, it would be very hard to guess. 

And a \textit{to} PP is only the most common complement. The following either were previously possible (marked with \textsuperscript{†}) or still are.

\begin{itemize}
    \item \textsuperscript{†}\textit{listen of}: `hear tell of'
    \item \textsuperscript{†}\textit{listen on}: equivalent to `listen to'
    \item \textit{listen for}, \textit{listen after}: `be eager or make an effort to catch the sound of', `endeavour to hear or hear of'
    \item \textit{listen out}: `listen for a sound'
    \item \textit{listen in}: `listen to a broadcast program'
    \item \textit{listen up}: `listen carefully', `pay attention'
\end{itemize}

Even with very similar meanings, there is sometimes a choice between two or more valid options: while there may be very slight nuanced differences among them, each of the following is possible \textit{speak about}/\textit{of}/\textit{on}/\textit{to prepositions}. But other choices are more questionable or downright ungrammatical (e.g., *\textit{speak above/at/in/towards/over prepositions}).

I hope it's becoming clear that it's not a simple thing to guess whether a Comp:PP is possible or which preposition might head it and that it's important that you keep this in mind when teaching complementation (See Section \ref{sec:verb-complementation}). Some folks will argue that the choice of preposition is ``motivated''. They're not wrong, and understanding the motivation may (or may not) lead to better recall, but it's simply cruel to expect students to be able to work out the motivations.

At the same time, it's useful to keep in mind that an attempt like \textit{I'm happy on my grades}, while ungrammatical (or at best non-Standard) is still likely to convey the intended sense.

\subsubsection*{Meanings} \label{sec:preposition-meanings}

To understand how the choice of prepositions could be motivated, we need to understand how their meanings expand through analogy.\footnote{For a deeper dive into this idea, see \citet{lakoff1980} and \citet{tyler2003}.} The basic meanings are well covered by good learner dictionaries, like the \textit{Longman dictionary of contemporary English}, and teachers should refer to these regularly. For extended meanings, consider the following chain of analogy for \textit{up}.

The basic meaning of \textit{up} is `away from the ground', as in \textit{The balloon went up.} This is extended to `a higher/larger amount or value', as in \textit{The price has gone up.} From there, we get the idea of `conscious/functioning', presumably from amount of function, as in \textit{The system's back up.} If something is functioning, then it is `prepared/able', as in \textit{Are you feeling up to it.} And from being able to complete something, we get the meaning of `completion/totality', as in \textit{They ate it up.} 

This chain of development may not be entirely historical, but when it's laid out that way we can see how (possibly) one might get from `away from the ground' to `completion/totality'. But the motivation is not one that a typical learner or teacher of English could be expected to work out on their own. Prepositions can be complex and confusing.

For \textit{to}, \citet{tyler2003} give the following meanings and examples:

\ea \label{ex:to-meanings}
    \ea \textbf{location:} \textit{In this picture, Diana is standing to my left.}
    \ex \textbf{contact:} \textit{Apply the soap directly to the stain for best results.}
    \ex \textbf{attachment:} \textit{He added a fence to the garden.}
    \ex \textbf{event:} \textit{The captain went to the boaters' rescue.}
    \ex \textbf{comparison:} \textit{The design of this sweater is inferior to that one.}
    \z
\z

\subsubsection*{Uses}

Even when the meaning of a preposition is the most basic one, the use of that meaning depends on culturally-determined perspectives.

\begin{quote}
    In using the prepositions, we are concerned not so much with objective measurements, i.e., with the actual dimensions of the things to which we are referring, as with how we imagine them to be at the time of speaking. \citep{close1992}
\end{quote}

Consider \textit{at} meaning `something or someone's exact point'. We use this \textit{at} for the following:

\begin{itemize}
    \item clock times: \textit{at 12:01} but not \textit{at January 8}
    \item weekends in the UK: \textit{at the weekend} (with the stress on \textit{end}), but not in North America, where we say \textit{on the weekend} (with the stress on \textit{week})
    \item nights, usually without a determiner: \textit{at night}, but not *\textit{at morning}
    \item meals, also without a determiner: \textit{at dinner}, but not *\textit{at rice}
    \item ages: \textit{at the age of 27}, but not *\textit{at 2023}
\end{itemize}

While these patterns appear idiomatic in modern English~-- with preposition selection often depending on syntactic factors like determiner presence rather than clear conceptual mapping~-- their historical development likely reflects cognitive metaphors of time as space. The original metaphorical extension of spatial \textit{at} to temporal domains suggests early English speakers conceptualized certain temporal references (clock times, nights, meals) as points. This cognitive foundation helps explain why these particular temporal expressions crystalized with \textit{at}, even though modern speakers may no longer actively access this metaphorical mapping.

The UK/US weekend distinction (\textit{at}/\textit{on}) particularly illustrates this process: while we might reconstruct the UK usage as reflecting a conceptualization of the weekend as the endpoint of a linear week, and the US usage as treating the weekend as a platform for activities, modern speakers likely use these prepositions conventionally rather than through active conceptual mapping.

Other things that are (at least sometimes) based on the metaphor of surfaces include: days (\textit{on Monday}, \textit{on the 12th}, \textit{on your birthday}), specified weeks (\textit{on the week of January 5th}), and mornings, afternoons, and evenings of specified days (\textit{on Monday morning}, \textit{on the evening of January 5th}). In contrast, we conceptualize the following things as containers and use \textit{in} for them: mornings, afternoons, and evenings generally (\textit{in the morning}), months and seasons (\textit{in January}, \textit{in }(\textit{the})\textit{ summer}), and years (\textit{in 2023}). There is no rhyme or reason to these choices. They are just what we, the community of English speakers or some part of it, have chosen. 

The same confusion applies to physical places. We can arrive \textit{at} destination cities for airline flights, transit stations and stops, places where we engage in regular activities such as school or work, street addresses, beginnings and ends, tops and bottoms, and levels and stages. In contrast, \textit{on} gets used for streets and transit lines, sides and borders, tops (without a determiner; contrast \textit{on top} vs \textit{at the top}), trips, journeys, and vacations. Where's the logic?

We think of provinces, states, and countries as containers, and so we use \textit{in}. The same holds for destination cities for airline flights (even though we sometimes think of them as points), for individual buildings and rooms, and for doors, windows, and mirrors.

Media also gets pulled in to this unsystematic system. Shows are \textit{on TV}, \textit{on the radio}, and \textit{on the internet}. We hear interesting ideas \textit{on podcasts}. Text appears \textit{on pages and screens}, and your photos are saved \textit{on your computer or your phone}. But we see moving performances \textit{in movies, programs, and series}. We learn about the world \textit{in books, magazines, newspapers, and emails}, and our images appear \textit{in photographs and paintings}.

At the risk of dragging this \uline{on} much longer than any reasonable person would want, we contact people \uline{at} their phone number or email address, follow them \uline{at} a distance, and look at them \uline{at} an angle. And we do things \uline{at} certain rates, frequencies, and prices. Are these all clearly points? When we are under medical care, we may need to be \uline{on} medication. When we drive our cars run \uline{on} gasoline, diesel, or electricity. When we're relevant, we're \uline{on} topic, and when we're organized, we're \uline{on} schedule. Are these all clearly surfaces? 

Anyone learning English would struggle to find systematicity in this hodgepodge. And the list goes \uline{on} and \uline{on}.

\subsubsection*{\textit{In} in English and Italian}

Even the core meanings of a central prepositions may differ across languages. The preposition \textit{in} in English and its Italian counterpart (also \textit{in}) are used extensively in each language and share a core `container' meaning, `enclosed or surrounded by something'.

\ea \label{ex:in-containment}
    \ea Physical location: \textit{in the house}/\textit{in casa}
    \ex Temporal context: \textit{in the morning}/\textit{in mattinata}
    \ex Abstract space: \textit{in linguistics}/\textit{in linguistica}
    \z
\z

But English \textit{in} has senses that are not shared with Italian. For instance, larger locations like cities are usually \textit{in} in English but \textit{a} (`at') in Italian. English also uses \textit{in} for smaller vehicles like cars, where Italian would use \textit{su} (`on'), as in buses or boats for English. Conversely, Italian prefers \textit{in} for large vehicles that would normally attract \textit{on} in English. These are issues of metaphor selection.

A more fundamental difference appears in (\ref{ex:in-vs-to}), showing the Italian use of \textit{in} for movement over a path to a point.

\ea Movement towards a place: \textit{I'm going to the bank.}/\textit{Vado in banca.}\label{ex:in-vs-to}
\z

\subsubsection*{Salience}
To make matters worse, prepositions aren't very salient: they tend to be small words, often pronounced without stress. As a result, in both print and speech, they get overlooked.

\begin{itemize}
    \item   \textit{Even \uline{if} I spend 10 hours, I can't fix it.} $\rightarrow$ \textit{Even I spend 10 hours, I can't fix it.}\footnote{Although this conditional  \textit{if} is a preposition, there is also a subordinator \textit{if} used in certain interrogatives clauses such as \textit{I wonder if she lives here.} See Chapter \ref{ch:subord-coord}.}
    \item \textit{That depends \uline{on} your ability.} $\rightarrow$ \textit{That depends your ability.}
\end{itemize}

\subsection{Phrasal verbs}

One area that illustrates the difficulties of preposition choice and meaning is phrasal verbs.

\textsc{Phrasal verbs} include examples like \textit{break up}, \textit{run into}, \textit{give away}, \textit{take off}, \textit{put on}, \textit{look up}, \textit{come across}, and \textit{turn down}.\footnote{\textit{CGEL} \citep{Huddleston2002} eschews the term \textsc{phrasal verbs} in favour of \textsc{prepositional verbs}.} These constructions consist of a verb combined with a preposition, and together they often take on a meaning different from the individual words.

To be clear, they're often not \textsc{phrases} in our technical sense. Nevertheless, the term \textsc{phrasal verb} is so widely used in language teaching, that I'll use it here too. 

\subsubsection*{Characteristics of Phrasal Verbs}

\begin{itemize}
    \item The meaning of the phrasal verb often cannot be deduced from the meanings of its individual parts.
    \item The preposition can often be separated from the verb, especially when the object is a pronoun: \textit{take \uline{them} off} vs. \textit{take off your shoes}.
    \item Phrasal verbs can be transitive (requiring an object) or intransitive (not requiring an object).
\end{itemize}

\subsubsection*{Idiomatic meanings}
Some phrasal verbs have meanings that are generally clear to anyone who knows the basic meanings of the verb and the associated preposition. Consider \textit{look at something}: there's a thing, it's at some point, and you have your gaze trained on that point. There's no clear reason why you shouldn't just say \textit{look it}, but if someone says \textit{look at it}, the meaning is not hard to perceive. 

On the other hand, if a plane takes off, it leaves the ground, which has very little to do with taking. If you come across something, you find it by chance, and you probably didn't move towards or across it. Carrying out a plan typically involves no carrying, and, while looking up a word often involves looking, it's hard to see how \textit{look} + \textit{up} becomes `check in a reference source'.

That said, many of these phrasal verbs need not take up much of your attention because most are actually not all that frequent. They may feel quite common, but if you check how often they appear in a relevant collection of text, you'll see that very few individual phrasal verbs are particularly common. For more on word frequency generally, see Section \ref{sec:word-frequency}.

\subsubsection*{Movable particles}

One aspect of objects in VP is that nothing can typically come between the head verb and the object, as shown in (\ref{ex:quicklycookies}).

\ea
    \ea[]{\textit{\myuline{Quickly}, they ate the cookies.}}\label{ex:quicklycookiesa}
    \ex[]{\textit{They \myuline{quickly} ate the cookies.}}\label{ex:quicklycookiesb}
    \ex[*]{\textit{They ate \myuline{quickly} the cookies.}}\label{ex:quicklycookiesc}
    \ex[]{\textit{They ate the cookies \myuline{quickly}.}}\label{ex:quicklycookiesd}
    \z\label{ex:quicklycookies}
\z

\noindent
This is true when the intervening constituent  is a PP, as in  (\ref{ex:morningcookies}), where (\ref{ex:morningcookiesb}) is, if anything, worse than (\ref{ex:quicklycookiesc}).

\ea
    \ea[]{\textit{\myuline{In the morning}, they ate the cookies.}}\label{ex:morningcookiesa}
    \ex[*]{\textit{They ate \myuline{in the morning} the cookies.}}\label{ex:morningcookiesb}
    \ex[]{\textit{They ate the cookies \myuline{in the morning}.}}\label{ex:morningcookiesd}
    \z\label{ex:morningcookies}
\z

But there are quite a few phrasal verbs that either allow or require that a PP occur after the head but before the object. These PPs are usually among the most common prepositions like \textit{in}, \textit{on}, \textit{up}, etc. and they usually appear without any complement, as in (\ref{ex:pickup}). These PPs function as \textsc{particles} in the VP. The examples with the particle after the object are in (\ref{ex:pickupa}), and the examples with the particle before the object are in (\ref{ex:pickupb}).

\begin{multicols}{2}
\ea \label{ex:pickup}
    \ea \label{ex:pickupa}
        \ea[]{\textit{Pick the pen \myuline{up}.}}
        \ex[]{\textit{Put the pen \myuline{down}.}}
        \ex[]{\textit{Get the pen \myuline{out}.}}
        \ex[]{\textit{Push the pen \myuline{in}.}}
    \z
    \ex \label{ex:pickupb}
    \ea[]{\textit{Pick \myuline{up} the pen.}}
        \ex[]{\textit{Put \myuline{down} the pen.}}
        \ex[]{\textit{Get \myuline{out} the pen.}}
        \ex[]{\textit{Push \myuline{in} the pen.}}
        \z
    \z
\z
\end{multicols}

Interestingly, when the object is a pronoun, it can't typically undergo this alternation, as shown in (\ref{ex:pickit}).\footnote{This is another categorial test for pronouns.}

\begin{multicols}{2}
\ea \label{ex:pickit}
    \ea
        \ea[]{\textit{Pick it up.}}
        \ex[]{\textit{Put it down.}}
        \ex[]{\textit{Get it out.}}
        \ex[]{\textit{Push it in.}}
    \z
    \ex
    \ea[*]{\textit{Pick up it.}}
        \ex[*]{\textit{Put down it.}}
        \ex[*]{\textit{Get out it.}}
        \ex[*]{\textit{Push in it.}}
        \z
    \z
\z
\end{multicols}

The examples in (\ref{ex:chairoff}) are superficially similar to those in (\ref{ex:pickup}, \ref{ex:pickit}, \& \ref{ex:takeoff}), but \textit{off} functions as the head of the Mod:PP \textit{off your chair}, in this case, and not as a particle. That's why the preposition and the NP can't be reversed, as shown in (\ref{ex:chairoffb}).

\begin{multicols}{2}
\ea
    \ea \label{ex:takeoff}
        \ea[]{\textit{Take off your jacket.}}
        \ex[]{\textit{Take your jacket off.}}
    \z
    \ex \label{ex:chairoff}
    \ea[]{\textit{Fall }[\textit{off your chair}].}
        \ex[*]{\textit{Fall your chair off.}}\label{ex:chairoffb}
        \z
    \z
\z
\end{multicols}

Notice that you can only fall off your chair because you're on it, but you're not on your jacket, so \textit{off your jacket} is not a phrase in (\ref{ex:chairoff}).

\subsection{Fronting and stranding} \label{sec:preposition-stranding}

In Chapter \ref{ch:questions}, we will consider \textsc{interrogative} clauses, and \textsc{relative} clauses are covered in Chapter \ref{ch:relatives}. It's hard to clearly address preposition \textsc{fronting} and \textsc{stranding} until you have at least some idea about these two structures, but fronting and stranding are too important to be left out of this chapter entirely.

\textsc{Fronting} refers to the process by which elements, often PPs or their objects, are moved to the beginning of a clause, while \textsc{stranding} is where part of the phrase is left behind. This is commonly seen in questions and relative clauses. For example, consider (\ref{ex:fronting}).

\ea \label{ex:fronting}
    \ea[]{\textit{You disagree }[\textit{about what}]\textit{?}\hfill [in situ]}\label{ex:frontinga}
    \ex[]{\textit{\uline{What} do you disagree }[\textit{about} \uline{\phantom{what}} ]\textit{?}\hfill [fronting with stranding]}\label{ex:frontingb}
    \ex[]{[\textit{\uline{About what}}]\textit{ do you disagree \uline{\phantom{about what}}?}\hfill [fronting]}\label{ex:frontingc}
    \z
\z

(\ref{ex:frontinga}) is how you would ask this if you hadn't heard clearly, were asking a clarifying question, or if you were expressing surprise: the full \textit{about what} PP appears in the VP after the head verb. In (\ref{ex:frontingb}), which is the normal way to ask the question, the NP \textit{what} has been \textsc{fronted} and the preposition \textit{about} is \textsc{stranded}. In (\ref{ex:frontingc}), which is a very formal way of asking the question, the whole PP is fronted.

The structure of these phrases can be difficult to grasp when just written out in text, as in this chapter. That's why it's often useful to show them in a graphic format that linguists call \textsc{syntax trees} or just \textsc{trees}. Such trees are the focus of the next chapter.

\section{Summary}

This chapter explored the characteristics and functions of adverbs and prepositions:

\begin{itemize}
    \item \textit{Adverbs} typically denote manner, frequency, and degree. They often end in \textit{--ly} and can be modified by \textit{very}.
    \item \textit{Adverb phrases} (AdvPs) usually function as modifiers but rarely as complements.
    \item \textit{Prepositions} primarily denote spatial, temporal, and other relations. They head preposition phrases (PPs) which can function as both modifiers and complements.
    \item Many words traditionally classified as adverbs or ``conjunctions'' (e.g., \textit{before}, \textit{after}, \textit{while}) are actually prepositions.
    \item Prepositions can take various complements, not just noun phrases, challenging the traditional view.
    \item \textit{Phrasal verbs} combine verbs with prepositions to create idiomatic meanings.
    \item Preposition choice and usage present significant challenges for English learners due to their often arbitrary nature and cultural determination.
\end{itemize}

The complexities of adverb and preposition usage in English often defy simple explanations. Teachers should be prepared to address learners' questions about these aspects of grammar without oversimplifying or overpromising easy solutions.

Nevertheless, presenting prepositions as taking a wide variety of complements, like verbs, will go a long way to eliminating most of the ``exceptions'' found in the traditional analysis and simplify the treatment of modification and complementation in clauses and other phrases.

\newpage

\begin{tcolorbox}[title=Self-check, colback=white]

\begin{enumerate}
    \item Identify the preposition in the sentence: \textit{The cat sat under the table.}
    \item Which of the following is a preposition? (a) \textit{over} (b) \textit{loudly} (c) \textit{sing}
    \item Supply an appropriate preposition: 
    \begin{itemize}
        \item \textit{She lives \uline{~~~~~} the city.}
    \item \textit{She graduated \uline{~~~~~} Harvard U.}
    \item \textit{He arrived \uline{~~~~~} 3 o'clock.}
    \item \textit{She is interested \uline{~~~~~} learning languages.}
    \item \textit{She is proficient \uline{~~~~~} English.}
    \item \textit{She is fond~\uline{~~~~~} chocolate.}
    \item \textit{She is good \uline{~~~~~} mathematics.}
    \item \textit{She relies \uline{~~~~~} her friends for support.}
    \item \textit{She is allergic \uline{~~~~~} peanuts.}
    \item \textit{She is scared \uline{~~~~~} spiders.}
    \item \textit{She is different~\uline{~~~~~} her sister.}
    \item \textit{She is married \uline{~~~~~} him.}
    \item Identify the adverb in the sentence: \textit{She sang happily at the concert.}
    \end{itemize}
    \item Which of the following is an adverb? (a) \textit{blue} (b) \textit{quickly} (c) \textit{apple}
    \item Supply an appropriate adverb: \textit{She completed the task \uline{~~~~~}.}
\end{enumerate}
\end{tcolorbox}

\begin{tcolorbox}[title=Answer key, colback=white]
    \begin{tabular}{lll}
1. under & 2. (a) over & 3. in \\
4. from & 5. at & 6. (a) from \\
7. in & 8. in & 9. of \\
10. at & 11. on & 12. to \\
13. of & 14. from & 15. to \\
16. happily & 17. (b) quickly & 18. quickly \\
19. (b) quickly & 20. beautifully & \\
\end{tabular}

\end{tcolorbox}