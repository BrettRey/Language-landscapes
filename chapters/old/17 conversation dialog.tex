\chapter{Conversations} \label{ch:conversations}

\epigraph{Words bounce. Words, if you let them, will do what they want to do and what they have to do.}{}

\begin{adjustwidth}{0pt}{1.1in} 
\setlength{\marginparwidth}{1.6in}
\setlength{\marginparsep}{-0.8in}
\noindent
Setting: Jim's sitting in the staff lounge, where his colleague Nao sees him.

\begin{sloppypar}
\begin{dialogue}
\item[Nao] Oh sorry, am I \annotate{interrupting}{Turn taking (\ref{sec:turn-arch})} something?

\item[Jim] No no, just~-- [closes laptop] I was reviewing this chapter on conversation, \annotate{actually}{Discourse marker (\ref{sssec:topic-man})}. \\\annotate{Kind of}{Hedging (\ref{subsec:face})} perfect timing, huh?

\item[Nao] [leans forward] \annotate{Oh?}{Common ground (\ref{sec:common-ground})} What kind of~-- 

\item[Jim] \hfill\annotate{It's for}{Overlap (\ref{sec:turn-arch})} a textbook I'm\\ writing. But I'm stuck because... [sighs,\annotate{gestures}{Embodied action (\ref{sec:bodies})} vaguely] it feels too abstract, \annotate{you know}{Alignment (\ref{sec:common-ground})}? \\\annotate{Like}{Example preface (\ref{sssec:topic-man})}, all formal terms and~-- 

\item[Nao] \hspace{3cm} \annotate{And you're not actually showing how it works}{Collaborative completion (\ref{subsec:grammar-interaction})}.\\

\item[Jim] \annotate{Exactly}{Alignment (\ref{sec:common-ground})}! \\\annotate{Like}{Topic shift (\ref{sssec:topic-man})} all this stuff about \textsc{turn-taking} and \textsc{repair strategies} and~-- [\annotate{points at laptop}{Deictic gesture (\ref{sec:bodies})}]

\item[Nao] [\annotate{nodding}{Back-channel (\ref{sec:common-ground})}] \\\annotate{Mm-hmm}{Common ground (\ref{sec:common-ground})}. The old ``show-don't-tell'' \annotate{problem}{Metacomment (\ref{subsec:knowledge})}.\\

\item[Jim] \annotate{Right?}{Checking understanding (\ref{sec:common-ground})} \\\annotate{I mean}{Reformulating (\ref{sec:common-ground})}... \\\\\annotate{Wait}{Side-sequence opener (\ref{subsec:side-seq})}. What if we just had a conversation about conversation? \annotate{Like}{Clarifying (\ref{sssec:topic-man})}, demonstrate the moves while talking?

\item[Nao] [tilts head] You mean a... \annotate{meta-conversation}{Metacomment (\ref{sssec:topic-man})}?
\end{dialogue}
\end{sloppypar}
\end{adjustwidth}

%%%%%%%%%%%%%%%%%%%%page break

\begin{sloppypar}
\begin{dialogue}
\item[Jim] \uline{Yeah!} Look what just happened: you finished my sentence~-- that's \uline{collaborative completion}. And when I \uline{paused} before ``like''? That's~-- 

\item[Nao] [raises eyebrows] \uline{Turn-yielding}? [\uline{sips coffee}]

\item[Jim] Yes! And see how we're both using these physical cues~-- 

\item[Nao] The embodied stuff? [mimics his pointing] Gestures, posture~-- 

\item[Jim] Exactly. Plus the confirmations that build \uline{common ground}...

\item[Nao] Speaking of which~-- [\uline{glances at his cup}] your coffee's probably cold.

\item[Jim] Oh! True. I'll get a fresh one. Need anything?

\item[Nao] I'm good. But~-- record this when you get back.

\item[Jim] Us?

\item[Nao] Sure. We're enacting every concept: turn-taking, repair, body language...

\item[Jim] And shared understanding~-- like you reading the cold coffee cue.

\item[Nao] Context plus observation. You haven't touched it since I sat down.

\item[Jim] Mind if I...? [holds up phone]

\item[Nao] Go ahead. Oh, and note how different text chat is~-- handling overlaps, gestures...

\item[Jim] \uline{In digital space?} That's a whole section~-- [\uline{adjusts phone}]

\item[Nao] Recording?

\item[Jim] Yep. So, digital platforms: typing indicators as \uline{online back-channels}?

\item[Nao] Right. They replace nods and \textit{mm-hmms}. 

[A crash at the counter; both look.]

\item[Jim] Ouch. That sounded expensive.

\item[Nao] Perfect example of \uline{external interruption}.

\item[Jim] And how we re-enter the talk~-- gestures anchoring attention.

\item[Nao] We'll be unstoppable, analysing everything.

\item[Jim] Occupational hazard. Ready to help me turn this into the textbook example?

\item[Nao] As long as my coffee stays hot.

\item[Jim] Unlike mine. Some patterns are hard to break.
\end{dialogue}

\end{sloppypar}

\section{Introduction: The Hidden Life of Conversation} \label{sec:intro}

Take a moment to consider your last conversation. Perhaps it was with a colleague about lesson plans, a friend about weekend plans, or a stranger about the weather. On the surface, these interactions might seem straightforward~-- just people exchanging information or maintaining social bonds. But have you ever noticed how a simple ``how are you?" can lead to anything from a quick ``fine" to a twenty-minute discussion of life's complexities? Or how the same words can mean entirely different things depending on who says them, when, and how?

Conversation is deeply woven into the fabric of human life, yet like fish in water, we rarely notice its true nature. We talk about the ``rules of conversation" as if they were like the rules of chess, clearly defined and deliberately followed. But real conversation is more like a flock of birds wheeling through the sky~-- complex patterns emerge not from a rulebook, but from countless small adjustments between individuals.

Think about how we learn to converse. Nobody sits down with a toddler and explains turn-taking mechanics~-- how we know when it's time our turn to speak~-- or repair strategies. Instead, these patterns emerge naturally through interaction. But somehow, across vastly different cultures and contexts, conversations show remarkably similar structural properties. It's as if we're all playing jazz without ever having learned music theory.

This self-organizing nature of conversation challenges our common assumptions. We often think of conversation as primarily about exchanging information or maintaining social relationships. While these functions are important, they barely scratch the surface. When we talk, we're not just sharing facts or being friendly~-- we're actively constructing reality together, maintaining our identities, and coordinating our social world in real-time.

Consider what happens when someone asks ``What did you do yesterday?" This isn't just a request for information. It's an invitation to construct a narrative, to present a particular version of yourself, to establish common ground, and perhaps to build or maintain a relationship. The answer might involve not just words but gestures, facial expressions, and subtle adjustments based on the listener's responses. All of this happens in split-second timing, usually without conscious thought.

The complexity deepens when we consider how conversations work across different contexts. A business meeting, a family dinner, and a chat between friends might all follow different ``rules'', but they're all recognizably conversation. Even more intriguingly, these patterns emerge similarly across languages and cultures, though with many local variations.

This seeming paradox~-- that universal patterns emerge from local interactions without central control~-- is at the heart of understanding conversation. It's why prescriptive ``rules of conversation" often miss the point. Real conversation isn't about following rules; it's about participating in a dynamic, collaborative process of meaning-making.

In this chapter, we'll consider this hidden life of conversation, looking at how turns are constructed and negotiated, how understanding is built and repaired, how bodies participate in talk, and how relationships are maintained through talk. We'll see how digital technologies are reshaping these patterns, and we'll consider what all this means for language teaching.

\section{Beyond Turn-Taking} \label{sec:beyond-turns}

Most attempts to explain conversation start with a simple observation: people take turns talking. But this observation leads many teachers and textbooks astray. They present conversation as a set of rules about when to speak and when to stay quiet, as if talk were a kind of verbal traffic light. This misses something fundamental about how conversation works.

\subsection{Timescales and Structure} \label{subsec:timescales}

When we look closely at recordings of natural conversation, we find that talk is organized simultaneously on multiple timescales \citep{Schegloff1977,Fauviaux2023}. At the finest grain, we have micro-adjustments in timing~-- tiny pauses, overlaps, and speed-ups that last fractions of a second. A speaker might speed up to hold the floor or pause briefly to invite response. These adjustments are far from random. As Schegloff demonstrates in his work on conversation analysis, they are a fundamental part of how interaction is organized, reflecting participants' attentiveness to the ongoing flow of talk \citep{Schegloff1982}.

But these micro-patterns nest inside larger structures. A single turn at talk might last several seconds and contain multiple complete sentences~-- or it might be as brief as a grunt of acknowledgement. These turns themselves cluster into recognizable sequences: a question followed by an answer, a greeting followed by a return greeting, an invitation followed by acceptance or declination \citep{Sacks1973}.

Moving up another level, we find stretches of talk that appear to hang together around particular activities or concerns. These aren't like chapters in a book with clear boundaries. Instead, they emerge, develop, and shade into one another through various practices that participants use. A speaker might link to an earlier thread of conversation (\textit{speaking of cats...}), transform the current activity (\textit{that reminds me...}), or initiate a completely new direction (\textit{oh, before I forget...}).

At an even broader scale, conversations themselves occur within recognizable social situations~-- catching up with an old friend, a job interview, a parent--teacher conference. These situations shape participants' expectations about appropriate behaviour, turn length, and level of formality. They also influence how people interpret ambiguous utterances. The phrase \textit{that's interesting} means something quite different in a job interview than in casual conversation with a friend.

This layered organization creates a kind of ordered complexity. At each level, patterns emerge from the interaction of simpler elements below. 

\subsubsection{Topic management}\label{sssec:topic-man}

A topic isn't just a collection of turns, and a conversation isn't just a collection of topics. Each level has its own organizational principles that emerge from~-- but can't be reduced to~-- the levels below.

For language teachers, this multi-level organization has important implications. When we teach conversation, we often focus on just one level~-- usually turns and simple sequences. We might drill question-answer pairs or teach turn-taking signals. But real conversational competence requires understanding how these levels interact. A student might master the mechanics of turn-taking but still struggle with topic management or miss activity-type cues that signal how formal or casual they should be.

Looking at actual recordings of conversation reveals how these levels interact. Consider this excerpt from the British National Corpus (\citeyear{BNC_KC8}):

\ea
\begin{dialogue}
\item[Gillian] I think it's just Friday night and he's... yeah, basically... I don't think Billy's telling him enough [cough] to be honest.

\item[Robert] Why?

\item[Gillian] I just don't.

\item[Robert] [yawn] What, he knows something's happened. But he doesn't know what it is...

\item[Gillian] He doesn't seem to be involved in that many... discussions about the company does he? [yawn] Or what's happening. I mean he was moaning about erm or those trade accounts wasn't he?

\item[Robert] Mm... well it's not for me to say. I don't know anything more about this business than he does.

\item[Gillian] Oh no.

\item[Robert] [...] go as far as to say we probably know less about what's going on than he does.

\item[Gillian] Do we? Oh. Such is life... \\I had a word with Steph on the phone.

\item[Robert] Oh?

\item[Gillian] She's getting fat... [...]

\item[Robert] [...]

\item[Gillian] Yeah. She goes to hospital on Thursday. Eighteen week appointment.
\end{dialogue}
\z

This brief exchange shows multiple organizational levels at work. At the micro level, we see the precise timing of turns marked by coughs, yawns, and brief responses like \textit{Oh} and \textit{Mm}. At the sequence level, we see question-answer pairs (like \textit{Why?} followed by \textit{I just don't}). At the topic level, we see how the conversation shifts from a discussion about Billy's situation at work to Steph's pregnancy through a clear topic transition (\textit{Such is life... I had a word with Steph on the phone}). And at the activity level, we see how the casual familial discussion enables Gillian and Robert to cover both workplace concerns and intimate personal matters, as well as comfortable topic transitions.

Compare this with how the same basic content might appear in a textbook dialogue:

\ea
\begin{dialogue}
\item[Robert] How is Billy doing at work?

\item[Gillian] I'm worried about him. I don't think he's getting enough information from his manager.

\item[Robert] Really? Why do you think that?

\item[Gillian] Well, he's not included in many important meetings about the company. For example, he was complaining about the trade accounts last week.

\item[Robert] I see. But I don't really know much about the situation. In fact, I probably know less than he does.

\item[Gillian] Oh, I understand. Let's change the subject. I spoke to Stephanie yesterday.

\item[Robert] Oh? How is she?

\item[Gillian] She's doing well. She's pregnant, you know. She has her 18-week doctor's appointment on Thursday.

\item[Robert] That's wonderful news!
\end{dialogue}
\z

The contrast between these two versions of essentially the same conversation reveals much about the gap between idealized and authentic interaction. While the textbook dialogue might appear clearer and easier to follow, it lacks many features that make real conversation work: the subtle timing adjustments marked by coughs and yawns, the collaborative construction of meaning through incomplete utterances, and the organic flow of topic transitions. Notice how in the authentic conversation, Gillian's topic shift to Steph emerges naturally through \textit{Such is life...}, while in the textbook dialogue, the transition is explicitly flagged with \textit{Let's change the subject}. Similarly, the textbook version smooths out the hesitations, repairs, and incremental additions that characterize authentic speech, replacing them with complete, well-formed sentences, adding explanatory context that real participants don't need~-- like labeling the subject change or explicitly stating relationships that are already clear to the speakers.

Think of smoothing out a dirt path into a paved road. The dirt path naturally formed where people actually wanted to walk, with gentle curves around obstacles and slopes that matched the terrain. The ``improved'' paved version, while cleaner and more geometric, often forces unnatural angles and ignores the wisdom embedded in those organic curves~-- just like how textbook dialogues iron out the natural rhythms that make real conversations work. These natural rhythms, as \citet{wilson2005} found, create waves of coordinated timing, from split-second adjustments to longer cycles of engagement and disengagement. When these organic patterns align, conversation flows naturally, like walking a well-worn path. When they don't, we get that uncomfortable feeling of being `out of step'~-- like trying to navigate an artificially straightened route that fights against the natural contours of human interaction.

\subsection{Cultural Organization of Talk} \label{subsec:cultural-org}

Many language teachers emphasize what they call ``turn-taking rules'', often presenting these as universal principles. They tell students to wait for a pause before speaking, to avoid interrupting, and to give clear signals when yielding the floor. While well-intentioned, this advice reflects just one cultural model of how talk should be organized~-- typically a middle-class, Western European or North American one.

Looking at conversations across cultures reveals a much richer picture. In many Caribbean communities, for instance, what \citet{reisman1974} calls ``contrapuntal conversation'' is common~-- multiple people speaking simultaneously, weaving their voices together like instruments in an ensemble. From a Western perspective, this might look like chaos or rudeness. But participants experience it as a highly coordinated form of interaction, one that creates involvement and builds solidarity.

Similar patterns appear in other cultural contexts. Consider this excerpt from Terry Gross' interview with Donna Summer on \textit{Fresh Air}:

\ea
\begin{dialogue}
\item[Donna SUMMER] \textit{So I always felt like I had this sort of~-- people to answer to and my children, and my child at the time, was one of them. And I felt that in the future I didn't want her to say, ``Mom, well, you did it.'' You know? But, you know\dots} 
\item[GROSS] \phantom{~~~~~~~~~~~}\textit{Did she ever say that?} 
\item[SUMMER] \textit{Yeah, she did. Yes, she did, much\dots}
\item[GROSS] \phantom{~~~~~~~~~~~~~~~~~~~~~~}\textit{What was your comeback?} 
\item[SUMMER] \phantom{Yeah, she did. Yes, she did, much}\dots\textit{to my chagrin.\\I just told her it was a different time, and, you know, I came from a totally different life than her.}
\end{dialogue}
\z

This interaction shows several cultural patterns at work. Gross, from a New-York Jewish background where \citet{tannen1984} documents a ``high-involvement style'', uses quick follow-up questions that overlap with Summer's speech. Rather than appearing rude, these overlaps signal engagement and help maintain conversational momentum. Summer, in turn, uses tag questions like \textit{you know?} to check for understanding and manage turn-taking. The overall effect is of a coordinated dance between interviewer and interviewee, each adapting to the other's conversational style.

These differences extend beyond timing patterns. Consider how Korean speakers often ask about the age of their interlocutor to decide whether to use honorific verb forms that encode complex social relationships, or how Japanese conversation involves frequent use of back-channel responses like \textit{hai} and \textit{ee} that signal attention and understanding \citep{maynard1986}. Even within English-speaking communities, we find significant geographic, cultural, and situational variation in conversational practices and expectations.

The implications for language teaching go beyond simple awareness of difference. When students from one conversational culture enter another, they face what \citet{gumperz1982} calls ``crosstalk''~-- systematic mismatches in conversational expectations that can lead to serious misunderstandings. A student whose first language treats overlapping speech as normal might be seen as ``pushy'' or ``rude'' by English speakers who expect sequential turn-taking. Conversely, a student from a culture that values longer silences between turns might appear ``unengaged'' or ``slow to respond'' in an certain contexts.

While it would be unrealistic to attempt to learn what conversational style to adopt in each situation, it can be very useful to be made aware that such differences exist. This awareness can help students navigate unfamiliar interactional patterns with more confidence and less anxiety. Rather than trying to master every local convention, they can develop a more flexible approach, understanding that when conversation feels awkward or uncomfortable, the issue may be cultural differences in conversational style rather than personal failings or language deficiency. For teachers, this means spending less time prescribing ``correct" conversational behaviour and more time helping students recognize and reflect on different patterns of interaction they encounter.

\subsection{Grammar in Interaction} \label{subsec:grammar-interaction}

Most discussions of grammar treat it as something that exists primarily in sentences~-- isolated, written units of language constructed with careful attention to rules. But real-time interaction reveals grammar doing different kinds of work. Consider these recorded exchanges from \citet{lerner1991}:

\ea
   \ea
   \begin{dialogue}
       \item[Rich] \textit{if you bring it intuh them}
       \item[Carol] \textit{ih don't cost yuh nothing}
   \end{dialogue}
   \ex
   \begin{dialogue}
       \item[David] \textit{so if one person said he couldn't invest}
       \item[Kerry] \textit{then I'd have ta wait}
   \end{dialogue}
   \ex
   \begin{dialogue}
       \item[Dan] \textit{when the group reconvenes in two weeks}
       \item[Roger] \textit{they're gunna issue strait jackets}
   \end{dialogue}
   \z
\z

In each case, speakers collaborate to build a complex sentence across turns. The first speaker produces a dependent clause (\textit{if...} or \textit{when...}) that projects a particular grammatical shape for what must follow. The second speaker then completes the structure, often starting with words like \textit{then} that mark the completion of the compound structure. As \citet{lerner1991} demonstrates, these `compound turn-constructional units' aren't accidents~-- they're systematic practices that speakers use to coordinate their contributions.

The grammar here isn't just organizing words into sentences~-- it's providing resources for real-time collaboration. Certain grammatical structures, like \textit{if-then} constructions, create natural pivot points where another speaker can jump in to complete the emerging utterance. We find similar patterns with relative clauses (\textit{the guy who...}), reported speech (\textit{she told me that...}), and list structures (\textit{first...second...}). These patterns show how deeply grammar is woven into the fabric of real-time interaction.

We find similar patterns with other grammatical structures, as in this example from \citet{szczepek2000}. 

\ea 
\begin{dialogue}
    \item[CO] \textit{and people \textsc{also};} \\
    \textit{who've never been \textsc{close} friends of hers;} \\
    \textit{but who'd \textsc{bend over backwards};} \\
    \textit{for this \textsc{woman.}}
    \item[AL] \textit{but are \textsc{tired of bending over backwards.}}
    \item[CO] but they \textsc{\textit{still do.}} \\
    \textit{we \textsc{all still do.}}
\end{dialogue}
\z

This exchange shows how grammar supports the collaborative development of a complex idea. After CO completes what could be a finished turn~-- it makes a complete point, forms a complete syntactic unit, and ends with a falling pitch and pause~-- AL continues the thought. Though AL's addition lacks its own subject, it fits smoothly into the established pattern. The grammar here provides a framework for speakers to build meaning together: CO sets up a structure that AL can extend, and then CO acknowledges and elaborates on this extension, moving from \textit{they} to the more inclusive \textit{we}.

Grammar also helps participants manage understanding in real time. When we examine recordings of natural conversation, we find speakers constantly adjusting their utterances based on how others are responding. They might break a complex sentence into chunks, checking for understanding after each piece. Or they might start with a simple structure and elaborate it incrementally as needed:

\ea
\begin{dialogue}
\item[A] \textit{Pass me that thing}
\item[B] \textit{Which--}
\item[A] \textit{The blue one. On the shelf. Next to the lamp.}
\end{dialogue}
\z

Here the grammar of reference unfolds dynamically as the speakers work to establish shared understanding. Rather than producing a complete noun phrase like \textit{the blue thing on the shelf next to the lamp}, speaker A builds the reference piece by piece, responding to B's evident uncertainty.

For language teachers, this view of grammar-in-interaction has important implications. When we teach grammar only through written sentences or scripted dialogues, we miss crucial aspects of how grammar actually works in conversation. Students need opportunities to experience how grammatical resources support real-time interaction~-- how they can be used to project what's coming next, to collaborate in building meanings, and to repair problems when they arise.

\section{The Architecture of Turns} \label{sec:turn-arch}

Consider this brief exchange from my classroom:

\ea
\begin{dialogue}
\item[Teacher] \textit{What did people used to think about dinosaurs?}
\item[Student] \textit{They think they crawl like} [demonstrates with hand movement]
\item[Teacher] \textit{They thought they walked like this?} [imitates movement]
\item[Student] \textit{Yes, like, like crocodile. But now they think more like bird.}
\item[Teacher] \textit{Right, more like birds. Standing upright.}
\end{dialogue}
\z

At first glance, this might seem like a simple exchange about changing views of dinosaur posture. But look closer. Each contribution fits precisely into the flow of talk. The student's initial response links back to the question with \textit{they}, the teacher's reframing offers both past tense and confirmation, and understanding emerges through a combination of words and gesture.

When we hear someone speaking English, we can usually tell when they've finished what they're trying to say, even if we weren't paying attention to the meaning. We might be checking our phone or watching a bird out the window, but we still know when it's our turn. How do we do this?

The answer lies in how speakers construct their turns at talk. Each turn is built from recognizable pieces that conversation analysts call turn-constructional units (TCUs). A TCU might be as brief as \textit{mm-hmm} or as complex as a multi-clause sentence. What matters isn't the length but that other participants can recognize when it's complete.

Consider these responses to a party invitation:

\ea
\begin{dialogue}
\item[A] \textit{You coming tonight?}
\item[B1] \textit{Can't.}
\item[A] \textit{Oh, okay.}
\end{dialogue}
\z

\ea
\begin{dialogue}
\item[A] \textit{You coming tonight?}
\item[B2] \textit{Can't.} \\
    \textit{Got this thing with my sister.} \\
    \textit{She needs a ride to the airport.}
\end{dialogue}
\z

B1's single-word response forms a complete TCU. It answers the question and A treats it as adequate. B2's response contains three TCUs, each potentially complete on its own, but built together to give a fuller explanation. The speaker may have sensed that \textit{can't} alone would seem too abrupt, or that \textit{got this thing} needed elaboration.

TCUs often align with grammatical units, but they don't have to. In language classrooms, we often see partial units doing important interactional work:

\ea
\begin{dialogue}
\item[Teacher] \textit{What's the capital of France?}
\item[Student] \textit{Um... Par...}
\item[Teacher] \textit{Pa...}
\item[Student] \textit{Paris!}
\end{dialogue}
\z

The student's \textit{um} and incomplete \textit{Par} aren't grammatically complete, but they function as meaningful TCUs here. They show the student is working on the answer, and the teacher treats them as legitimate contributions by offering support.

This points to something crucial: turns aren't just grammatical units~-- they're tools for action. Even a grunt can be a complete TCU if it does recognizable work in context. Conversely, a perfectly formed sentence might be treated as incomplete if it leaves some projected action unfinished.

For language teachers, this means rethinking how we present spoken English. Instead of focusing only on complete, grammatical sentences, we need to help students:
\begin{itemize}[noitemsep]
    \item Recognize when others' turns are potentially complete
    \item Build turns that others can follow and project
    \item Use different types of TCUs appropriately:
        \begin{itemize}[noitemsep]
            \item Brief responses (\textit{uh-huh}, \textit{right}, \textit{okay})
            \item Partial repeats (\textit{You what?}, \textit{On Tuesday?})
            \item Incremental additions (\textit{In Toronto. Last summer.})
            \item Collaborative completions (\textit{A: It was like... B: A disaster?})
        \end{itemize}
    \item Fit their turns to sequential contexts
\end{itemize}

Understanding turn architecture helps explain why some learners struggle with conversation despite good grammar. They might build turns that native speakers find hard to project, leading to awkward interruptions or long silences. Think about a student who responds to \textit{How are you?} with \textit{I am feeling quite well today because the weather is pleasant and I completed my assignment early}. The grammar is perfect, but the turn is longer and more complex than English speakers expect in this context. A simple \textit{Good, thanks} would better fit the sequential environment.

The goal isn't perfect control of turn construction~-- even native speakers have overlaps, interruptions, and awkward transitions. Rather, it's helping students understand that turns aren't just about saying things correctly~-- they're about saying things in ways that other participants can follow and respond to appropriately. This might mean starting with simple TCUs and gradually building up to more complex ones as students become more comfortable with how turns work in conversation.

\section{Building Common Ground} \label{sec:common-ground}

Watch a parent and young child looking at a picture book together. You'll often hear exchanges like this:

\ea
\begin{dialogue}
\item[Parent] \textit{What's that?} [points]
\item[Child] \textit{A buh buh.}
\item[Parent] \textit{Yes, a bird! What's the bird doing?}
\item[Child] \textit{Eee!}
\item[Parent] \textit{Eating, that's right. The bird is eating some seeds.}
\end{dialogue}
\z

This might seem far removed from adult conversation, but it illustrates something fundamental about how we communicate: understanding isn't just transferred from one mind to another~-- it's built together, piece by piece, through constant checking, confirming, and repairing when things go wrong.

Adult conversations may be more sophisticated, but they rely on similar processes. Consider this exchange between colleagues:

\ea
\begin{dialogue}
\item[A] \textit{Did you see that email from Sandra?}
\item[B] \textit{About the meeting?}
\item[A] \textit{No, the budget thing.}
\item[B] \textit{Oh right, yeah. The cuts.}
\item[A] \textit{Pretty bad, huh?}
\item[B] \textit{Twenty percent across the board.}
\item[A] \textit{And that's just the start.}
\end{dialogue}
\z

Like the parent and child, these speakers work together to establish what they're talking about. They check understanding (\textit{About the meeting?}), repair misunderstandings when they arise (\textit{No, the budget thing}), and build shared interpretations (\textit{Pretty bad, huh?}). This process of establishing and maintaining shared understanding~-- what psychologists call \textsc{common ground}~-- is central to all conversation.

When you're talking with a friend, you probably don't think consciously about how you show you're following along. You nod, make little sounds like \textit{mm-hmm}, maybe repeat key words, or ask brief questions. These aren't just social niceties~-- they're part of what \citet{clark1991} calls `grounding'~-- the moment-by-moment work of establishing mutual understanding.

Let's look at how this works in practice. Here's a recording of someone giving directions:

\ea
\begin{dialogue}
\item[A] \textit{So you go down Bank Street}
\item[B] \textit{mm-hmm}
\item[A] \textit{past the library}
\item[B] \textit{right}
\item[A] \textit{and there's this big red building}
\item[B] \textit{the old post office?}
\item[A] \textit{yeah, exactly, and just after that}
\item[B] \textit{mm-hmm}
\item[A] \textit{there's a little side street}
\end{dialogue}
\z

B isn't just being polite here. Each response shows not only that B is listening but that specific pieces of information have been understood. \textit{mm-hmm} acknowledges receipt, \textit{right} claims understanding, and \textit{the old post office?} demonstrates understanding by offering relevant additional information.

When understanding breaks down, speakers have various ways to repair the problem. Sometimes they catch their own troubles:

\ea
\begin{dialogue}
\item[Teacher] \textit{Turn to page fifty-- sorry, fifteen.}
\item[Student] \textit{Fifteen?}
\item[Teacher] \textit{Yes, fifteen.}
\end{dialogue}
\z

Other times, listeners signal trouble:

\ea
\begin{dialogue}
\item[A] \textit{We need to defenestrate the problem.}
\item[B] \textit{Defen--?}
\item[A] \textit{Oh, sorry~-- throw it out. Defenestrate means throw out a window.}
\item[B] \textit{Ah, okay. Bit formal!}
\item[A] \textit{Yeah, showing off really} [laughs]
\end{dialogue}
\z

Notice how participants work together to repair understanding. B signals trouble by partially repeating the problem word with rising intonation. A not only provides the meaning but acknowledges the word choice might have been inappropriate. B confirms understanding while gently teasing about the language choice, and A accepts this with self-deprecating humor.

For language learners, all this presents several challenges:

\begin{itemize}[noitemsep]
   \item Recognizing grounding signals in English (they may differ from their L1)
   \item Producing appropriate signals while processing unfamiliar language
   \item Identifying when repair is needed and how to initiate it
   \item Managing face concerns when signaling non-understanding
   \item Monitoring for understanding while speaking English
   \item Managing all of this across cultural differences
\end{itemize}

Think about a student who keeps nodding and saying \textit{yes} while understanding little. They've learned this can be a safe response, but it short-circuits the real work of grounding. Or consider a student who remains silent while processing, unaware that this silence might be interpreted as confusion or disengagement by English speakers who expect frequent back-channel responses.

As teachers, we can help by making these processes more explicit. We might:

\begin{itemize}[noitemsep]
   \item Practice common grounding expressions (\textit{I see}, \textit{got it}, \textit{you mean...?})
   \item Show how intonation affects meaning (\textit{mm-hmm} with rising vs. falling pitch)
   \item Demonstrate various repair strategies:
       \begin{itemize}[noitemsep]
           \item Open class repair (\textit{Sorry?}, \textit{Pardon?})
           \item Specific repair (\textit{What does X mean?}, \textit{When you say X...})
           \item Checking understanding (\textit{So you mean...}, \textit{Is it like...})
       \end{itemize}
   \item Practice monitoring for and responding to signs of trouble
   \item Work on managing repair while maintaining face
\end{itemize}

The goal isn't to make these processes mechanical~-- they should eventually become as automatic in English as they are in students' first languages. But some initial explicit attention to how understanding is built, monitored, and repaired can help students participate more effectively in English conversation while they develop this automaticity.

\section{Bodies in Conversation} \label{sec:bodies}

When we analyze conversation in textbooks or research papers, we often reduce it to words on a page. Even our recordings capture just voices. But real conversation is thoroughly embodied~-- hands gesture, heads nod, bodies lean and turn, eyes meet or avoid, faces display understanding or confusion.

On the one hand, these elements are nowhere near as important for conveying a message as some popular ideas would have you think. After all, we have very little problem understanding people on the radio, telephone, or in print. On the other hand, they aren't just add-ons to the ``real'' conversation happening in words. As \citet{goodwin2000} demonstrates, bodily actions are fundamental to how we construct and coordinate our interactions.

To see how deeply embodied conversation is, try having a discussion while keeping your body completely still~-- no nodding, no hand movements, no facial expressions, no shifts in posture. You'll likely find it remarkably difficult, and your partner will probably find it quite unsettling. Or try discussing a complex spatial arrangement (like how to rearrange furniture in a room) while keeping your hands still. Even on the phone, where our conversation partner can't see us, we tend to gesture as we speak.

The term `body language' is often used for these phenomena, but it's misleading. It suggests that bodily actions are just a kind of add-on code that translates mental states into visible signals~-- crossed arms mean defensiveness, leaning forward means interest, and so on. But actual embodied interaction is far more sophisticated. Consider this interaction recorded at a construction site \citep{needed}:

\begin{dialogue}
\item[Worker 1] \textit{So we need to [points to beam] move all these and then--} 
\item[Worker 2] \textit{Yeah, all along here [traces line with finger] up to the--}
\item[Worker 1] \textit{Right, but first we gotta [mimes lifting motion] all the old ones}
\item[Worker 2] \textit{Oh yeah, those'll have to come out}
\end{dialogue}

The verbal part of this exchange is almost incomprehensible without the bodily actions, but note how precisely coordinated the words and actions are. Worker 1's pointing gesture doesn't just happen anywhere~-- it occurs exactly when needed to specify which beam, and the lifting gesture precisely coincides with the relevant part of the utterance. These aren't random movements; they're part of a sophisticated system for building meaning in real time.

Different kinds of gestures do different kinds of work in conversation. \citet{mcneill1992} identifies several major types:
\begin{itemize}[noitemsep]
   \item Pointing (deictic) gestures, like Worker 1's beam-pointing above, connect talk to the physical environment
   \item Iconic gestures represent physical features or actions~-- like when we show the size of a fish or demonstrate how something moves
   \item Metaphoric gestures represent abstract concepts~-- watch people's hands as they discuss ideas `building on each other' or time `flowing forward'
   \item Beat gestures mark rhythm and emphasis, often synchronized with the prosodic structure of speech
\end{itemize}

The timing of these gestures is remarkably precise. When someone explains a spiral staircase by saying ``it goes up like [hand spirals upward] to the second floor,'' the spiraling gesture begins exactly with `like' and completes just as they say `up'. This precise timing reflects deep connections between speech and gesture systems. Even more intriguingly, listeners process speech and gesture as an integrated package~-- in experimental studies, people often cannot remember whether they got a particular piece of information from words or gestures \citep{needed}.

Bodies also create and manipulate conversational space. Different types of interaction create characteristic spatial arrangements. At a coffee shop, you might notice:
\begin{itemize}[noitemsep]
   \item Friends catching up: Seated at 90-degree angles, bodies oriented toward a shared space between them, often leaning in during emotional moments or key points
   \item Business meetings: Directly facing, laptops or documents creating a shared workspace that participants manage together, bodies more upright and formal
   \item Quick greetings: Standing, bodies at 45 degrees, partly oriented toward exits, maintaining what \citet{kendon1990} calls `temporary with' arrangements that signal the interaction's brief nature
   \item Service encounters: Customer and server separated by counter, maintaining clear roles through spatial positioning, but often leaning in to create momentary shared spaces for checking details or making choices
\end{itemize}

These arrangements aren't static~-- people work constantly to maintain and adjust them as interaction proceeds. Watch a group of friends walking and talking: they'll continuously adjust their pace and positioning to maintain what Kendon calls an `F-formation'~-- an arrangement that lets everyone see each other's faces and gestures easily.

The way we use our bodies in conversation varies significantly across cultures. In many Mediterranean cultures, speakers use larger gesture spaces and more frequent touch than typical in Northern Europe. Japanese conversation involves frequent head movement and facial displays of attention but relatively less gesture. Even within cultures, these patterns vary by context~-- the same person who gestures expansively while telling stories to friends might use minimal movement in a job interview.

For language teachers, understanding embodied interaction matters deeply. Our students need to navigate not just verbal norms but bodily ones. A student who maintains too much or too little eye contact, who gestures too much or too little, or who doesn't recognize culturally appropriate patterns of space and touch may face social difficulties regardless of their verbal fluency.

Teaching itself is deeply embodied. Watch experienced teachers:
\begin{itemize}[noitemsep]
   \item Their hands trace intonation patterns in the air
   \item Their bodies demonstrate grammatical relationships~-- stepping forward for future time, using hand levels to show comparative forms
   \item Their faces display confirmation or confusion to encourage self-correction
   \item Their posture and positioning manage turn-taking and participation patterns
\end{itemize}

Understanding how bodies work in conversation can help us both teach more effectively and help our students become more comfortable navigating the embodied aspects of their new language. While we can't teach a complete system of bodily behavior (nor should we try), we can help students notice and understand how bodies participate in creating meaning and managing interaction.

\section{Information and Action} \label{sec:info-action}
\subsection{Action Sequences} \label{subsec:action-seq}
\subsection{Knowledge Construction} \label{subsec:knowledge}
\subsection{Side Sequences and Corrections} \label{subsec:side-seq}

\section{Social Dimensions} \label{sec:social}
\subsection{Face and Politeness} \label{subsec:face}
\subsection{Power and Identity} \label{subsec:power}
\subsection{Institutional Talk} \label{subsec:institutional}

\section{Digital Reshaping} \label{sec:digital}
\subsection{New Turn Patterns} \label{subsec:new-turns}
\subsection{Multiple Conversations} \label{subsec:multiple}
\subsection{Digital Repair} \label{subsec:digital-repair}