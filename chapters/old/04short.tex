\chapter{Adverbs and prepositions}

\epigraph{Slowly, carefully, precisely\\Between here and there}{}

\noindent
While traditional grammatical analysis drew seemingly clear lines between preposition phrases and adverb phrases, this apparent simplicity masked significant problems. The rigid classifications often led to contradictions and failed to capture important distinctions in how these elements actually function in English. If you've ever felt confused about whether a word like \textit{before} is an adverb, a ``conjunction'', or a preposition depending on the sentence, you're not alone.

Many dictionaries and traditional grammars adhere to what we might call the ``NP-complement view'' of prepositions---defining them primarily as words that must precede and govern a noun phrase. This often leads to categorizing words like \textit{before} or \textit{down} as adverbs when they appear without a following noun phrase (\textit{I've seen him before}, \textit{He fell down}) or as ``conjunctions'' when followed by a clause (\textit{before you arrived}). But this approach just creates unnecessary complexity and often contradicts the dictionaries' own examples \autocite{reynolds2025}.

This chapter aims to untangle this confusion by presenting a clearer, more consistent framework based on syntactic behaviour---a ``flexible-complement view.'' This view recognizes prepositions as heads of prepositional phrases (PPs) that can take various types of complements (including noun phrases, other PPs, clauses, or even no complement at all). We argue that adopting this perspective not only aligns better with linguistic evidence but also simplifies the grammar, ultimately providing a more robust and pedagogically useful understanding for ESL teachers. We will explore the characteristics and functions of adverbs and prepositions, examine the evidence for the flexible-complement view, address common misclassifications, and discuss specific challenges these categories present for language learners. Our goal is to equip you with the tools to navigate this complex landscape with confidence.

\section{Adverbs and Adverb Phrases (AdvPs)}
The most typical adverbs are those like \textit{happily} where you start with an adjective and you add \textit{--ly}. But there are many others. These include: \textit{again}, \textit{best}, \textit{due}, \textit{even}, \textit{far}, \textit{hard}, \textit{just}, \textit{long}, \textit{maybe}, \textit{never}, \textit{often}, \textit{pretty}, \textit{quite}, \textit{right}, \textit{seldom}, \textit{thus}, and \textit{very}.

\subsection{Characteristics of Adverbs}

\subsubsection*{What they denote}

The meanings of adverbs as a group are quite diverse.

\begin{itemize}
    \item \textbf{Manner:} \textit{easily}, \textit{quickly}, \textit{silently}
    \item \textbf{Time and frequency:} \textit{already}, \textit{always}, \textit{often}, \textit{rarely}, \textit{temporarily}
    \item \textbf{Degree:} \textit{almost}, \textit{thoroughly}, \textit{very}
    \item \textbf{Order:} \textit{initially}, \textit{next}, \textit{last}
    \item \textbf{Certainty:} \textit{definitely}, \textit{likely}, \textit{probably}
    \item \textbf{Domain:} \textit{politically}, \textit{scientifically}, \textit{technically}
    \item \textbf{Evaluation:} \textit{fortunately}, \textit{happily}, \textit{well}
    \item \textbf{Speech act:} \textit{confidentially}, \textit{frankly}, \textit{honestly}
    \item \textbf{Connective:} \textit{furthermore}, \textit{however}, \textit{thus}.
    \item etc.
\end{itemize}

\subsubsection*{Morphological characteristics}
Many adverbs share morphological properties with adjectives:
\begin{itemize}
    \item \textbf{Adjective + \textit{--ly}:} The following are quite common examples: \textit{exactly}, \textit{recently}, \textit{usually}, \textit{quickly}, \textit{clearly}, \textit{particularly}, \textit{completely}.
    \item \textbf{Grade:} Like adjectives, adverbs are often gradable: \textit{well}, \textit{better}, \textit{best}; \textit{fast}, \textit{faster}, \textit{fastest}; \textit{more skillfully}, \textit{least helpfully}, etc.
    \item \textbf{Modifiable by \textit{very}:} Also like adjectives, adverbs can often be modified by \textit{very}: \textit{very well}, \textit{very quickly}, \textit{very often}, \textit{very hard}, \textit{very long}, \textit{very soon}, etc.
\end{itemize}

\subsection{Characteristics and Functions of Adverb Phrases (AdvPs)} \label{sec:advps}

An adverb phrase (AdvP) is a phrase headed by an adverb. The underlined adverbs are the heads of the following adverb phrases: \textit{\uline{clearly}}, \textit{\uline{very}}, \textit{very \uline{clearly}}, \textit{not so \uline{well}}, etc.

\subsubsection*{AdvPs as modifiers}

Most typically, AdvPs function as modifiers in a wide range of contexts:

\begin{itemize}
    \item \textbf{In VPs:} \textit{She {\ob}sings \uline{beautifully}\cb}. \textit{He {\ob}\uline{never} goes to the store\cb}.
    \item \textbf{In AdjPs:} \textit{The movie was {\ob}\uline{incredibly} boring\cb}. \textit{She is {\ob}\uline{almost} ready\cb}.
    \item \textbf{In other AdvPs:} \textit{He runs {\ob}\uline{very} quickly\cb}.  \textit{She sings {\ob}\uline{quite} beautifully\cb}.
    \item \textbf{In clauses:} \textit{\uline{Hopefully}, she will arrive soon.} \textit{\uline{Surprisingly}, he didn't enjoy the concert.}
    \item \textbf{In PPs:} \textit{He lives {\ob}\uline{right} in the city centre\cb}. \textit{She sat {\ob}\uline{just} behind the curtain\cb.}
    \item \textbf{Before NPs (rarely):} \textit{{\ob}\uline{Even} the thought of it{\cb} is frightening.} \textit{Everyone's presence was comforting, {\ob}\uline{particularly} yours\cb.}
    \item \textbf{But not modifying Ns within NPs:} *\textit{{\ob}The \uline{quickly} runner{\cb} won.} *\textit{She bought {\ob}the \uline{beautifully} dress\cb.}
\end{itemize}

\subsubsection*{AdvPs and complements} \label{sec:AdvPs+Comps}
It's quite unusual for AdvPs to function as complements, but they do occur occasionally, as in \textit{She treated it \uline{carefully}}. Similarly, AdvPs rarely license complements themselves, though exceptions exist, such as \textit{\uline{independently} of his parents}.

\section{Prepositions and Preposition Phrases (PPs)}

So, what are prepositions then? Traditionally, and in many dictionaries, they are defined narrowly~-- essentially as words like \textit{in}, \textit{on}, or \textit{with} that require a following noun phrase (like \textit{in the box}, \textit{on the table}, \textit{with a friend}). This ``NP-complement view'' often forces words that look and behave similarly into different categories depending on what follows them. For instance, \textit{before} might be called a preposition in \textit{before dinner} (with an NP), a ``conjunction'' in \textit{before you arrived} (with a clause), and an adverb in \textit{I've seen him before} (with no complement).

But this approach creates unnecessary complexity and leads to inconsistencies, even within dictionary entries themselves, which often list examples contradicting their own definitions \autocite{reynolds2025}. A more consistent and linguistically sound approach~-- the ``flexible-complement view'' adopted here and supported by \textit{The Cambridge Grammar of the English Language} \autocite{Huddleston2002}~-- treats prepositions as heads of prepositional phrases (PPs) that can license a variety of complements, or sometimes none at all.

Under this view, \textit{before} is simply a preposition, regardless of whether it is followed by an NP (\textit{before dinner}), a clause (\textit{before you arrived}), a PP (\textit{before now}), or nothing (\textit{I left before}). Similarly, words like \textit{down} (in \textit{He fell down}), \textit{apart} (in \textit{They live apart}), or \textit{because} (in \textit{because it rained}) are also analyzed as prepositions, heading PPs that may lack an overt complement (intransitive prepositions) or take complements other than NPs.

This unified analysis significantly simplifies the grammar. Instead of needing three categories for words like \textit{before}, we have one: preposition. This provides a clearer and more coherent picture for you as a teacher and, ultimately, for your students. The rest of this section details the characteristics of prepositions and PPs under this flexible-complement view, provides further evidence for distinguishing them from adverbs, and discusses common challenges they pose for learners.

\subsection{Characteristics of Prepositions and PPs}\label{sec:preps}

Having established the flexible-complement view, let's examine the characteristics of prepositions (P) and the prepositional phrases (PPs) they head.

\subsubsection*{Characteristics of Prepositions}

English prepositions typically exhibit the following properties:

\begin{itemize}
    \item \textbf{Denotation:} Prepositions primarily denote relations. These relations are often spatial (\textit{in}, \textit{on}, \textit{under}, \textit{between}), temporal (\textit{before}, \textit{after}, \textit{during}, \textit{until}), or directional (\textit{to}, \textit{from}, \textit{towards}, \textit{away}), but they can also express cause (\textit{because}, \textit{due}), comparison (\textit{like}, \textit{as}), and many other abstract relationships. Often, a single preposition can denote different types of relations depending on context (e.g., \textit{before 6:00} vs. \textit{before the mirror}).
    \item \textbf{Head Function:} Prepositions always function as the head of a prepositional phrase (PP). Remember, my technical meaning of \textsc{phrase} allows for single-word phrases, so \textit{in}, \textit{before}, or \textit{home} can each constitute a PP on their own when used intransitively.
    \item \textbf{Form and Structure:} Most prepositions have only one form and lack prefixes or suffixes.
    \ea \textit{at}, \textit{by}, \textit{from}, \textit{home}, \textit{in}, \textit{now}, \textit{on}, \textit{of}, \textit{over}, \textit{through}, \textit{to}, \textit{under}, \textit{where}, \textit{with} \label{ex:bareprep}
    \z
    But there are exceptions. Some have prefixes or suffixes 
    \ea  \textit{apart}, \textit{beside}, \textit{ceilingward}, \textit{depending} \label{ex:compoundprep}
    \z
    or are made of multiple words.
    \ea \textit{in front}, \textit{on board}, \textit{so as} \label{ex:complexprep}
    \z
    And a few are gradable, like adjectives/adverbs are.
    \ea 
        \ea \textit{near}, \textit{nearer}, \textit{nearest}
        \ex\textit{far}, \textit{farther}, \textit{farthest}
        \ex\textit{close}, \textit{closer}, \textit{closest}
        \z
    \z
\end{itemize}



\subsubsection*{Characteristics of Prepositional Phrases}\label{sec:pp-characteristics}

Prepositional phrases (PPs), being headed by prepositions, have their own characteristics, particularly regarding modification:

\begin{itemize}
    \item \textbf{Internal Modifiers:} PPs can themselves contain modifiers that specify or intensify the relation expressed by the head preposition. Common modifying AdvPs include \textit{even}, \textit{just}, and \textit{almost} (\textit{\uline{almost} to the door}, \textit{\uline{even} without it}, \textit{\uline{just} before noon}).
    \item \textbf{Distinctive Modifiers:} Crucially for distinguishing PPs from AdvPs, PPs readily accept modification by AdvPs headed by \textit{right}, \textit{straight}, \textit{clear}, and \textit{way}~-- modifiers that typically cannot modify adverbs or AdvPs (\textit{\uline{right} in front of the house}, \textit{\uline{straight} up the tree}, \textit{\uline{clear} to the horizon}, \textit{\uline{way} out west}).
    \item \textbf{Measure Phrase Modifiers:} PPs denoting time or space often take measure phrases (typically NPs) as modifiers (\textit{\uline{two hours} before he arrived}, \textit{\uline{a kilometre} down the road}).
    \item \textbf{Lack of Degree Modification:} Unlike AdjPs and many AdvPs, PPs are almost never modified by degree adverbs like \textit{very}, \textit{too}, or \textit{so} (*\textit{\uline{very} in the room}, *\textit{\uline{so} down the well}, *\textit{\uline{too} on the table}).
\end{itemize}

\subsection{Functions and Complementation of PPs}\label{sec:pp-functions-complementation}

Prepositional phrases perform two main syntactic functions in larger structures: they act as modifiers or as complements.

\subsubsection*{PPs as Modifiers and Complements}\label{sec:pps-as-mods-comps}

PPs frequently function as modifiers, adding information about time, place, manner, etc., within various phrase types:

\begin{itemize}
    \item \textbf{Modifying NPs:} \textit{the book {\ob}on the shelf\cb}, \textit{the man {\ob}with the hat\cb}
    \item \textbf{Modifying VPs:} \textit{slept {\ob}beside a tree\cb}, \textit{studies {\ob}during the day\cb}
\end{itemize}

PPs also commonly function as complements, selected by specific heads:

\begin{itemize}
    \item \textbf{In VPs:} \textit{She is {\ob}in the garden\cb}. \textit{He talked {\ob}to us\cb}. \textit{I received it {\ob}from them\cb}.
    \item \textbf{In NPs:} \textit{knowledge {\ob}of finance\cb}, \textit{interest {\ob}in movies\cb}, \textit{desire {\ob}for love\cb}.
    \item \textbf{In AdjPs:} \textit{eager {\ob}for work\cb}, \textit{proud {\ob}of you\cb}, \textit{happy {\ob}about the change\cb}.
    \item \textbf{In other PPs:} \textit{down {\ob}from the cottage\cb}, \textit{due {\ob}to the time\cb}, \textit{except {\ob}for that\cb}.
\end{itemize}

\subsubsection*{Variety in Prepositional Complementation}\label{sec:variety-pp-complementation}

A key aspect of the flexible-complement view introduced in \S3.1 is recognizing that prepositions~-- like verbs~-- can license a wide variety of complement types, not just the traditionally expected noun phrase. While a transitive preposition (one licensing an NP object) is the most frequent type, prepositions can also take other phrases or clauses, or simply be intransitive (appear without an overt complement). This variety is often where traditional analyses and dictionary entries become inconsistent. Consider the range:

\begin{itemize}
    \item \textbf{Comp:NP (Object):} The most typical case. \\ \textit{on {\ob}the table\cb}, \textit{with {\ob}John\cb}, \textit{about {\ob}lunch\cb}
    \item \textbf{Comp:PP:} Some prepositions take another PP as their complement. \\ \textit{from {\ob}under the bed\cb}, \textit{out {\ob}of the box\cb}, \textit{until {\ob}after midnight\cb}
    \item \textbf{Comp:Clause:} Many words traditionally called ``conjunctions'' are better analyzed as prepositions taking clausal complements. \\ \textit{before {\ob}I arrive\cb}, \textit{after {\ob}the party ended\cb}, \textit{since {\ob}you asked\cb}, \textit{about {\ob}whether it works\cb}, \textit{thinking about {\ob}what you said\cb}
    \item \textbf{Comp:AdjP:} A smaller set of prepositions license AdjP complements. \\ \textit{from {\ob}bad to worse\cb}, \textit{at {\ob}best\cb}, \textit{for {\ob}certain\cb}, \textit{in {\ob}short\cb}, \textit{as {\ob}serious\cb} (in \textit{regarded as serious})
    \item \textbf{Comp:AdvP:} This is relatively rare. \\ \textit{until {\ob}recently\cb}, \textit{since {\ob}when\cb}?, \textit{by {\ob}far\cb}
    \item \textbf{Intransitive:} Many prepositions occur without any complement. Those traditionally classed as ``adverbs" in these uses (like \textit{before}, \textit{down}, \textit{in}, \textit{outside}) fit the syntactic patterns of prepositions (as discussed in \S\ref{sec:distinguishing-pps-advps}) and are better analyzed simply as prepositions heading a PP. \\ \textit{I've seen him {\ob}before\cb}. \textit{He looked {\ob}up\cb}. \textit{Come {\ob}on in\cb}. \textit{She lives {\ob}nearby\cb}. \textit{He went {\ob}aboard\cb}.
\end{itemize}

Recognizing this full range of complementation possibilities allows for a simpler more consistent analysis, avoiding the need to assign words like \textit{before} or \textit{since} to multiple categories based solely on what follows them. It aligns with the principle that lexical categories (like preposition) should be defined by their broader syntactic behaviour, not just by their most typical complement type.

\subsection{Distinguishing PPs from AdvPs: Key Syntactic Differences}\label{sec:distinguishing-pps-advps}

The flexible-complement view analyzes words traditionally called ``adverbs" (like \textit{before}, \textit{down}, \textit{in} when used alone) or ``conjunctions" (like \textit{before}, \textit{after}, \textit{since} when followed by a clause) as prepositions. A key justification for this comes from syntax: these words, and the phrases they head, consistently pattern with undisputed prepositions/PPs, and contrast sharply with undisputed adverbs/AdvPs. Here are some diagnostic tests that illustrate this distinction:

\subsubsection*{Test 1: Modification Patterns}\label{sec:test-modification}

As noted in \S\ref{sec:pp-characteristics}, PPs accept modification by a specific set of AdvPs headed by words like \textit{right}, \textit{just}, \textit{straight}, \textit{clear}, and \textit{way}. Typical adverbs and AdvPs do not accept these modifiers. Contrast the acceptable PPs in (\ref{ex:pp-mods}) with the unacceptable AdvPs in (\ref{ex:advp-mods}).

\ea\label{ex:pp-mods} % PP modification
    \ea \textit{right before lunch}
    \ex \textit{just inside the house}
    \ex \textit{straight down the road}
    \ex \textit{way up the tree}
    \z
\z
\ea\label{ex:advp-mods} % AdvP modification (ungrammatical)
    \ea *\textit{right quickly}
    \ex *\textit{just carefully}
    \ex *\textit{straight locally}
    \ex *\textit{way often}
\z
\z

Conversely, typical adverbs and AdvPs readily accept degree modifiers like \textit{very}, \textit{too}, and \textit{so}, whereas PPs generally do not (\S\ref{sec:pp-characteristics}).

\ea\label{ex:advp-mods2} 
    \ea \textit{very quickly}
    \ex \textit{too carefully}
    \ex \textit{so locally}
    \z
\z
\ea\label{ex:pp-mods2} % PP modification (ungrammatical)
    \ea *\textit{very before lunch}
    \ex *\textit{too inside the house}
    \ex *\textit{so down the road}
    \z
\z

Words like \textit{before}, \textit{inside}, \textit{down}, etc.~-- even when used without an overt complement~-- pattern with prepositions in accepting modifiers like \textit{right} and rejecting modifiers like \textit{very}.

\subsubsection*{Test 2: Function as Predicative Complement}\label{sec:test-predcomp}

PPs frequently function as predicative complements, especially after verbs like \textit{be}, \textit{seem}, \textit{remain}. AdvPs typically cannot function this way. Crucially, PPs headed by words often misclassified as ``adverbs" pattern with other PPs here.

\ea\label{ex:pp-predcomp} % PP as PredComp
    \ea \textit{The meeting is {\ob}at noon\cb}.
    \ex \textit{Your keys are {\ob}on the table\cb}.
    \ex \textit{The power remained {\ob}off\cb}. % 'off' here patterns as PP
    \ex \textit{Is he {\ob}in\cb} ? % 'in' here patterns as PP
    \z
\z
\ea\label{ex:advp-predcomp} % AdvP as PredComp (ungrammatical)
    \ea *\textit{The meeting is {\ob}easily\cb}.
    \ex *\textit{Your keys are {\ob}quickly\cb}.
    \ex *\textit{The power remained {\ob}locally\cb}.
    \ex *\textit{Is he {\ob}carefully\cb}?
    \z
\z

\subsubsection*{Test 3: Coordination Patterns}\label{sec:test-coordination}

Constituents generally coordinate only with constituents of the same category performing the same function. PPs coordinate readily with other PPs, but typically not with AdvPs. Again, words like \textit{up}, \textit{before}, \textit{in} pattern as PPs.

\ea\label{ex:pp-coord} % PP coordination
    \ea \textit{He looked {\ob}down the street{\cb} and {\ob}all around\cb}. (PP + PP)
    \ex \textit{Come {\ob}before lunch{\cb} or {\ob}after dinner\cb}. (PP + PP)
    \ex \textit{Is she {\ob}in{\cb} or {\ob}at lunch\cb}? (PP + PP)
    \z
\z
\ea\label{ex:advp-coord} % AdvP coordination
    \ea \textit{She works {\ob}quickly{\cb} and {\ob}efficiently\cb}. (AdvP + AdvP)
    \ex \textit{Speak {\ob}too loudly{\cb} or {\ob}too softly\cb}. (AdvP + AdvP)
    \z
\z
\ea\label{ex:mixed-coord} % Mixed coordination (ungrammatical)
    \ea *\textit{He looked {\ob}up{\cb} and {\ob}efficiently\cb}. (PP + AdvP)
    \ex *\textit{Come {\ob}before lunch{\cb} or {\ob}quickly\cb}. (PP + AdvP)
    \z
\z

\subsubsection*{Test 4: Modification within NPs}\label{sec:test-np-mod}

PPs commonly function as modifiers within NPs, appearing after the head noun (postmodifiers) or occasionally before (premodifiers). AdvPs cannot modify nouns within NPs. Words like \textit{in}, \textit{down}, \textit{abroad} pattern with PPs in this regard.

\ea\label{ex:pp-np-mod} % PP modifying N
    \ea \textit{the {\ob}door \uline{in the wall}\cb} (Postmodifier)
    \ex \textit{the {\ob}\uline{down} escalator\cb} (Premodifier)
    \ex \textit{his {\ob}trip \uline{abroad}\cb} (Postmodifier)
    \ex \textit{the {\ob}people \uline{inside}\cb} (Postmodifier)
    \z
\z
\ea\label{ex:advp-np-mod} % AdvP modifying N (ungrammatical)
    \ea *\textit{the {\ob}door \uline{quickly}\cb}
    \ex *\textit{the {\ob}\uline{carefully} escalator\cb}
    \ex *\textit{his {\ob}trip \uline{beautifully}\cb}
    \ex *\textit{the {\ob}people \uline{locally}\cb}
    \z
\z

\subsubsection*{Conclusion from Tests}\label{sec:tests-conclusion}

These syntactic tests consistently demonstrate that words like \textit{before}, \textit{down}, \textit{in}, \textit{up}, \textit{nearby}, \textit{abroad}, etc., behave like undisputed prepositions, even when they appear without an overt complement. They accept the kinds of modifiers PPs take, function in syntactic positions characteristic of PPs (like predicative complement), coordinate with other PPs, and can modify nouns within NPs. They fail tests characteristic of adverbs (e.g., modification by \textit{very}, inability to be a predicative complement of \textit{be}). This strong syntactic evidence justifies analyzing these words consistently as prepositions, regardless of their complementation pattern, supporting the flexible-complement view outlined in \S\ref{sec:pp-functions-complementation}.

\subsection{Common Challenges for Learners}\label{sec:learner-challenges-preps}

Even with a clear analysis, prepositions remain a significant hurdle for English language learners. Their complexity stems from several areas.

\subsubsection*{Challenge 1: Selecting Prepositions in Complements}\label{sec:selecting-preps-challenge}

One major difficulty lies in selecting the correct preposition when a verb, noun, or adjective licenses a PP complement. While sometimes motivated by spatial or temporal metaphors, the choice is often conventional or seemingly arbitrary from a learner's perspective. Consider the verb \textit{graduate}. Its complementation has evolved:

\ea\label{ex:graduate-comps-redux-2}
    \ea \textit{Humber graduated {\ob}100 students\cb.}\hfill(Older form; Obj:NP)
    \ex \textit{I graduated {\ob}from Humber\cb.} \hfill(Common 20th C. form; Comp:PP)
    \ex \textit{I graduated {\ob}Humber\cb.} \hfill(Common 21st C. form; Obj:NP)
    \z
\z

To a learner, the shift between needing \textit{from} and taking a direct object might seem unpredictable. Similarly, \textit{listen} historically could take a direct object (\textit{listen them}) but now typically requires \textit{to} (\textit{listen {\ob}to them\cb})~-- a choice motivated perhaps by \textit{give ear to} but not easily guessed. Verbs may license multiple prepositions with subtle meaning differences (\textit{speak about}/\textit{of}/\textit{on}/\textit{to prepositions}) while rejecting others that seem semantically plausible (*\textit{speak at}/\textit{in prepositions}). It's crucial to remember this apparent arbitrariness when teaching complementation patterns (\S\ref{sec:verb-complementation}) and assessing student attempts like *\textit{I'm happy on my grades}~-- which, while non-standard, likely conveys the intended meaning.

\subsubsection*{Challenge 2: Understanding Meanings and Uses}\label{sec:meaning-use-challenge}

Beyond complement selection, the varied and extended meanings of prepositions pose another challenge. Core spatial meanings often extend metaphorically in ways that aren't obvious (\S\ref{sec:preposition-meanings}). For instance, \textit{up} can extend from `away from ground' (\textit{balloon went up}) to `higher amount' (\textit{price went up}) to `functioning' (\textit{system's up}) to `completion' (\textit{eat it up}). While traceable, this path isn't easily predicted.

Furthermore, the conventional uses of prepositions for time and place often rely on culturally specific conceptualizations (\S\ref{sec:preposition-meanings}). We use \textit{at} for points in time (clock times, night, meals, ages) but \textit{on} for surfaces (days, specific dates, streets) and \textit{in} for containers (mornings generally, months, seasons, years, cities, countries, rooms). These groupings aren't always logically transparent~-- why \textit{at night} but \textit{in the morning}? Why \textit{on TV} but \textit{in movies}? Why \textit{arrive at} a city but live \textit{in} it? Even related languages differ; English uses \textit{in} for cars, while Italian uses \textit{su} (`on'), but Italian uses \textit{in} for larger vehicles where English often uses \textit{on} (buses, trains). These conventions must largely be learned case-by-case.

Adding to the difficulty, prepositions are often unstressed and phonologically reduced in speech, making them hard for learners to perceive accurately (\textit{listen to} often sounds like /lɪsənə/).

\subsubsection*{Challenge 3: ``Phrasal Verbs''}\label{sec:phrasal-verbs-challenge}

One area that illustrates the difficulties of preposition choice and meaning is ``phrasal verbs''~-- a common term in language teaching, though often imprecise. These constructions (\textit{break up}, \textit{run into}, \textit{give away}, \textit{take off}, \textit{put on}, \textit{look up}) consist of a verb combined with one or more prepositions, where the combination often leads to a new meaning. To be clear, despite the name, they are frequently \emph{not} phrases in the technical syntactic sense (a head and its dependents). The combination often functions more like a single vocabulary item in terms of meaning, even if composed of syntactically separate parts.\footnote{\textit{CGEL} \autocite{Huddleston2002} eschews the term \textsc{phrasal verbs} in favour of \textsc{prepositional verbs}.}

\paragraph*{Idiomatic meanings}
Some verb-preposition combinations are relatively transparent (\textit{look at something}), but many phrasal verbs are idiomatic~-- their meaning cannot be easily deduced from the parts. If a plane \textit{takes off}, it leaves the ground, which has little to do with taking. If you \textit{come across} something, you find it by chance. \textit{Looking up} a word often involves looking, but it's hard to see how \textit{look} + \textit{up} becomes `check in a reference source'. While these feel common, it's worth remembering that most individual phrasal verbs are not extremely frequent (\S\ref{sec:word-frequency}), though the pattern itself is common.

\paragraph*{Movable particles}
A key syntactic feature distinguishing many phrasal verbs from simple V + PP sequences is the behaviour of the preposition as a \textsc{particle} that can often move after an NP object. This particle, typically an intransitive preposition, functions as a complement within the VP, alongside the object NP. Compare:

\ea\label{ex:particle-movement-3}
    \ea \textit{She {\ob}picked \uline{up} the book\cb}. / \textit{She {\ob}picked the book \uline{up}\cb}.
    \ex \textit{They {\ob}took \uline{off} their coats\cb}. / \textit{They {\ob}took their coats \uline{off}\cb}.
    \z
\z
When the object is a pronoun, though, the particle must follow it:

\ea\label{ex:particle-pronoun-3}
    \ea \textit{She {\ob}picked it \uline{up}\cb}. / *\textit{She {\ob}picked \uline{up} it\cb}.
    \ex \textit{They {\ob}took them \uline{off}\cb}. / *\textit{They {\ob}took \uline{off} them\cb}.
    \z
\z

This contrasts crucially with sequences where the preposition heads a PP consisting of a Head + Object and functioning as a modifier or complement. The preposition in such PPs cannot be placed after the object, because the object is \emph{in the PP} not in the VP:

\ea\label{ex:particle-vs-pp}
    \ea \textit{He fell {\ob}off his chair\cb}. (PP = \textit{off his chair})
    \ex *\textit{He fell {\ob}his chair off\cb}.
    \z
\z

Distinguishing verb-particle constructions from verb + prepositional phrase sequences requires attention to both the idiomaticity of the meaning and these syntactic movement patterns.

\subsubsection*{Challenge 4: Preposition Stranding and Fronting}\label{sec:stranding-fronting-challenge}

Finally, learners encounter structures where prepositions are separated from the NP they logically relate to, a phenomenon common in questions and relative clauses (\S\ref{sec:preposition-stranding}). In \textit{What are you talking {\ob}about\cb}?, the preposition \textit{about} is ``stranded'' at the end, separated from the interrogative word \textit{what} it relates to. The more formal alternative involves ``fronting'' the entire PP: \textit{{\ob}About what{\cb} are you talking}?

\ea\label{ex:stranding-redux-2}
    \ea \textit{\uline{Who} did you give it {\ob}to\cb} ? (Stranding)
    \ex \textit{{\ob}To whom{\cb} did you give it} ? (Fronting)
    \z
\z

While stranding is common, especially in speech, the patterns governing when it's possible or preferred over fronting are complex and will be explored further in chapters \ref{ch:questions} and \ref{ch:relatives}. Introducing the concepts here highlights another way prepositions interact intricately with English syntax.

% --- Optional Practice Box ---
% \begin{tcolorbox}[title=Practice: Preposition Challenges, colback=white]
% \begin{enumerate}
%    \item Choose the correct preposition: \textit{I apologized (for / at / to) being late.}
%    \item Explain the difference: \textit{She turned on the light} vs. \textit{She turned on her colleagues}.
%    \item Rewrite using particle movement if possible: \textit{He put on his hat.}
%    \item Rewrite using pied-piping: \textit{Which hotel are you staying at?}
%    \item Why might a learner say *\textit{arrive to the city} instead of \textit{arrive in/at the city}?
% \end{enumerate}
% \end{tcolorbox}
% --- End Practice Box ---

\section{Summary}\label{sec:summary-ch4}

This chapter explored the characteristics and functions of adverbs and prepositions, aiming to provide a clearer framework than traditional analyses often offer, especially regarding words that seem to shift categories.

We saw that adverbs typically denote manner, time, degree, etc., and head adverb phrases (AdvPs). The primary function of AdvPs is as \textsc{Modifiers} within various phrase types~-- including VPs, AdjPs, PPs, and other AdvPs~-- but generally not within NPs modifying the noun directly. AdvPs rarely function as \textsc{Complements}.

In contrast, prepositions primarily denote relationships (spatial, temporal, and abstract) and always head PPs. Adopting a ``flexible-complement view'', we saw that prepositions license a wide variety of \textsc{Complements}~-- including NPs (Objects), other PPs, Clauses, AdjPs, and occasionally AdvPs. Crucially, many prepositions can also be used without any overt complement following them (uses often mislabeled as "adverbs" traditionally). PPs function widely as both \textsc{Modifiers} (e.g., in NPs and VPs) and \textsc{Complements} (e.g., in VPs, NPs, AdjPs).

We examined several syntactic tests (\S\ref{sec:distinguishing-pps-advps})~-- involving modification patterns (like accepting \textit{right} but not \textit{very}), function as predicative complements, coordination possibilities, and modification within NPs~-- that consistently distinguish PPs from AdvPs and support analyzing words like \textit{before}, \textit{down}, or \textit{since} as prepositions regardless of their complementation pattern.

Finally, we reviewed common challenges these categories pose for language learners (\S\ref{sec:learner-challenges-preps}), including the often arbitrary selection of prepositions, the complexity of their meanings and uses, the idiomatic nature and syntactic variability of phrasal verbs, and the phenomenon of preposition stranding. Understanding these distinctions and complexities provides a more robust foundation for teaching these crucial aspects of English grammar.