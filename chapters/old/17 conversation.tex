\chapter{Conversations} \label{ch:conversations}

\epigraph{Words bounce. Words, if you let them, will do what they want to do and what they have to do.}{}

\section{The nature of conversation}

Conversation is a fundamental human activity that serves a wide range of purposes. If you're like me, you may have believed~-- you may still believe~-- that the main purpose of a conversation is to exchange information and ideas. But conversations are usually about coordinating and about building and maintaining social relationships. At its core, conversation is a collaborative and interactive process, one that requires participants to coordinate their actions and intentions in real time. It is through conversation that we express our thoughts and feelings, negotiate our identities and roles, and make sense of the world around us.

But what exactly do we mean by conversation? The term is often used interchangeably with related concepts such as dialogue, discussion, or talk, but there are some key features that distinguish conversation from other forms of verbal communication. One of the most basic is the idea of \textsc{turn-taking}. In a conversation, participants take turns speaking and listening, with each person's contribution building on and responding to what came before. This is different from, say, a lecture or a monologue, where one person speaks for an extended period without interruption or feedback from others.

Another key feature of conversation is its spontaneity and unpredictability. While conversations can certainly be planned or structured to some degree, they are fundamentally open-ended and emergent, with the direction and outcome determined by the moment-to-moment choices and reactions of the participants. This is what distinguishes conversation from, for example, a scripted dialogue in a play or movie, where the content and sequence of turns is predetermined.

Conversation is also deeply \textsc{context-dependent}, shaped by the specific participants, setting, purpose, and cultural norms of the interaction. The way we converse with a close friend over coffee is very different from the way we converse with a colleague in a business meeting or with a stranger on the street. In each case, we draw on a different set of assumptions, expectations, and communicative strategies to handle the social and interpersonal dynamics at play.

Despite this context-dependence, there are some universal features of conversation that seem to cut across cultural and linguistic boundaries. One of these is the use of \textsc{adjacency pairs}, or matched sets of utterances that conventionally go together, such as question-answer, greeting-greeting, or offer-acceptance/refusal. These pairs create a normative framework for conversation, allowing participants to anticipate and plan their contributions in relation to what came before and what is likely to come next.

Another universal feature of conversation is the need for \textsc{coordination} and \textsc{cooperation} between participants. Conversation is not a solo activity, but a joint endeavor that requires each person to take into account the knowledge, perspective, and communicative needs of their interlocutor(s). This involves a constant process of \textsc{monitoring} and \textsc{adjustment}, as participants track the flow of the conversation and adapt their own contributions accordingly. It also involves a willingness to \textsc{collaborate} in the face of trouble or misunderstanding, working together to repair breakdowns and maintain a shared understanding of the interaction.

At the same time, conversation is also a site of \textsc{negotiation} and even \textsc{contestation}, as participants work to assert their own identities, opinions, and goals within the interaction. Conversation is not always a smooth or harmonious process, but can involve disagreement, competition, and conflict. The way these tensions are managed and resolved (or not) is itself a key aspect of conversational competence.

For language learners, developing conversational skills involves much more than simply acquiring a set of linguistic forms or memorizing dialogues. It requires a deep understanding of the nature of conversation itself, including its interactive, context-dependent, and interpersonal dimensions. It means learning to navigate the complex web of social and cultural norms that shape conversational interaction in the target language, from the appropriate ways to take turns and introduce topics to the subtle cues that signal agreement, disagreement, or shifts in stance.

One framework that can be helpful for understanding the multidimensional nature of conversation is the concept of \textsc{communicative competence}, first proposed by sociolinguist Dell \citet{hymes1972}. Hymes argued that to be a competent communicator in a given language, it is not enough to know the grammar and vocabulary of that language. One must also know how to use the language appropriately and effectively in different social contexts, taking into account factors such as the participants, the setting, the purpose of the interaction, and the norms and expectations of the community.

Hymes identified four key dimensions of communicative competence:

\begin{itemize}[noitemsep]
    \item \textit{Linguistic competence}: knowledge of the language code, including grammar, vocabulary, and pronunciation.
    \item \textit{Sociolinguistic competence}: knowledge of how to use language appropriately in different social contexts, taking into account factors such as the relationship between speakers, the setting, and the purpose of the interaction.
    \item \textit{Discourse competence}: knowledge of how to produce and interpret larger stretches of language, such as narratives, explanations, or arguments, and how to maintain coherence and cohesion across turns and topics.
    \item \textit{Strategic competence}: knowledge of how to use language to achieve specific communicative goals, such as expressing politeness, persuading others, or repairing misunderstandings.
\end{itemize}

For language learners, developing communicative competence means moving beyond a narrow focus on linguistic forms to consider the broader social, cultural, and interpersonal dimensions of language use. It means learning to adapt one's communication to different contexts and interlocutors, and to use language creatively and flexibly to achieve a range of communicative purposes.

This is where the role of conversation in language learning becomes crucial. Conversation provides a rich and authentic context for learners to develop and practice their communicative competence, as they engage in real-time interaction with other speakers of the language. Through conversation, learners can experiment with different ways of expressing themselves, test out their understanding of social and cultural norms, and receive immediate feedback on their communicative choices.

However, not all conversation is equally valuable for language learning. In many traditional language classrooms, conversation is often limited to highly structured and artificial exchanges, such as memorized dialogues or teacher-led question-and-answer sequences. While these activities may provide some practice with basic linguistic forms, they do little to develop learners' ability to engage in more spontaneous, open-ended, and authentic interaction.

To truly harness the power of conversation for language learning, teachers need to create opportunities for learners to engage in what linguist Michael \citet{long1983} has called ``negotiation of meaning". This refers to the collaborative process by which participants in a conversation work to understand each other and to make themselves understood, using clarification requests, confirmation checks, and comprehension checks. Research has shown that when learners engage in this kind of negotiation, they are more likely to notice gaps in their own linguistic knowledge, receive valuable input and feedback, and produce more complex and accurate language.

Creating opportunities for meaningful negotiation requires a shift away from teacher-centered, form-focused instruction towards more learner-centered, communicative approaches. This might involve activities such as information gaps, problem-solving tasks, role plays, or simulations, where learners have to use the language to achieve a real communicative purpose. It also means encouraging learners to take risks, to experiment with the language, and to learn from their mistakes, rather than striving for perfect accuracy at all times.

Of course, the role of conversation in language learning extends far beyond the classroom. In today's globalized and interconnected world, learners have more opportunities than ever to engage in authentic conversation with speakers of the target language, whether through study abroad, online interaction, or local community engagement. These experiences can be invaluable for developing learners' communicative competence, as they provide exposure to a wide range of communicative contexts, styles, and norms.

However, engaging in conversation outside the classroom can also be daunting for learners, particularly those who are not yet confident in their language skills. This is where the role of the teacher becomes crucial, in helping learners to develop the strategies and confidence they need to seek out and make the most of these opportunities. This might involve teaching learners how to initiate and maintain conversations, how to ask for help or clarification, or how to cope with communication breakdowns. It also means fostering a classroom environment that values risk-taking, experimentation, and learning from mistakes.

Ultimately, the goal of language learning is not just to acquire a set of linguistic forms, but to be able to use the language effectively and appropriately for real-world communication. And conversation is at the heart of this goal. By understanding the nature of conversation, and by providing learners with rich and varied opportunities to engage in authentic communicative interaction, language teachers can help learners to develop the skills and confidence they need to become truly proficient communicators in the target language.

This is not to say that conversation is the only or even the most important aspect of language learning. Learners also need to develop their skills in reading, writing, and other modes of communication, and to acquire a solid foundation in the linguistic forms and structures of the language. But conversation provides a crucial bridge between these more abstract forms of knowledge and the real-world demands of communicative language use.

Moreover, conversation is not just a means to an end in language learning, but an end in itself. The ability to engage in meaningful, authentic conversation is a key part of what it means to be a competent language user, and a source of great personal and social satisfaction. Through conversation, we connect with others, express our identities and values, and make sense of the world around us. It is a fundamental human activity, one that is at the very core of what it means to be a social and communicative being.

As language teachers, then, we have a responsibility to recognize and celebrate the centrality of conversation in language learning, and to create opportunities for learners to develop their conversational skills in all their richness and complexity. This means moving beyond a narrow focus on linguistic forms and structures, to consider the broader social, cultural, and interpersonal dimensions of language use. It means providing learners with authentic, engaging, and meaningful opportunities to use the language for real communicative purposes. And it means fostering a classroom environment that values experimentation, risk-taking, and learning from mistakes, as learners work to become confident and competent communicators in the target language.

\section{The building blocks of conversation}

Conversation, in all its varied forms and contexts, is built on a foundation of fundamental structures and techniques that allow us to coordinate what we're talking about, why we're conversing, what we say, when and how we say it. These building blocks of conversation include turn-taking, topic management, repair, discourse markers, and nonverbal communication. We've already covered some elements of this in Chapters \ref{ch:coherence} and \ref{ch:pragmatics}, but others have been left until now.

\section{Fundamental Concepts in Conversation Analysis}

One of the most basic and universal features of conversation is the system of \textsc{turn-taking} that governs who speaks when and for how long. In any conversation, participants employ a complex set of cues and conventions to determine when it is appropriate to take a turn, yield a turn, or backchannel (provide brief verbal or nonverbal feedback while another speaker is talking). These cues can include verbal signals such as intonation, pacing, and the use of specific words or phrases, as well as nonverbal signals such as gaze, gestures, and body position.

For example, in many English conversations, a speaker may signal that they are coming to the end of their turn by using a falling intonation, slowing down their pace, or using a tag question such as \textit{you know?} or \textit{right?} At this point, another participant may take the cue to begin their own turn, often by providing a brief acknowledgement or agreement before launching into their own contribution. Overlapping speech, where two or more participants talk at the same time, is also common in many casual conversations, and can serve various functions such as showing enthusiasm, solidarity, or attempts to gain the floor.

\begin{dialogue}
 \item[Donna SUMMER] \textit{So I always felt like I had this sort of~-- people to answer to and my children, and my child at the time, was one of them. And I felt that in the future I didn't want her to say, ``Mom, well, you did it.'' You know? But, you know\dots} \\
 \item[GROSS] \textit{Did she ever say that?} \\
 \item[SUMMER] \textit{Yeah, she did. Yes, she did, much\dots}\\
 \item[GROSS] \textit{What was your comeback?} \\
 \item[SUMMER] \dots\textit{to my chagrin. I just told her it was a different time, and, you know, I came from a totally different life than her.}
\end{dialogue}

In this example, Donna Summers uses the tag question \textit{you know?} to check for understanding and agreement and to signal the end of her speaking turn, while Terry Gross' interjections and follow-up questions help maintain the flow of the conversation.

Terry Gross is a Jewish woman from New York, and in many New York Jewish conversations, fast-paced, overlapping speech is the norm, and participants may compete for the floor in a lively, even combative style of interaction. Failing to jump in and assert one's voice in this context may be seen as a lack of engagement or conviction. For instance, a family dinner might involve multiple people talking at once, with interruptions seen as a sign of interest and involvement.

However, the specific norms and expectations around turn-taking can vary significantly across cultures and contexts. In contrast to the high-engagement style of the New-York Jewry, communities such as the Amish may expect very long pauses to demonstrate consideration and respect and to show that responses are measured and thoughtful rather than impulsive. \citet{Enninger1991} reports at least 11 silent pauses longer than 20 seconds in a 40-minute conversation among three Amish adults.

Each style is likely to be preferred and even expected by those who are used to it, and when they meet those with the opposite style, they often experience them rude. Before jumping to such a conclusion, though, it's important to consider that it may be simply a matter of turn-taking style.

Understanding these cultural variations can help us avoid assuming that our own turn-taking style is normal and natural.

Effective conversation is also often said to involve understanding and using \textsc{backchanneling}. These brief responses, such as \textit{mm-hmm}, \textit{yeah}, and \textit{I see}, show the speaker that the listener is engaged and following along. They help maintain the flow of conversation and provide feedback without taking the floor. For example, the bold text in the following conversation is backchanneling.

\begin{quote}
    \textbf{A:} \textit{Are you graduating from IEC this semester?} \\
    \textbf{B:} \textit{Yeah} \\
    \textbf{A:} \textit{\textbf{Very exciting}} \\
    \textbf{B:} \textit{Yeah} \\
    \textbf{A:} \textit{When is graduation? }\\
    \textbf{B:} \textit{16 December} \\
    \textbf{A:} \textit{\textbf{Oh}, \textbf{ok} }\\
    \textbf{A:} \textit{And Monday is your oldest son's surgery for his eye} \\
    \textbf{B:} \textit{nnn mhmmm} \\
    \textbf{A:} \textit{Is he nervous?} \\
    \textbf{B:} \textit{No} \\
    \textbf{A:} \textit{\textbf{No}}
\end{quote}

In this dialogue from \citet[105]{ShelleyGonzalez2013}, Speaker A uses backchanneling (\textit{Very exciting} to show a positive evaluation. \textit{Oh} shows understanding of the date, while \textit{ok} acknowledges that and signals a topic change. The final \textit{no}) echoes B's answer to show engagement and perhaps to encourage Speaker B to continue sharing information. In many speech communities, this kind of response demonstrates active listening and helps to maintain a smooth conversational flow.

But here too, there is significant variance among language communities. In some Indigenous communities, for instance, there may be an expectation of less and different backchanneling. Certainly, that was the case in the 17th century, when René-Robert Cavelier, Sieur de La Salle, a French explorer and fur trader in the great lakes areas of North America, found the backchanneling of the Haudenosaunee in important meetings remarkable. He wrote that that community members would attentively listen to the speaker, regardless of the speech's duration, and ``content themselves with saying from time to time: You speak well, you are right \dots~and they give these indications of consent to all the speeches'' \citep[131]{duval2024}.

This form of back-channelling, which values equality and respect in communication, contrasts with the more interruption-tolerant and feedback-oriented norms prevalent in many English-speaking contexts.

\textsc{Adjacency pairs} are another key building block of conversation closely related to turn-taking. These are paired utterances that conventionally go together, such as question-answer, greeting-greeting, or offer-acceptance/refusal. The first part of an adjacency pair (e.g., a question) creates a strong expectation for the second part (e.g., an answer), and participants will often go to great lengths to provide the expected response, even if it is delayed or reformulated.

For example, if one participant asks \textit{What did you do last weekend?}, there is a strong pressure for the other participant to provide some kind of answer, even if it is not a direct or complete one. A response such as \textit{Oh, not much} or \textit{I can't really remember} still functionally serves as a second pair part, even if it does not fully satisfy the expectations of the question. Failing to provide any kind of answer (e.g., by changing the subject abruptly or remaining silent) would often be a marked and potentially face-threatening move.

There's also some expectation that the initiator of adjacency pairs will alternate. Consider again the conversation above. Speaker B does little to keep the conversation going. They answer questions, but only in the most minimal way, leaving all the work of maintaining the conversation to A. Perhaps, this is because B wishes the conversation to end. But many language learners find themselves overwhelmed by the need to process what their interlocutor is saying and to formulate a response. In such cases, the minimal responses may be due to these processing challenges. In other cases, the feeling of being overwhelmed may simply be due to a confusion about the purposes of conversation or a lack of awareness of how conversations work and one's own role in maintaining them. In any case, the conversation above could be much more balanced if B were to contribute more by elaborating and showing an interest in A's situation.

\begin{dialogue}
    \item[A] Are you graduating from IEC this semester?
    \item[B] Yeah, I am! It's been a long road but I'm excited to finally have my degree. How about you, are you graduating too or still have a ways to go?
    \item[A] I have one more semester after this. Congrats though, that's a huge accomplishment!
    \item[B] Thanks! I'm definitely ready to be done. Do you have any fun plans for winter break to look forward to before next semester?
    \item[A]  My family and I are going to Colorado for a ski trip, so I'm excited about that. When is your actual graduation ceremony?
    \item[B] Oh nice, Colorado is beautiful! The graduation is on December 16th. My whole family is coming into town to celebrate.
    \item[A]  That's wonderful, I'm sure they're all so proud of you. What a special day that will be.
    \item[B] Thanks, yeah, it will be great to have everyone together. I'm a bit nervous though because my oldest son is actually having eye surgery two days later, on the 18th.
    \item[A]  Oh gosh, that's a lot to have going on at once. Eye surgery~-- is it serious? How is he feeling about it?
    \item[B] It's a procedure to correct his vision that he's needed for a while. He's a trooper though and is mostly just eager to get it over with and hopefully see better. How are your kids doing?
    \item[A]  They're doing well, thanks for asking! Making me play endless rounds of Monopoly over break. I'm glad your son is handling the surgery well, I'm sure that puts your mind at ease a bit.
    \item[B] Haha, Monopoly~-- bet that gets heated! But yes, I'm proud of how he's dealing with it. It's still nerve-wracking as a parent though.
\end{dialogue}

\textsc{Topic management} is another key dimension of conversation that requires coordination and negotiation between participants. In any conversation, participants need to work together to establish, maintain, and shift topics in a way that is coherent, relevant, and engaging for all involved. This involves asking questions, providing new information, making connections to previous topics, and signalling boundaries between topics.

For example, a common way to introduce a new topic is to use a ``pre-sequence'' or ``pre-announcement'' that signals the upcoming shift and checks for the other participant's interest or alignment. A speaker might say something like \textit{Oh, did you hear about what happened at the office today?} or \textit{I've been meaning to ask you something\dots} These moves give the other participant a chance to express interest, ask for clarification, or even decline to take up the new topic if it is not relevant or appropriate.

Maintaining a topic, on the other hand, involves asking follow-up questions, providing additional details or examples, and expressing agreement or alignment with the other speaker's contributions. A participant who consistently introduces new topics without allowing for sufficient development or exploration of the current one may be seen as uncooperative or self-centered. Similarly, one who merely provides perfunctory answers without any elaboration or questioning. This is a style that many beginning language learners often fall into because they are so focused on just getting a few words out and may not be able to think about doing more to sustain the topic. This could also be encouraged by teachers' questioning style in class (see Chapter \ref{ch:questions}).

Shifting topics can be a delicate dance, as it requires participants to strike a balance between novelty and coherence. Abrupt or unrelated shifts (e.g., suddenly changing the subject from the weather to one's medical history) can be jarring and disruptive to the flow of conversation. More skillful topic shifts often involve a gradual transition or a link to some element of the previous topic. For instance, a speaker might pivot from talking about a recent movie they saw to a related topic such as the actor's career or the broader genre of film.

\textsc{Repair} refers to the techniques that participants use to address problems or hitches in the flow of conversation, such as misunderstandings, mishearings, or errors. These can range from simple requests for clarification (e.g., \textit{What did you say?}) to more complex negotiations of meaning (e.g., \textit{When you say ``put it off,'' do you mean postpone it or cancel it altogether?}).

Repair is often initiated by the listener who notices a problem, but it can also be self-initiated by the speaker who realizes they have made an error or been unclear. Self-repair often involves a brief pause, a filler (such as \textit{uh} or \textit{um}), and a restatement or clarification of the problematic utterance. Other-initiated repair often involves a question or request for clarification, which the original speaker then addresses in their subsequent turn.

The way repair is handled can vary significantly across cultures and contexts. In some communities, direct other-initiated repair (e.g., \textit{No, that's not right. What I meant was\dots}) may be seen as face-threatening or rude, while in others it may be seen as a sign of engagement and collaboration. Likewise, the threshold for what counts as a problem in need of repair can vary widely. In some contexts, minor slips or disfluencies may be glossed over without comment, while in others they may be explicitly flagged and corrected.

\textsc{Discourse markers}, as discussed in Section \ref{sec:discourse-markers}, are words or phrases that function to signal the structure and flow of conversation, as well as the speaker's stance or attitude towards what is being said. They include items such as \textit{well}, \textit{you know}, \textit{I mean}, \textit{like}, and \textit{actually}, which often appear at the beginning or end of turns and serve to mark boundaries, indicate hesitation, or soften the force of an utterance.

For example, starting a turn with \textit{well} often signals that the speaker is about to say something that contrasts with or qualifies the previous utterance, as in 

A: \textit{I thought the movie was great. }

B: \textit{Well, I thought the ending was a bit disappointing.}

Using \textit{you know} or \textit{like} in the middle of a turn can serve as a hedge or an appeal for understanding, as in \textit{It's just, you know, it's been a really tough week.} And ending a turn with \textit{I mean} can serve to clarify or modify the speaker's previous statement, as in \textit{I don't think he's a bad person, I mean, he just made a mistake.}

Discourse markers are often seen as ``filler words'' or ``verbal tics,'' and language learners may be advised to avoid them in order to sound more fluent or proficient. While it is true that we tend to prefer prepared talks that have lower numbers of discourse markers, research has shown that they serve important interactional and pragmatic functions in conversations, and that their appropriate use is a sign of communicative competence.

Finally, \textsc{nonverbal communication} plays a small role in shaping the meaning and flow of conversation. This includes elements such as facial expressions, eye contact, gestures, posture, and proxemics (use of space). These nonverbal cues can serve to reinforce, complement, or even contradict the verbal content of an utterance, and they can vary significantly across cultures and contexts.

The role of nonverbal communication has often been overblown. Listening to any conversation on the radio should persuade you that your comprehension is hardly affected at all by lack of visual access to nonverbal communication clues. Conversely, watch enough TED talks or product launches, and the obviously coached use of gestures will seem not just lacking in communicative content, but annoying and distracting. At least, it does to me.

\textsc{Personal space} is another conversational aspect that varies between linguistic communities. For example, in many Arab cultures, it is common for speakers to stand or sit very close to each other and to use frequent touching as a sign of intimacy and engagement. In contrast, in many Northern European cultures, speakers tend to maintain a larger distance and to avoid touching outside of very close relationships. Likewise, the amount and type of eye contact that is considered appropriate can vary widely. In many North American contexts, direct eye contact is seen as a sign of honesty and confidence, while in many East Asian contexts, prolonged direct eye contact may be seen as confrontational or disrespectful.

Gestures and facial expressions can also carry significant meaning in conversation. A raised eyebrow, a head tilt, or a hand wave can all serve to signal interest, confusion, or emphasis. However, the specific meanings attached to these gestures can vary across cultures. In some communities, a head nod may signal agreement, while in others it may simply indicate that the listener is following along. Likewise, a smile may signal happiness, politeness, or even embarrassment, depending on the context and the cultural norms.

But very little of this will come as a surprise to most adults, and usually teaching English-language learners about them is of very little value. What may be useful is simply reminding people to be aware that their own expectations may not be shared, to avoid taking offence, and to try to assume good faith, rather than rudeness, in their conversational partners.

Ultimately, the building blocks of conversation~-- turn-taking, adjacency pairs, topic management, repair, discourse markers, and nonverbal communication~-- are what allow participants to coordinate their interactions and co-construct meaning in real time. By understanding how these elements work together, and by developing the skills to deploy them effectively in different cultural and communicative contexts, language learners can become more adept at (English) conversation.

For language teachers, this means going beyond a focus on grammar and vocabulary to emphasize the interactional and pragmatic dimensions of language use. It means providing opportunities for learners to observe and analyze authentic conversations, to practice managing turns and topics in a range of contexts, and to reflect on their own communication strategies and challenges. It means creating a classroom environment that values active listening, collaboration, and intercultural awareness.

As we move through an increasingly globalized and interconnected world, the ability to engage in diverse conversational styles and norms is becoming ever more important. By mastering the building blocks of conversation, and by developing the flexibility and adaptability to use them effectively across contexts, language learners can become not just proficient speakers, but truly skilled communicators, able to build meaningful connections with others across all kinds of linguistic and cultural boundaries.

\section{Conversational styles and cultural norms}

Conversation is not merely an exchange of words, but a complex dance of styles and norms shaped by the cultural backgrounds of the participants. Every community has its own unique ways of using language to communicate, establish relationships, and negotiate social hierarchies. These conversational styles and norms can vary widely across cultures, leading to potential misunderstandings and communication breakdowns when individuals from different backgrounds interact. In this section, we will explore some of the key dimensions of conversational style and the role of culture in shaping the norms of interaction.

One of the most influential frameworks for understanding conversational style comes from the work of sociolinguist Deborah Tannen. In her book \textit{You Just Don't Understand: Women and Men in Conversation}, \citet{Tannen1990} proposes that men and women often exhibit different conversational styles, which she terms ``report talk" and ``rapport talk," respectively. According to Tannen, men tend to use conversation primarily to exchange information and establish status, while women use it to build and maintain relationships. For example, a man might tell a story to showcase his achievements or expertise, while a woman might share a similar experience to find common ground and build solidarity.

While Tannen's framework has been critiqued for oversimplifying gender differences and neglecting the role of other factors such as power and context, it nonetheless highlights how conversational approaches and goals can vary across social groups. It also points to the potential for miscommunication when people with different conversational styles interact. 

Beyond gender, cultural background plays a significant role in shaping conversational styles and norms. One key dimension of cultural variation is the degree of directness or indirectness in communication. In some cultures, such as the United States and Germany, it is generally expected that speakers will express their thoughts and opinions directly and explicitly. Beating around the bush or hinting at one's meaning is often seen as evasive or insincere. In contrast, in many Asian cultures, such as Japan and China, indirectness is highly valued as a way of maintaining harmony and avoiding loss of face. Speakers may employ indirect questions, hedging, or even silence to convey their meaning without stating it outright.

For example, if a Japanese subordinate is unsatisfied with a decision made by their boss, they are unlikely to say so directly. Instead, they might make a vague statement like これはちょっと難しいですね (\textit{kore wa chotto muzukashii desu ne}), which translates roughly to `This is a bit difficult, isn't it?' The boss is expected to read between the lines and infer the subordinate's true feelings. An American in the same situation, however, might be more likely to state their objection explicitly, such as \textit{I don't think this is the right decision because\dots} Failure to recognize and adapt to these differences in directness can lead to significant misunderstandings and tensions in cross-cultural communication.

Another important dimension of cultural variation in conversational style is the distinction between high-context and low-context communication. In high-context cultures, such as many in Asia, the Middle East, and Latin America, much of the meaning in conversation is conveyed through shared assumptions, nonverbal cues, and the overall context of the interaction. Speakers may use fewer words and rely more on inference and interpretation to get their point across. In low-context cultures, such as many in Northern Europe and North America, meaning is conveyed more explicitly through the literal content of the message. Speakers tend to use more words and provide more detailed explanations to ensure understanding.

For instance, in a high-context culture like China, a business meeting may begin with extensive small talk and relationship-building activities before any actual business is discussed. The seating arrangements, gift-giving rituals, and even the choice of restaurant for the meeting may all convey important messages about the status and intentions of the participants. In a low-context culture like the United States, on the other hand, a meeting is more likely to begin with a direct statement of the agenda and a focus on the task at hand. Participants may engage in some brief small talk, but it is seen as secondary to the main purpose of the meeting.

Dealing with these differences in communication context can be challenging for language learners, who may be used to relying on explicit verbal cues to interpret meaning. They may find high-context communication frustratingly vague or indirect, or they may offend their interlocutors by being too blunt or task-focused in a high-context setting. Developing intercultural competence requires learning to read the subtle cues and contextual factors that shape meaning in different cultural settings.

Politeness strategies are another area where cultural norms can vary significantly. In many Western cultures, politeness is often associated with individual autonomy and the avoidance of imposition. Speakers may use hedging (e.g., ``I think," ``perhaps"), apologizing, or giving others options to minimize the perceived imposition of a request or opinion. In many non-Western cultures, however, politeness is more often associated with showing respect for social hierarchy and maintaining group harmony. Speakers may rely more on honorifics, self-effacement, or indirect speech acts to signal their deference to others and avoid direct confrontation.

For example, in many Korean business settings, it is considered polite for a junior employee to use honorific forms of address and grammar when speaking to a senior colleague, even if they are on friendly terms outside of work. A failure to use these forms would be seen as disrespectful and insubordinate. In contrast, in many American workplaces, the use of first names and informal speech is seen as a way of creating a sense of egalitarianism and teamwork, even across hierarchical levels. A Korean employee who insists on using honorifics in this context may be perceived as distant or uncooperative.

Cultural differences in politeness norms can also lead to misunderstandings around the use of direct speech acts such as requests, invitations, or refusals. In many Asian cultures, a direct refusal (e.g., ``No, I can't come to your party") is considered face-threatening and impolite. Instead, speakers may make excuses, express regret, or even remain silent to convey refusal. For example, a Japanese speaker might respond to an unwanted invitation by saying 考えておきます (\textit{kangaete okimasu}), which literally means `I'll set aside (some time) and think about it,' but probably conveys a polite refusal. An adult learner Japanese, though, might interpret this response as a straightforward expression of a genuine intention to decide later. This misinterpretation isn't just about missing politeness. It's about missing the conventional implicature (see Section \ref{sec:implicature}). The Japanese speaker isn't being evasive; they're following a conventional pattern where \textit{kangaete okimasu} in this context has become a recognized indirect refusal strategy.

The cross-cultural asymmetry is particularly interesting here: English speakers learning Japanese often struggle with these conventionalized indirectness patterns, while Japanese speakers learning English sometimes over-apply indirectness where directness is expected.

Navigating politeness norms across cultures requires not just linguistic knowledge, but also a deep understanding of the cultural values and social relationships that underpin them. Language learners need to develop the ability to interpret indirect speech acts, to use appropriate forms of address and honorifics, and to adjust their level of directness to match the expectations of their interlocutors. They also need to cultivate a sense of empathy and openness to different ways of expressing politeness and respect.

Hofstede's (\citeyear{hofstede2001}) cultural dimensions theory, particularly his concept of power distance, provides a useful framework for understanding how these conversational norms reflect deeper cultural values. Power distance refers to the extent to which less powerful members of organizations and institutions accept and expect that power is distributed unequally. In high power distance cultures, such as Malaysia, Saudi Arabia, and the Philippines, conversational patterns often reflect and reinforce hierarchical relationships. For example, subordinates may be expected to speak only when spoken to in meetings, use elaborate honorifics, and avoid contradicting superiors. In contrast, low power distance cultures like Denmark, New Zealand, and Israel tend to have more egalitarian conversational patterns, with more informal address forms and greater acceptance of subordinates challenging their superiors' views.

This dimension helps explain why, for instance, a Vietnamese student might be reluctant to ask questions in class (reflecting high power distance norms) while a Dutch student might readily engage in debate with their professor (reflecting low power distance norms). Understanding these underlying cultural values can help language learners anticipate and adapt to different conversational expectations across cultures.

Within the Canadian context, conversational styles and norms can vary significantly across regions, communities, and individual backgrounds. While Canadians are often characterized as speaking with excessive politeness and indirectness (stereotype of the apologetic Canadian), there is significant diversity in how these norms are enacted in practice. For example, in some Indigenous communities, silence may be used as a sign of respect or a way of giving others space to speak, rather than as an indication of disengagement or lack of understanding. In Québécois French, the use of informal pronouns (\textit{tu}) and first names is more widespread than in European French, reflecting a more egalitarian and familiar approach to social relationships.

And as Canada becomes increasingly multicultural, with large immigrant populations from Asia, Africa, the Middle East, and Latin America, the conversational landscape is becoming ever more complex and diverse. Language learners in Canada need to develop not just proficiency in English or French, but also the intercultural competence to cope with the multiple conversational styles and norms that they may encounter in their daily lives. This requires a willingness to step outside one's own cultural comfort zone, to listen actively and observe carefully, and to adapt one's own communication style as needed to build bridges of understanding.

\section{Conversation in different contexts}

There are so many conversations and types of conversations that it's hard to say anything coherent or useful about conversations as a group. The way we converse with friends and family in casual settings is markedly different from how we communicate in professional or academic environments. And the way you converse with your friends and family may be quite different from how I converse with mine. On top of this, the rise of digital communication has added new layers of complexity to the conversational landscape, with online interactions having their own unique features and norms.

\textsc{Casual conversation} among friends and family is perhaps the most fundamental and universal form of conversational interaction. These exchanges can be characterized by a high degree of informality, with relaxed turn-taking norms and a wide range of acceptable topics. For instance, a conversation between two close friends might move fluidly from discussing the latest sports scores to sharing deeply personal experiences, all within the space of a few minutes. But most of all, these conversations are typically built on a deep and broad reservoir of shared context. 

In these conversations, language usage often includes references to events, traditions, habits, and situations that might be unintelligible to outsiders but are rich in meaning for the participants. For example, references to that one summer at the lake or a catchphrase from a beloved television show can encapsulate complex memories and emotions in a few words. Alternatively, many things can be left unsaid that would have to be made explicit in other conversations. \textit{Tomo's still in bed} can be correctly interpreted as a caution to be quiet or an injunction to go wake him. The tone of these conversations can be warm and supportive, with use of humor, slang, and other markers of familiarity.

For language learners, engaging in casual conversation can be an excellent way to build fluency and confidence, as it provides a low-stakes environment for practicing new vocabulary and grammatical structures. The problem is finding contexts where such conversations are possible.

Even within the realm of casual conversation, though, there can be significant cultural variations. For example, in some cultures, such as many Arab societies, it is common for people to engage in extensive small talk and inquiries about one's health and family before moving on to the main topic of conversation. In contrast, in many Northern European cultures, such as Germany or Finland, people tend to be more direct and eschew lengthy preambles. Germans are likely to consider beating around the bush a waste of time, while Greeks may find it offensive to come right to the point.

Moving into the realm of \textsc{professional and institutional discourse}, we find that conversation takes on a very different character. In these settings, the primary purpose of talk is often to achieve specific goals or outcomes, whether that's making a sale, delivering a report, or reaching a decision in a meeting. As a result, the content and structure of these conversations are typically much more focused and agenda-driven than in casual settings. Turns are often allocated based on specific roles and hierarchies, and there may be explicit rules or protocols governing who can speak when and for how long.

For example, in a typical business meeting, the chair or facilitator usually opens the discussion, sets out the agenda, and then invites specific individuals to speak to each item in turn. Participants are expected to stay on topic, to be concise in their contributions, and to avoid interrupting or talking over one another.

Educational settings, such as classrooms and universities, present yet another distinct conversational context. Here, the primary purpose of talk is usually to facilitate learning, whether through teacher-led instruction, student presentations, or collaborative discussions. The structure of these conversations is often highly choreographed, with the teacher or instructor playing a central role in initiating and guiding the discourse.

A common pattern in classroom interaction is the IRF exchange, which stands for Initiation-Response-Feedback. In this model, the teacher initiates a question or prompt, a student responds, and the teacher then provides feedback or evaluation. For example:

\begin{dialogue}
\item[Teacher] What's the capital of France?
\item[Student] Paris.
\item[Teacher] That's right. Paris is the capital of France.
\end{dialogue}

While this structure can be effective for certain types of knowledge-checking or recall tasks, it has been criticized for limiting student agency and real communicative interaction. In more student-centred classrooms, there is often a greater emphasis on open-ended discussion, problem-solving tasks, and peer-to-peer interaction. Language learners in these settings need to develop skills in expressing opinions, asking questions, and collaborating with others to construct meaning.

Like other conversational contexts, educational settings are not immune to cultural variation. In many East Asian classrooms, for instance, there is a strong emphasis on respect for the teacher and avoidance of loss of face. Students may be hesitant to ask questions or express opinions that could be seen as challenging the teacher's authority. In contrast, in many Western educational settings, particularly at the university level, students are often encouraged to engage in critical thinking and to challenge and debate ideas openly.

Finally, the rise of online and computer-mediated communication has added a new dimension to the conversational landscape. From email and instant messaging to social media and video conferencing, digital technologies have created new spaces and modalities for interaction that come with their own unique features and challenges.

One key aspect of online conversation is its potential for asynchronicity, or the ability for participants to engage in conversation at different times. In an email exchange, for instance, there may be significant delays between turns, with participants composing and responding to messages at their own pace. This can allow for more thoughtful and reflective communication, but it can also lead to misunderstandings or breakdowns in coherence if the context of a message is lost over time.

Synchronous online communication can take two main forms: text-based chat and video conferencing, each with their own advantages and challenges. In text-based chat, the pressure to respond quickly and keep the conversation flowing can lead to a higher incidence of typos, fragmentary utterances, and interrupted turns. The lack of nonverbal cues such as facial expressions and gestures can make it harder to convey nuance and build rapport.

The language of chats also tends to be more informal and abbreviated than in face-to-face settings. Acronyms, emojis, and other forms of internet slang are common, as are non-standard grammatical constructions. For language learners, this can present both opportunities and challenges. On the one hand, participating in online forums or chat rooms can provide exposure to authentic, colloquial language use. On the other hand, it can be difficult to keep up with the rapid pace and unconventional forms of online talk.

Video conferencing platforms like Zoom and Meet, on the other hand, preserve many of the nonverbal aspects of face-to-face conversation, provided participants have their cameras on. Users can see facial expressions, gestures, and even some aspects of body language, though the 2D nature of video and potential technical issues like lag can still impact the natural flow of conversation. Video calls also introduce unique challenges, such as difficulties with turn-taking due to slight audio delays, the cognitive load of constantly seeing oneself on screen, and what has come to be known as 'Zoom fatigue' from extended periods of video interaction.

