\chapter{Pragmatics} \label{ch:pragmatics}

\epigraph{We learn to read\\
the weather in each other's voices,\\
the barometric pressure\\
of an almost-spoken word}{}

\section{Introduction to Pragmatics}

Consider this exchange:

\begin{dialogue}
\item[A] \textit{I'm on my way.}
\item[B] \textit{The door's open.}
\end{dialogue}

Nothing in the literal meaning of B's response addresses A's statement, but the exchange makes perfect sense. We can imagine a context in which A is likely going to B's house, and B is saying A can come right in without knocking. The study of how we make such inferences~-- how we derive meaning beyond what is literally said~-- is \textsc{pragmatics}.

Even a seemingly straightforward utterance like \textit{it's cold} can mean radically different things depending on context. Imagine the following situations:

\ea
    \ea[]{[At home, near an open window] \textit{It's cold.}\hfill[Request to close window]}
    \ex[]{[Tasting soup] \textit{It's cold.}\hfill[Complaint about food]}
    \ex[]{[Looking at weather forecast] \textit{It's cold.}\hfill[Warning about weather]}
    \z
\z

The same three words can function as a request, a complaint, or a warning. The linguistic form remains constant, but the meaning shifts dramatically based on context. This is the domain of pragmatics: how context shapes meaning.

But pragmatics isn't just about context. When I say \textit{some students arrived late}, you understand that not all students were late, even though \textit{some} literally means `at least one, possibly all'. Similarly, if you ask \textit{do you have the time?} I understand that you are asking an open question (`What time is it?'), not a closed question (`Do you or do you not know what time it is?'). And when I respond \textit{three fifteen}, you know I mean it's currently 3:15, not that the baseball score is 3 to 15 or that the number of students in the school is 315. These meanings arise not from the words themselves, but from shared assumptions about how we use language.

The study of these shared assumptions forms another key aspect of pragmatics. We expect speakers to be informative but not overly so, truthful, relevant, and clear. When they appear to violate these expectations, we search for additional meaning. If someone asks \textit{is Pat a good teacher?} and receives the response \textit{Pat has neat handwriting}, the apparent irrelevance triggers a search for implied meaning~-- in this case, that Pat is not a particularly good teacher.

So while semantics deals with the relationship between linguistic forms and their literal meanings, pragmatics examines how context, shared knowledge, and principles of communication combine to create meaning in actual language use. For language teachers and learners, understanding pragmatics is crucial because so much of what we communicate goes beyond what we literally say.

\subsection{Why Pragmatics Matters for Language Teaching}


Crucially for you, this may be the chapter where the learners' instincts are often better than the teachers'. After all, learners bring with them sophisticated pragmatic knowledge from their first language. They know that literal meanings aren't always intended meanings, that social relationships affect how we speak, and that context shapes interpretation. They've been making these kinds of pragmatic inferences their whole lives.

Recall the discussion of subject dropping in Section \ref{sec:subject-dropping}. When a student says \textit{Got it} or \textit{Need help?}, they're using perfectly standard English. But teachers, perhaps focused on an artificially formal register or misunderstanding what constitutes Standard English, may incorrectly mark these as errors. Similarly, the insistence on ``complete sentences'' (Section \ref{sec:complete-sentences}) often reflects teachers' misconceptions about standard usage rather than any real grammatical principle. A response like \textit{On the table} to the question \textit{Where's my phone?} is entirely standard~-- in fact, answering with a complete sentence like \textit{Your phone is on the table} could seem oddly formal or even uncooperative.

The issue isn't always that learners are breaking rules of Standard English, but rather that teachers sometimes operate under various misconceptions: conflating formality with accuracy, idealizing the written language, and failing to appreciate that more isn't always better. This may stem from their own education, where formal academic language was prioritized, or from textbooks that present an artificially rigid version of the language. It might reflect an over-reaction against genuine student errors, such as dropping heads, determiners, and other elements required in a particular context.

It commonly stems from a desire to get students to practice particular forms. Consider this example from my own language learning experience: In my Spanish class last week, the lesson was on possessive pronouns, and the teacher was getting us to practice by asking us, in effect, \textit{Is this your computer?} \textit{Is that her pencil?}, etc. The most natural response in such cases, is \textit{sí} or \textit{no, es mío}, but if students responded that way, they wouldn't be practicing the target forms, so our teacher had us saying the Spanish equivalent of \textit{yes, it is my computer}, and \textit{no, it isn't her pencil. It's my pencil.}

Keep this in mind as we explore the various aspects of English pragmatics in this chapter. Ask yourself whether you're insisting on more elaborate forms because they're more appropriate or just because they contain the structure you want students to practice. There's nothing wrong with the latter~-- explicit practice of forms is often necessary~-- but don't confuse it with teaching appropriate language use. And don't let it lead you to mark pragmatically appropriate responses as errors just because they don't showcase the target structure.

\section{Cooperation and Implicature}

Consider this exchange from Section \ref{sec:fallacy-of-monosemy}:

\begin{dialogue}
\item[Child] \textit{Can I go to the bathroom?}
\item[Adult] \textit{I don't know, can you?}
\end{dialogue}

This tired form of hazing dressed up as a joke depends on affecting to misunderstand the child's intended meaning. The child is making a request for permission, but the adult pretends to interpret \textit{can} literally, as a question about ability. The adult is being uncooperative, violating the normal expectations we have about how conversation works.

Here's a similarly uncooperative conversation:

\begin{dialogue}
\item[A] \textit{Do you know where JB 110 is?}
\item[B] \textit{Yes.}
\end{dialogue}

Compare that to how most people would react:

\begin{dialogue}
\item[A] \textit{Do you know where JB 110 is?}
\item[B] \textit{That's building J, there. And }B\textit{ is for basement.}
\end{dialogue}

B's response doesn't literally answer the yes/no question that was asked, but it's exactly what we expect. B understands that A is asking for directions, not conducting a survey of people's knowledge of campus buildings. This kind of inference~-- deriving unstated meaning from what is said~-- is called \textsc{implicature}.

Implicature generally works because we assume speakers are trying to be helpful. When someone says \textit{I hear you're a good cook}, they're usually angling for an invitation. When a colleague mentions \textit{I didn't receive your email}, they're asking you to send it again. When someone responds to \textit{How are you?} with \textit{I've had better days}, they're saying things aren't going well.

The philosopher Paul Grice formalized these expectations into what he called the \textsc{cooperative principle}: speakers try to make their contributions appropriate to the conversation. This principle breaks down into four basic maxims that we unconsciously follow and expect others to follow:

\ea
   \ea[]{Be appropriately informative (not too little, not too much)}
   \ex[]{Be truthful}
   \ex[]{Be relevant}
   \ex[]{Be clear}
   \z
\z

Much of what we call implicature arises when these maxims appear to be violated. If I ask \textit{how was the movie?} and you respond \textit{well, the popcorn was good}, I understand you didn't enjoy the film. Your seemingly irrelevant response triggers a search for hidden meaning. Similarly, if I say \textit{some students passed the test}, you understand that not all students passed~-- if everyone had passed, saying \textit{some} would violate the maxim of being appropriately informative.

Grice didn't see the maxims as prescriptive rules about how we ought to speak. Rather, he meant them as descriptions of what speakers typically do and what listeners typically expect. When we deviate from these expectations, he observed, we usually do so deliberately, with the intention of creating additional meaning:

\ea
   \ea[]{[Asked \textit{How did you get here?}]\\ \textit{I placed one foot in front of the other, repeatedly transferring my weight from the back foot to the front foot, while maintaining my balance and adjusting my trajectory as needed to avoid obstacles and ensure arrival at my intended destination.}\hfill[too much information]}
   \ex[]{[About a student's excuse] \\\textit{Yes, and I'm the Queen of England.}\hfill[not true]}
   \ex[]{[Asked if they liked the gift] \\\textit{Well, it's Thursday today.}\hfill[not relevant]}
   \ex[]{[Asked for travel directions] \\\textit{Uh, at the thing where you turn, there's this building, you know the one I mean, and then you kind of go that way for a bit.}\hfill[not clear]}
   \z
\z

In each case, the speaker is overtly breaching normal expectations about communication, but in a way that creates clear meaning. The absurdly detailed description of walking mocks the obvious question. The deliberately vague directions suggest irritation or perhaps inability to help. The claim to be the Queen of England marks the student's excuse as equally incredible. And the complete non-sequitur about Thursday effectively communicates dislike of the gift without saying so directly.

These principles of cooperation operate in all languages, but the ways speakers flout them vary considerably. What counts as ``too much information'' in one culture might be ``appropriately thorough'' in another. What seems ``admirably direct'' in some contexts might seem ``unnecessarily blunt'' in others. 

Even routine greetings may work differently: English speakers expect \textit{how are you?} to receive only a brief, positive response. I recently heard of an interaction somebody had with a Russian in which the reply was ``Fine, like everyone in Canada.''

Indonesian speakers use \textit{sudah mandi?} (`have you showered yet?') as a similar kind of greeting that requires no literal answer about bathing (something I didn't at first understand when I was in Ambon). The extent to which speakers mark their non-cooperation varies too. English speakers often use elaborate irony or sarcasm~-- \textit{Oh, GREAT, that's EXACTLY what I needed right now}~-- while speakers of other languages might prefer more subtle signals.

\section{Speech Acts and Performatives}

Consider what happens when a judge says \textit{I sentence you to five years}. These words don't just describe something~-- they do something. The act of sentencing is accomplished through the utterance itself. Or consider a priest saying \textit{I now pronounce you husband and wife}. Again, the words themselves perform the action. These are clear examples of what J. L. \citet{austin1962} called \textsc{performative utterances}.

But speech accomplishes actions in less obvious ways too. When we say things like \textit{I promise to return it tomorrow} or \textit{I apologize for being late}, we're not just making statements~-- we're performing the acts of promising and apologizing. Even apparently simple statements like \textit{it's cold in here} often perform actions~-- in this case, perhaps, requesting that someone close a window.

Austin's key insight was that all utterances perform actions, not just those with explicit performative verbs like \textit{promise}, \textit{apologize}, or \textit{sentence}. When we say \textit{the bus leaves at six}, we might be:

\ea
    \ea[]{Making a prediction\hfill[about when the bus will actually leave]}
    \ex[]{Stating a fact\hfill[about the schedule]}
    \ex[]{Issuing a command\hfill[to a driver]}
    \ex[]{Making a suggestion\hfill[about when to leave]}
    \ex[]{Offering an excuse\hfill[for not staying longer]}
    \z
\z

The same speech act can be realized by different syntactic constructions, depending on the context and the speaker's intentions. This becomes particularly clear when we look at requests/directions. Consider these ways of directing someone to close a window:

\ea
   \ea[]{\textit{Please close the window.}\hfill[imperative directive]}
   \ex[]{\textit{Would you mind closing the window?}\hfill[interrogative directive]}
   \ex[]{\textit{It's freezing in here.}\hfill[declarative directive]}
   \ex[]{\textit{I hereby request that you close the window.}\hfill[declarative directive]}
   \z
\z

Each accomplishes the same basic task~-- trying to get the window closed~-- but does so through different types of speech acts. John \citet{searle1979} organized these acts into five basic categories:

\begin{itemize}
   \item \textbf{Assertives} commit the speaker to the truth of something: asserting, stating, claiming, reporting, describing
   \item \textbf{Directives} try to get someone to do something: requesting, ordering, suggesting, inviting
   \item \textbf{Commissives} commit the speaker to future action: promising, threatening, offering, refusing
   \item \textbf{Expressives} express feelings or attitudes: thanking, apologizing, congratulating, complaining
   \item \textbf{Declarations} bring about changes through utterances: declaring war, performing marriages, naming ships
\end{itemize}

Any utterance may perform three distinct levels of acts. When I say \textit{It's cold}, I perform:
\begin{itemize}
   \item A \textbf{locutionary act}: producing words with a certain meaning
   \item An \textbf{illocutionary act}: making a statement, request, etc.
   \item A \textbf{perlocutionary act}: having some effect on the hearer
\end{itemize}

The locutionary act is simply saying that the temperature is low. The illocutionary act might be requesting that someone close a window. And the perlocutionary act would be actually getting them to close it. Of course, perlocutionary acts can fail~-- my hint about the temperature might be ignored~-- but the illocutionary act of requesting is still performed whether or not it achieves its intended effect.

These distinctions help explain many of the examples we saw in the previous section. When someone responds to \textit{How did you get here?} with an absurdly detailed description of walking, they're performing multiple acts:

\ea
    \ea[]{Locutionary: describing the physical process of walking}
    \ex[]{Illocutionary: criticizing the obviousness of the question}
    \ex[]{Perlocutionary: discourage further obvious questions}
    \z
\z

The speaker says too much to perform an expressive act (criticism) rather than the representative act (describing) that was ostensibly requested. This interaction of cooperative principles and speech acts is central to how we create and interpret meaning in conversation.


\begin{tcolorbox}[title=Practice: Analyzing Speech Acts, colback=white, parbox]
\setlength{\parindent}{1.5em}
\noindent For each of the following exchanges, identify:\\
\phantom{~~~}1. The literal meaning (locutionary act)\\
\phantom{~~~}2. The intended meaning (illocutionary act)\\
\phantom{~~~}3. The expected outcome (perlocutionary act)\\
\phantom{~~~}4. Any relevant cultural considerations
\begin{enumerate}
    \item In response to ``Would you like some coffee?'':\\
    ``I've already had five cups today.''
    
    \item Teacher to student arriving late:\\
    ``Nice of you to join us.''
    
    \item At a dinner party:\\
    Host: ``Would anyone like the last piece?''\\
    Guest: ``Oh, I couldn't.''
    
    \item In an office:\\
    ``Do you have a minute?''
    
    \item On the phone:\\
    ``I should let you go.''
\end{enumerate}

Consider how these exchanges might differ in other languages/cultures you're familiar with. How would you help students understand and produce appropriate responses in English?
\end{tcolorbox}

\section{Politeness and Face}\label{sec:politeness}

Consider these ways of asking someone to move:
\ea
    \ea[]{\textit{Move.}\hfill[imperative]}
    \ex[]{\textit{Could you move, please?}\hfill[modal interrogative]}
    \ex[]{\textit{I wonder \uline{if you'd mind moving a bit}?}\hfill[subordinate interrogative\footnote{See Section \ref{sec:sub-interrog}.}]}
    \ex[]{\textit{You might be more comfortable over there.}\hfill[suggestion]}
    \z
\z

All convey basically the same message, but they differ dramatically in politeness. The imperative is direct to the point of rudeness in most contexts, while the subordinate interrogative is almost apologetic in its indirectness. Understanding these differences requires understanding the concept of \textsc{face}.

\subsection{Face and Face-Threatening Acts}

\textsc{Face}, in pragmatics, refers to our public self-image~-- how we want others to see us and how we see ourselves. It has two aspects:
\begin{itemize}
    \item \textbf{Positive face}: our desire to be liked, appreciated, and approved of
    \item \textbf{Negative face}: our desire for autonomy and freedom from imposition
\end{itemize}

Many speech acts inherently threaten either positive or negative face. Requests threaten negative face by imposing on the hearer. Criticisms threaten positive face by expressing disapproval. Refusals threaten both faces~-- they deny a request (threatening the asker's positive face) while also creating social discomfort (threatening the refuser's positive face).

Consider these refusals:
\ea \label{ex:refusals}
    \ea[]{\textit{No.}\hfill[bare negative]}
    \ex[]{\textit{I'm afraid I can't.}\hfill[regret expression]}
    \ex[]{\textit{I'd love to, but I'm away.}\hfill[positive preface + excuse]}
    \ex[]{\textit{Sounds great! Unfortunately, I'm away. Maybe next time?}\\\hfill[enthusiasm + regret + future possibility]}
    \z
\z

The bare negative in (\ref{ex:refusals}a) does nothing to mitigate the face threat. Each subsequent example adds more face-saving elements: expressions of regret, positive feelings, explanations, and suggestions for future interaction. The complex structure in (d) works to preserve both the speaker's and hearer's face.

\subsection{Politeness Strategies}

English speakers manage face threats in two main ways. Sometimes we try to minimize the imposition on others (negative politeness), and sometimes we try to emphasize social bonds and shared goals (positive politeness).

Consider this exchange between colleagues:

\begin{dialogue}
\item[A] \textit{I was wondering if you might possibly have time to look at this report?}
\item[B] \textit{Let's go through it together~-- we can figure out what needs changing.}
\end{dialogue}

A's request shows classic negative politeness: the past progressive \textit{was wondering}, the past-tense modal \textit{might} (see Section \ref{sec:tense-aspect}), the hedge about possibly having time. Each element works to minimize the imposition. It's as if A is saying ``I know this threatens your autonomy, and I'm doing everything I can to show I respect that.''

B's response, in contrast, exemplifies positive politeness. Instead of maintaining distance, B emphasizes connection and shared purpose through \textit{let's} and \textit{we}. By framing it as a collaborative activity rather than a favour, B reduces the face threat while strengthening social bonds.

These patterns appear throughout English conversation. When we want to maintain distance and respect autonomy, we often embed our main point within layers of possibility and tentativeness:

\ea
    \ea[]{\textit{Please, pass the salt.}\hfill[polite imperative]}
    \ex[]{\textit{Could you pass the salt?}\hfill[past-tense modal request]}
    \ex[]{\textit{Would you mind passing the salt?}\hfill[conditional reaction question]}
    \ex[]{\textit{I don't suppose you'd mind passing the salt?}\\\hfill[negative conditional-reaction confirmation question]}
    \z
\z

\noindent Each layer adds another buffer against the face threat. It's like wrapping a package in multiple layers of soft padding. The more formal or potentially imposing the situation, the more layers we tend to add:

\ea[]{\textit{I was wondering if you might possibly have a moment to discuss my paper at some point this week, if it's not too much trouble?}}
\z

But when we want to emphasize closeness and shared goals, we do the opposite. Instead of wrapping our meaning in protective layers, we actively work to reduce distance:

\ea
    \ea[]{\textit{Let's grab lunch.}\hfill[shared decision]}
    \ex[]{\textit{You must try this cake~-- it's amazing!}\hfill[shared enthusiasm]}
    \ex[]{\textit{I know you're busy, but this will only take a minute.}\hfill[shared understanding]}
    \z
\z

These forms suggest shared perspective and experience. \textit{Let's} presents it as a joint decision rather than one person's. \textit{You must try} presumes shared tastes and interests. \textit{I know you're busy} acknowledges the other person's situation before making a request.

These strategies become particularly clear in how we handle disagreement:

\begin{dialogue}
\item[Negative] \textit{I don't suppose we might want to consider a slightly different approach?}
\item[Positive] \textit{You've got a great point there, and maybe we could also think about...}
\end{dialogue}

The negative strategy maintains distance through hedging and tentativeness, while the positive strategy emphasizes agreement and shared purpose even while disagreeing.

Different cultures and contexts favor different mixes of these strategies. Some Asian cultures, for example, tend to use more negative politeness in professional contexts, maintaining greater distance even between close colleagues. American English speakers often use more positive politeness in professional settings, actively working to reduce distance. Neither approach is inherently more polite~-- they're just different ways of managing face threats.

Even within one culture, individuals and situations vary considerably. Compare these ways of asking someone to quiet down in a library:

\ea
    \ea[]{\textit{Shhh!}\hfill[minimal politeness - emergency]}
    \ex[]{\textit{Could you keep it down a bit?}\hfill[standard negative politeness]}
    \ex[]{\textit{I know you're excited about the topic, but others are trying to study.}\\\hfill[elaborate positive politeness]}
    \z
\z

The appropriate choice depends on factors like:
\begin{itemize}
    \item How severe the disturbance is
    \item What social distance exists between speakers
    \item Whether this is a first request or a repeat
    \item What power dynamics are in play
\end{itemize}

For language learners, the key is recognizing these patterns and understanding when to use them. A request that's appropriately polite between friends (\textit{Hey, grab that for me?}) might seem shockingly rude to a stranger. Conversely, the elaborate politeness appropriate for formal requests (\textit{I wonder if I might trouble you...}) can seem artificially distant among friends.

\subsection{Power, Distance, and Imposition}\label{sec:power}

The choice of politeness strategy depends on three main factors:
\begin{itemize}
    \item \textbf{Power}: relative social status of speaker and hearer
    \item \textbf{Distance}: degree of familiarity between participants
    \item \textbf{Imposition}: how big a request or face threat is involved
\end{itemize}

Consider how a request to borrow a pen might vary:
\ea \label{ex:pen-requests}
    \ea[]{\textit{Gimme a pen.}\hfill[to close friend]}
    \ex[]{\textit{Could I borrow a pen?}\hfill[to classmate]}
    \ex[]{\textit{I wonder if I might borrow a pen?}\hfill[to professor]}
    \ex[]{\textit{Would it be possible to borrow a pen for the exam?}\hfill[to examiner]}
    \z
\z

As power difference and social distance increase, so does the complexity of the request structure, even though the basic imposition remains the same.

The relationship between these factors can be seen in the interaction patterns of modal verbs:
\ea \label{ex:modal-patterns}
    \ea[]{\textit{can/will} \(\rightarrow\) low distance, equal power}
    \ex[]{\textit{could/would} \(\rightarrow\) medium distance/power difference}
    \ex[]{\textit{might/wonder if could} \(\rightarrow\) high distance/power difference}
    \z
\z

\subsection{Cultural Variations}

Languages differ not only in their politeness markers but in their basic assumptions about face. Some cultures emphasize positive face (maintaining harmony), while others focus more on negative face (respecting autonomy). Some treat certain speech acts as more face-threatening than others.

For instance, many Asian languages have elaborate honorific systems marking social relationships, while English relies more on indirect structures and hedging devices. Some languages readily use imperatives where English prefers modal questions. Others may find English's frequent use of \textit{please} and \textit{thank you} excessive.

Consider how different languages handle this classroom exchange:
\ea \label{ex:class-exchange}
    \ea[]{English: \textit{Could you open your books to page 50, please?}}
    \ex[]{Japanese: \textit{50ページを開きなさい} `open page 50'}
    \z
\z

The English version uses a modal question + please, while the Japanese uses an imperative without seeming impolite.

\begin{tcolorbox}[title=Cross-cultural Pragmatic Failure, colback=white]
When learners transfer L1 politeness strategies directly to English, it can lead to pragmatic failure:
\begin{itemize}
    \item Being too direct (\textit{Give me pen})
    \item Being too indirect (\textit{If it's not too much trouble, I'm hoping that maybe...})
    \item Using inappropriate register (\textit{Yo, Professor!})
    \item Missing expected markers (\textit{I want coffee} vs \textit{I'd like a coffee, please})
\end{itemize}

Help students understand that:
\begin{itemize}
    \item Politeness norms vary by culture
    \item Direct translation often doesn't work
    \item Context determines appropriate forms
    \item Practice with authentic situations helps
\end{itemize}
\end{tcolorbox}

\section{Context and Meaning}

Consider what happens when someone says \textit{the door's open}. Without more information, we have no way to know whether this is:

\begin{itemize}
    \item A warning about security
    \item An invitation to enter
    \item A complaint about a draft 
    \item An explanation for noise
    \item A suggestion to close it
\end{itemize}

And yet, in actual conversation, we rarely have trouble determining which meaning is intended. When A says \textit{I'm on my way} and B responds \textit{the door's open}, we understand B is telling A they can walk right in. When someone gets up to investigate a noise and says \textit{oh, the door's open}, we know they've found the source. 

What's interesting isn't just that context determines meaning~-- that's obvious~-- but how quickly and reliably speakers and hearers converge on the same interpretation. Consider a longer example:

\ea
   \textbf{A:} \textit{Did you get my email?}\\
   \textbf{B:} \textit{My laptop died.}\\
   \textbf{A:} \textit{So you need me to send it again?}\\
   \textbf{B:} \textit{No, I saw it on my phone. I'll respond tonight.}
\z

B's first response isn't a simple \textit{no}, which would be true but misleading. Nor is it a full explanation like \textit{yes, I got it on my phone, but I haven't responded because my laptop died and I prefer to write longer emails on a proper keyboard}. Instead, B offers just one piece of relevant context~-- the laptop problem~-- trusting A to work out its implications. A proposes one interpretation, which B then corrects while filling in the rest of the story.

This kind of collaborative meaning-making is the norm in conversation. Consider another example:

\ea
   \textbf{A:} \textit{Coffee?}\\
   \textbf{B:} \textit{I've had five cups already.}\\
   \textbf{A:} \textit{Tea then?}\\
   \textbf{B:} \textit{Water's fine.}
\z

A doesn't ask \textit{are you declining my offer of coffee because you've already had too much caffeine, or is there some other reason you're mentioning your coffee consumption?} The meaning is clear from context. Similarly, B doesn't spell out \textit{no thank you, I don't want tea either because I've already had too much caffeine today, but I would accept a non-caffeinated beverage}. Each turn provides just enough information for the other person to infer the relevant meaning.



\section{Context and Pragmatic Competence}
\subsection{Types of Context}
When we interpret language, we draw on several types of context:

\begin{itemize}
    \item \textbf{Linguistic context}: the surrounding text or discourse
    \item \textbf{Physical context}: the immediate environment
    \item \textbf{Social context}: relationships, roles, and cultural norms
    \item \textbf{Cognitive context}: shared knowledge and assumptions
\end{itemize}

Consider these classroom exchanges:
\ea
    \ea[]{\textit{Could you open that?}\hfill[physical - window/door/book?]}
    \ex[]{\textit{As I was saying before the break...}\hfill[linguistic - previous discourse]}
    \ex[]{\textit{Professor, may I ask a question?}\hfill[social - student role]}
    \ex[]{\textit{You know what this means, right?}\hfill[cognitive - shared knowledge]}
    \z
\z

Each utterance relies on different contextual knowledge for interpretation. Without the physical context, we can't know what \textit{that} refers to. Without the previous discourse, we can't follow what was being said. Without understanding classroom roles and relationships, we might misread the formality level. And without shared knowledge, we can't grasp what \textit{this} means.

For language teachers, understanding these context types helps explain why seemingly simple expressions can cause confusion. When a student asks \textit{What page?} instead of \textit{What page are we on?}, they're relying on shared physical and linguistic context. When they say \textit{Teacher!} instead of \textit{Excuse me, Professor}, they may be transferring social context expectations from their first language.

\subsection{Deixis and Reference}
\textsc{Deixis} refers to words whose meaning depends entirely on context. English uses several types of deictic expressions:

\begin{itemize}
    \item \textbf{Person deixis}: \textit{I}, \textit{you}, \textit{we}, \textit{they}
    \item \textbf{Time deixis}: \textit{now}, \textit{then}, \textit{today}, \textit{last week}
    \item \textbf{Place deixis}: \textit{here}, \textit{there}, \textit{this}, \textit{that}
    \item \textbf{Social deixis}: \textit{sir}, \textit{ma'am}, titles
    \item \textbf{Discourse deixis}: \textit{above}, \textit{below}, \textit{the following}
\end{itemize}

Languages differ significantly in their deictic systems. Japanese, for instance, has a three-way distinction in demonstratives (\textit{kono}/\textit{sono}/\textit{ano}) where English has two (\textit{this}/\textit{that}). Many Asian languages have complex systems of social deixis marking status relationships that English largely lacks.

Consider this natural exchange demonstrating deictic reference:
\ea
    \textbf{A:} \textit{Put it there.}\\
    \textbf{B:} \textit{Here?}\\
    \textbf{A:} \textit{No, there.}\\
    \textbf{B:} \textit{Oh, there.}
\z

Despite using the same words (\textit{here}/\textit{there}), each person understands exactly what location is meant because they share a physical context and can use gesture and gaze to disambiguate the reference.

Teachers' classroom language is particularly rich in deixis:
\ea
    \ea[]{\textit{Look at this example here.}}
    \ex[]{\textit{Turn to the next page.}}
    \ex[]{\textit{Who can answer this question?}}
    \z
\z

Such utterances rely heavily on shared attention and physical co-presence for interpretation.

\subsection{Shared Knowledge and Common Ground}
Successful communication depends on what linguists call \textsc{common ground}~-- the knowledge, beliefs, and assumptions shared by speaker and hearer. Consider these exchanges:

\ea
    \ea[]{\textbf{A:} \textit{The usual?}\\
          \textbf{B:} \textit{Please.}}
    \ex[]{\textbf{A:} \textit{Did you hear about Pat?}\\
          \textbf{B:} \textit{No, what happened?}}
    \ex[]{\textbf{A:} \textit{Remember that thing?}\\
          \textbf{B:} \textit{Which thing?}}
    \z
\z

In (a), both participants must know what ``the usual'' refers to. In (b), they must share knowledge of who Pat is. In (c), B signals that there isn't enough shared context to identify ``that thing.''

Common ground comes from various sources:
\begin{itemize}
    \item \textbf{Physical copresence}: sharing the same environment
    \item \textbf{Linguistic copresence}: sharing the same discourse
    \item \textbf{Community membership}: sharing cultural knowledge
    \item \textbf{Personal history}: sharing experiences
\end{itemize}

Speakers constantly assess and adjust to their audience's knowledge:
\ea
    \ea[]{\textit{You know the coffee shop on Main?}\hfill[checking common ground]}
    \ex[]{\textit{There's this coffee shop on Main~-- it's next to the library}\\\hfill[establishing common ground]}
    \ex[]{\textit{The coffee shop}\hfill[assuming common ground]}
    \z
\z

In classrooms, teachers often need to explicitly build common ground rather than assuming it exists. This might involve:
\begin{itemize}
    \item Establishing shared reference points
    \item Creating shared experiences through activities
    \item Explicitly teaching cultural context
\end{itemize}

\subsection{Pragmatic Failure and Misunderstanding}
When pragmatic expectations aren't met, communication can break down in various ways:

\begin{itemize}
    \item \textbf{Pragmalinguistic failure}: using language that doesn't convey the intended meaning
    \item \textbf{Sociopragmatic failure}: violating social or cultural norms
    \item \textbf{Context failure}: misreading or lacking necessary context
\end{itemize}

Consider these examples of pragmatic failure:
\ea
    \ea[*]{\textit{I have 25 years} (intending `I'm 25-years old')\hfill[pragmalinguistic]}
    \ex[]{[To professor] \textit{Hey bro, got a minute?}\hfill[sociopragmatic]}
    \ex[]{[Without context] \textit{That's what she said!}\hfill[context]}
    \z
\z

For language learners, pragmatic failure can be more serious than grammatical errors because:
\begin{itemize}
    \item Grammar errors are usually recognized as learning errors
    \item Pragmatic errors may be interpreted as rudeness
    \item Feedback on pragmatic errors is often indirect or absent
    \item L1 pragmatic patterns can be deeply ingrained
\end{itemize}

\begin{tcolorbox}[title=Teaching Pragmatic Competence, colback=white]
To help students develop pragmatic competence:
\begin{itemize}
    \item Expose them to authentic language use
    \item Highlight context-dependent meanings
    \item Practice with real-world scenarios
    \item Discuss cross-cultural differences
    \item Provide explicit feedback on pragmatic issues
\end{itemize}
Remember that students' L1 pragmatic knowledge is an asset~-- help them understand how to apply it appropriately in English.
\end{tcolorbox}