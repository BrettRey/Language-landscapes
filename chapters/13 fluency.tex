\chapter{Fluency and flow} \label{ch:fluency}

\epigraph{Before words were work,\\
they were water, were wind, were\\
breath becoming song}{}
\is{fluency|(}

When folks speak about being fluent in a language, what they usually mean is proficient. In this chapter, I want to focus on one special aspect of proficiency: flow. I mean fluent in the way that a dancer is fluent, a flock or birds, a school of fish, Lionel Messi dribbling, Oscar Peterson soloing, or a skilled chef moving through the kitchen. Of course, these are also proficient, but fluency admits the possibility of error, and that's part of the distinction I want to draw between fluency and proficiency. For now, think of a fluent speaker as one whose speech flows, even if it may be ungrammatical at times. A fluent reader may not understand everything, but their reading is effortless. A fluent language user can still be learning the language, but they have automatized enough that they proceed with few stops or hesitations. They can get it out, and they can take it in.\footnote{The ideas in this section are largely due to \citet{Robinson2011}, \citet{Segalowitz2010}, and \citet{Skehan2003}.}

At the same time, what they're saying may not be all that profound, and the language could be quite simple. It's less about the content, and all about the flow.

\section{What is fluency?} \label{sec:what-is-fluency}
\is{fluency!definition|(}\is{complexity|(}\is{accuracy|(}\is{attention|(}\is{proficiency|(}

Recall that technical terms are often drawn from common words but have more or less differing meanings. For us, technically, ``\textsc{fluency} can be described as the ability to process language receptively and productively at a reasonable speed'' \citep[11]{Nation2014b}. Let's be careful about ability though. Linguistic proficiency is purely internal, but our ability to apply it runs smack into the world around us. Knowing more vocabulary, for example, is an internal, linguistic factor that will typically increase your fluency via your proficiency, but being in a noisy distracting environment is an external factor that will typically decrease your ability to process language. In such an environment, you may become slow or halting, which is to say disfluent.

Almost everything in this book so far has been about accuracy and to a lesser extent complexity, but knowing the landscape of a language isn't of much use unless one can traverse it with ease, with fluency.

To better crystallize this perhaps abstract concept, consider Figure \ref{fig:CAFtriangle}, which shows that these three aspects of proficiency~-- complexity, accuracy, and fluency~-- are interrelated through our limited attention span. Just as a piece of string can form different triangles but its length remains constant, our attention can be distributed differently across these three aspects, but at any moment it remains finite. When we focus attention on one aspect, we necessarily have less for the others. A rider going through difficult terrain has to focus on navigating complexity, reducing their capacity for speed. An archer who wishes to shoot accurately will pause to control their breathing, trading off speed for precision. Just as the string can only stretch so far in one direction before pulling back from others, our attention can only extend so far toward one aspect of language before drawing away from the others.

\begin{figure}[ht]
\centering
\begin{tikzpicture}
  % Define nodes for the triangle's vertices
  \node (A) at (0,0) {};
  \node (B) at (4,0) {};
  \node (C) at (2,3.46) {}; % Adjusted for a visually isosceles appearance

  % Draw arrows with labels
  \draw[{Latex[length=3mm]}-{Latex[length=3mm]}] (A) -- (B) node[midway, below] {Fluency/flow};
  \draw[{Latex[length=3mm]}-{Latex[length=3mm]}] (B) -- (C) node[midway, right] {Complexity};
  \draw[{Latex[length=3mm]}-{Latex[length=3mm]}] (C) -- (A) node[midway, left] {Accuracy};
\end{tikzpicture}
\caption{Interrelationship between complexity, accuracy, and fluency}
\label{fig:CAFtriangle}
\end{figure}


The relationships run in reverse. If you would like to be~-- or would like your students to be~-- more fluent, consider reducing the pressure for absolute accuracy. This may feel counterintuitive for teachers and students accustomed to valuing accuracy above all else. Grades often reflect this emphasis, as accuracy is easily measured. And, as English students improve, they're expected to face more complex linguistic challenges. But, as with obstacle courses, those added challenges slow us down. They may make us better climbers, but for increased running speed we need practice without obstacles. If you want to improve your language fluency, find ways to use the language without the constant worry about perfect grammar.

Another way is to reduce the pressure to be complex. Using familiar vocabulary and simpler grammatical structures allows us to focus on the flow, on strengthening fluency itself. 

As we develop proficiency, we gradually automatize certain aspects of language use, requiring less conscious attention. This is like learning to stretch our perimeter~-- the total attention available increases through practice. But during the learning process itself, we often need to focus predominantly on one aspect. Figure \ref{fig:CAFtriangle2} shows how we might distribute attention when prioritizing fluency development.

\begin{figure}[ht]
\centering
\begin{tikzpicture}
  % Define nodes for the triangle's vertices
  \node (A) at (0,0) {};
  \node (B) at (6,0) {};
  \node (C) at (3,1) {}; % Adjusted for a visually isosceles appearance

  % Draw arrows with labels
  \draw[{Latex[length=3mm]}-{Latex[length=3mm]}] (A) -- (B) node[midway, below] {Fluency/flow};
  \draw[{Latex[length=3mm]}-{Latex[length=3mm]}] (B) -- (C) node[midway, right] {Complexity};
  \draw[{Latex[length=3mm]}-{Latex[length=3mm]}] (C) -- (A) node[midway, left] {Accuracy};
\end{tikzpicture}
\caption{Focus more on fluency and less on accuracy \& complexity.}
\label{fig:CAFtriangle2}
\end{figure}
\is{complexity|)}\is{accuracy|)}\is{attention|)}

Of course, the focus should not always and only be on fluency. But it should be on fluency often enough, certainly much more than has traditionally been the case.

And when it is, the internal experience of fluency should be almost a lack of any experience at all. There is no effort, no focusing of attention. Language production or comprehension simply happens the way a leaf floats downstream in a river, without planning, without decisions.

\subsection{Fluency vs proficiency}
\is{fluency!vs proficiency|(}

Forgive me for going on, but I often find that people confound fluency and proficiency, even once the distinction has been clearly set out. And this is understandable; the two are related concepts. Nevertheless, they are distinct. High linguistic proficiency makes fluency much easier over a broad range of situations, tasks, and topics, but someone with low proficiency may still be quite fluent in limited situations. Consider Table \ref{tab:fluency_proficiency}.\footnote{For an example of low proficiency and fluency, see \href{https://youtu.be/48yr93hwvCk?t=21}{https://youtu.be/48yr93hwvCk?t=21}. For low proficiency with much higher fluency, see \href{https://youtu.be/m0UGhSufSJk?t=45}{https://youtu.be/m0UGhSufSJk?t=45}. And for high proficiency with low fluency, see this interview of Ted Kennedy \href{https://youtu.be/e5TkhNWPspM?t=76}{https://youtu.be/e5TkhNWPspM?t=76}.}

\setlength{\extrarowheight}{4pt} % Adjust the 5pt as needed to increase the space
\begin{table}[ht]
\centering
\begin{tabular}{lp{0.35\textwidth}p{0.35\textwidth}}
 & \uline{\textbf{Fluent}} & \uline{\textbf{Proficient}} \\
\textbf{Definition} & Flowing production or processing & High ability to flow, not always actualized \\
\textbf{Scope} & Particular to one skill in a particular situation (e.g., reading) & General skills, grammatical and lexical knowledge \\
\textbf{Consistency} & Varies by occasion & Typically stable or growing \\
\end{tabular}
\caption{Comparison of Fluency and Proficiency}
\label{tab:fluency_proficiency}
\end{table}

Fluency is characterized by flowing production or processing of language. This aspect of linguistic ability is specific to one skill at a time, such as speaking, reading, or listening, and is highly variable depending on the situation. For instance, a person might be more fluent in casual conversation but less so in a formal presentation. Fluency is marked by a natural ease in using language, where communication occurs with minimal hesitations or disruptions. This doesn't necessarily imply complete accuracy or complexity in language use. As expert language users, it's important to note that you and I may not always exhibit fluency, especially in unfamiliar contexts or when using specialized terminology.

On the other hand, proficiency encompasses a broader range of skills and knowledge. It includes a general ability in various aspects of language, from grammar and vocabulary to the ability to understand and produce complex language structures. Proficiency tends to be more stable or progressively improving over time. It is a measure of one's overall capacity in a language, encompassing both the depth and breadth of linguistic understanding. Anyone able to read this book in English is proficient, meaning you have a comprehensive grasp of English, including its rules, nuances, and complexities.

The relationship between fluency and proficiency is dynamic. While high proficiency can facilitate fluency, it doesn't automatically guarantee it. Conversely, a speaker might demonstrate fluency in certain contexts despite having lower overall proficiency.
\is{fluency!vs proficiency|)}\is{fluency!definition|)}\is{proficiency|)}

\section{Dimensions of Fluency}
\is{dimensions of fluency|(}

Fluency, as I've hinted, is not a monolithic construct. It's a multi-faceted beast, with tentacles reaching into various domains of language use. When we talk about fluency, we're really talking about a constellation of fluencies, each with its own unique characteristics and challenges. Let's untangle this knot and look at each dimension in turn.

\subsection{Speaking Fluency}
\is{speaking fluency|(}\is{automaticity|(}\is{proceduralization|(}

The most obvious manifestation of fluency (or disfluency) is in speech. We've all experienced those magic moments when the words just flow, when our tongues seem to have a mind of their own, when we're in the zone. But we've also all had those other moments, the ones where we stumble, hesitate, grope for words, start and stop like a sputtering engine.

Fluent speech employs a smooth, rapid delivery with few pauses, hesitations, or false starts. It's the difference between a jazz musician effortlessly riffing and a novice haltingly plunking out a tune. Consider the following examples:

\ea
    \ea[]{\textit{So then I said to him, I said, ``Look, if you're not happy with the arrangement, we can always renegotiate. But we need to do it soon, before the project deadline.'' And he just sort of nodded and said, ``Yeah, okay, let's talk tomorrow.'' So, I think it's going to work out.}}\label{ex:fluent}
    \ex[]{\textit{So then I said to him, I said, um, ``Look, if you're not, uh, happy with the, the arrangement, we can always, you know, rene\dots, renegotiate. But we need to do it, uh, soon, before the, um, project deadline.'' And he just sort of, uh, nodded and said, ``Yeah, okay, let's, let's talk tomorrow.'' So, I, I, I think it's, uh, it's going to work out.}}\label{ex:disfluent}
    \z
\z

The first example (\ref{ex:fluent}) flows smoothly, with few pauses or hesitations. The speaker seems to know exactly what they want to say and how to say it. In contrast, the second example (\ref{ex:disfluent}) has a bunch of filled pauses (\textit{um}, \textit{uh}), repeated words and phrases, and false starts (\textit{rene}--, \textit{renegotiate}). The speaker seems to be grasping for words, struggling to formulate their thoughts in real time. Not that this is bad. In fact, it's quite normal. It's just less fluent than (\ref{ex:fluent}).

But fluency isn't just about speed or smoothness. It's about \textsc{automaticity}, the ability to retrieve and articulate words and phrases without conscious effort. It's about \textsc{proceduralization}, the gradual transformation of declarative knowledge (``knowing that'') into procedural skill (``knowing how'').\is{cognitive load}

Think about the first time you tried to drive a car (if you drive). Everything required conscious effort: checking the mirrors, signaling, shifting gears, monitoring your speed. But over time, these actions became automatic, proceduralized. You could do them without thinking, freeing up cognitive resources for higher-level tasks like navigating or conversing with passengers. If you're like me, you may have set out to drive one place and instead found yourself on your regular route to work, a route you've taken so many times, it has become automatic.

The same is true of speaking a language: as lower-level processes like word retrieval and articulation become more automatic, more fluent, you're able to devote more attention to the content and nuances of your message.

And our main cognitive limitation is that: attention. Linguists used to be very focused on trying imagine grammars that use as little memory as possible on the assumption that our memory for individual words and expressions was quite limited. The thinking was ``why remember \textit{jump}, \textit{jumped}, \textit{jumps}, \textit{jumping} along with all the other regular verb forms when you can just remember \textit{jump} and then produce the rest on the fly with rules?'' It seems more likely that memory retrieval may sometimes be a problem, but ``storage capacity'' is practically unlimited \citep{Erman2000, JackendoffAudring2018}.

\subsection{Listening Fluency}
\is{listening fluency|(}

But speaking is only half the conversation. The other half is listening, and it comes with its own fluency demands. Listening fluency is about processing language in real-time, about keeping up with the flow of speech as it comes at you. It's about parsing the stream of sound into meaningful chunks, and interpreting the message, all on the fly. For fluent listeners, this all happens effortlessly, automatically. They're able to follow the thread of conversation, to anticipate where it's going, to read between the lines.

Obviously, listening is somewhat different from speaking in that we control very little of what happens. The contents, form, volume, and speed of the text~-- I'll call it a text even though it's not written~-- are all dictated by the speaker. We're hardly even in control of our ability to understand it. If someone within earshot is speaking a language you know, you're generally going to understand what they're saying, whether you want to or not. Speakers can, in a sense, hijack our brains.

Because of the unique elements of speech, listening fluency is often reduced to language comprehension, though the two are not exactly the same. You've almost certainly listened to an English speaker with an unfamiliar accent. In such cases, your listening comprehension may drop significantly below your potential language comprehension. Similarly, there are people who have learned English entirely through print, who can read with some fluency, but who cannot parse the speech stream. They're unfamiliar with how the language sounds.

If you have trouble picturing this, pull up a subtitled YouTube video in a language you don't understand. The subtitles should be a direct transcription, not a summary or a translation. Then, turn the subtitles off and count the words. That's all. It doesn't have to be a long segment. Even just a few seconds is fine. Just try to count the number of words, and when you're done, turn the subtitles back on and check. You'll probably find that you could not reliably separate the text into individual words. When we speak, in any language, we just run the words all together.

Because of this, most beginning students can benefit from listening while reading. In other words, they should have access to recordings or computer-generated text-to-speech output while they read. Or, in a classroom, a teacher can read to them as they follow along. This provides visual clues about word boundaries and sometimes helps word recognition.

The main problem, though, is vocabulary knowledge. You can't recognize a word you don't know. Even once someone is able to parse the speech stream reasonably well, they will still be unfamiliar with literally thousands of words and expressions. 

The role of context in listening fluency is often overstated. Fluent listeners do make use of contextual cues, especially in understanding reference. They use their knowledge of the topic, the speaker's intentions, and the overall discourse to fill in gaps and make predictions about what will be said next. But comprehension is mostly driven by fluent word recognition for which guessing and inference is no substitute.
\is{listening fluency|)}

\subsection{Reading Fluency}\label{sec:reading-fluency}
\is{reading!fluency|(}\is{Simple View of Reading|(}\is{decoding|(}\is{language comprehension|(}

Then there's reading fluency. On the surface, it seems a world apart from the fast-paced back-and-forth of conversation. But under the hood, many of the same processes are going on with one extra twist: word decoding.

In the Simple View of Reading \citep{Gough1986}, fluent reading depends on two factors: 1) smooth, accurate word decoding/recognition, and 2) language comprehension.

\begin{equation}
    RC = D\times LC
\end{equation}

In this formula, $RC$ is reading comprehension, $D$ is decoding ability, and $LC$ is language comprehension. In other words, fluent reading is mostly the same as fluent listening, as long as your reading fluency is not degraded by slow or inaccurate decoding.

Both fluent listening and fluent reading are about comprehension, about constructing a mental model of the text as you go, about following the flow of ideas, the arc of the argument or narrative. Fluent readers make it look easy, skimming across the surface of the text, dipping in and out as needed. But for struggling readers, it's a laborious slog. If the problem is strictly a reading problem, then it's because they're bogged down by an inability to automatically decode the individual words. When decoding is effortful, it requires at lot of attention, and attention is what is limited. With more attention going to figuring out what is printed on the page, little is available to integrate ideas and follow the narrative.

Of course, $LC$ may be low even if $D$ is high. For example, I can decode in Ojibwe, Indonesian, Croatian, and Finnish, all of which use the Roman alphabet, with relative fluency, but I know hardly any vocabulary or grammar in those languages, and so my language/listening comprehension would be near zero. As a result, my reading comprehension would similarly be near zero. \il{Ojibwe}\il{Indonesian}\il{Croatian}\il{Finnish}

Because English has a somewhat opaque orthography, few students are able to fully transfer their existing decoding ability from another language, even if it has a Roman-alphabetic script. That said, most people who read in a Germanic or Romance language will have a strong head start in decoding English, and anyone who is literate in a Roman-alphabetic script will also have be able to transfer some skills. Those who have little or no experience reading alphabetic scripts will necessarily have low reading fluency, even if they have good listening fluency in English.
\is{language comprehension|)}\is{decoding|)}\is{Simple View of Reading|)}\is{reading!fluency|)}

\subsection{Writing Fluency}
\is{writing!fluency|(}

Finally, there's writing fluency. In some ways, it's the inverse of reading. Instead of taking in and making sense of someone else's words, you're spinning your own out of the ether. But the core challenges are similar: translating thoughts into language, crafting phrases and sentences that capture your meaning, building a coherent structure.

A major difference between writing and speech, though, is that writing is often expected to be more thoughtful, developed, organized, and complex. All of those require attention, that most limited of cognitive resources, and typically reduce fluency.\is{cognitive load}

For less fluent writers, crafting even a simple paragraph can be a start-stop affair, a matter of grinding out a few words or sentences, then pausing to reread, revise, rethink. They may struggle to find the right words, to vary their sentence structures, to maintain a consistent tone or point of view. The end result often lacks coherence and flow, reading more like a patchwork of ideas than a seamless narrative or argument.
\is{writing!fluency|)}

\subsection{Putting It All Together}

These fluencies, though distinct, are deeply intertwined. Progress in one often fuels progress in the others. As your spoken vocabulary grows, so does your listening comprehension. As your reading fluency develops, so does your sense of what ``sounds right'' in your own writing.

And of course, the more you engage with a language across all these modalities, the more automatic and effortless it all becomes. Every conversation, every page read, every paragraph written, is another step on the path to overall fluency.

But the relationships are complex and sometimes counterintuitive. You can be a fluent speaker but a halting reader, a prodigy at writing but tongue-tied in conversation. Fluency is not a single skill but a tapestry of skills, woven together in unique ways for each individual.

And that, in a nutshell, is the multidimensional nature of fluency. It's a complex construct, but one that's essential to grasp if we want to effectively foster and assess it in our students. So keep these dimensions in mind as we move on to exploring the factors that influence fluency development and the strategies we can use to nurture it in our classrooms. The journey to fluency is a long and winding one, but with a clear understanding of the destination, we can help our students navigate it with confidence and skill.
\is{dimensions of fluency|)}





\subsection{Fluency across skills}
\is{fluency!across skills|(}

We most naturally associate fluency with spoken fluency, but the concept also applies to reading, writing, speaking, and listening.\footnote{And signing when it comes to sign languages. \il{sign languages}} If you read fluently, you get caught up in the story and ideas, while the words on the page recede from awareness. It's effortless. If your reading is disfluent, though, you stop and think, you reread, you may not get the ideas, and you may have to look up words. It's effortful. I regularly experience both kinds of reading.

A fluent writer experiences a smooth flow of ideas during the act of writing. This process involves being able to articulate thoughts with ease, without being hindered by constant self-editing or grappling for the right words. It's about the writer's ability to let their ideas spill onto the page in a coherent and consistent manner, almost as if they're taking dictation from a muse. This doesn't necessarily mean that the first draft is perfect or doesn't require editing. However, in fluent writing, the initial phase, at least, is fluid and unencumbered by excessive pauses or blocks.

In contrast, a disfluent writing process involves frequent stops, struggles to find the right words, constant revisions while writing, and a general feeling of laboriousness. This can result in a writing experience that is fragmented and frustrating, often making it difficult to maintain a consistent train of thought or a coherent argument. Writing is probably the skill in which most English speakers have the least experience of fluency, even considering how little we write relative to the other skills. Little of this book, for instance, was written fluently.

It is in speaking, perhaps, that fluency is most easily appreciated. It involves the ability to express thoughts spontaneously and coherently without undue pauses, hesitations, or self-corrections. A fluent speaker can navigate conversations smoothly, adjusting their speech in real-time to suit the context and audience. This doesn't mean they never make mistakes; rather, they can continue speaking fluidly despite occasional errors. Disfluent speech, on the other hand, is typically slow and is marked by frequent stops, filler words, and a lack of smooth progression in ideas, which can impede effective communication.

Listening fluency is harder to conceptualize because the listener is not in control of the flow. It refers to the ability to understand spoken language in real time. Fluent listeners can comprehend the gist and specific details of what's being said, even when the speech is fast or involves complex ideas. They can also interpret tone, nuance, and implied meanings without much conscious effort. In contrast, a less fluent listener may frequently misinterpret or miss out on parts of the conversation, requiring additional clarifications or repetitions. Disfluent listening likely involves giving up and tuning out.

In the early 1990s, during my time in Tokyo, I had diligently practiced a handful of Japanese phrases, honing them to a level where I could articulate them with a semblance of fluency. This was in the time before cell phones with translators and GPS in everyone's pockets, in a city where finding a building in an unfamiliar location can be quite a challenge. Frequently, I found myself relying on these rehearsed phrases to ask for directions. And it was here that my limited display of spoken fluency would collide with the reality of my listening disfluency. Because, though I could make myself understood, the responses I elicited, typically rapid and detailed, far exceeded my listening fluency. \il{Japanese}

These interactions usually ended with me nodding and smiling in feigned understanding, often little wiser about the directions I had been given. I would then set off in the general direction indicated and hope for the best.
\is{fluency!across skills|)}

\section{Measures of fluency}
\is{measures of fluency|(}

\subsection{Speaking and listening}
\is{speech rate|(}

A variety of tasks may be used to measure speaking fluency, Spontaneous speech tasks: Participants might be asked to describe a picture, tell a story, provide directions, or engage in a conversation based on a specific prompt. Researchers analyze the transcript afterward, looking at elements like words per minute, pauses, hesitations, and grammatical complexity.
Retelling tasks: Participants might listen to a short story or passage and then be asked to retell it in their own words. This measures their ability to process information and produce smooth, connected speech.
Timed reading tasks: Participants may be asked to read a passage aloud as quickly and accurately as possible. This provides insight into both reading fluency and speaking fluency.

The average speaking rate for native English speakers in the United States is around 150 words per minute (wpm) for general conversation. But this rate varies depending on the context and purpose of the speech. For instance, presentations typically have a speaking rate of 100--150 wpm, while conversational speech is 120--150 wpm. Audiobooks are read at a pace of 150--160 wpm, which is considered the upper range that people comfortably hear and vocalize words. Radio hosts and podcasters also speak at a rate of 150--160 wpm. In contrast, auctioneers and commentators can speak much faster, ranging from 250 to 400 wpm.

Speech rate is not a static measure and can be influenced by various factors such as nervousness, urgency, mental fatigue, the complexity of the words, the complexity of content, verbal pauses, and event-driven pauses. These factors can affect the overall speaking rate and vary among individuals based on their environment, upbringing, culture, and even geographical location. For example, people in London typically speak faster than those from Yorkshire. Additionally, if English is not the speaker's first language, they usually speak a bit slower

Interestingly, speech rate is strongly correlated with vocabulary knowledge, suggesting that one way to increase speech rate is simply to learn a lot of useful vocabulary (See, \ref{sec:vocab-size-coverage}).

Along with the obvious implications for speaking, this has implications for listening. A recent study of recorded lecture comprehension in competent anglophones, which employed audio, slides, and video of the speaker, found no significant difference between the $1\times$ viewing speed (about 160 wpm) and $2\times$, and only minor degradation at $2.5\times$ \citep{Murphy2021}.
\is{speech rate|)}

\subsection{Reading}
\is{reading!speed|(}

\citet{Carver1992} found that for proficient adult readers of English, the optimal reading speed is about 300 words per minute, with a roughly linear increase in average reading speed from about 120 words per minute in grade 2. When the adults are reading for learning, their speed slows to about 200 words per minute, and when reading very deliberately, for example when knowing they will be tested on the details of a passage, they commonly slow to rates of roughly 140 words per minute. For folks reading in a second language, it's typically much slower. Students studying in an English-medium university or high-school and able to score well on a reading comprehension test may read in the 80--120 words-per-minute range \citep[292]{Grabe2009}. Japanese first-year university students may start reading simple English stories as slowly as 80 words per minute \citep{McLean2017}. Reading out loud is typically much slower than silent reading at about 170 words per minute, comparable to general conversation speeds.
\is{reading!speed|)}

\subsection{Writing}
\is{writing!speed|(}

There is no lower bound on typical writing speeds. People who deem themselves to be engaged in writing, may produce not a single written word for long periods of time. This is because we often include planning and researching as part of our writing. And that's perfectly fair.

At the upper bound, some professional romance writers can produce 2,500 words per hour, or about 40 words per minute \citep{Ha2017}. But that's highly unusual and can only be sustained for an hour or two per day.

In a more typical case, \citet{VanWaes2015} observed Master's students at the University of Antwerp who were studying foreign languages at the Common European Framework of Reference for Languages B2 level. They compared their first-language writing speed to that of the language they were studying. They were to write a short text about a recent experience, such as what they'd done on the weekend. Their first language speed was about 31.5 words per minute, or 34.5 if you count things they ended up deleting. The second language writing was about four words per minute slower. Overall, this is less than one quarter of typical speaking speeds.
\is{writing!speed|)}

\section{Factors affecting fluency}
\is{vocabulary knowledge|(}

Overall, the factor most highly correlated with fluency across all skills is vocabulary knowledge.  In reading about 50\% of the variance in fluency is accounted for by vocabulary knowledge \citep{Grabe2010}. As I've said, listening fluency is a bit hard to conceptualize, but vocabulary knowledge accounts for roughly 30--45\% of the variance in listening comprehension \citep{Zhang2022a}, which is roughly the same as for reading. In speaking, the relationship is more modest, with vocabulary knowledge accounting for around 15\% of spoken fluency \citep{Uchihara2019a,Liu2020}.

Writing is so variable that it's difficult to identify correlations. There are many bottlenecks to writing from the speed of your printing/writing/typing, to your level of distractedness. Remember that in a conversation, there's a good deal of social pressure to participate, while in writing, if you start daydreaming, you're typically the only one who can get the writing restarted. Obviously, proficiency is a limiting factor.
\is{vocabulary knowledge|)}

\subsection{Cognitive processing and affective factors}
\is{affective factor|(}\is{repair!fluency|(}\is{cognitive load}

Cognitive processing and affective factors may have a stronger influence on L2 repair fluency than L1 repair patterns \citep{Peltonen2024}.

Repairs are generally more common in the L2 than L1, but individual preferences for repair types vary. Some people rely more heavily on false starts or repetitions, while others favor replacements or reformulations. For language teachers, this suggests the need for a nuanced understanding of repairs in student speech. Rather than viewing all repairs as disfluencies to be eliminated, teachers might consider how different repair types serve as coping mechanisms for students as they navigate the demands of L2 speech production. By acknowledging the role of anxiety and attentional control in repair behavior, teachers can create more supportive classroom environments that encourage students' fluency development.

At the same time, it's crucial to avoid overemphasizing repair reduction at the expense of fostering students' overall communicative competence and confidence.
\is{repair!fluency|)}\is{affective factor|)}

\is{fluency|)}